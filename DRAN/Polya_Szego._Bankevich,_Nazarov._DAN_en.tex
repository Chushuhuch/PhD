\documentclass[
aps,%
12pt,%
final,%
notitlepage,%
oneside,%
onecolumn,%
nobibnotes,%
nofootinbib,% 
superscriptaddress,%
noshowpacs,%
centertags]%
{revtex4}
%\documentclass[12pt,russian]{article}
%\renewcommand{\baselinestretch}{1}
\renewcommand{\baselinestretch}{2}

\begin{document}

\large

\title{A generalization of the P\'olya-Szeg\"o inequality for one-dimensional functionals}

\author{\firstname{S.~V.}~\surname{Bankevich}}
\email{Sergey.Bankevich@gmail.com}
\affiliation{%
St. Petersburg State University
}%
\author{\firstname{A.~I.}~\surname{Nazarov}}
\email{al.il.nazarov@gmail.com}
\affiliation{%
St. Petersburg State University
}%

\maketitle

\begin{center}
\small Presented by Academician I.A.~Ibragimov 12.11.2010 �.
\end{center}

\section{Introduction}
It is well known that for a nonnegative measurable function $u$, defined in
$\Omega = [-1, 1]$, the so-called layer cake representation holds (see, 
for example, \cite[Ch. 1]{LL}):
$$u(x) = \int_0^\infty \chi_{\{x \in \Omega :\ u(x) > t \}}\, dt.$$ 
For a measurable set $E \subset \Omega$ let us define the sets 
$$ E^* = [1 - \abs{E}, 1], \qquad 
E^\circ = (-\frac{\abs{E}}{2}, \frac{\abs{E}}{2}).$$ 
We define the monotone rearrangement and the Steiner symmetrization of a function $u$ as follows 
(see \cite[Ch. 10]{HLP}): 
$$u^*(x) = \int_0^\infty \chi_{\{x \in \Omega :\ u(x) > t\}^*}\, dt; \qquad 
u^\circ = \int_0^\infty \chi_{\{x \in \Omega :\ u(x) > t\}^\circ}\, dt.$$
It is easy to see that 
\begin{equation} 
\label{relation} 
u^*(x) = u^\circ(\frac{x + 1}{2}). 
\end{equation} 

Let $F: \mathbb R \times \mathbb R \to \bar{\mathbb R}_+$ be a continuous 
function which is even and convex with respect to the second argument.
Also let $a: \Omega \times \mathbb R \to \bar{\mathbb R}_+$ be a continuous function. 
We introduce the functional 
\begin{equation} 
\label{functional} 
I(u) = \int_\Omega F(u(x), \abs{a(x, u(x)) u'(x)}) dx
\end{equation} 
defined on the set ${\cal M}$ of functions $u \in \W(\Omega)$, for which 
the integral in (\ref{functional}) converges. 

Monotonicity properties of the functionals under rearrangements were studied
by a number of authors (see e.g. \cite{Kawohl} and references therein). In particular, 
it is well known that if $a \equiv const$, then 
\begin{subequations} 
\begin{equation} 
\label{toprovemon} 
I(u^*) \leqslant I(u).
\end{equation} 
If, in addition, $u(\pm 1) = 0$, then 
\begin{equation} 
\label{toprovesymm} 
I(u^\circ) \leqslant I(u). 
\end{equation} 
\end{subequations} 

In the paper \cite{Br} the behavior of the functional (\ref{functional}) and its 
multidimensional analogue under the Steiner symmetrization was studied. Namely, it was
shown that if $a$ is even and convex in $x$, then the inequality 
(\ref{toprovesymm}) holds 
for any piecewise linear function $u$. Further, the author of \cite{Br} claims that
for $u \in {\cal M}$ this inequality can be obtained by a simple approximation. 
However, the Lavrentiev phenomenon (see e.g. \cite[Ch. 4]{BGG}) could arise
for the functional (\ref{functional}). This possibility was not taken into
account and was not excluded in \cite{Br}, and thus, the proof of
(\ref{toprovesymm}) for a general case contains a gap.

In Section \ref{symmSection} of this paper, we establish the absence of the
Lavrentiev phenomenon for functionals of the
form (\ref{functional}) with a weight $a$ which is even and convex in $x$. Thus, 
we prove the inequality (\ref{toprovesymm}) in full generality. Note that for 
the multi-dimensional case the problem remains open.

The bulk of our paper is devoted to the more difficult inequality 
(\ref{toprovemon}). For Lipschitz functions $u$, we prove it under only
necessary conditions on the weight function. For arbitrary function
$u \in {\cal M}$ we succeed in doing it only under a certain additional restriction.

In our proof we modify the construction from \cite{ASC}. In that paper
the absence of the Lavrentiev phenomenon was proved for functionals (\ref{functional})
with $a \equiv const$ and a general function $F$.

Note that in the paper \cite{Lan} the inequality (\ref{toprovemon}) was considered
under the additional restriction $u(-1) = \min \limits_\Omega u $. However,
both the class of functions $F$ and the class of admissible weights in that
paper are non-optimal. Our results for this case formulated at the
end of Section \ref{sobolSection} are stronger than the ones in \cite{Lan}. 

\section{Necessary conditions on the weight} 

Recall that the function $a$ is assumed to be nonnegative. 

\begin{prop} 
Suppose the inequality (\ref{toprovemon}) holds for any strictly 
convex $F$ and any piecewise linear function $u$. Then the weight $a$ is even in
$x$ and satisfies
\begin{equation} 
\label{almostConcave} 
\forall s \leqslant t \in [-1, 1], \quad \forall u \in \mathbb R 
\qquad a(s, u) + a(t, u) \geqslant a(1 - t + s, u). 
\end{equation} 
\end{prop} 

\rem If $a$ is concave and even in $x$, then it satisfies 
(\ref{almostConcave}). 

\begin{prop} 
\label{period} 
If for the function $a$, satisfying 
(\ref{almostConcave}), there exist $x_0 \in [-1, 1] $ and $u_0 \in \mathbb R$
such that $a(x_0, u_0) = 0$, then $a(1, u_0) = 0$ and function
$a(\cdot, u_0)$ is periodic on $[x_0, 1]$. 
\end{prop} 

\rem Let the assumptions of Proposition \ref{period} hold.
If, in addition,
the function $a(\cdot, u_0)$ is even, then it is periodic on the whole
interval $[-1, 1]$, and the period is a divisor of $x_0 +1$.

\section{The result for piecewise Lipschitz functions} 

\begin{defin}
Let $a$ be a continuous function defined on $[-1, 1] \times \mathbb R$, and let
$a(\cdot, u)$ have a finite number of zeros for any $u$. We define ${\cal W}(a)$ 
as a set of functions $u$, piecewise absolutely continuous on $[-1, 1]$ such that 
if $\hat{x}$ is a point of discontinuity, then $a(\hat{x}, \cdot) \equiv 0$ 
on the interval $[u(\hat{x}-), u(\hat{x }+)]$. We say that a sequence 
$u_n \in {\cal W}(a)$ converges to $u \in {\cal W}(a)$, if $u_n$ are absolutely 
continuous on every interval $\Omega_k$, where $u$ is absolutely continuous and
$u_n \to u$ in $\W(\Omega_k)$ for every $k$.
\end{defin}

For functions $u \in {\cal W}(a)$ we define the integral in (\ref{functional}) as 
the sum of integrals taken over the intervals where $u$ is absolutely continuous.
We also define the set $\widetilde{{\cal M}}(a)$ of functions 
$u \in {\cal W}(a)$, for which this integral converges. 

Obviously, any piecewise Lipschitz function $u \in {\cal W}(a)$ belongs to
$\widetilde{\cal M}(a)$. 

\begin{thm}
\label{linthm}
Let the weight function $a$ be even in $x$ and satisfy
(\ref{almostConcave}). Then, if $u \in {\cal W}(a)$ 
is piecewise linear between any two discontinuity points, then 
$u^* \in {\cal W}(a)$, $u^*$ is piecewise linear between the discontinuity points, and 
the inequality (\ref{toprovemon}) holds.
\end{thm} 

Theorem \ref{linthm} can be proved by direct computation using 
Proposition \ref{period}.

\begin{prop} 
\label{lowersemi} 
The functional $I(u)$ is weakly semi-continuous from below in ${\cal W}(a)$. 
\end{prop} 

\begin{thm} 
\label{lipthm} 
Let the weight function $a$ satisfy the assumptions of Theorem \ref{linthm}. Then for 
any piecewise Lipschitz function $u \in {\cal W}(a)$, the inequality (\ref{toprovemon}) holds.
\end{thm} 

\begin{proof} 
For any piecewise Lipschitz function $u$ there exists a sequence of
piecewise linear functions $u_n$, such that
\begin{equation} 
\label{convergence} 
u_n \to u\ \text{in}\ {\cal W}(a)\quad \text{and}\quad I(u_n) \to I(u). 
\end{equation} 
Then the result of \cite[Theorem 1, part 2]{Br}, the formula (\ref{relation}) and Theorem
\ref{linthm} imply $u_n^*\rightharpoondown u^*$. Applying Proposition 
\ref{lowersemi} completes the proof. 
\end{proof} 

\section{Generalization to the Sobolev functions} 
\label{sobolSection} 

\begin{thm} 
\label{Wapprox} 
Let the weight function $a$ satisfy the assumptions of Theorem \ref{linthm},
and let it decrease
in $x$ on $[0, 1]$. Then for any function $u \in \widetilde{\cal M}(a)$ 
the inequality (\ref{toprovemon}) holds.
\end{thm} 

To prove this, we extend the construction from \cite[Theorem 2.4]{ASC} to the case of
monotone weight. This allows us to construct a sequence of Lipschitz 
functions $u_n$, satisfying (\ref{convergence}). The result follows from 
Theorem \ref{lipthm} and Proposition \ref{lowersemi}. 

\begin{thm} 
\label{main} 
Let $a(x, u) = \mathfrak{a}(x) \mathfrak{A}(u)$, where $\mathfrak{a}$ is even 
and satisfies (\ref{almostConcave}). Then for any function 
$u \in \widetilde{\cal M}(a)$ the inequality (\ref{toprovemon}) holds.
\end{thm}

The theorem is proved in several steps.
First, (\ref{toprovemon}) is proved under the additional assumption that 
$\mathfrak{a}$ is piecewise monotone and is constant in neighbourhoods
of every non-zero local minimum. Then (\ref{toprovemon}) is proved for all
piecewise monotone functions $\mathfrak{a}$,
and, finally, without additional restrictions. 

\medskip 

Now let the function $u$ be subject to the additional condition 
\begin{equation} 
\label{restrictU} 
u(-1) = \min\limits_{[-1, 1]} u \geqslant 0. 
\end{equation} 

\begin{prop} 
If the inequality (\ref{toprovemon}) holds for any piecewise 
linear function $u$ subject to the condition (\ref{restrictU}) and for any strictly 
convex $F$, then the weight $a$ satisfies the condition (\ref{almostConcave}). 
\end{prop} 

\rem If $a(\cdot, u)$ is a nonincreasing function, then it satisfies 
(\ref{almostConcave}). 

\begin{thm} 
\label{restthm} 
Let any of the two following conditions hold:
\begin{enumerate} 
\item $a(\cdot, u)$ is a nonincreasing function for all $u$.
\item $a(x, u) = \mathfrak{a}(x) \mathfrak{A}(u)$, where $\mathfrak{a}$ 
satisfies (\ref{almostConcave}). 
\end{enumerate} 
Then for any function $u \in \widetilde{\cal M}(a)$, satisfying the condition 
(\ref{restrictU}), the inequality (\ref{toprovemon}) holds.
\end{thm} 

\medskip 

In the paper \cite{Lan} the inequality (\ref{toprovemon}) is considered
for functions subject to (\ref{restrictU}) and for functionals with the
integrand of the form $F(\abs{u'}) a(x)$. 
It is assumed in \cite{Lan} that $a$ is a nondecreasing function. Note that for 
$F(t) = t^p$ the
integrand in \cite{Lan} can be reduced to the form (\ref{functional}), and the restrictions
on the weight in Theorem \ref{restthm} are considerably weaker than in \cite{Lan}.
Besides, the additional restrictions on the growth of $F$ are imposed in \cite{Lan}.

\section{The inequality for the symmetrization}
\label{symmSection}

Recall that in \cite{Br} the inequality (\ref{toprovesymm}) was proved
for Lipschitz functions $u$, satisfying $u(\pm 1) = 0$, if the weight $a$ is
even and convex in $x$. 

\begin{prop} 
If the inequality (\ref{toprovesymm}) holds for any 
piecewise linear function $u$ with $u(\pm 1) = 0$, then the weight function
$a(\cdot, u)$ is even and convex in $x$.
\end{prop} 

\begin{thm} 
\label{symm} 
Let the function $a(\cdot, u)$ be even and convex for all $u \in \mathbb R$. 
Then for every $u \in \Wf(-1, 1)$ the inequality (\ref{toprovesymm}) holds.
\end{thm}

Similarly to Theorem \ref{main}, we proceed in two steps.
First, we prove (\ref{toprovesymm}) for the functions $a$, which are constant in $x$
in some neighborhood
of zero, and then we do it for the general case.

\begin{acknowledgments}
This work was supported by RFBR grant 09-01-00729 and Federal Scientific and Innovation Program.
\end{acknowledgments}

\begin{thebibliography}{99}
\large
\bibitem{LL} E.~Lieb, M.~Loss. Analysis. Second edition. Providence, Rhode Island; AMS, 2001. xxii+346 p.
\bibitem{HLP} G.~Hardy, J.~E.~Littlewood, G.~P\'olya. Inequalities. Second edition. Cambridge;
Cambridge University Press, 1952. xii+324 p.
\bibitem{Kawohl} B.~Kawohl. Rearrangements and convexity of level sets in PDE,
Lecture notes in mathematics 1150. Berlin; Springer Verlag, 1985. 134 p.
\bibitem{Br} F.~Brock. // Calc. Var. 1999. V.~8. P.~15--25.
\bibitem{BGG} G.~Buttazzo, M.~Giaquinta, S.~Hildebrandt. One-Dimentional Variational Problems.
An Introduction. New York; Oxford University Press, 1998. viii+262 p.
\bibitem{ASC} G.~Alberti,~F.~Serra~Cassano // Ser. Adv. Math. Appl. Sci. 1994.
V.~18. P.~1--17.
\bibitem{Lan} R.~Landes // Math.~Nachr., V.~280, Iss.~5--6. P.~560--570.

\end{thebibliography}
\end{document}
