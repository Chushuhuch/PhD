
\section*{Общая характеристика работы}

\newcommand{\actuality}{\underline{\textbf{\actualityTXT}}}
\newcommand{\progress}{\underline{\textbf{\progressTXT}}}
\newcommand{\aim}{\underline{{\textbf\aimTXT}}}
\newcommand{\tasks}{\underline{\textbf{\tasksTXT}}}
\newcommand{\novelty}{\underline{\textbf{\noveltyTXT}}}
\newcommand{\influence}{\underline{\textbf{\influenceTXT}}}
\newcommand{\methods}{\underline{\textbf{\methodsTXT}}}
\newcommand{\defpositions}{\underline{\textbf{\defpositionsTXT}}}
\newcommand{\reliability}{\underline{\textbf{\reliabilityTXT}}}
\newcommand{\probation}{\underline{\textbf{\probationTXT}}}
\newcommand{\contribution}{\underline{\textbf{\contributionTXT}}}
\newcommand{\publications}{\underline{\textbf{\publicationsTXT}}}

\textbf{Актуальность темы исследования.}
Перестановки играют значимую роль в вариационном исчислении.
Впервые симметричная перестановка (симметризация) была введена Штейнером в 1836 году.
Штейнер работал над доказательством изопериметрического неравенства (задачей Дидоны)
о максимальной площади плоской фигуры с фиксированным периметром.
Штейнер используя симметризацию доказал в \cite{Steiner}, что если максимум существует, он достигается на круге.
Только в 1879 году Вейерштрасс доказал существование максимума методами вариационного исчисления.

Примерно во время появления доказательства в своей книге \cite{Rayleigh} лорд Рэлей сформулировал гипотезу
о том, что среди всех плоских мембран заданной площади, закреплённых по краю, наименьшей основной частотой обладает круг
(а точнее, предположил, что выполняется некоторая оценка первого собственного числа через меру области).
Математически эта задача сводится к нахождению минимума первого собственного числа задачи Дирихле для оператора Лапласа,
которая имеет вариационную природу.
Гипотеза Рэлея была доказана независимо Фабером (\cite{Faber}) и Краном (\cite{Krahn})
с использованием симметризации и получила в дальнейшем название неравенства Крана-Фабера.
Отметим, что другая гипотеза Рэлея о наименьшей основной частоте закреплённой пластины
была доказана лишь в 1995 году Надирашвили \cite{Nadirashvili}
с использованием варианта симметризации, предложенного ранее Пойа и Сегё.

Впоследствии изучение свойств перестановок получило дальнейшее развитие в работах Пойа и Сегё, подытоженное в классическом труде \cite{PS_book}.
<<Изопериметрические неравенства в математической физике>>.
В книге при помощи симметризации доказано множество соотношений между различными геометрическими и физическими характеристиками областей,
такими как уже упомянутые периметр, площадь, основная частота мембраны, основная частота закреплённой пластины,
а также моментом инерции, жёсткостью кручения, ёмкостью и другими.
Эти соотношения позволяют не только сформулировать утверждения относительно наиболее выгодных форм области
с точки зрения разнообразных величин,
но и оценить сложные для вычисления величины через те, которые получить просто.

В частности, в книге \cite{PS_book} доказано так называемое неравенство Пойа-Сегё, состоящее в следующем.
Пусть функция $u: \Real^n \to \Real_+$ (здесь и далее $\Real_+ = [0, \infty)$) гладкая и финитная, тогда выполнено неравенство
\begin{equation}
\label{eq:intro_PS_power}
\int\limits_{\Real^n} \abs{ \nabla \symm{u}(x) }^p dx \le \int\limits_{\Real^n} \abs{ \nabla u(x) }^p dx,
\end{equation}
где $\symm{u}$ --- симметричная перестановка функции $u$.
И даже более общее утверждение:
для $u \ge 0$ и для любой выпуклой $F: \Real_+ \to \Real_+$, $F \ge 0$ выполнено
\begin{equation}
\label{eq:intro_PS}
\int\limits_{\Real^n} F( \abs{ \nabla \symm{u}(x) } ) dx \le \int\limits_{\Real^n} F( \abs{ \nabla u(x) } ) dx.
\end{equation}

Также, поскольку это неравенство может применяться для нахождения функций, доставляющих минимум функционала,
особый интерес представляет вопрос, когда (\ref{eq:intro_PS}) превращается в равенство.
Только в 1988 году Бразерс и Зимер (\cite{BroZiem}) установили условия,
при которых из равенства в (\ref{eq:intro_PS}) следует совпадение $u$ и $\symm{u}$ с точностью до параллельного переноса.

В течение 80-х годов вышло несколько публикаций об обобщении неравенства Пойа-Сегё на функционалы вида
$$
\int F(u) G( \norm{ \nabla u } ) dx.
$$
Далее, в работе \cite{Kawohl1986} неравенство Пойа-Сегё распространено
при некоторых (необходимых, судя по всему) ограничениях на функционалы вида
$$
\int F(x', u) G( \norm{ \nabla u } ) dx,
$$
где норма $\norm{\cdot}$ --- некоторая взвешенная норма с весами, зависящими от $x'$.
Доказательство дано для гладких функций $u$.
Также, аналогичные неравенства были получены для монотонной перестановки.
%\mytodo{Можно сослаться на Каволя и Таленти.}

Отметим ещё связанное с перестановками понятие поляризации,
которое исползьзуется для доказательства многих утверждений изопериметрического характера
(см., например, \cite{Dubinin1985, SolyninZalgaller2004}).

\textbf{Степень разработанности темы исследования.}
Существенно сложнее оказалось распространить неравенство на более общую зависимость от
функции и от переменной, по которой происходит перестановка.
Значительную роль в решении этого вопроса сыграла работа \cite{Brock}.
В ней для липшицевых функций при некоторых условиях на весовую функцию был получен аналог неравенства (\ref{eq:intro_PS}).
\begin{equation}
\label{eq:intro_brock}
\int\limits_\Omega F( x', \symm{u}(x), \norm{ \mathcal{D} \symm{u} } ) dx \le \int\limits_\Omega F( x', u(x), \norm{ \mathcal{D} u } ) dx,
\end{equation}
где $x = (x_1, \dots, x_n) = (x', x_n)$,
$$
\mathcal{D} u = ( a_1( x', u( x ) ) D_1 u,\ \dots,\ a_{n - 1}( x', u( x ) ) D_{n - 1} u,\ a( x, u( x ) ) D_n u ).
$$
Однако в переносе этого неравенства на общий случай содержится пробел.
В \cite{Brock} доказано, что если последовательность функций сходится в $\W(\Real^n)$,
то подпоследовательность из симметризаций этих функций сходится там же слабо.
Ввиду этого факта доказательство можно вести по следующей схеме.
\begin{itemize}
    \item Неравенство доказывается для кусочно линейных функций $u$.
    \item Доказывается, что функционал слабо полунепрерывен снизу.
    \item Находится последовательность кусочно линейных функций $u_n$,
        приближающих предельную функцию $u$ в смысле пространств Соболева ($u_n \to u$ в $\W(\Real^n)$)
        и в смысле функционала ($I(u_n) \to I(u)$).
        После чего можно написать
        $$
        I(\symm{u}) \le \varliminf I(\symm{u_n}) \le \lim I(u_n) = I(u).
        $$
\end{itemize}

Автор работы \cite{Brock} постулировал существование $u_n$ по существу без доказательства.
Между тем, приближение функции в смысле функционала регулярными (в частности липшицевыми) функциями нельзя назвать простым вопросом.
Известно множество примеров, когда даже инфимум функционала по естественной области определения функционала
отличается от инфимума по множеству регулярных функций,
в том числе и в одномерном случае.
Для таких функционалов говорят о возникновении эффекта Лаврентьева.

В 1927 г. М. А. Лаврентьев обнаружил (\cite{Lavrentieff1927}), что для интегрального функционала с выпуклым по производной и коэрцитивным интегрантом
инфимум по абсолютно непрерывным функциям может быть строго меньше инфимума по липшицевым функциям.
В \cite{Mania1934} дан более простой пример, для которого возникает эффект Лаврентьева в одномерном случае.

В работе \cite{Zhikov1995} получено знаменитое логарифмическое условие отсутствия эффекта Лаврентьева в многомерном случае,
а также приведены простые примеры на плоскости, для которых эффект Лаврентьева имеет место.
\mytodo{можно ещё добавить. или нет}

В статье \cite{ASC} показано, что для функционалов вида
$$
\int_{-1}^1 F(u(x), u'(x)) dx
$$
можно найти последовательность регулярных функций $u_n$, приближающих $u$ и в $\W[-1, 1]$, и в смысле функционала.
В частности, для таких функционалов эффект Лаврентьева отсутствует.

В статье \cite{EspositoTrombetti2007} пробел в работе \cite{Brock} частично закрыт для функционалов схожей структуры
при помощи тонких результатов геометрической теории функций, полученных в работе \cite{CianchiFusco2006},
и приближения лагранжиана снизу.

Отметим ещё работу \cite{Landes}, в которой рассматривается неравенство Пойа-Сегё с весом для монотонных перестановок
в двумерном случае при условии, что функция $u$ закреплена на левом краю прямоугольника.
Неравенство доказано при условии степенного роста интегранта по производной и убывания веса по $y$.
Заметим, что условие на вес довольно ограничительны.

\textbf{Цели и задачи.}
Целью диссертации является обобщение неравенства Пойа-Сегё
как на более общие функционалы и формы зависимости от свободной переменной, функции и её производной,
так и на случай монотонной перестановки, которая также представляет серьёзный интерес.
Основной задачей является ввести зависимость от переменной, по которой происходит перестановка.
%В главе \ref{chapt:multi} неравенство (\ref{eq:intro_brock}) переносится на случай монотонной перестановки.
%Получены необходимые условия и доказана их достаточность при степенном ограничении на рост лагранжиана по производной.
%В главе \ref{chapt:unbounded} неравенства для монотонной и симметричной перестановок доказываются в одномерном случае
%лишь при необходимых условиях.
%В частности, в одномерном случае закрыт пробел в работе \cite{Brock}.
%В главе \ref{chapt:variable} исследуется обобщение неравенства Пойа-Сегё (\ref{eq:intro_PS_power})
%на случай переменного показателя суммирования.

\textbf{Научная новизна.}
Выносимые на защиту положения являются новыми.% и получены автором лично.

\textbf{Теоретическая и практическая значимость работы.}
Работа носит теоретический характер.
Результаты представляют интерес для специалистов по вариационному исчислению и уравнениям в частных производных.

\textbf{Методология и методы исследования.}
При доказательстве основных результатов диссертации были использованы
классические методы вариационного исчисления, математического и функционального анализа,
а также обобщение метода аппроксимации в смысле функционала, разработанного в \cite{ASC}.
%Используемые неклассические методы аппроксимации разработаны в статье \cite{ASC}, а также в статьях предшественников.
%Методы доказательства кусочно линейных вариантов неравенств стандартны и в частности разработаны в \cite{Brock}.
В главе 2 использован разработанный автором оригинальный метод
аппроксимации непрерывной функции многих переменных функциями с конечным числом монотонных участков при некоторых ограничениях.

\textbf{Положения, выносимые на защиту.}
\begin{itemize}
    \item Получены необходимые условия на вес для выполнения неравенства Пойа-Сегё с весом для монотонной перестановки.
    \item Доказано неравенство Пойа-Сегё с весом для монотонной перестановки в случае ограниченного (степенного) роста интегранта .
    \item Доказано неравенство Пойа-Сегё с весом в одномерном случае без ограничений, лишь при необходимых условиях.
    \item Доказана необходимость условий, налагаемых в работе \cite{Brock} на вес для выполнения
        неравенства Пойа-Сегё с весом для симметризации.
    \item В одномерном случае закрыт пробел в работе \cite{Brock}: доказано неравенство Пойа-Сегё с весом для симметризации
        без дополнительных ограничений.
    \item Представлены необходимые и достаточные условия выполнения неравенства Пойа-Сегё с весом для монотонной перестановки
        на функциях, закреплённых на левом конце.
        Неравенство доказано в многомерном случае для интегрантов ограниченного роста по производной
        и в одномерном случае без дополнительных ограничений.
    \item Представлены необходимые и достаточные условия выполнения неравенства Пойа-Сегё с переменным показателем
        суммирования в одномерном случае.
        Показано, что прямое многомерное обобщение отсутствует.
\end{itemize}

\textbf{Степень достоверности и апробация.}
Все результаты диссертации снабжены подробными доказательствами и опубликованы в ведущих научных изданиях.
Результаты диссертации докладывались на следующих семинарах и конференциях:
\begin{itemize}
    \item
        Международная конференция по дифференциальным уравнениям и динамическим системам (Суздаль, 2006).
    \item
        Международная конференция <<Теория приближений>> (Санкт-Петербург, 2010).
    \item
        Международная конференция по дифференциальным уравнениям и динамическим системам (Суздаль, 2010).
    \item
        Международная конференция <<Дифференциальные уравнения и смежные вопросы>>,
        посвящённая 110-летию И. Г. Петровского (Москва, 2011).
    \item
        Международная школа “Variational Analysis and Applications” (Эриче, Италия, 2012).
    \item
        Международная конференция по дифференциальным уравнениям и динамическим системам (Суздаль, 2016).
    \item
        Cеминар им. В.И. Смирнова по математической физике в Санкт-Петербургском отделении математического института
        им. В.А.Стеклова РАН (Санкт-Петербург, рук: Н. Н. Уральцева, А. И. Назаров, Т. А. Суслина).
    \item
        Городской семинар по конструктивной теории функций (Санкт-Петербург, рук.: М. А. Скопина)
    \item
        Семинар по теории функций многих действительных переменных и ее приложениям к задачам математической физики
        (Семинар Никольского) (Москва, рук.: О. В. Бесов).
\end{itemize}

\textbf{Публикации.}
Результаты диссертации опубликованы в работах \cite{Bankevich2011, Bankevich2015, Bankevich2016, Bankevich2018},
\cite{Suzdal2006, Approximation2010, Suzdal2010, Petrovskii2011, Suzdal2016}.
Работы \cite{Bankevich2011, Bankevich2018} опубликованы в журналах из перечня ВАК.
Работы \cite{Bankevich2015, Bankevich2016} опубликованы в изданиях,
удовлетворяющем достаточному условию включения в перечень ВАК ---
журнал <<Calculus of Variations and Partial Differential Equations>> и переводная версия журнала
<<Записки научных семинаров Ленинградского отделения математического института им. В.А. Стеклова АН СССР>>
(<<Journal of Mathematical Sciences>>) входят в систему цитирования Scopus.

Работы \cite{Bankevich2011, Bankevich2015} написаны в неразделимом соавторстве,
за исключением оригинального метода аппроксимации, предложенного автором.

\textbf{Объем и структура работы.}
Диссертация состоит из введения, четырёх глав, содержащих 20 параграфов, заключения и списка литературы.
Полный объём диссертации составляет 73 страницы, включая 6 рисунков и 2 таблицы.
Список литературы содержит 61 наименование.

\section*{Содержание работы}

\underline{Во введении} описаны актуальность темы исследования и степень ее разработанности,
поставлены цели и задачи, аргументирована научная новизна, достоверность,
теоретическая и практическая значимость результатов, перечислены использованные методы,
выносимые на защиту положения, публикации и доклады по теме диссертации, кратко изложена структура работы.

\underline{В главе 0} диссертации введены обозначения, используемые в работе,
а также приведены используемые известные факты со ссылками на источники.

Напомним определения перестановок.
Пусть $\Omega = \omega \times (-1, 1)$,
где $\omega$ --- ограниченная область в $\Real^{n - 1}$ с липшицевой границей.
Обозначим $x = ( x_1, \dots, x_{n - 1}, y ) = ( x', y )$.

Для измеримой неотрицательной функции $u$, заданной на $\overline{\Omega}$ выполнена теорема о послойном представлении
(см. \cite[теорема 1.13]{LiebLoss}), состоящая в следующем.
Пусть $\mathcal{A}_t(x') := \{ y \in [-1,1] :\ u( x', y ) > t \}$.
Тогда имеет место равенство
$$
u(x', y) = \int_0^\infty \charf{\mathcal{A}_t(x')}(y) dt,
$$
где $\charf{A}$ --- характеристическая функция множества $A$.

Определим симметричную перестановку измеримого множества $E \subset [-1, 1]$ и
симметричную перестановку (симметризацию по Штейнеру) неотрицательной функции $u \in \W(\overline{\Omega})$.
\begin{eqnarray*}
\symm{E} := [-\frac{\meas{E}}{2}, \frac{\meas{E}}{2}]; \qquad
\symm{u}(x', y) = \int\limits_0^\infty \charf{ \symm{( \mathcal{A}_t(x') )} }(y) dt.
\end{eqnarray*}

В тех же условиях определим монотонную перестановку множества $E$ и функции $u \in \W(\overline{\Omega})$.
\begin{eqnarray*}
\mon{E} := [1 - \meas E, 1]; \qquad
\mon{u}(x', y) = \int\limits_0^\infty \charf{ \mon{ \mathcal{A}_t(x') } }(y) dt.
\end{eqnarray*}

\underline{В главе 1} диссертации изучается неравенство, аналогичное неравенству (\ref{eq:intro_brock}),
с монотонной перестановкой вместо симметризации.

Определим множество $\mathfrak{F}$ непрерывных функций $F: \overline{\omega} \times \Real_+ \times \Real_+ \to \Real_+$,
выпуклых и строго возрастающих по третьему аргументу, удовлетворяющих $F( \cdot, \cdot, 0 ) \equiv 0$.

Рассмотрим функционал
$$
\IWg( u ) = \int\limits_\Omega F( x', u(x), \norm{ \mathcal{D} u } ) dx,
$$
где $F \in \mathfrak{F}$,
$\norm{\cdot}$ --- некоторая норма в $\Real^n$, симметричная по последней координате,
$$\mathcal{D} u = ( a_1( x', u( x ) ) D_1 u,\ \dots,\ a_{n - 1}( x', u( x ) ) D_{n - 1} u,\ a( x, u( x ) ) D_n u )$$
--- градиент $u$ с весом (обратите внимание, что только вес при $D_n u$ зависит от $y$),
$a( \cdot, \cdot ): \overline{\Omega} \times \Real_+ \to \Real_+$
и $a_i( \cdot, \cdot ): \overline{\omega} \times \Real_+ \to \Real_+$ --- непрерывные функции.
Здесь и далее индекс $i$ пробегает от $1$ до $n - 1$.

Рассмотрим неравенство
\begin{equation}
\label{eq:intro_to_prove_weighted}
\IWg( \mon{u} ) \le \IWg( u )
\end{equation}

В \S 1.1 вводятся необходимые обозначения.

В \S 1.2 устанавливаются условия, необходимые для выполнения неравенства (\ref{eq:intro_to_prove_weighted}).

\begin{thmIntro}
%\label{thm:necessary_conditions_weighted}
\textbf{\textup{i)}}
Если неравенство $(\ref{eq:intro_to_prove_weighted})$ выполняется для некоторой $F \in \mathfrak{F}$ и произвольной кусочно линейной $u$,
то вес $a$ чётен по $y$, то есть $a(x', y, v) \equiv a(x', -y, v)$.

\textbf{\textup{ii)}}
Если неравенство $(\ref{eq:intro_to_prove_weighted})$ выполняется для произвольной $F \in \mathfrak{F}$
и произвольной кусочно линейной $u$, то вес $a$ удовлетворяет неравенству
\begin{equation}
\label{eq:intro_almostConcave}
a(x', s, v) + a(x', t, v) \ge a(x', 1 - t + s, v), \qquad x' \in \overline{\omega},\ -1 \le s \le t \le 1,\ v \in \Real_+.
\end{equation}
\end{thmIntro}

Также получены необходимые условия в случае закреплённых на левом конце функций.

\begin{thmIntro}
Если неравенство $(\ref{eq:intro_to_prove_weighted})$ выполняется для произвольной $F \in \mathfrak{F}$
и произвольной кусочно линейной $u$, закреплённой на левом конце: $u(\cdot, -1) \equiv 0$,
то вес $a$ удовлетворяет неравенству $(\ref{eq:intro_almostConcave})$.
\end{thmIntro}

В \S 1.3 доказывается неравенство (\ref{eq:intro_to_prove_weighted}) для кусочно линейных $u$.

\begin{lmIntro}
%\label{lm:weighted_linear}
Пусть функция $a(x', \cdot, u)$ чётна и удовлетворяет условию $(\ref{eq:intro_almostConcave})$.
Тогда, если $u$ --- неотрицательная кусочно линейная функция, то $\IWg( u ) \ge \IWg( \mon{u} )$.
\end{lmIntro}

\begin{lmIntro}
Пусть функция $a(x', \cdot, u)$ удовлетворяет условию $(\ref{eq:intro_almostConcave})$.
Тогда, если $u$ --- неотрицательная кусочно линейная функция, удовлетворяющая $u(\cdot, -1) \equiv 0$,
то $\IWg( u ) \ge \IWg( \mon{u} )$.
\end{lmIntro}

В \S 1.4 устанавливается слабая полунепрерывность функционала $\IWg$
и доказывается теорема, которая в дальнейшем является основой для предельных переходов.

\begin{thmIntro}
%\label{thm:uplift}
Пусть $B \subset A \subset \W(\overline{\Omega})$.
Предположим, что для каждого $u \in A$ найдётся последовательность $u_k \in B$ такая,
что $u_k \to u$ в $\W(\overline{\Omega})$ и $\IWg( u_k ) \to \IWg( u )$.
Тогда

\textbf{\textup{i)}}
Если для любой функции $v \in B$ выполнено $\IWg( \symm{v} ) \le \IWg( v )$,
то для любой функции $u \in A$ будет выполнено $\IWg( \symm{u} ) \le \IWg( u )$.

\textbf{\textup{ii)}}
Если для любой функции $v \in B$ выполнено $\IWg( \mon{v} ) \le \IWg( v )$,
то для любой функции $u \in A$ будет выполнено $\IWg( \mon{u} ) \le \IWg( u )$.
\end{thmIntro}

В \S 1.5 неравенство (\ref{eq:intro_to_prove_weighted}) доказывается для интегрантов с ограниченным ростом по производной.

\begin{thmIntro}
%\label{thm:bounded_growth}
Пусть функция $a(x', \cdot, u)$ чётна и удовлетворяет условию $(\ref{eq:intro_almostConcave})$.
Тогда

\textbf{\textup{i)}} Неравенство (\ref{eq:intro_to_prove_weighted}) верно для произвольной неотрицательной $u \in Lip(\overline{\Omega})$.

\textbf{\textup{ii)}} Предположим, что для любых $x' \in \overline{\omega}, z \in \Real_+, p \in \Real$
функция $F$ удовлетворяет неравенству
$$F( x', z, p ) \le C ( 1 + |z|^{q^*} + |p|^q ),$$
где $\frac{1}{q^*} = \frac{1}{q} - \frac{1}{n}$, если $q < n$, либо $q^*$ любое в противном случае.
Если $q \le n$, то дополнительно предположим, что веса $a$ и $a_i$ ограничены.
Тогда неравенство (\ref{eq:intro_to_prove_weighted}) верно для произвольной неотрицательной $u \in W{}^1_q(\overline{\Omega})$.
\end{thmIntro}

\begin{thmIntro}
Пусть функция $a(x', \cdot, u)$ удовлетворяет условию $(\ref{eq:intro_almostConcave})$.
Тогда

\textbf{\textup{i)}} Неравенство (\ref{eq:intro_to_prove_weighted}) верно для произвольной неотрицательной $u \in Lip(\overline{\Omega})$,
удовлетворяющей $u(\cdot, -1) \equiv 0$.

\textbf{\textup{ii)}} Предположим, что для любых $x' \in \overline{\omega}, z \in \Real_+, p \in \Real$
функция $F$ удовлетворяет неравенству
$$F( x', z, p ) \le C ( 1 + |z|^{q^*} + |p|^q ),$$
где $\frac{1}{q^*} = \frac{1}{q} - \frac{1}{n}$, если $q < n$, либо $q^*$ любое в противном случае.
Если $q \le n$, то дополнительно предположим, что веса $a$ и $a_i$ ограничены.
Тогда неравенство (\ref{eq:intro_to_prove_weighted}) верно для произвольной неотрицательной $u \in W{}^1_q(\overline{\Omega})$,
удовлетворяющей $u(\cdot, -1) \equiv 0$.
\end{thmIntro}

Глава 2 диссертации посвящена снятию условия ограниченного роста с интегранта.
Это удаётся сделать только в одномерном случае, поэтому далее $u \in \W[-1, 1]$ и
$$
\IWg( u ) = \int\limits_{-1}^1 F( u(x), a(x, u(x)) \abs{ u'(x) } )\, dx.
$$

В \S 2.1 формулируется одномерный вариант задачи.

В \S 2.2 удаётся распространить результат статьи \cite{ASC} на случай функционала $\IWg$
и доказать отсутствие эффекта Лаврентьева в случае кусочной монотонности веса.

\begin{lmIntro}
Пусть $a$ --- непрерывная функция, $a(\cdot, u)$ возрастает на $[-1, 0]$ и убывает на $[0, 1]$ для всех $u \ge 0$.
Тогда для любой функции $u \in \W[-1, 1]$, $u \ge 0$, найдётся последовательность $\{u_k\} \subset Lip[-1, 1]$, удовлетворяющая
\begin{equation}
u_k \to u \text{ в } \W[-1, 1] \quad \text{ и } \quad \IWg(u_k) \to \IWg(u).
\end{equation}
\end{lmIntro}

\begin{thmIntro}
Пусть функция $a$ непрерывна, чётна, убывает на $[0, 1]$ и удовлетворяет неравенству $(\ref{eq:intro_almostConcave})$.
Тогда для любой $u \in \W[-1, 1]$ выполнено $\IWg( \symm{u} ) \le \IWg( u )$.
\end{thmIntro}

В \S 2.3 доказано несколько важных свойств весовых функций, удовлетворяющих необходимым условиям.
В частности установлена структура множества нулей весовых функций.

В \S 2.4 с веса снимается требование монотонности и, тем самым,
неравенство (\ref{eq:intro_to_prove_weighted}) доказано в наиболее общем виде.

\begin{thmIntro}
%\label{thm:unbounded_growth}
Пусть $F \in \mathfrak{F}$, функция $u \in \W[-1, 1]$ неотрицательна,
и весовая функция $a: [-1, 1] \times \Real_+ \to \Real_+$ непрерывна
и допустима для $u$.
Тогда справедливо неравенство $(\ref{eq:intro_to_prove_weighted})$.
\end{thmIntro}

В \S 2.5 завершается доказательство для функций, закреплённых на левом конце.

\begin{thmIntro}
Пусть $F \in \mathfrak{F}$, функция $u \in \W[-1, 1]$ неотрицательна, $u(-1) = 0$,
весовая функция $a: [-1, 1] \times \Real_+ \to \Real_+$ непрерывна и удовлетворяет неравенству $(\ref{eq:intro_almostConcave})$.
Тогда справедливо неравенство $(\ref{eq:intro_to_prove_weighted})$.
\end{thmIntro}

В \S 2.6 доказано, что условия, накладываемые на вес в работе \cite{Brock},
являются необходимыми в случае симметричной перестановки.

\begin{thmIntro}
Если неравенство $(\ref{eq:intro_brock})$ выполняется для произвольной $F \in \mathfrak{F}$ и произвольной кусочно линейной $u$,
то вес $a$ --- чётная и выпуклая по первому аргументу функция.
\end{thmIntro}

И наконец, в \S 2.7 закрывается пробел в работе \cite{Brock} в одномерном случае.
\begin{thmIntro}
Пусть $F \in \mathfrak{F}$, функция $u \in \W[-1, 1]$ неотрицательна,
и непрерывная весовая функция $a: [-1, 1] \times \Real_+ \to \Real_+$ чётна и выпукла по первому аргументу.
Тогда справедливо неравенство $(\ref{eq:intro_brock})$.
\end{thmIntro}

\underline{В главе 3} диссертации рассмотрено обобщение неравенства Пойа-Сегё
на случай переменного показателя суммирования.
А именно, рассматриваются два функционала:
\begin{eqnarray*}
\J(u) &=& \int\limits_{-1}^1 |u'(x)|^{p(x)} dx \\
\I(u) &=& \int\limits_{-1}^1 ( 1 + | u'(x) |^2 )^{\frac {p(x)}{2}} dx.
\end{eqnarray*}

Подобные функционалы возникают при моделировании электрореологических жидкостей (см., напр., \cite{RuzickaModelingPaper, RuzickaModeling}).
В частности, в настоящее время активно изучаются эллиптические уравнения с $p(x)$-лапласианом в качестве главного члена.
Литература в этой области обширна, см. напр.
\cite{Ruzicka, Zhikov2017} и ссылки оттуда.
%http://www.mathnet.ru/php/archive.phtml?wshow=paper&jrnid=tm&paperid=735&option_lang=rus

В \S 3.1 ставится задача и вводятся обозначения.

В \S 3.2 получены условия, необходимые для выполнения неравенств
$\J(\symm{u}) \le \J(u)$ и $\I(\symm{u}) \le \I(u)$.

\begin{thmIntro}
Пусть $\J(\symm{u}) \le \J(u)$ выполнено для любой кусочно линейной функции $u \ge 0$.
Тогда $p(x) \equiv const$.
\end{thmIntro}

То есть изучение аналога неравенства Пойа-Сегё для функционала $\I$ теряет смысл.

\begin{thmIntro}
\label{thm:intro_necessary_variable}
Если неравенство $\I(\symm{u}) \le \I(u)$ выполняется для всех кусочно линейных $u \ge 0$,
то $p$ чётна и выпукла.
Более того, выпукла следующая функция:
$$
K(s, x) = s ( 1 + s^{-2} )^{\frac {p(x)}{2}}, \qquad s > 0,\ x \in [-1, 1].
$$
\end{thmIntro}

В \S 3.3 показано, что условия, необходимые для выполнения неравенства $\I(\symm{u}) \le \I(u)$,
являются и достаточными.

\begin{lmIntro}
Пусть $p$ чётна, а $K$ выпукла по совокупности переменных.
Тогда для любой кусочно линейной функции $u \in \Wf[-1, 1]$ выполнено $\I(\symm{u}) \le \I(u)$.
\end{lmIntro}

\begin{thmIntro}
Пусть $p$ чётна, а $K$ выпукла по совокупности переменных.
Тогда для любой функции $u \in \Wf[-1, 1]$ выполнено $\I(\symm{u}) \le \I(u)$.
\end{thmIntro}

Условие выпуклости функции $K$ есть на самом деле неявное условие на функцию $p$.
В \S 3.4 приведены некоторые явные достаточные условия выполнения неравенства $\I(\symm{u}) \le \I(u)$.

Выпуклость функции $K$ равносильна неотрицательности гессиана $K$ (если есть выпуклость по какому-нибудь направлению: $K$ всегда выпукла по $s$),
которая в свою очередь сводится к условию
$$
q q'' \ge q'^2 B(w, q),
$$
где $w = {1 \over s^2}$, $q(x) = p(x) - 1$,
$$
B(w, q) = \frac{
q ( 4 w - ( w + 3 ) \ln( w + 1 ) ) - {w - 1 \over w} \ln( w + 1 ) + 4 {w \over \ln( w + 1 )} - 4
}{
2 ( q w + 1 )
} \cdot {q \over q + 1}.
$$

\begin{thmIntro}
\label{thm:intro_sufficient}
Пусть $p(x)\ge1$ --- чётная непрерывная функция на $[-1, 1]$.

\textbf{\textup{i)}}
Если функция $(p(x)-1)^{0.37}$ выпукла, то неравенство $\I(\symm{u}) \le \I(u)$ выполнено для любой неотрицательной $u \in \Wf[-1, 1]$.

\textbf{\textup{ii)}}
Если $p(x) \le 2.36$ для всех $x \in [-1, 1]$ и функция $\sqrt{p(x) - 1}$ выпукла,
то неравенство $\I(\symm{u}) \le \I(u)$ выполнено для произвольной неотрицательной $u \in \Wf[-1, 1]$.
\end{thmIntro}

В \S 3.5 описаны численно-аналитические методы для получения оценок,
на которых основаны выводы теоремы \ref{thm:intro_sufficient}.
Пусть ${\mathcal B}(q) \equiv \sup\limits_{w > 0} B(w, q)$. Тогда

\begin{eqnarray}
&& \sup\limits_{q \ge 0}{\mathcal B}(q) = \limsup\limits_{q \to +\infty}{\mathcal B}(q) \le 0.63; \\
&& \sup\limits_{0 \le q \le 1.36}{\mathcal B}(q) \le 0.5.
\end{eqnarray}

Наконец, в \S 3.6 показано, что прямое распространение неравенства $\I(\symm{u}) \le \I(u)$
на многомерный случай несодержательно.

\begin{thmIntro}
Если
$\int\limits_{\Omega} ( 1 + | \nabla \symm{u}(x) |^2 )^{ \frac{p(x)}{2} } dx
\le \int\limits_{\Omega} ( 1 + | \nabla u(x) |^2 )^{ \frac{p(x)}{2} } dx$
для любой неотрицательной функции $u \in \Wf(\overline{\Omega})$,
то $p(x',y)$ не зависит от $y$.
\end{thmIntro}

\mytodo{убрать?}

\underline{В заключении} перечисляются основные результаты работы, которые заключаются в следующем.
В диссертации получены разноплановые обобщения неравенства Пойа-Сегё
с введением зависимости интегранта от переменной, по которой происходит перестановка.
Неравенство перенесено на случай монотонных перестановок с зависимостью от свободной переменной в виде веса при производной.
Также неравенство обобщено на случай переменного показателя суммирования в оригинальном неравенстве Пойа-Сегё.

Более детально,
получены необходимые условия на вес для выполнения неравенства Пойа-Сегё с весом для монотонной перестановки.
Доказано неравенство Пойа-Сегё с весом для монотонной перестановки в случае ограниченного (степенного) роста интегранта .
Доказано неравенство Пойа-Сегё с весом в одномерном случае без ограничений, лишь при необходимых условиях.
Доказана необходимость условий, налагаемых в работе \cite{Brock} на вес для выполнения неравенства Пойа-Сегё с весом для симметризации.
В одномерном случае закрыт пробел в работе \cite{Brock}:
доказано неравенство Пойа-Сегё с весом для симметризации без дополнительных ограничений.
Представлены необходимые и достаточные условия выполнения неравенства Пойа-Сегё с весом для монотонной перестановки
на функциях, закреплённых на левом конце.
Неравенство доказано в многомерном случае для интегрантов ограниченного роста по производной
и в одномерном случае без дополнительных ограничений.
Представлены необходимые и достаточные условия выполнения неравенства Пойа-Сегё с переменным показателем
суммирования в одномерном случае.
Показано, что прямое многомерное обобщение отсутствует.

Работа поддержана грантом РФФИ 18-01-00472.

\urlstyle{rm}                               % ссылки URL обычным шрифтом

\begingroup
    \let\clearpage\relax
    \printbibliography[keyword={othercites}, title={ }]

%    \vspace*{10mm}
    \chapter*{Публикации автора по теме диссертации}
    \addcontentsline{toc}{chapter}{Публикации автора по теме диссертации}

    \nocite{Bankevich2011, Bankevich2015, Bankevich2016, Bankevich2018}
    \nocite{Suzdal2006, Approximation2010, Suzdal2010, Petrovskii2011, Suzdal2016}

%    \vspace*{-10mm}

    \printbibliography[heading=subbibintoc, keyword={authorpapersVAK}, title={Публикации в рецензируемых изданиях}]

%    \vspace*{-10mm}

    \printbibliography[heading=subbibintoc, keyword={proceedings}, title={Прочие публикации}]
    % \printbibliography[keyword={authorpapersVAK}, title={Публикации в рецензируемых изданиях}]

    \printbibliography[keyword={conference}, title={Прочие публикации}]
\endgroup

\urlstyle{tt}                               % возвращаем установки шрифта ссылок URL


%%\newpage
%При использовании пакета \verb!biblatex! список публикаций автора по теме
%диссертации формируется в разделе <<\publications>>\ файла
%\verb!../common/characteristic.tex!  при помощи команды \verb!\nocite!
%
%\ifdefmacro{\microtypesetup}{\microtypesetup{protrusion=false}}{} % не рекомендуется применять пакет микротипографики к автоматически генерируемому списку литературы
%\ifnumequal{\value{bibliosel}}{0}{% Встроенная реализация с загрузкой файла через движок bibtex8
%  \renewcommand{\bibname}{\large \authorbibtitle}
%  \nocite{*}
%  \insertbiblioauthor           % Подключаем Bib-базы
%  %\insertbiblioother   % !!! bibtex не умеет работать с несколькими библиографиями !!!
%}{% Реализация пакетом biblatex через движок biber
%  \ifnumgreater{\value{usefootcite}}{0}{
%%  \nocite{*} % Невидимая цитата всех работ, позволит вывести все работы автора
%  \insertbiblioauthorcited      % Вывод процитированных в автореферате работ автора
%  }{
%  \insertbiblioauthor           % Вывод всех работ автора
%%  \insertbiblioauthorgrouped    % Вывод всех работ автора, сгруппированных по источникам
%%  \insertbiblioauthorimportant  % Вывод наиболее значимых работ автора (определяется в файле characteristic во второй section)
%  \insertbiblioother            % Вывод списка литературы, на которую ссылались в тексте автореферата
%  }
%}
%\ifdefmacro{\microtypesetup}{\microtypesetup{protrusion=true}}{}

