%%% Основные сведения %%%
\newcommand{\thesisAuthorLastName}{Банкевич}
\newcommand{\thesisAuthorOtherNames}{Сергей Викторович}
\newcommand{\thesisAuthorInitials}{С.\,В.}
\newcommand{\thesisAuthor}             % Диссертация, ФИО автора
{%
    \texorpdfstring{% \texorpdfstring takes two arguments and uses the first for (La)TeX and the second for pdf
        \thesisAuthorLastName~\thesisAuthorOtherNames% так будет отображаться на титульном листе или в тексте, где будет использоваться переменная
    }{%
        \thesisAuthorLastName, \thesisAuthorOtherNames% эта запись для свойств pdf-файла. В таком виде, если pdf будет обработан программами для сбора библиографических сведений, будет правильно представлена фамилия.
    }
}
\newcommand{\thesisAuthorShort}        % Диссертация, ФИО автора инициалами
{\thesisAuthorInitials~\thesisAuthorLastName}
%\newcommand{\thesisUdk}                % Диссертация, УДК
%{517.988.38}
\newcommand{\thesisTitle}              % Диссертация, название
{О монотонности интегральных функционалов \\ при перестановках}
\newcommand{\thesisSpecialtyNumber}    % Диссертация, специальность, номер
{01.01.02}
\newcommand{\thesisSpecialtyTitle}     % Диссертация, специальность, название
{Дифференциальные уравнения, динамические системы и оптимальное управление}
\newcommand{\thesisDegree}             % Диссертация, ученая степень
{кандидата физико-математических наук}
\newcommand{\thesisDegreeShort}        % Диссертация, ученая степень, краткая запись
{канд. физ.-мат. наук}
\newcommand{\thesisCity}               % Диссертация, город написания диссертации
{Санкт-Петербург}
\newcommand{\thesisYear}               % Диссертация, год написания диссертации
{2018}
\newcommand{\thesisOrganization}       % Диссертация, организация
{Санкт-Петербургский государственный университет}
\newcommand{\thesisOrganizationShort}  % Диссертация, краткое название организации для доклада
{СПбГУ}

\newcommand{\thesisInOrganization}     % Диссертация, организация в предложном падеже: Работа выполнена в ...
{Санкт-Петербургском государственном университете}

\newcommand{\supervisorFio}            % Научный руководитель, ФИО
{Назаров Александр Ильич}
\newcommand{\supervisorRegalia}        % Научный руководитель, регалии
{доктор физико-математических наук, профессор}
\newcommand{\supervisorFioShort}       % Научный руководитель, ФИО
{А.~И.~Назаров}
\newcommand{\supervisorRegaliaShort}   % Научный руководитель, регалии
{д.ф.м.н., проф.}


\newcommand{\opponentOneFio}           % Оппонент 1, ФИО
{Степанов Владимир Дмитриевич}
\newcommand{\opponentOneRegalia}       % Оппонент 1, регалии
{член-корреспондент РАН, доктор физико-математических наук, профессор}
\newcommand{\opponentOneJobPlace}      % Оппонент 1, место работы
{Федеральное государственное автономное образовательное учреждение высшего образования <<Российский университет дружбы народов>>}
\newcommand{\opponentOneJobPost}       % Оппонент 1, должность
{главный научный сотрудник}

\newcommand{\opponentTwoFio}           % Оппонент 2, ФИО
{Сурначёв Михаил Дмитриевич}
\newcommand{\opponentTwoRegalia}       % Оппонент 2, регалии
{доктор физико-математических наук}
\newcommand{\opponentTwoJobPlace}      % Оппонент 2, место работы
{Федеральное государственное учреждение <<Федеральный исследовательский центр Институт прикладной математики им. М.В.~Келдыша Российской академии наук>>}
\newcommand{\opponentTwoJobPost}       % Оппонент 2, должность
{старший научный сотрудник}

\newcommand{\leadingOrganizationTitle} % Ведущая организация, дополнительные строки
{Федеральное государственное бюджетное образовательное учреждение высшего образования
 «Владимирский государственный университет имени Александра Григорьевича и Николая Григорьевича Столетовых»}

\newcommand{\defenseDate}              % Защита, дата
{14 июня 2018 г.~в~14 часов}
\newcommand{\defenseCouncilNumber}     % Защита, номер диссертационного совета
{Д\,212.232.49}
\newcommand{\defenseCouncilTitle}      % Защита, учреждение диссертационного совета
{Санкт-Петербургском государственном университете}
\newcommand{\defenseCouncilAddress}    % Защита, адрес учреждение диссертационного совета
{198504, Россия, Санкт-Петербург, Старый Петергоф, Университетский пр., дом 28, ауд. 405.}
\newcommand{\defenseCouncilPhone}      % Телефон для справок
{\todo{+7~(0000)~00-00-00}}

\newcommand{\defenseSecretaryFio}      % Секретарь диссертационного совета, ФИО
{Чурин Ю. В.}
\newcommand{\defenseSecretaryRegalia}  % Секретарь диссертационного совета, регалии
{доктор физико-математических наук, доцент}            % Для сокращений есть ГОСТы, например: ГОСТ Р 7.0.12-2011 + http://base.garant.ru/179724/#block_30000

\newcommand{\synopsisLibrary}          % Автореферат, название библиотеки
{Научной библиотеке им. М. Горького Санкт-Петербургского государственного университета по адресу:
 199034, Санкт-Петербург, Университетская наб., 7/9, а также на сайте
 https://disser.spbu.ru/files/disser2/disser/2b2QTcvhkz.pdf}
\newcommand{\synopsisDate}             % Автореферат, дата рассылки
{<<\underline{\hspace{5mm}}>> \underline{\hspace{2cm}} 2018 года}

% To avoid conflict with beamer class use \providecommand
\providecommand{\keywords}%            % Ключевые слова для метаданных PDF диссертации и автореферата
{}
