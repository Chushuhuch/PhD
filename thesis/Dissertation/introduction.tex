\chapter*{Введение}							% Заголовок
\addcontentsline{toc}{chapter}{Введение}	% Добавляем его в оглавление

\newcommand{\actuality}{}
\newcommand{\progress}{}
\newcommand{\aim}{{\textbf\aimTXT}}
\newcommand{\tasks}{\textbf{\tasksTXT}}
\newcommand{\novelty}{\textbf{\noveltyTXT}}
\newcommand{\influence}{\textbf{\influenceTXT}}
\newcommand{\methods}{\textbf{\methodsTXT}}
\newcommand{\defpositions}{\textbf{\defpositionsTXT}}
\newcommand{\reliability}{\textbf{\reliabilityTXT}}
\newcommand{\probation}{\textbf{\probationTXT}}
\newcommand{\contribution}{\textbf{\contributionTXT}}
\newcommand{\publications}{\textbf{\publicationsTXT}}

%{\actuality}
Перестановки играют значимую роль в вариационном исчислении.
Впервые симметричная перестановка (симметризация) была введена Штейнером в 1836 году.
Штейнер работал над доказательством изопериметрического неравенства (задачей Дидоны)
о максимальной площади плоской фигуры с фиксированным периметром.
Штейнер используя симметризацию доказал в \cite{Steiner}, что если максимум существует, он достигается на круге.
Только в 1879 году Вейерштрасс доказал существование максимума методами вариационного исчисления.

%J.W.S. Rayleigh, "The theory of sound" , London (1894/96) pp. 339–340 (Edition: Second)
Примерно во время появления доказательства в своей книге \cite{Rayleigh} лорд Рэлей сформулировал гипотезу
о том, что среди всех плоских мембран заданной площади, закреплённых по краю, наименьшей основной частотой обладает круг
(а точнее, предположил, что выполняется некоторая оценка первого собственного числа через меру области).
Математически эта задача сводится к нахождению минимума первого собственного числа задачи Дирихле для оператора Лапласа,
которая имеет вариационную природу.
Гипотеза Рэлея была доказана независимо Фабером (\cite{Faber}) и Краном (\cite{Krahn}) гипотеза была доказана
с использованием симметризации и получила в дальнейшем название неравенства Крана-Фабера.
%G. Faber, "Beweis, dass unter allen homogenen Membranen von gleicher Fläche und gleicher Spannung die kreisförmige den tiefsten Grundton gibt" Sitzungsber. Bayer. Akad. Wiss. München, Math.-Phys. Kl. (1923) pp. 169–172
%E. Krahn, "Über eine von Rayleigh formulierte Minimaleigenschaft des Kreises" Math. Ann. , 94 (1925) pp. 97–100
Отметим, что другая гипотеза Рэлея о наименьшей основной частоте закреплённой пластины
была доказана лишь в 1995 году Надирашвили \cite{Nadirashvili}.
%N.S. Nadirashvili, "Rayleigh's conjecture on the principal frequency of the clamped plate" Arch. Rational Mech. Anal., 129 (1995) pp. 1–10

Впоследствии изучение свойств перестановок получило дальнейшее развитие в работах Пойа и Сегё, подытоженное в классическом труде \cite{PS_book}.
<<Изопериметрические неравенства в математической физике>>.
В книге при помощи симметризации доказано множество соотношений между различными геометрическими и физическими характеристиками областей,
такими как уже упомянутые периметр, площадь, основная частота мембраны, основная частота закреплённой пластины,
а также моментом инерции, жёсткостью кручения, ёмкостью и другими.
Эти соотношения позволяют не только сформулировать утверждения относительно наиболее выгодных форм области
с точки зрения разнообразных величин,
но и оценить сложные для вычисления величины через те, которые получить просто.

В частности, в книге \cite{PS_book} доказано так называемое неравенство Пойа-Сегё, состоящее в следующем.
Пусть функция $u: \Real^n \to \Real_+$ (здесь и далее $\Real_+ = [0, \infty)$) гладкая и финитная, тогда выполнено неравенство
$$
\int\limits_{\Real^n} \abs{ \nabla \symm{u}(x) }^p dx \le \int\limits_{\Real^n} \abs{ \nabla u(x) }^p dx,
$$
где $\symm{u}$ --- симметричная перестановка функции $u$.
И даже более общее утверждение:
для $u \ge 0$ и для любой выпуклой $F: \Real_+ \to \Real_+$, $F \ge 0$ выполнено
\begin{equation}
\label{eq:intro_PS}
\int\limits_{\Real^n} F( \abs{ \nabla \symm{u}(x) } ) dx \le \int\limits_{\Real^n} F( \abs{ \nabla u(x) } ) dx.
\end{equation}

Также, поскольку это неравенство может применяться для нахождения функций, доставляющих минимум функционала,
особый интерес представляет вопрос, когда (\ref{eq:intro_PS}) превращается в равенство.
Только в 1988 году Бразерс и Зимер (\cite{BroZiem}) установили условия,
при которых из равенства в (\ref{eq:intro_PS}) следует совпадение $u$ и $\symm{u}$ с точностью до параллельного переноса.

\todo{Ещё истории про обобщение неравенства. Добавление зависимости от $u$.}

\mytodo{Огромная литература. См. обзор.}

Аналогичные неравенства были получены для монотонной перестановки.
\mytodo{Можно сослаться на Каволя и Таленти.}

Существенно сложнее оказалось добавить зависимость от переменной, по которой происходит перестановка.
Значительную роль в решении этого вопроса сыграла работа \cite{Brock}.
В ней для липшицевых функций при некоторых условиях на весовую функцию был получен аналог неравенства (\ref{eq:intro_PS}):
\begin{equation}
\label{eq:intro_brock}
\int\limits_\Omega F( x', \symm{u}(x), \norm{ \mathcal{D} \symm{u} } ) dx \le \int\limits_\Omega F( x', u(x), \norm{ \mathcal{D} u } ) dx,
\end{equation}
где $x = (x_1, \dots, x_n) = (x', x_n)$,
$$
\mathcal{D} u = ( a_1( x', u( x ) ) D_1 u,\ \dots,\ a_{n - 1}( x', u( x ) ) D_{n - 1} u,\ a( x, u( x ) ) D_n u ).
$$
Однако в переносе этого неравенства на общий случай содержится пробел.
В \cite{Brock} доказано, что если последовательность функций сходится в $\W(\Real^n)$,
то подпоследовательность из симметризаций этих функций сходится там же слабо
(см. также \cite{AlmgrenLieb}, \cite{Burchard} для более детального анализа сходимости перестановок сходящейся последовательности).
Ввиду этого факта доказательство можно вести по следующей схеме.
\begin{itemize}
    \item Неравенство доказывается для кусочно линейных функций $u$.
    \item Доказывается, что функционал слабо полунепрерывен снизу.
    \item Находится последовательность кусочно линейных функций $u_n$,
        приближающих предельную функцию $u$ в смысле пространств Соболева ($u_n \to u$ в $\W(\Real^n)$)
        и в смысле функционала ($I(u_n) \to I(u)$).
        После чего можно написать
        $$
        I(\symm{u}) \le \varliminf I(\symm{u_n}) \le \lim I(u_n) = I(u).
        $$
\end{itemize}

Автор \cite{Brock} постулировал существование $u_n$ по существу без доказательства.
Между тем, приближение функции в смысле функционала регулярными (в частности липшицевыми) функциями нельзя назвать простым вопросом.
Известно множество примеров, когда даже инфимум функционала по естественной области определения функционала
отличается от инфимума по множеству регулярных функций,
в том числе и в одномерном случае.
Для таких функционалов говорят о возникновении эффекта Лаврентьева.
\todo{Эффект Лаврентьева. Литература}

В статье \cite{ASC} показано, что для функционалов вида
$$
\int_{-1}^1 F(u(x), u'(x)) dx
$$
можно найти последовательность регулярных функций $u_n$, приближающих $u$ и в $\W[-1, 1]$, и в смысле функционала.
В частности, для таких функционалов эффект Лаврентьева отсутствует.

\todo{Добавить про закреплённые}

Диссертация состоит из введения и четырёх глав.

В главе \ref{chapt:external} диссертации введены обозначения, используемые в работе,
а также приведены используемые известные факты со ссылками на источники.

Напомним определения перестановок.
Пусть $\Omega = \omega \times (-1, 1)$,
где $\omega$ --- ограниченная область в $\Real^{n - 1}$ с липшицевой границей.
Обозначим $x = ( x_1, \dots, x_{n - 1}, y ) = ( x', y )$.

Для измеримой неотрицательной функции $u$, заданной на $\overline{\Omega}$ выполнена теорема о послойном представлении
(см. \cite[теорема 1.13]{LiebLoss}), состоящая в следующем.
Пусть $\mathcal{A}_t(x') := \{ y \in [-1,1] :\ u( x', y ) > t \}$.
Тогда имеет место равенство
$$
u(x', y) = \int_0^\infty \charf{\mathcal{A}_t(x')}(y) dt,
$$
где $\charf{A}$ --- характеристическая функция множества $A$.

Определим симметричную перестановку измеримого множества $E \subset [-1, 1]$ и
симметричную перестановку (симметризацию по Штейнеру) неотрицательной функции $u \in \W(\overline{\Omega})$:
\begin{eqnarray*}
\symm{E} := [-\frac{\abs{E}}{2}, \frac{\abs{E}}{2}]; \qquad
\symm{u}(x', y) = \int\limits_0^\infty \charf{ \symm{( \mathcal{A}_t(x') )} }(y) dt.
\end{eqnarray*}

В тех же условиях определим монотонную перестановку множества $E$ и функции $u \in \W(\overline{\Omega})$:
\begin{eqnarray*}
\mon{E} := [1 - \meas E, 1]; \qquad
\mon{u}(x', y) = \int\limits_0^\infty \charf{ \mon{ \mathcal{A}_t(x') } }(y) dt.
\end{eqnarray*}

В главе \ref{chapt:multi} диссертации изучается неравенство, аналогичное неравенству (\ref{eq:intro_brock}),
с монотонной перестановкой вместо симметризации.

Определим множество $\mathfrak{F}$ непрерывных функций $F: \overline{\omega} \times \Real_+ \times \Real_+ \to \Real_+$,
выпуклых и строго возрастающих по третьему аргументу, удовлетворяющих $F( \cdot, \cdot, 0 ) \equiv 0$.

Рассмотрим функционал:
$$
\IWg( u ) = \int\limits_\Omega F( x', u(x), \norm{ \mathcal{D} u } ) dx,
$$
где $F \in \mathfrak{F}$,
$\norm{\cdot}$ --- некоторая норма в $\Real^n$, симметричная по последней координате,
$$\mathcal{D} u = ( a_1( x', u( x ) ) D_1 u,\ \dots,\ a_{n - 1}( x', u( x ) ) D_{n - 1} u,\ a( x, u( x ) ) D_n u )$$
--- градиент $u$ с весом (обратите внимание, что только вес при $D_n u$ зависит от $y$),
$a( \cdot, \cdot ): \overline{\Omega} \times \Real_+ \to \Real_+$
и $a_i( \cdot, \cdot ): \overline{\omega} \times \Real_+ \to \Real_+$ --- непрерывные функции.
Здесь и далее индекс $i$ пробегает от $1$ до $n - 1$.

Рассмотрим неравенство
\begin{equation}
\label{eq:intro_to_prove_weighted}
\IWg( \mon{u} ) \le \IWg( u )
\end{equation}

В \S\ref{sec:notations_weighted} вводятся необходимые обозначения.

В \S\ref{sec:necessary_weighted} устанавливаются условия, необходимые для выполнения неравенства (\ref{eq:intro_to_prove_weighted}):

\begin{thm}
%\label{thm:necessary_conditions_weighted}
\textbf{1.}
Если неравенство $(\ref{eq:intro_to_prove_weighted})$ выполняется для некоторой $F \in \mathfrak{F}$ и произвольной кусочно линейной $u$,
то вес $a$ чётен по $y$, то есть $a(x', y, v) \equiv a(x', -y, v)$.

\textbf{\textup{ii)}}
Если неравенство $(\ref{eq:intro_to_prove_weighted})$ выполняется для произвольной $F \in \mathfrak{F}$
и произвольной кусочно линейной $u$, то вес $a$ удовлетворяет неравенству
\begin{equation}
\label{eq:intro_almostConcave}
a(x', s, v) + a(x', t, v) \ge a(x', 1 - t + s, v), \qquad x' \in \overline{\omega},\ -1 \le s \le t \le 1,\ v \in \Real_+.
\end{equation}
\end{thm}

В \S\ref{sec:linearn} доказывается неравенство (\ref{eq:intro_to_prove_weighted}) для кусочно линейных $u$:

\begin{lm}
%\label{lm:weighted_linear}
Пусть функция $a(x', \cdot, u)$ чётна и удовлетворяет условию $(\ref{eq:intro_almostConcave})$.
Тогда, если $u$ --- неотрицательная кусочно линейная функция, то $\IWg( u ) \ge \IWg( \mon{u} )$.
\end{lm}

В \S\ref{sec:wlsc} устанавливается слабая полунепрерывность функционала $\IWg$
и оформляем в виде теоремы рассуждения, которые будем использовать для предельного перехода:

\begin{thm}
%\label{thm:uplift}
Пусть $B \subset A \subset \W(\overline{\Omega})$.
Предположим, что для каждого $u \in A$ найдётся последовательность $u_k \in B$ такая,
что $u_k \to u$ в $\W(\overline{\Omega})$ и $\IWg( u_k ) \to \IWg( u )$.
Тогда

\textbf{\textup{i)}}
Если для любой функции $v \in B$ выполнено $\IWg( \symm{v} ) \le \IWg( v )$,
то для любой функции $u \in A$ будет выполнено $\IWg( \symm{u} ) \le \IWg( u )$.

\textbf{\textup{ii)}}
Если для любой функции $v \in B$ выполнено $\IWg( \mon{v} ) \le \IWg( v )$,
то для любой функции $u \in A$ будет выполнено $\IWg( \mon{u} ) \le \IWg( u )$.
\end{thm}

В \S\ref{sec:sobolev_bounded} неравенство (\ref{eq:intro_to_prove_weighted}) доказывается для интегрантов с ограниченным ростом по производной:

\begin{thm}
%\label{thm:bounded_growth}
Пусть функция $a(x', \cdot, u)$ чётна и удовлетворяет условию $(\ref{eq:intro_almostConcave})$.
Тогда

\textbf{\textup{i)}} Неравенство (\ref{eq:intro_to_prove_weighted}) верно для произвольной неотрицательной $u \in Lip(\overline{\Omega})$.

\textbf{\textup{ii)}} Предположим, что для любых $x' \in \overline{\omega}, z \in \Real_+, p \in \Real$
функция $F$ удовлетворяет неравенству
$$F( x', z, p ) \le C ( 1 + |z|^{q^*} + |p|^q ),$$
где $\frac{1}{q^*} = \frac{1}{q} - \frac{1}{n}$, если $q < n$, либо $q^*$ любое в противном случае.
Если $q \le n$, то дополнительно предположим, что веса $a$ и $a_i$ ограничены.
Тогда неравенство (\ref{eq:intro_to_prove_weighted}) верно для произвольной неотрицательной $u \in W{}^1_q(\overline{\Omega})$.
\end{thm}

Глава \ref{chapt:unbounded} диссертации посвящена снятию условия ограниченного роста с интегранта.
Это удаётся сделать только в одномерном случае, поэтому далее $u \in \W[-1, 1]$ и
$$
\IWg( u ) = \int\limits_{-1}^1 F( u(x), a(x, u(x)) \abs{ u'(x) } )\, dx.
$$

В \S\ref{sec:notations_unbounded} формулируется одномерный вариант задачи.

В \S\ref{sec:monotone_weight} удаётся распространить результат статьи \cite{ASC} на случай функционала $\IWg$
и доказать отсутствие эффекта Лаврентьева в случае кусочной монотонности веса.

\begin{lm}
Пусть $a$ --- непрерывная функция, $a(\cdot, u)$ возрастает на $[-1, 0]$ и убывает на $[0, 1]$ для всех $u \ge 0$.
Тогда для любой функции $u \in \W[-1, 1]$, $u \ge 0$, найдётся последовательность $\{u_k\} \subset Lip[-1, 1]$, удовлетворяющая
\begin{equation}
u_k \to u \text{ в } \W[-1, 1] \quad \text{ и } \quad \IWg(u_k) \to \IWg(u).
\end{equation}
\end{lm}

\begin{thm}
Пусть функция $a$ непрерывна, чётна, убывает на $[0, 1]$ и удовлетворяет неравенству $(\ref{eq:intro_almostConcave})$.
Тогда для любой $u \in \W[-1, 1]$ выполнено $\IWg( \symm{u} ) \le \IWg( u )$.
\end{thm}

В \S\ref{sec:weight_properties} доказано несколько важных свойств весовых функций, удовлетворяющих необходимым условиям.
В частности установлена структура множества нулей весовых функций.

В \S\ref{sec:general_sobolev} с веса снимается требование монотонности и, тем самым,
неравенство (\ref{eq:intro_to_prove_weighted}) доказано в наиболее общем виде:

\begin{thm}
%\label{thm:unbounded_growth}
Пусть $F \in \mathfrak{F}$, функция $u \in \W[-1, 1]$ неотрицательна,
и весовая функция $a: [-1, 1] \times \Real_+ \to \Real_+$ непрерывна
и допустима для $u$.
Тогда справедливо неравенство $(\ref{eq:intro_to_prove_weighted})$.
\end{thm}

В \S\ref{sec:sobolev_pinned} завершается доказательство для функций, закреплённых на левом конце.

В \S\ref{sec:necessary_symm} доказано, что условия, накладываемые на вес в работе \cite{Brock},
являются необходимыми в случае симметричной перестановки:

\begin{thm}
Если неравенство $(\ref{eq:intro_brock})$ выполняется для произвольной $F \in \mathfrak{F}$ и произвольной кусочно линейной $u$,
то вес $a$ --- чётная и выпуклая по первому аргументу функция.
\end{thm}

И наконец, в \S\ref{sec:sobolev_symm} закрывается пробел в работе \cite{Brock} в одномерном случае:
\begin{thm}
Пусть $F \in \mathfrak{F}$, функция $u \in \W[-1, 1]$ неотрицательна,
и непрерывная весовая функция $a: [-1, 1] \times \Real_+ \to \Real_+$ чётна и выпукла по первому аргументу.
Тогда справедливо неравенство $(\ref{eq:intro_brock})$.
\end{thm}

В главе \ref{chapt:variable} диссертации рассмотрено обобщение неравенства Пойа-Сегё
на случай переменного показателя суммирования.
\mytodo{Здесь про $p(x)$-лапласиан.}
А именно, рассматриваются два функционала:
\begin{eqnarray*}
\J(u) &=& \int\limits_{-1}^1 |u'(x)|^{p(x)} dx \\
\I(u) &=& \int\limits_{-1}^1 ( 1 + | u'(x) |^2 )^{\frac {p(x)}{2}} dx.
\end{eqnarray*}

В \S\ref{sec:notations_variable} ставится задача и вводятся обозначения.

В \S\ref{sec:necessary_variable} получены условия, необходимые для выполнения неравенств
$\J(\symm{u}) \le \J(u)$ и $\I(\symm{u}) \le \I(u)$.

\begin{thm}
Пусть $\J(\symm{u}) \le \J(u)$ выполнено для любой кусочно линейной функции $u \ge 0$.
Тогда $p(x) \equiv const$.
\end{thm}

То есть изучение аналога неравенства Пойа-Сегё для функционала $\I$ теряет смысл.

\begin{thm}
\label{thm:intro_necessary_variable}
Если неравенство $\I(\symm{u}) \le \I(u)$ выполняется для всех кусочно линейных $u \ge 0$,
то $p$ чётна и выпукла.
Более того, выпукла следующая функция:
$$
K(s, x) = s ( 1 + s^{-2} )^{\frac {p(x)}{2}}, \qquad s > 0,\ x \in [-1, 1].
$$
\end{thm}

В \S\ref{sec:inequality_variable} показано, что условия, необходимые для выполнения неравенства $\I(\symm{u}) \le \I(u)$,
являются и достаточными.

\begin{lm}
Пусть $p$ чётна, а $K$ выпукла по совокупности переменных.
Тогда для любой кусочно линейной функции $u \in \Wf[-1, 1]$ выполнено $\I(\symm{u}) \le \I(u)$.
\end{lm}

\begin{thm}
Пусть $p$ чётна, а $K$ выпукла по совокупности переменных.
Тогда для любой функции $u \in \Wf[-1, 1]$ выполнено $\I(\symm{u}) \le \I(u)$.
\end{thm}

Условие выпуклости функции $K$ есть на самом деле неявное условие на функцию $p$.
В \S\ref{sec:sufficient_variable} приведены некоторые явные достаточные условия выполнения неравенства $\I(\symm{u}) \le \I(u)$.

Выпуклость функции $K$ есть неотрицательность гессиана $K$ (плюс выпуклость по какому-нибудь направлению),
которая в свою очередь сводится к условию
$$
q q'' \ge q'^2 B(w, q),
$$
где $w = {1 \over s^2}$, $q(x) = p(x) - 1$.

\begin{thm}
\label{thm:intro_sufficient}
Пусть $p(x)\ge1$ --- чётная непрерывная функция на $[-1, 1]$.

\textbf{\textup{i)}}
Если функция $(p(x)-1)^{0.37}$ выпукла, то неравенство $\I(\symm{u}) \le \I(u)$ выполнено для любой неотрицательной $u \in \Wf[-1, 1]$.

\textbf{\textup{ii)}}
Если $p(x) \le 2.36$ для всех $x \in [-1, 1]$ и функция $\sqrt{p(x) - 1}$ выпукла,
то неравенство $\I(\symm{u}) \le \I(u)$ выполнено для произвольной неотрицательной $u \in \Wf[-1, 1]$.
\end{thm}

В \S\ref{sec:calculations} описаны численно-аналитические методы для получения оценок,
на которых основаны выводы теоремы \ref{thm:intro_sufficient}.
Пусть ${\mathcal B}(q) \equiv \sup\limits_{w > 0} B(w, q)$. Тогда

\begin{eqnarray}
&& \sup\limits_{q \ge 0}{\mathcal B}(q) = \limsup\limits_{q \to +\infty}{\mathcal B}(q) \le 0.63; \\
&& \sup\limits_{0 \le q \le 1.36}{\mathcal B}(q) \le 0.5.
\end{eqnarray}

Наконец, в \S\ref{sec:multi_variable} показано, что прямое распространение неравенства $\I(\symm{u}) \le \I(u)$
на многомерный случай несодержательно:

\begin{thm}
Если
$\int\limits_{\Omega} ( 1 + | \nabla \symm{u}(x) |^2 )^{ \frac{p(x)}{2} } dx
\le \int\limits_{\Omega} ( 1 + | \nabla u(x) |^2 )^{ \frac{p(x)}{2} } dx$
для любой неотрицательной функции $u \in \Wf(\overline{\Omega})$,
то $p(x',y)$ не зависит от $y$.
\end{thm}


%Обзор, введение в тему, обозначение места данной работы в
%мировых исследованиях и~т.\:п., можно использовать ссылки на~другие
%работы\ifnumequal{\value{bibliosel}}{1}{~\autocite{Gosele1999161}}{}
%(если их~нет, то~в~автореферате
%автоматически пропадёт раздел <<Список литературы>>). Внимание! Ссылки
%на~другие работы в разделе общей характеристики работы можно
%использовать только при использовании \verb!biblatex! (из-за технических
%ограничений \verb!bibtex8!. Это связано с тем, что одна
%и~та~же~характеристика используются и~в~тексте диссертации, и в
%автореферате. В~последнем, согласно ГОСТ, должен присутствовать список
%работ автора по~теме диссертации, а~\verb!bibtex8! не~умеет выводить в одном
%файле два списка литературы).
%При использовании \verb!biblatex! возможно использование исключительно
%в~автореферате подстрочных ссылок
%для других работ командой \verb!\autocite!, а~также цитирование
%собственных работ командой \verb!\cite!. Для этого в~файле
%\verb!Synopsis/setup.tex! необходимо присвоить положительное значение
%счётчику \verb!\setcounter{usefootcite}{1}!.
%
%Для генерации содержимого титульного листа автореферата, диссертации
%и~презентации используются данные из файла \verb!common/data.tex!. Если,
%например, вы меняете название диссертации, то оно автоматически
%появится в~итоговых файлах после очередного запуска \LaTeX. Согласно
%ГОСТ 7.0.11-2011 <<5.1.1 Титульный лист является первой страницей
%диссертации, служит источником информации, необходимой для обработки и
%поиска документа>>. Наличие логотипа организации на титульном листе
%упрощает обработку и поиск, для этого разметите логотип вашей
%организации в папке images в формате PDF (лучше найти его в векторном
%варианте, чтобы он хорошо смотрелся при печати) под именем
%\verb!logo.pdf!. Настроить размер изображения с логотипом можно
%в~соответствующих местах файлов \verb!title.tex!  отдельно для
%диссертации и автореферата. Если вам логотип не~нужен, то просто
%удалите файл с логотипом.
%
%\ifsynopsis
%Этот абзац появляется только в~автореферате.
%Для формирования блоков, которые будут обрабатываться только в~автореферате,
%заведена проверка условия \verb!\!\verb!ifsynopsis!.
%Значение условия задаётся в~основном файле документа (\verb!synopsis.tex! для
%автореферата).
%\else
%Этот абзац появляется только в~диссертации.
%Через проверку условия \verb!\!\verb!ifsynopsis!, задаваемого в~основном файле
%документа (\verb!dissertation.tex! для диссертации), можно сделать новую
%команду, обеспечивающую появление цитаты в~диссертации, но~не~в~автореферате.
%\fi
%
%% {\progress}
%% Этот раздел должен быть отдельным структурным элементом по
%% ГОСТ, но он, как правило, включается в описание актуальности
%% темы. Нужен он отдельным структурынм элемементом или нет ---
%% смотрите другие диссертации вашего совета, скорее всего не нужен.
%
%{\aim} данной работы является \ldots
%
%Для~достижения поставленной цели необходимо было решить следующие {\tasks}:
%\begin{enumerate}
%  \item Исследовать, разработать, вычислить и~т.\:д. и~т.\:п.
%  \item Исследовать, разработать, вычислить и~т.\:д. и~т.\:п.
%  \item Исследовать, разработать, вычислить и~т.\:д. и~т.\:п.
%  \item Исследовать, разработать, вычислить и~т.\:д. и~т.\:п.
%\end{enumerate}
%
%
%{\novelty}
%\begin{enumerate}
%  \item Впервые \ldots
%  \item Впервые \ldots
%  \item Было выполнено оригинальное исследование \ldots
%\end{enumerate}
%
%{\influence} \ldots
%
%{\methods} \ldots
%
%{\defpositions}
%\begin{enumerate}
%  \item Первое положение
%  \item Второе положение
%  \item Третье положение
%  \item Четвертое положение
%\end{enumerate}
%В папке Documents можно ознакомиться в решением совета из Томского ГУ
%в~файле \verb+Def_positions.pdf+, где обоснованно даются рекомендации
%по~формулировкам защищаемых положений.
%
%{\reliability} полученных результатов обеспечивается \ldots \ Результаты находятся в соответствии с результатами, полученными другими авторами.
%
%
%{\probation}
%Основные результаты работы докладывались~на:
%перечисление основных конференций, симпозиумов и~т.\:п.
%
%{\contribution} Автор принимал активное участие \ldots
%
%%\publications\ Основные результаты по теме диссертации изложены в ХХ печатных изданиях~\cite{Sokolov,Gaidaenko,Lermontov,Management},
%%Х из которых изданы в журналах, рекомендованных ВАК~\cite{Sokolov,Gaidaenko},
%%ХХ --- в тезисах докладов~\cite{Lermontov,Management}.
%
%\ifnumequal{\value{bibliosel}}{0}{% Встроенная реализация с загрузкой файла через движок bibtex8
%    \publications\ Основные результаты по теме диссертации изложены в XX печатных изданиях,
%    X из которых изданы в журналах, рекомендованных ВАК,
%    X "--- в тезисах докладов.%
%}{% Реализация пакетом biblatex через движок biber
%%Сделана отдельная секция, чтобы не отображались в списке цитированных материалов
    \begin{refsection}[vak,papers,conf]% Подсчет и нумерация авторских работ. Засчитываются только те, которые были прописаны внутри \nocite{}.
%        %Чтобы сменить порядок разделов в сгрупированном списке литературы необходимо перетасовать следующие три строчки, а также команды в разделе \newcommand*{\insertbiblioauthorgrouped} в файле biblio/biblatex.tex
%        \printbibliography[heading=countauthorvak, env=countauthorvak, keyword=biblioauthorvak, section=1]%
%        \printbibliography[heading=countauthorconf, env=countauthorconf, keyword=biblioauthorconf, section=1]%
%        \printbibliography[heading=countauthornotvak, env=countauthornotvak, keyword=biblioauthornotvak, section=1]%
%        \printbibliography[heading=countauthor, env=countauthor, keyword=biblioauthor, section=1]%
%        \nocite{%Порядок перечисления в этом блоке определяет порядок вывода в списке публикаций автора
%                vakbib1,vakbib2,%
%                confbib1,confbib2,%
%                bib1,bib2,%
%        }%
%        \publications\ Основные результаты по теме диссертации изложены в~\arabic{citeauthor}~печатных изданиях,
%        \arabic{citeauthorvak} из которых изданы в журналах, рекомендованных ВАК,
%        \arabic{citeauthorconf} "--- в~тезисах докладов.
    \end{refsection}
%    \begin{refsection}[vak,papers,conf]%Блок, позволяющий отобрать из всех работ автора наиболее значимые, и только их вывести в автореферате, но считать в блоке выше общее число работ
%        \printbibliography[heading=countauthorvak, env=countauthorvak, keyword=biblioauthorvak, section=2]%
%        \printbibliography[heading=countauthornotvak, env=countauthornotvak, keyword=biblioauthornotvak, section=2]%
%        \printbibliography[heading=countauthorconf, env=countauthorconf, keyword=biblioauthorconf, section=2]%
%        \printbibliography[heading=countauthor, env=countauthor, keyword=biblioauthor, section=2]%
%        \nocite{vakbib2}%vak
%        \nocite{bib1}%notvak
%        \nocite{confbib1}%conf
%    \end{refsection}
%}
%При использовании пакета \verb!biblatex! для автоматического подсчёта
%количества публикаций автора по теме диссертации, необходимо
%их~здесь перечислить с использованием команды \verb!\nocite!.
% % Характеристика работы по структуре во введении и в автореферате не отличается (ГОСТ Р 7.0.11, пункты 5.3.1 и 9.2.1), потому её загружаем из одного и того же внешнего файла, предварительно задав форму выделения некоторым параметрам
%
%\textbf{Объем и структура работы.} Диссертация состоит из~введения, четырёх глав, заключения и~двух приложений.
%%% на случай ошибок оставляю исходный кусок на месте, закомментированным
%%Полный объём диссертации составляет  \ref*{TotPages}~страницу с~\totalfigures{}~рисунками и~\totaltables{}~таблицами. Список литературы содержит \total{citenum}~наименований.
%%
%Полный объём диссертации составляет
%\formbytotal{TotPages}{страниц}{у}{ы}{}, включая
%\formbytotal{totalcount@figure}{рисун}{ок}{ка}{ков} и
%\formbytotal{totalcount@table}{таблиц}{у}{ы}{}.   Список литературы содержит
%\formbytotal{citenum}{наименован}{ие}{ия}{ий}.

\mytodo{bigl, bigr}

\mytodo{Свои работы отдельно. Тезисы тоже считаются.}

\mytodo{Во введении кроме исторической части ещё изложение результатов.}

\mytodo{пройти по всем ссылкам и посмотреть, что все факты правильно называются: предложение вместо леммы}

\mytodo{проверить орфографию}

\mytodo{Make use of \textbackslash norm, \textbackslash abs, \textbackslash meas, \textbackslash set and other commands}

\mytodo{fix discontinuity in graphs}

\mytodo{introduce a section with formulations of external assertions, если нельзя их всех сделать в тексте, как сейчас во многих случаях сделано}

\mytodo{широкую библиографию по эффекту Лаврентьева. Посмотреть кроме письма ещё в BGH}

Пусть $\Omega = \omega \times [-1,1]$,
где $\omega$ --- ограниченная область в $\Real^{n - 1}$ с липшицевой границей.
Обозначим $x = ( x_1, \dots, x_{n - 1}, y ) = ( x', y )$.

Напомним теорему о послойном представлении измеримой неотрицательной функции $u$, заданной на $\Omega$
(см. \cite[Теорема 1.13]{LiebLoss}).
Положим $\mathcal{A}_t(x') := \{ y \in [-1,1] :\ u( x', y ) > t \}$.
Тогда имеет место равенство
$$
u(x', y) = \int_0^\infty \charf{\mathcal{A}_t(x')}(y) dt,
$$
где $\charf{A}$ --- характеристическая функция множества $A$.

Определим симметричную перестановку измеримого множества $E \subset [-1, 1]$ и
симметричную перестановку (симметризацию по Штейнеру) неотрицательной функции $u \in \W(\Omega)$:
\begin{eqnarray*}
\symm{E} := [-\frac{\abs{E}}{2}, \frac{\abs{E}}{2}]; \qquad
\symm{u}(x', y) = \int\limits_0^\infty \charf{ \symm{( \mathcal{A}_t(x') )} }(y) dt.
\end{eqnarray*}

В тех же условиях определим монотонную перестановку множества $E$ и функции $u \in \W(\Omega)$:
\begin{eqnarray*}
\mon{E} := [1 - \meas E, 1]; \qquad
\mon{u}(x', y) = \int\limits_0^\infty \charf{ \mon{ \mathcal{A}_t(x') } }(y) dt.
\end{eqnarray*}

Возьмём выпуклую чётную функцию $F$ и рассмотрим функционал
\begin{equation}
\label{PS_functional}
\IPS( u ) = \int_\Omega F( \norm{ \nabla u } ) dx.
\end{equation}
Для такого функционала хорошо известно классическое неравенство Пойа-Сегё: $\IPS( \symm{u} ) \le \IPS( u )$.

\mytodo{fix}
В данной диссертации мы рассматриваем обобщения неравенства Пойа-Сегё на более общие классы функционалов.
В первой и второй главах мы рассматриваем взвешенные аналоги классического функционала.

Определим множество $\mathfrak{F}$ непрерывных функций $F: \omega \times \Real_+ \times \Real_+ \to \Real_+$
(здесь и далее $\Real_+ = [0, \infty)$),
выпуклых и строго возрастающих по третьему аргументу, удовлетворяющих $F( \cdot, \cdot, 0 ) \equiv 0$.

Рассмотрим функционал:
\begin{equation}
\IWg( u ) = \int\limits_\Omega F( x', u(x), \norm{ \mathcal{D} u } ) dx,
\end{equation}
где $F \in \mathfrak{F}$,
$\norm{\cdot}$ --- некоторая норма в $\Real^n$, симметричная по последней координате,
то есть удовлетворяющая $\norm{( x', y )} = \norm{( x', -y )}$,
$$\mathcal{D} u = ( a_1( x', u( x ) ) D_1 u,\ \dots,\ a_{n - 1}( x', u( x ) ) D_{n - 1} u,\ a( x, u( x ) ) D_n u )$$
--- градиент $u$ с весом (обратите внимание, что только вес при $D_n u$ зависит от $y$),
$a( \cdot, \cdot ): \Omega \times \Real_+ \to \Real_+$ и $a_i( \cdot, \cdot ): \omega \times \Real_+ \to \Real_+$ --- непрерывные функции.
Здесь и далее индекс $i$ пробегает от $1$ до $n - 1$.
Очевидно, что при $a_i = a \equiv 1$ выполнено $\IWg \equiv \IPS$.

\mytodo{fix}
В первой главе мы рассматриваем аналог неравенства Пойа-Сегё для монотонной перестановки с функционалом (\ref{weighted_functional}):
\begin{equation}
\IWg( \mon{u} ) \le \IWg( u )
\end{equation}
Мы устанавливаем необходимые для выполнения неравенства условия на весовую функцию $a$.
Также мы доказываем неравенство при необходимых условиях и дополнительном ограничении на рост интегранта по производной.

Во второй главе мы снимаем требование ограничения роста,
и также доказываем аналогичный результат для симметричной перестановки:
\begin{equation}
\IWg( \symm{u} ) \le \IWg( u ),
\end{equation}
тем самым закрывая пробел в работе \cite{Brock}.

\mytodo{оставить здесь только мотивацию/историю}