\section{Доказательство неравенства (\ref{eq:to_prove_weighted}) для функций, закреплённых на левом конце}
\label{sec:sobolev_pinned}

Рассмотрим теперь случай, когда функция $u$ удовлетворяет дополнительному условию $u(-1) = 0$.
\begin{thm}
Пусть $F \in \mathfrak{F}$, функция $u \in \W[-1, 1]$ неотрицательна, $u(-1) = 0$,
весовая функция $a: [-1, 1] \times \Real_+ \to \Real_+$ непрерывна и удовлетворяет неравенству $(\ref{eq:almostConcave})$.
Тогда справедливо неравенство $(\ref{eq:to_prove_weighted})$.
\end{thm}

\begin{proof}
Мы следуем схеме доказательства теоремы \ref{thm:unbounded_growth},
но вместо $(H1)$ и $(H7)$ накладываем следующие условия на вес:

\bigskip
\noindent
$(H1')$ $a(x, v)$ удовлетворяет неравенству (\ref{eq:almostConcave}), а также $\IWg(a, u) < \infty$.

\bigskip
\noindent
$(H7')$ Выполнено условие $(H7)$, и $a(\cdot, v) \equiv 0$ в некоторой $v$-окрестности нуля.

\bigskip
\textbf{Шаг 1.}
\textit{Пусть $u \in \W[-1, 1]$, выполнено $u(-1) = 0$, и вес $a$ удовлетворяет условиям $(H1'), (H2)-(H6), (H7')$.
Тогда выполняется неравенство (\ref{eq:to_prove_weighted}).}

Для доказательства будем приближать функцию $u$ так же, как и в первом шаге доказательства теоремы \ref{thm:unbounded_growth},
с заменой $u$ в некоторой окрестности точки $x = -1$ на линейную так, чтобы $u_n(-1) = 0$.

\bigskip
\textbf{Шаг 2.}
\textit{Пусть вес $a$ удовлетворяет условиям $(H1'), (H2)-(H6)$.
Тогда выполняется неравенство (\ref{eq:to_prove_weighted}).}

Для доказательства добавим в множество $W$ из второго шага доказательства теоремы \ref{thm:unbounded_growth} точку $0$
и повторим рассуждение.

\medskip

Дальнейшие шаги проходят без изменений.
\end{proof}
