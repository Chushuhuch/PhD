\section{Доказательство неравенства (\ref{toprove_weighted}) для произвольных весов}
\label{moveForth}

Теперь мы хотим избавиться от условия монотонности веса по $x$.
Будем это делать в несколько этапов.

Для начала отметим, что все свойства функции $a$ интересуют нас лишь в окрестности графиков функций $u$, $\mon{u}$.

Мы вводим несколько ограничений на весовую функцию.
Каждое следующее, будучи добавленным к предыдущим, задаёт более узкий класс весов. 

\bigskip

\smallskip
\noindent
$(H1)$ $a(x, v)$ чётна по $x$ и удовлетворяет неравенству (\ref{almostConcave}), а также $\IWg(a, u) < \infty$.
\smallskip

\bigskip
\noindent
$(H2)$ На множестве $v \in [\min u(x), \max u(x)]$, для которых $a(\cdot, v) \not\equiv 0$,
количество нулей функций $a(\cdot, v)$ ограничено константой, не зависящей от $v$.

\bigskip
\noindent
$(H3)$ Если $a(x_0, u(x_0)) = 0$ для некоторого $x_0$, то $a(\cdot, u(x_0)) \equiv 0$.
Кроме того, выполнено $\lim_{k \to \infty} D_k(a, U(a)) = 0$, где
$$U(a) := \{ v \in [\min u(x), \max u(x)]: a(\cdot, v) \not \equiv 0 \},$$
\begin{equation}
\label{bigD}
D_k(a, U) := \sup \limits_{v \in U}
\frac{\max \limits_{\abs{x_1 - x_2} \le \frac{2}{k}} \abs{a(x_1, v) - a(x_2, v)}}
{\min \limits_{\dist (x, u^{-1} (v)) \le \frac{2}{k}} a(x, v)}.
\end{equation}

\bigskip
\noindent
$(H4)$ Найдётся такое чётное $k$, что $a(\cdot, v)$ линейны для каждого $v$ на участках
$[-1 + \frac{2i}{k}, -1 + \frac{2(i + 1)}{k}]$.

\bigskip
\noindent
$(H5)$ Множество $v \in \Real$, для которых $a(\cdot, v)$ имеет участки постоянства,
отличается от множества $v \in \Real$ таких, что $a(\cdot, v) \equiv 0$,
лишь на множество меры $0$.

\bigskip
\noindent
$(H6)$ Отрезок $[-1, 1]$ можно разбить на конечное число промежутков,
на каждом из которых в $v$-окрестности графика $u(x)$ вес $a$ не меняет монотонности по $x$.

\bigskip
\noindent
$(H7)$ Пусть $x_1 < x_2 < x_3$,
и на $[x_1, x_2]$ вес $a(\cdot, v)$ в $v$-окрестности графика функции $u$ убывает,
а на $[x_2, x_3]$ возрастает.
Тогда в некоторой окрестности точки $u(x_2)$ имеем $a(\cdot, v) \equiv 0$.
\todo{Выглядит как бред. Проверить. Вроде беда с параметрами $a$}

\bigskip

Вес, удовлетворяющий условию $(H1)$, мы будем называть допустимым для заданной функции $u(x)$.

\medskip

Теперь мы можем сформулировать основное утверждение.
\begin{thm}
\label{unbounded_growth_thm}
Пусть $F \in \mathfrak{F}$, функция $u \in \W(-1, 1)$ неотрицательна,
и весовая функция $a: [-1, 1] \times \Real_+ \to \Real_+$ непрерывна
и допустима для $u$.
Тогда справедливо неравенство $(\ref{toprove_weighted})$.
\end{thm}

Мы докажем неравенство (\ref{toprove_weighted}) при условиях $(H1)-(H7)$,
а затем будем постепенно избавляться от них.

Для доказательства нам потребуются следующие факты.

\begin{prop}
\label{levelDerivative}
\textrm{\cite[Theorem 6.19]{LiebLoss} }
Для любой $u \in \W(-1, 1)$ и произвольного множества $A \subset \Real$ нулевой меры выполнено
$u'(x) = 0$ для почти всех $x \in u^{-1}(A)$.
\end{prop}

\begin{lm}
\label{zeroApprox}
Пусть $u \in \W(-1, 1)$ неотрицательна.
И пусть замкнутое множество $W \subset \Real_+$ таково, что множество всех $v \in W$, для которых $a(\cdot, v) \not\equiv 0$, имеет меру ноль.
Тогда найдётся возрастающая последовательность весов $\mathfrak b_{\ell}$ такая, что

1) $\mathfrak b_{\ell}(\cdot, v) \rightrightarrows a(\cdot, v)$ для почти всех $v$;

2) $\mathfrak b_{\ell}(\cdot, v) \equiv 0$ для каждого $v$ в некоторой (зависящей от $\ell$) окрестности $W$;

3) $\IWg(\mathfrak b_{\ell}, u) \to \IWg(a, u)$ и $\IWg(\mathfrak b_{\ell}, \mon{u}) \to \IWg(a, \mon{u})$.
\end{lm}

\begin{rem}
Если $a$ допустимы для $u$, то и $\mathfrak b_{\ell}$ тоже.
\end{rem}

\begin{proof}
Возьмём $\rho(d) := \min(1, \max(0, d))$,
$$
\mathfrak b_{\ell}(x, v) := a(x, v) \cdot \rho(\ell \dist(v, W) - 1) \le a(x, v).
$$
Вес $\mathfrak b_{\ell}$ равен нулю в $\left(\frac{1}{\ell}\right)$-окрестности $W$.
Кроме того, $\mathfrak b_{\ell} \equiv a$ вне $\left(\frac{2}{\ell}\right)$-окрестности $W$,
а также $\mathfrak b_{\ell}(x, v)$ возрастает по $\ell$.
Тем самым \todo{тем самым или это ещё один очевидный факт?}, $\mathfrak b_{\ell}(\cdot, v) \rightrightarrows a(\cdot, v)$ для почти всех $v$.
По теореме о монотонной сходимости имеем
$\IWg(u^{-1}(\Real_+ \setminus W), \mathfrak b_{\ell}, u) \nearrow \IWg(u^{-1}(\Real_+ \setminus W), a, u)$.

Разобьем множество $W$ на два: $W_1 := \{v \in W: a(\cdot, v) \equiv 0\}$ и $W_2 = W \setminus W_1$.
Тогда
$$
\begin{aligned}
\IWg(u^{-1}(W_1), \mathfrak b_{\ell}, u) &= \IWg(u^{-1}(W_1), a, u),\\
\IWg(u^{-1}(W_2), \mathfrak b_{\ell}, u) &= \int\limits_{x \in u^{-1}(W_2)} F \bigl( u(x), \mathfrak b_{\ell}(x, u(x)) |u'(x)| \bigr) \, dx.
\end{aligned}
$$
При этом, по предложению \ref{levelDerivative}, почти всюду на $u^{-1}(W_2)$ выполнено $u'(x) = 0$.
То есть
$$
\IWg(u^{-1}(W_2), \mathfrak b_{\ell}, u) = \int\limits_{x \in u^{-1}(W_2)} F \bigl( u(x), 0 \bigr) \, dx = 0.
$$
Аналогично $\IWg(u^{-1}(W_2), a, u) = 0$, откуда $\IWg(\mathfrak b_{\ell}, u) \to \IWg(a, u)$.
Вторая часть пункта \textit{3)} доказывается так же.
\end{proof}

Перейдем к доказательству теоремы.

\bigskip
\textbf{Шаг 1.}
\textit{Пусть $u \in \W(-1, 1)$, и вес $a$ удовлетворяет условиям $(H1)-(H7)$.
Тогда выполняется неравенство (\ref{toprove_weighted}).}

Разобьем отрезок $[-1, 1]$ на отрезки $\Delta_k = [\hat{x}_k, \hat{x}_{k + 1}]$, состоящие из двух частей.
В левой части каждого отрезка вес $a$ будет возрастать по $x$ в окрестности графика $u(x)$,
в правой же будет убывать.
Согласно замечанию \ref{monotone_weight_appr_rem}
на каждом таком отрезке можно повторить схему из леммы \ref{monotone_weight_appr_lm},
приближая функцию $u$ липшицевыми функциями $u_n$.
Это даёт $\IWg(\Delta_k, a, u_n) \to \IWg(\Delta_k, a, u)$.

Однако при такой аппроксимации функции $u_n$ могут иметь разрывы в точках $\hat{x}_k$.

Заметим теперь, что согласно условию $(H7)$ можно выбрать точки $\hat{x}_k$ так,
что $a \equiv 0$ в $(x, v)$-окрестности точек $(\hat{x}_k, u(\hat{x}_k))$.

Изменим теперь функции $u_n$ в окрестности точек $\hat{x}_k$ на линейные,
сделав $u_n$ непрерывными на $[-1, 1]$.
В силу вышесказанного, интегралов $\IWg(\Delta_k, a, u_n)$ это не изменит,
и мы получаем $\IWg(a, u_n) \to \IWg(a, u)$ и $u_n \to u$ в $\W(-1, 1)$.

По лемме \ref{uplift} получаем (\ref{toprove_weighted}).

\bigskip

\textbf{Шаг 2.}
\textit{Пусть вес $a$ удовлетворяет условиям $(H1)-(H6)$.
Тогда выполняется неравенство (\ref{toprove_weighted}).}

Применим лемму \ref{zeroApprox}.
В качестве множества $W$ возьмем множество всех $v$,
при которых происходит переход графика $u(x)$ из промежутка,
в котором вес убывает по $x$, в промежуток, в котором вес возрастает.
Очевидно, получившиеся функции $\mathfrak b_{\ell}$ удовлетворяют $(H1)-(H7)$.
Из шага 1 имеем $\IWg(\mathfrak b_{\ell}, \mon{u}) \le \IWg(\mathfrak b_{\ell}, u)$.
Переходя к пределу, получаем требуемое неравенство (\ref{toprove_weighted}).

\bigskip

\textbf{Шаг 3.}
\textit{Пусть вес $a$ удовлетворяет условиям $(H1)-(H5)$.
Тогда выполняется неравенство (\ref{toprove_weighted}).}

Рассмотрим абсциссы точек излома функции $a$ и ординаты, для которых $a$ имеет участки постоянства.
Эти абсциссы и ординаты определяют деление прямоугольника $[-1, 1] \times [\min u(x), \max u(x)]$
на более мелкие, внутри которых вес $a$ не меняет монотонности.
Однако, количество мелких прямоугольников может оказаться бесконечным.
Кроме того, если функция пересекает горизонтальную границу прямоугольника,
монотонность в $v$-окрестности точки пересечения может меняться.

Возьмем множество $W$ точек $v$, для которых вес $a$ имеет участки постоянства по $x$.
В соответствии с $(H5)$ множество $v \in W$, для которых $a(\cdot, v) \not\equiv 0$, имеет нулевую меру.

Применив лемму \ref{zeroApprox}, построим последовательность весов $\mathfrak b_{\ell}$.
У каждого из них количество участков монотонности конечно,
поскольку между соседними по $v$ участками строгой монотонности
присутствует полоса нулевых значений веса шириной по крайней мере $\frac{2}{k}$.

Тем самым, вес $\mathfrak  b_{\ell}$ может менять монотонность вдоль графика $u$
либо в точках $x = -1 + \frac{2 i}{k}$, либо в тех местах, где график пересекает полосу нулевых значений веса.
Ясно, что таких пересечений может быть лишь конечное число, поскольку
$\int |u'|$ увеличивается как минимум на $\frac{2}{\ell}$ во время такого перехода, а $u' \in \Ls(-1, 1)$.

Мы получили, что $\mathfrak b_{\ell}$ удовлетворяют $(H1)-(H6)$.
Из шага 2 имеем $\IWg(\mathfrak b_{\ell}, \mon{u}) \le \IWg(\mathfrak b_{\ell}, u)$.
Переходя к пределу, получаем (\ref{toprove_weighted}).

\bigskip
\textbf{Шаг 4.}
\textit{Пусть вес $a$ удовлетворяет условиям $(H1)-(H3)$.
Тогда выполняется неравенство (\ref{toprove_weighted}).}

Предположим, что функция  $a$ удовлетворяет $(H1)-(H3)$, в частности $\IWg(a, u) < \infty$.

Зафиксируем произвольное четное $k$.
По точкам $a(-1 + \frac{2i}{k}, v)$ для каждого $v$ построим кусочно линейную по $x$ интерполяцию.
Получившаяся функция $a_k(x, v)$ непрерывна, четна по $x$
и по лемме \ref{piecewiseLinearConcave} удовлетворяет неравенству (\ref{almostConcave}).
Кроме того, $a_k \to a$ при $k \to \infty$, причем сходимость равномерная на компактах.
Однако неравенство $a_k(x, u(x)) \le a(x, u(x))$ не обязано выполняться,
и потому веса $a_k$ могут не быть допустимыми для $u$.

Возьмем $\mathfrak c_k := (1 - D_k(a_k, U(a_k))) a_k$, где $D_k$ определены в (\ref{bigD}).
Числа $D_k(a_k, U(a_k))$ положительны и стремятся к нулю, поэтому $\mathfrak c_k \to a$ при $k \to \infty$.
Покажем, что $\mathfrak c_k(x, u(x)) \le a(x, u(x))$.
Возьмем некоторое число
$x \in [-1 + \frac{2i}{k}, -1 + \frac{2(i + 1)}{k}] =: [x_i, x_{i + 1}]$.
Тогда $\mathfrak c_k(x, u(x)) \le \max( \mathfrak c_k(x_i, u(x)), \mathfrak c_k(x_{i + 1}, u(x)) )$,
поскольку $\mathfrak c_k$ кусочно линейны по $x$.
Далее,
\begin{multline*}
\mathfrak c_k(x_i, u(x)) = ( 1 - D_k(a_k, U(a_k))) \cdot a(x_i, u(x)) \\
\le a(x_i, u(x)) - \frac{a(x_i, u(x)) - a(x, u(x))}{a(x_i, u(x))} \cdot a(x_i, u(x)) = a(x, u(x)).
\end{multline*}
Аналогично, $\mathfrak c_k(x_{i + 1}, u(x)) \le a(x, u(x))$.
Тем самым, $\mathfrak c_k(x, u(x)) \le a(x, u(x))$ для любого $x$, и $\mathfrak c_k$ являются допустимыми для $u$.
То есть функции $\mathfrak c_k$ удовлетворяют $(H1)-(H4)$.

При заданном $k \in \Nat$, будем приближать функцию $\mathfrak c_k =: \mathfrak c$ весами, удовлетворяющими $(H1)-(H5)$.
Рассмотрим вспомогательную функцию $\Lambda(x) = 1 - \abs{x}$, удовлетворяющую условию (\ref{almostConcave}).

Возьмем
$$
t(v):=D_k(\mathfrak c, U(\mathfrak c)) \cdot \max\{\tau \ge 0: \forall x \in u^{-1}(v) \quad \tau \Lambda(x) \le \mathfrak c(x, u(x))\}.
$$
Функция $t$ зависит от $k$, но мы будем опускать это в записи.

Ясно, что максимальное $\tau$ равно нулю только если $c(\cdot, v) \equiv 0$, иначе нарушается условие $(H3)$.
Функция $t$ может не быть непрерывной. Однако, несложно видеть, что она полунепрерывна снизу.
Возьмем теперь
$$
\tilde{t}(v) := \inf_{w \in u([-1, 1])} \{t(w) + |v - w|\}.
$$
Очевидно, что $\tilde{t} \le t$, и множества нулей функций $t$ и $\tilde{t}$ совпадают.

Покажем, что $\tilde{t}$ непрерывна (и даже липшицева).
Зафиксируем некоторое $v_1$.
Тогда найдутся сколь угодно малое $\eps > 0$ и $w_1 \in u([-1, 1])$,
удовлетворяющие $\tilde{t}(v_1) = t(w_1) + |v_1 - w_1| - \eps$.
Для любого $v_2$ имеем $\tilde{t}(v_2) \le t(w_1) + |v_2 - w_1|$.
И, тем самым, $\tilde{t}(v_2) - \tilde{t}(v_1) \le |v_1 - v_2| + \eps$.
В силу произвольности $v_1$, $v_2$ и $\eps$, получаем, что $\tilde{t}$ непрерывна.

При $\alpha \in [0, 1]$ функция $\mathfrak d_\alpha(x, v) := \mathfrak c(x, v) + \alpha \Lambda(x) \tilde{t}(v)$
чётна $x$, удовлетворяет неравенству (\ref{almostConcave}) согласно лемме \ref{maxSumConcave},
и не превосходит $a(x, v)$ по построению функции $\tilde{t}$.
Таким образом, $\mathfrak d_\alpha$ --- допустимый вес.
И теперь очевидно, что $\mathfrak d_\alpha$ удовлетворяет условиям $(H1)-(H4)$.

Покажем, что найдётся последовательность $\alpha_j \searrow 0$, что $\mathfrak d_{\alpha_j}$ не имеет горизонтальных участков,
кроме $v$, для которых $\mathfrak d_{\alpha_j}(\cdot, v) \equiv 0$, и множества меры ноль.
Обозначим множество $\alpha$, ``плохих'' на участке $[x_i, x_{i + 1}]$:
\begin{multline*}
A_i := \bigl \{\alpha \in [0, 1]: \\
\meas \{v \in [\min u, \max u]: \frac{\mathfrak c(x_{i + 1}, v) - \mathfrak c(x_i, v))}{\frac{2}{k}} + \alpha \chi_i \tilde{t} (v) = 0 \} > 0 \bigr \},
\end{multline*}
где $\chi_i = 1$ если $[x_i, x_{i + 1}] \subset [0, 1]$, и $\chi_i = -1$, если $[x_i, x_{i + 1}] \subset [-1, 0]$.

Рассмотрим функцию
$$
\begin{aligned}
h_i(v) = & \frac{\mathfrak c(x_{i + 1}, v) - \mathfrak c(x_i, v)}{\tilde{t} (v)} & \text{ при } \tilde{t} (v) \neq 0 & \\
h_i(v) = & 0 & \text{ при } \tilde{t} (v) = 0 &.
\end{aligned}
$$
Тогда $\card(A_i) = \card(\{ \alpha \in [0, 1]: \meas \{ v \in [\min u, \max u]: h_i(v) \pm \frac{2}{k} \alpha = 0 \} > 0 \}).$
Значит $\card(A_i) \le \aleph_0$, а также $\card(\cup_i A_i) \le \aleph_0$.
Тем самым, найдётся последовательность весов $\mathfrak d_{\alpha_j} \searrow \mathfrak c$, удовлетворяющих $(H1)-(H5)$.
Из шага 3 имеем $\IWg(\mathfrak d_{\alpha_j}, \mon{u}) \le \IWg(\mathfrak d_{\alpha_j}, u)$.
Переходя к пределу, получаем $\IWg(\mathfrak c, \mon{u}) \le \IWg(\mathfrak c, u)$.

Далее, при $x \in [-1, 1]$ имеем
\begin{equation}
\label{step4Conv}
F \bigl( u(x), \mathfrak c_k(x, u(x)) |u'(x)| \bigr) \to F \bigl( u(x), a(x, u(x)) |u'(x)| \bigr)
\end{equation}
при $k \to \infty$.
Кроме того, $F \bigl( u(x), a(x, u(x)) |u'(x)| \bigr)$ является суммируемой мажорантой для левой части соотношения (\ref{step4Conv}).
По теореме Лебега о мажорируемой сходимости, получаем $\IWg(\mathfrak c_k, u) \to \IWg(a, u)$.
Поскольку $\IWg(\mathfrak c_k, \mon{u}) \le \IWg(\mathfrak c_k, u)$, лемма \ref{uplift} даёт неравенство (\ref{toprove_weighted}).

\bigskip
\textbf{Шаг 5.}
\textit{Пусть вес $a$ удовлетворяет лишь условию $(H1)$.
Тогда выполняется неравенство (\ref{toprove_weighted}).}

Будем строить приближение для $a$ весами, удовлетворяющими $(H1)-(H2)$.
Воспользуемся леммой \ref{zeroApprox} с множеством $W = \set{ v \in \Real_+: a(\cdot, v) \equiv 0 }$.
Введем обозначение
$$
Z_a(v) := \{ x \in [-1, 1]: a(x, v) = 0 \}.
$$
Заметим, что множества $Z_{\mathfrak b_{\ell}}(v)$ совпадают либо с $Z_{a}(v)$, либо с $[-1, 1]$.

Покажем, что  $\mathfrak b_{\ell}$ удовлетворяет $(H2)$.
Действительно, в противном случае найдётся последовательность $v_m$, для которой
$m < \card(Z_{\mathfrak b_{\ell}})(v_m) < \infty$.
После перехода к подпоследовательности имеем $v_m \to v_0$.
Из части 2 леммы \ref{periodicity} следует,
что множества $Z_{\mathfrak b_{\ell}}(v_m) = Z_{a}(v_m)$ периодические с периодом не более $\frac{2}{m - 1}$.
Возьмем некоторый $x \in [-1, 1]$.
Для каждого $m$ найдётся $x_m$ такой, что
$\abs{x - x_m} \le \frac{1}{m - 1}$ и $a(x_m, v_m) = 0$.
Но $a(x_m, v_m) \to a(x, v_0)$.
Тем самым, $a(x, v_0) = 0$.

Отсюда $Z_{a}(v_0) = [-1, 1]$.
Но это означает, что для любого $v$, для которого $\abs{v - v_0} \le \frac{1}{\ell}$,
выполнено $\mathfrak b_{\ell}(\cdot, v) \equiv 0$,
что противоречит $\card(Z_{\mathfrak b_{\ell}})(v_m) < \infty$.

Зафиксируем теперь $\ell \in \Nat$, обозначим $\mathfrak b_{\ell} =: \mathfrak b$
и приблизим функцию $\mathfrak b$ весами, удовлетворяющими $(H1)-(H3)$.
Из $(H2)$ следует, что найдётся множество $T \subset [-1, 1]$
состоящее из конечного числа элементов, такое, что
если $x \not\in T$ и $\mathfrak b(x, v) = 0$ для некоторого $v$, то $\mathfrak b(\cdot, v) \equiv 0$.

Вновь воспользуемся леммой \ref{zeroApprox} с множеством $W = u(T) \cup \mon{u}(T)$.
Полученные при помощи леммы веса $\mathfrak c_j$ удовлетворяют $(H1)-(H2)$,
поскольку отличаются от $\mathfrak b$ лишь домножением на непрерывный множитель,
меньший единицы и зависящий только от $v$.

Из непрерывности $u$ следует, что для достаточно больших $k$ найдутся $j = j(k)$ такие, что
$$
u \Bigl( \Bigl\{ x \in [-1, 1]: dist(x, T) \le \frac{4}{k} \Bigr\} \Bigr) \subset \Bigl\{ v \in \Real_+: dist(v, u(T)) \le \frac{1}{2j} \Bigr\},
$$
и $j(k) \to \infty$ при $k \to \infty$.
Отсюда $\min\limits_{dist(x, u^{-1}(v)) \le \frac{2}{k}} c_j(x, v) > 0$ для всех $v \in U(c_j)$.
Более того, при $v \in U(c_j)$
$$
\frac{\max\limits_{\abs{x_i - x_{i + 1}} \le \frac{2}{k}} \abs{\mathfrak c_j(x_i, v) - \mathfrak c_j(x_{i + 1}, v)}}
{\min\limits_{\dist(x, u^{-1}(v)) \le \frac{2}{k}} \mathfrak c_j(x, v)}
=\frac{\max\limits_{\abs{x_i - x_{i + 1}} \le \frac{2}{k}} \abs{\mathfrak b(x_i, v) - \mathfrak b(x_{i + 1}, v)}}
{\min\limits_{\dist(x, u^{-1}(v)) \le \frac{2}{k}} \mathfrak b(x, v)}.
$$
При этом, знаменатель второй дроби при $v \in U(\mathfrak c_j)$ отделен от нуля.
Тем самым, $D_k(\mathfrak c_j, U(\mathfrak c_j))$ ограничена.

Поскольку $D_k$ не меняется при домножении первого аргумента на коэффициент, не зависящий от $x$,
и $U(\mathfrak c_j) \nearrow U(\mathfrak b)$, имеем
$$
D_k(\mathfrak c_j, U(\mathfrak c_j)) = D_k(\mathfrak b, U(\mathfrak c_j)) \le D_k(\mathfrak b, U(\mathfrak b)) \to 0
$$
при $k \to \infty$.

Таким образом, веса $\mathfrak c_{j(k)}$ удовлетворяют $(H1)-(H3)$.
Из шага 4 имеем $\IWg(\mathfrak c_{j(k)}, \mon{u}) \le \IWg(\mathfrak c_{j(k)}, u)$.
Переходя к пределу, получаем $\IWg(\mathfrak b_{\ell}, \mon{u}) \le \IWg(\mathfrak b_{\ell}, u)$,
а затем и неравенство (\ref{toprove_weighted}).

Тем самым, теорема \ref{unbounded_growth_thm} доказана.
\hfill $\square$
