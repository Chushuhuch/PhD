\section{Случай монотонной перестановки}
\label{moveForth}

Now we want to get rid of the monotonicity restriction on the weight.
We do this in several steps.

To begin, we note that all properties of the function $\mathfrak a$ are of interest
only in the neighborhood of the graphs of functions $u$ and $\mon{u}$.

We introduce the following conditions, each of which, when added to the previous ones, defines a smaller class of weight functions:

\bigskip

\smallskip
\noindent
$(H1)$ $\mathfrak a(x, v)$ satisfies (\ref{almostConcave}), is even in $x$ and $I(\mathfrak a, u) < \infty$.
\smallskip

\bigskip
\noindent
$(H2)$ the number of zeros of $\mathfrak a(\cdot, v)$ is bounded by a constant independent of $v$
for all $v \in [\min u(x), \max u(x)]$ such that $\mathfrak a(\cdot, v) \not \equiv 0$.

\bigskip
\noindent
$(H3)$ If $\mathfrak a(x_0, u(x_0)) = 0$ for some $x_0$, then $\mathfrak a(\cdot, u(x_0)) \equiv 0$.
Moreover, $\lim\limits_{k \to \infty} D_k(\mathfrak a, U(\mathfrak a)) = 0$, where
$$U(\mathfrak a) := \{ v \in [\min u(x), \max u(x)]: \mathfrak a(\cdot, v) \not \equiv 0 \},$$
\begin{equation}
\label{bigD}
D_k(\mathfrak a, U): = \sup \limits_{v \in U}
\frac{\max \limits_{\abs{x_1 - x_2} \le \frac{2}{k}} \abs{\mathfrak a(x_1, v) - \mathfrak a(x_2, v)}}
{\min \limits_{\dist (x, u^{-1} (v)) \le \frac{2}{k}} \mathfrak a(x, v)}.
\end{equation}

\bigskip
\noindent
$(H4)$ There exists an even $k$, such that $\mathfrak a(\cdot, v)$ are linear for each $v$ on each of the segments
$[-1 + \frac{2i}{k}, -1 + \frac{2(i + 1)}{k}]$.

\bigskip
\noindent
$(H5)$ The difference between the
set of $v \in \Real_+$, for which $\mathfrak a(\cdot, v)$ has segments of constant values,
and the set of $v \in \Real_+$ such that $\mathfrak a(\cdot, v) \equiv 0$
has zero measure.

\bigskip
\noindent
$(H6)$ The segment $[-1, 1]$ can be represented as a unity of touching segments
on each of which $\mathfrak a$ does not change the monotonicity with respect to $x$ in a $v$-neighborhood of the graph of the function $u$.

\bigskip
\noindent
$(H7)$ Let $x_1 < x_2 < x_3$,
let $\mathfrak a(\cdot, v)$ decrease for $x \in [x_1, x_2]$ in a $v$-neighborhood of the graph of the function $u$,
and let $\mathfrak a(\cdot, v)$ increase for $x \in [x_2, x_3]$ in a $v$-neighborhood of the graph of the function $u$.
Then we have $\mathfrak a(\cdot, v) \equiv 0$ in a $v$-neighborhood of $u(x_2)$.

\bigskip

The weights satisfying $(H1)$ will be called \textit{admissible for a given $u$}.

\medskip

Now we can formulate the main assertion of our work.
\begin{thm}
\label{unbounded_growth_thm}
Suppose $F \in \mathfrak{F}$, the function $u \in \W(-1, 1)$ is nonnegative,
and the weight function $\mathfrak a: [-1, 1] \times \Real_+ \to \Real_+$ is continuous
and admissible for $u$.
Then inequality $(\ref{toprove_weighted})$ holds.
\end{thm}

We prove inequality (\ref{toprove_weighted}) under conditions $(H1)-(H7)$,
and then get rid of extra conditions one by one.

To prove it, we need the following facts.

\begin{prop}
\label{levelDerivative}
\textrm{\cite[Theorem 6.19]{LiebLoss} }
For every $u \in \W(-1, 1)$ and for an arbitrary set $A \subset \Real_+$ of zero measure,
$u'(x) = 0$ almost everywhere in $u^{-1}(A)$.
\end{prop}

\begin{lm}
\label{zeroApprox}
Suppose that $u \in \W(-1, 1)$ is nonnegative.
Let a closed set $W \subset \Real_+$ be such that
the set of $v \in W$, for which $\mathfrak a(\cdot, v) \not\equiv 0$, has a zero measure.
Then there exists an increasing sequence of weights $\mathfrak b_{\ell}$, which satisfy

1) $\mathfrak b_{\ell}(\cdot, v) \rightrightarrows \mathfrak a(\cdot, v)$ for almost all $v$;

2) $\mathfrak b_{\ell}(\cdot, v) \equiv 0$ for every $v$ in some neighborhood of $W$ (the neighborhood depends on $\ell$);

3) $I(\mathfrak b_{\ell}, u) \to I(\mathfrak a, u)$ and $I(\mathfrak b_{\ell}, \mon{u}) \to I(\mathfrak a, \mon{u})$.
\end{lm}

\begin{rem}
If $a$ is admissible for $u$, then $b_{\ell}$ are also admissible.
\end{rem}

\begin{proof}
Take $\rho(d) := \min(1, \max(0, d))$,
$$\mathfrak b_{\ell}(x, v) := \mathfrak a(x, v) \cdot \rho(\ell \dist(v, W) - 1) \le \mathfrak a(x, v).$$
This weight is equal to zero in the $\left(\frac{1}{\ell}\right)$-neighborhood of $W$.
In addition, $\mathfrak b_{\ell} \equiv \mathfrak a$ outside the $\left(\frac{2}{\ell}\right)$-neighborhood of $W$ and
$\mathfrak b_{\ell}(x, v)$ increases in $\ell$.
Thus, $\mathfrak b_{\ell}(\cdot, v) \rightrightarrows \mathfrak a(\cdot, v)$ for almost all $v$.
By the monotone convergence theorem
$I(u^{-1}(\Real_+ \setminus W), \mathfrak b_{\ell}, u) \nearrow I(u^{-1}(\Real_+ \setminus W), \mathfrak a, u)$.

Divide the set $W$ into $W_1 := \{v \in W: \mathfrak a(\cdot, v) \equiv 0\}$ and $W_2 = W \setminus W_1$.
Then
$$
\begin{aligned}
I(u^{-1}(W_1), \mathfrak b_{\ell}, u) &= I(u^{-1}(W_1), \mathfrak a, u),\\
I(u^{-1}(W_2), \mathfrak b_{\ell}, u) &= \int\limits_{x \in u^{-1}(W_2)} F \bigl( u(x), \mathfrak b_{\ell}(x, u(x)) |u'(x)| \bigr) \, dx.
\end{aligned}
$$
By Proposition \ref{levelDerivative}, $u'(x) = 0$ almost everywhere on $u^{-1}(W_2)$.
Thus
$$I(u^{-1}(W_2), \mathfrak b_{\ell}, u) = \int\limits_{x \in u^{-1}(W_2)} F \bigl( u(x), 0 \bigr) \, dx = 0.$$
Similarly, $I(u^{-1}(W_2), \mathfrak a, u) = 0$. Hence $I(\mathfrak b_{\ell}, u) \to I(\mathfrak a, u)$.
The second relation in \textit{3)} is proved by the same arguments.
\end{proof}

We proceed to the proof of the theorem.

\bigskip
\textbf{Step 1.} \textit{Let $u \in \W(-1, 1)$ and let the weight $\mathfrak a$ satisfy the conditions $(H1)-(H7)$.
Then inequality (\ref{toprove_weighted}) holds.}

Divide the segment $[-1, 1]$ into touching subsegments $\Delta_j$, each consisting of two parts.
On the left part of each $\Delta_j$ the weight $\mathfrak a$ increases in $x$ in a neighborhood
of the graph of $u(x)$. On the right part it decreases.
On each $\Delta_j$, we can apply the construction from the previous section
for approximating $u$ with Lipschitz functions $u_n$.
This gives us $I(\Delta_j, \mathfrak a, u_n) \to I(\Delta_j, \mathfrak a, u)$.

However, the approximating functions $u_n$ have discontinuities at the borders of the segments $\Delta_j$
(denote them by $\hat{x}_j$).

Note that according to the condition $(H7)$ one can choose points $\hat{x}_j$ so
that $\mathfrak a \equiv 0$ in $(x, v)$-neighborhoods of the points $(\hat{x}_j, u(\hat{x}_j))$.

Next, substitute functions $u_n$ in these neighborhoods of $\hat{x}_j$ with linear pieces
making $u_n$ continuous on $[-1, 1]$.
In view of the above, this does not change the integrals $I(\Delta_j, \mathfrak a, u_n)$,
and we get $I(\mathfrak a, u_n) \to I(\mathfrak a, u)$.

By Lemma \ref{uplift} we obtain (\ref{toprove_weighted}).

\bigskip

\textbf{Step 2.} \textit{Let the weight $\mathfrak a$ satisfy the conditions $(H1)-(H6)$.
Then inequality (\ref{toprove_weighted}) holds.}

We apply Lemma \ref{zeroApprox} using the following set $W$:
the set of all $v$, at which the graph of $u(x)$ traverses from a rectangle,
in which the weight decreases in $x$,
to a rectangle in which the weight increases.
Obviously, the resulting functions $\mathfrak b_{\ell}$ satisfy $(H1)-(H7)$.
By Step 1, $I(\mathfrak b_{\ell}, \mon{u}) \le I(\mathfrak b_{\ell}, u)$.
Passing to the limit, we obtain (\ref{toprove_weighted}).

\bigskip

\textbf{Step 3.} \textit{Let the weight $\mathfrak a$ satisfy the conditions $(H1)-(H5)$.
Then inequality (\ref{toprove_weighted}) holds.}

Consider abscissas of nodes of $\mathfrak a$
and ordinates, for which $\mathfrak a$ has constant pieces.
They define a division of the rectangle $[-1, 1] \times [\min u(x), \max u(x)]$
into smaller rectangles, in each of which the weight $\mathfrak a$ is monotone in $x$.
However, the number of rectangles can be infinite.
Also, if the graph of $u$ crosses a horizontal boundary of some rectangle,
monotonicity in the $v$-neighborhood of the point of intersection may change.

Consider a set $W$ containing all $v$, for which the weight $\mathfrak a$ has constant pieces.
Due to $(H5)$ the set of all  $v \in W$ such that $a(\cdot, v) \not\equiv 0$ has a zero measure.

We apply Lemma \ref{zeroApprox} and obtain a sequence of weights $\mathfrak b_{\ell}$.
We claim that each of them has only a finite number of monotonicity rectangles.
Indeed, any two vertically adjacent rectangles with different monotonicities
are separated by a stripe of $\frac{2}{\ell}$ width with zero values.

The weight $b_{\ell}$ can change monotonicity along the graph of $u$
either at the points $x = -1 + \frac{2 i}{k}$ or where the graph crosses a stripe of zero values.
Note that only a finite number of such crossings can arise since
$\int |u'|$ gains at least $\frac{2}{\ell}$ at any crossing and $u' \in \Ls(-1, 1)$.

Thereby, each $\mathfrak b_{\ell}$ satisfies $(H1)-(H6)$. By Step 2, $I(\mathfrak b_{\ell}, \mon{u}) \le I(\mathfrak b_{\ell}, u)$.
Passing to the limit, we obtain (\ref{toprove_weighted}).

\bigskip
\textbf{Step 4.} \textit{Let the weight $\mathfrak a$ satisfy the conditions $(H1)-(H3)$.
Then inequality (\ref{toprove_weighted}) holds.}

Suppose that the function $\mathfrak a$ satisfies $(H1)-(H3)$, in particular $I(\mathfrak a, u) < \infty$.

We fix an arbitrary even $k$.
For each $v$ we interpolate $\mathfrak a$ with piecewise linear functions
with nodes $( -1 + \frac{2i}{k}, \mathfrak a(-1 + \frac{2i}{k}, v) )$.
The resulting function $\mathfrak a_k(x, v)$ is continuous, even in $x$
and satisfies (\ref{almostConcave}) by Lemma \ref{piecewiseLinearConcave}.
In addition, $\mathfrak a_k \to \mathfrak a$ when $k \to \infty$,
moreover the convergence is uniform on compact sets.
However, the inequality $\mathfrak a_k(x, u(x)) \le \mathfrak a(x, u(x))$ can be violated,
and thus $\mathfrak a_k$ may be non-admissible for $u$.

Set $\mathfrak c_k := (1 - D_k(\mathfrak a_k, U(\mathfrak a_k))) \mathfrak a_k$, where $D_k$ is defined in (\ref{bigD}).
$D_k(\mathfrak a_k, U(\mathfrak a_k))$ are positive and tend to zero, thus $\mathfrak c_k \to \mathfrak a$ as $k \to \infty$.
We claim that $\mathfrak c_k(x, u(x)) \le \mathfrak a(x, u(x))$.

Indeed, consider some
$x \in [-1 + \frac{2i}{k}, -1 + \frac{2(i + 1)}{k}] =: [x_i, x_{i + 1}]$.
Then $\mathfrak c_k(x, u(x)) \le \max( \mathfrak c_k(x_i, u(x)), \mathfrak c_k(x_{i + 1}, u(x)) )$, because
$\mathfrak c_k$ is piecewise linear in $x$. Moreover,
\begin{multline*}
\mathfrak c_k(x_i, u(x)) = ( 1 - D_k(\mathfrak a_k, U(\mathfrak a_k))) \cdot \mathfrak a(x_i, u(x)) \\
\le \mathfrak a(x_i, u(x)) - \frac{\mathfrak a(x_i, u(x)) - \mathfrak a(x, u(x))}{\mathfrak a(x_i, u(x))} \cdot \mathfrak a(x_i, u(x)) = \mathfrak a(x, u(x)).
\end{multline*}
Similarly, $\mathfrak c_k(x_{i + 1}, u(x)) \le \mathfrak a(x, u(x))$.
Thus, $\mathfrak c_k(x, u(x)) \le \mathfrak a(x, u(x))$ for any $x$, and $\mathfrak c_k$ are admissible for $u$.
Thereby the functions $\mathfrak c_k$ satisfy $(H1)-(H4)$.

For a given $k \in \Nat$, we approximate the function $\mathfrak c_k =: \mathfrak c$ with weights satisfying $(H1)-(H5)$.
Consider the auxiliary function $\Lambda(x) = 1 - \abs{x}$, satisfying (\ref{almostConcave}).

Take
$$t(v):=D_k(\mathfrak c, U(\mathfrak c)) \cdot \max\{\tau \ge 0: \forall x \in u^{-1}(v) \quad \tau \Lambda(x) \le \mathfrak c(x, u(x))\}.$$
The function $t$ depends on $k$, but we omit this fact in presentation.

It is clear that the maximum $\tau$ is zero only if $\mathfrak c(\cdot, v) \equiv 0$,
since otherwise the condition $(H3)$ is violated.

Function $t$ may be discontinuous. However, it is easy to see that it is lower semicontinuous.
Next, we take
$$\tilde{t}(v) := \inf_{w \in u([-1, 1])} \{t(w) + |v - w|\}.$$
It is obvious that $\tilde{t} \le t$, and the set of zeros of $t$ and $\tilde{t}$ coincide.

We claim that $\tilde{t}$ is continuous (and even Lipschitz).
Indeed, take some $v_1$.
Then there is an arbitrarily small $\eps > 0$ and $w_1 \in u([-1, 1])$
satisfying $\tilde{t}(v_1) = t(w_1) + |v_1 - w_1| - \eps$.
For every $v_2$, we have $\tilde{t}(v_2) \le t(w_1) + |v_2 - w_1|$.
And thus $\tilde{t}(v_2) - \tilde{t}(v_1) \le |v_1 - v_2| + \eps$.
By the arbitrariness of $v_1$, $v_2$ and $\eps$, the claim follows.

For $\alpha \in [0, 1]$ the function $\mathfrak d_\alpha(x, v) := \mathfrak c(x, v) + \alpha \Lambda(x) \tilde{t}(v)$
is even in $x$, satisfies (\ref{almostConcave}) in concordance with Lemma \ref{maxSumConcave},
and does not exceed $\mathfrak a(x, v)$ due to the construction of the function $\tilde{t}$.
Thus, $\mathfrak d_\alpha$ is an admissible weight.
Also, it is obvious that $\mathfrak d_\alpha$ satisfies $(H1)-(H4)$.

Let us show that there exists a sequence $\alpha_j \searrow 0$
such that $\mathfrak d_{\alpha_j}(\cdot, v)$ has no segments of constant values,
unless $\mathfrak d_{\alpha_j}(\cdot, v) \equiv 0$ or $v$ belongs to a zero measure set.
We introduce the set of $\alpha$ which are ``bad'' on $[x_i, x_{i + 1}]$:
\begin{multline*}
A_i := \bigl \{\alpha \in [0, 1]: \\
\meas \{v \in [\min u, \max u]: \frac{\mathfrak c(x_{i + 1}, v) - \mathfrak c(x_i, v))}{\frac{2}{k}} + \alpha \chi_i \tilde{t} (v) = 0 \} > 0 \bigr \},
\end{multline*}
where $\chi_i = 1$ if $[x_i, x_{i + 1}] \subset [0, 1]$, and $\chi_i = -1$ if $[x_i, x_{i + 1}] \subset [-1, 0]$.

Consider the following function
$$
\begin{aligned}
h_i(v) = & \frac{\mathfrak c(x_{i + 1}, v) - \mathfrak c(x_i, v)}{\tilde{t} (v)} & \text{ if } \tilde{t} (v) \neq 0 & \\
h_i(v) = & 0 & \text{ if } \tilde{t} (v) = 0 &.
\end{aligned}
$$
We have $\card(A_i) = \card(\{ \alpha \in [0, 1]: \meas \{ v \in [\min u, \max u]: h_i(v) \pm \frac{2}{k} \alpha = 0 \} > 0 \}).$
Then $\card(A_i) \le \aleph_0$, and $\card(\cup_i A_i) \le \aleph_0$.
Thus, there exists a sequence of weights $\mathfrak d_{\alpha_j} \searrow \mathfrak c$, satisfying $(H1)-(H5)$.
By Step 3, $I(\mathfrak d_{\alpha_j}, \mon{u}) \le I(\mathfrak d_{\alpha_j}, u)$.
Passing to the limit, we get $I(\mathfrak c, \mon{u}) \le I(\mathfrak c, u)$.

Further, for $x \in [-1, 1]$ we have
\begin{equation}
\label{step4Conv}
F \bigl( u(x), \mathfrak c_k(x, u(x)) |u'(x)| \bigr) \to F \bigl( u(x), \mathfrak a(x, u(x)) |u'(x)| \bigr)
\end{equation}
as $k \to \infty$.
Moreover, $F \bigl( u(x), \mathfrak a(x, u(x)) |u'(x)| \bigr)$ is an integrable majorant
for the left-hand side in (\ref{step4Conv}).
By the Lebesgue theorem, we have $I(\mathfrak c_k, u) \to I(\mathfrak a, u)$.
Since $I(\mathfrak c_k, \mon{u}) \le I(\mathfrak c_k, u)$, Lemma \ref{uplift} proves inequality (\ref{toprove_weighted}).

\bigskip
\textbf{Step 5.} \textit{Let the weight $\mathfrak a$ satisfy only the condition $(H1)$.
Then inequality (\ref{toprove_weighted}) holds.}

We approximate $\mathfrak a$ by weights satisfying $(H1)-(H2)$.
To do this we apply Lemma \ref{zeroApprox} with $W = \{ v \in \Real_+: \mathfrak a(\cdot, v) \equiv 0 \}$.
Let us introduce the notation $$Z_{\mathfrak a}(v) := \{ x \in [-1, 1]: \mathfrak a(x, v) = 0 \}.$$
Note that the sets $Z_{\mathfrak b_{\ell}}(v)$ are either $Z_{\mathfrak a}(v)$ or $[-1, 1]$.

Let us show that $\mathfrak b_{\ell}$ satisfies $(H2)$.
Indeed, otherwise there is a sequence $v_m$, for which
$m < \card(Z_{\mathfrak b_{\ell}})(v_m) < \infty$.
After passing to a subsequence, we have $v_m \to v_0$.
Part 2 of Lemma \ref{periodicity} implies that the set $Z_{\mathfrak b_{\ell}}(v_m) = Z_{\mathfrak a}(v_m)$
is periodic with a period less or equal to $\frac{2}{m - 1}$.
Take some $x \in [-1, 1]$. For each $m$ there exists $x_m$ such that
$\abs{x - x_m} \le \frac{1}{m - 1}$ and $\mathfrak a(x_m, v_m) = 0$.
But $\mathfrak a(x_m, v_m) \to \mathfrak a(x, v_0)$.
Therefore, $\mathfrak a(x, v_0) = 0$.

Thus $Z_{\mathfrak a}(v_0) = [-1, 1]$.
But this means that for every $v$ such that $\abs{v - v_0} \le \frac{1}{\ell}$,
we have $\mathfrak b_{\ell}(\cdot, v) \equiv 0$,
which contradicts $\card(Z_{\mathfrak b_{\ell}})(v_m) < \infty$.

Now we fix $\ell \in \Nat$ and denote $\mathfrak b_{\ell} =: \mathfrak b$.
Let us approximate the function $\mathfrak b$ with weights satisfying $(H1)-(H3)$.
It follows from $(H2)$ that there exists a set $T \subset [-1, 1]$
consisting of a finite number of elements, such that
if $x \not\in T$ and $\mathfrak b(x, v) = 0$ for some $v$, then $\mathfrak b(\cdot, v) \equiv 0$.

We use Lemma \ref{zeroApprox} with $W = u(T) \cup \mon{u}(T)$.
The weights $\mathfrak c_j$, given by the Lemma, satisfy $(H1)-(H2)$,
since they are just $\mathfrak b$ multiplied by a factor less than one, which depends only on $v$.

For any $k$ sufficiently large, there exists $j = j(k)$ such that
$$u \Bigl( \Bigl\{ x \in [-1, 1]: dist(x, T) \le \frac{4}{k} \Bigr\} \Bigr) \subset \Bigl\{ v \in \Real_+: dist(v, u(T)) \le \frac{1}{2j} \Bigr\},$$
and $j(k) \to \infty$ as $k \to \infty$ by continuity of $u$.
This implies that $\min\limits_{dist(x, u^{-1}(v)) \le \frac{2}{k}} c_j(x, v) > 0$
for all $v \in U(c_j)$.
Moreover, for $v \in U(c_j)$ we have
$$
\frac{\max\limits_{\abs{x_i - x_{i + 1}} \le \frac{2}{k}} \abs{\mathfrak c_j(x_i, v) - \mathfrak c_j(x_{i + 1}, v)}}
{\min\limits_{\dist(x, u^{-1}(v)) \le \frac{2}{k}} \mathfrak c_j(x, v)}
=\frac{\max\limits_{\abs{x_i - x_{i + 1}} \le \frac{2}{k}} \abs{\mathfrak b(x_i, v) - \mathfrak b(x_{i + 1}, v)}}
{\min\limits_{\dist(x, u^{-1}(v)) \le \frac{2}{k}} \mathfrak b(x, v)}.
$$
Note, that the denominator of the right-hand side is separated from zero for $v \in U(\mathfrak c_j)$.
Thus, $D_k(\mathfrak c_j, U(\mathfrak c_j))$ is bounded.

Since $D_k$ does not change if we multiply the first argument by a positive factor independent of $x$,
and $U(\mathfrak c_j) \nearrow U(\mathfrak b)$, we have
$$D_k(\mathfrak c_j, U(\mathfrak c_j)) = D_k(\mathfrak b, U(\mathfrak c_j)) \le D_k(\mathfrak b, U(\mathfrak b)) \to 0$$
as $k \to \infty$.

Thus, the weights $\mathfrak c_{j(k)}$ satisfy $(H1)-(H3)$.
By Step 4, $I(\mathfrak c_{j(k)}, \mon{u}) \le I(\mathfrak c_{j(k)}, u)$.
Passing to the limit, we get $I(\mathfrak b_{\ell}, \mon{u}) \le I(\mathfrak b_{\ell}, u)$,
and consequently inequality (\ref{toprove_weighted}).

Thus, Theorem \ref{unbounded_growth_thm} is proved.
\hfill $\square$

\medskip

Now we consider the case where the function $u$ satisfies the additional condition $u(-1) = 0$.
\begin{thm}
Suppose that $F \in \mathfrak{F}$, the function $u \in \W(-1, 1)$ is nonnegative, $u(-1) = 0$,
and the weight function $\mathfrak a: [-1, 1] \times \Real_+ \to \Real_+$ is continuous
and satisfies $(\ref{almostConcave})$.
Then inequality $(\ref{toprove_weighted})$ holds.
\end{thm}

\begin{proof}
We follow the proof of Theorem \ref{unbounded_growth_thm},
but we change $(H1)$ and $(H7)$ to the following conditions:

\bigskip
\noindent
$(H1')$ $\mathfrak a(x, v)$ satisfies (\ref{almostConcave}), and $I(\mathfrak a, u) < \infty$.

\bigskip
\noindent
$(H7')$ Assumption $(H7)$ is satisfied and $\mathfrak a(\cdot, v) \equiv 0$ in some $v$-neighborhood of zero.

\bigskip
\textbf{Step 1.} \textit{Let $u \in \W(-1, 1)$, $u(-1) = 0$ and let the weight $\mathfrak a$ satisfy the conditions $(H1'), (H2)-(H6), (H7')$.
Then inequality (\ref{toprove_weighted}) holds.}

To prove this, we approximate the function $u$ in the same way as in the first step in the proof of Theorem \ref{unbounded_growth_thm},
changing $u$ in a neighborhood of $x = -1$ to a linear function with $u_n(-1) = 0$ preserved.

\bigskip
\textbf{Step 2.} \textit{Let the weight $\mathfrak a$ satisfy conditions $(H1'), (H2)-(H6)$.
Then inequality (\ref{toprove_weighted}) holds.}

To prove this, we add zero to the set $W$ from the second step in the proof of Theorem \ref{unbounded_growth_thm},
and repeat the rest of the proof.

\medskip

Further steps are unchanged.
\end{proof}

