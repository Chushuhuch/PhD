\section{Условия, необходимые для выполнения неравенства (\ref{toprove_symm})}

Нам потребуется вспомогательная
\begin{lm}
\label{almostConvexLm}
Если для функции $a \in C([-1, 1] \times \Real_+)$ выполнено соотношение
\begin{equation}
\label{almostConvex}
\forall s, t \in [-1, 1],\ \forall v \in \Real_+ \quad
a( s, v ) + a( t, v ) \ge a\Bigl( \frac{ s - t }{2}, v \Bigr) + a\Bigl( \frac{ t - s }{2}, v \Bigr),
\end{equation}
то она чётна и выпукла по первому аргументу.
\end{lm}

\begin{proof}
Предположим для начала, что $a(\cdot, v) \in C^1([-1, 1])$ при каждом $v$.
Зафиксируем произвольные $s \in [-1, 1]$ и $v \in \Real_+$ и рассмотрим функцию
$$b(x) := a( s, v ) + a( x, v ) - a( \frac{ s - x }{2}, v ) - a( \frac{ x - s }{2}, v ) \ge 0.$$
$x = -s$ является точкой минимума функции $b$, поскольку $b(-s) = 0$.
Значит,
$$
b'(-s) = a'_x( -s, v ) + \frac{1}{2} a'_x( s, v ) - \frac{1}{2} a'_x( -s, v ) = 0,
$$
то есть $a'_x( s, v ) = -a'_x( -s, v )$.
Тем самым, функция $a(\cdot, v)$ четна.

Рассмотрим теперь случай произвольной непрерывной $a$.

Продолжим $a( x, v ) := a( -1, v )$ при $x < -1$ и $a( x, v ) := a( 1, v )$ при $x > 1$.
Рассмотрим усреднение функции:
$$
a_\rho( x, v ) = \int_\Real \omega_\rho ( z ) a( x - z, v ) dz = \int_\Real \omega_\rho ( z ) a( x + z, v ) dz,
$$
где $\omega_\rho(z)$ --- усредняющее ядро с радиусом $\rho$.
Тогда для $-1 + \rho \le s, t \le 1 - \rho$
\begin{multline*}
a_\rho( s, v ) + a_\rho( t, v ) - a_\rho( \frac{ s - t }{2}, v ) - a_\rho( \frac{ t - s }{2}, v ) =
\\ \int_\Real \omega_\rho ( z ) \bigl( a( s - z, v ) + a( t + z, v ) - a( \frac{ s - t }{2} - z, v ) - a( \frac{ t - s }{2} + z, v ) \bigr) dz \ge 0.
\end{multline*}
Значит функция $a_\rho(\cdot, v)$ чётна на $[-1 + \rho, 1 - \rho]$.
Переходя к пределу при $\rho \to 0$, получаем, что функция $a(\cdot, v)$ чётна.

Наконец, для любых $s$, $t$ и $v$ имеем
$$
a( s, v ) + a( t, v ) = a( s, v ) + a( -t, v ) \ge 2 a\bigl( \frac{ s + t }{2}, v \bigr).
$$
\end{proof}

\begin{thm}
Если неравенство $(\ref{toprove_symm})$ выполняется для произвольной $F \in \mathfrak{F}$ и произвольной кусочно линейной $u$,
то вес $a$ --- чётная и выпуклая по первому аргументу функция.
\end{thm}

\begin{proof}
Докажем, что в условиях теоремы выполнено неравенство (\ref{almostConvex}).
Если это так, то ввиду леммы \ref{almostConvexLm} утверждение теоремы будет следовать.

Предположим, что неравенство (\ref{almostConvex}) не выполнено.
Тогда найдутся $-1 \le s < t \le 1$, $\eps, \delta > 0$ ($2 \eps < t - s$) и $v_0 \in \Real_+$,
такие, что для любого $0 \le z \le \eps$ и любого $v_0 \le v \le v_0 + \eps$ выполнено
\begin{equation}
\label{notConvex}
a(s + z, v + z) + a(t - z, v + z) + 2 \delta < a\Bigl(\frac{s - t}{2} + z, v + z \Bigr) + a\Bigl(\frac{t - s}{2} - z, v + z \Bigr).
\end{equation}

Рассмотрим функцию $u_2$, введенную в (\ref{parLinU}). Тогда
$$
\left\{
\begin{aligned}
\symm{u_2}(x) &= v_0, & x \in &[-1, \frac{s - t}{2}] \cup [\frac{t - s}{2}, 1]\\
\symm{u_2}(x) &= v_0 + x - \frac{s - t}{2}, & x \in &[\frac{s - t}{2}, \frac{s - t}{2} + \eps]\\
\symm{u_2}(x) &= v_0 + \eps, & x \in &[\frac{s - t}{2} + \eps, \frac{t - s}{2} - \eps]\\
\symm{u_2}(x) &= v_0 + \frac{t - s}{2} - x, & x \in &[\frac{t - s}{2} - \eps, \frac{t - s}{2}].
\end{aligned}
\right.
$$

Отсюда получаем
\begin{multline*}
0 \le \IWg(a, u_2) - \IWg(a, \overline{u_2}) \\
=\int_0^{\eps} F\bigl( u_2(s + z), \frac{a( s + z, u_2(s + z) )}{\eps} \bigr) dz + \int_0^{\eps} F\bigl( u_2(t - z), \frac{a(t - z, u_2(t - z))}{\eps} \bigr) dz \\
-\int_0^{\eps} F\bigl( \symm{u_2}(\frac{s - t}{2} + z), \frac{a( \frac{s - t}{2} + z, \symm{u_2}(\frac{s - t}{2} + z) )}{\eps} \bigr) dz \\
-\int_0^{\eps} F\bigl( \symm{u_2}(\frac{t - s}{2} - z), \frac{a( \frac{t - s}{2} - z, \symm{u_2}(\frac{t - s}{2} - z) )}{\eps} \bigr) dz
=: \Delta \IWg.
\end{multline*}

Возьмем $F(v, p) := f(p) := p + \gamma p^2$, где $\gamma > 0$.
Тогда
\begin{multline*}
\Delta \IWg = \int_0^{\eps} \bigl( f(\frac{a(s + z, v_0 + z)}{\eps}) + f(\frac{a(t - z, v_0 + z)}{\eps}) \\
- f(\frac{a(\frac{s - t}{2} + z, v_0 + z)}{\eps}) - f(\frac{a(\frac{t - s}{2} - z, v_0 + z)}{\eps}) \bigr) dz.
\end{multline*}

Обозначим
$$
A = \max \limits_{(x, v)} a(x, v), \qquad (x, v) \in [-1, 1 ] \times u_2([-1, 1] ).
$$
Если взять $\gamma := \frac{\delta / \eps}{(A / \eps)^2} > 0$,
то для $p \le \frac{ A }{\eps}$ имеем $p \le f( p ) \le p + \frac{\delta}{\eps}$, и
\begin{multline*}
\Delta \IWg \le \frac{1}{\eps} \int_0^{\eps} \bigl( a(s + z, v_0 + z) + a(t - z, v_0 + z) + 2 \delta
\\ - a(\frac{s - t}{2} + z, v_0 + z) - a(\frac{t - s}{2} - z, v_0 + z) \bigr) dz < 0
\end{multline*}
(последнее неравенство следует из (\ref{notConvex})).

Тем самым, мы пришли к противоречию, что завершает доказательство.
\end{proof}
