\section{Обозначения}

В этой главе мы рассматриваем одномерный случай задачи из первой главы.
Тем самым, пропадают весовые коэффициенты $a_i$,
вес $a = a(x, v) : [-1, 1] \times \Real_+ \to \Real_+$,
$\mathfrak{F}$ --- множество непрерывных функций $F: \Real_+ \times \Real_+ \to \Real_+$
выпуклых и строго возрастающих по второму аргументу, удовлетворяющих $F( \cdot, 0 ) \equiv 0$.
Рассматриваемый функционал имеет вид:
$$
\IWg( a, u ) = \int\limits_{-1}^1 F( u(x), a(x, u(x)) \abs{ u'(x) } )\, dx.
$$
Также мы будем использовать обозначение
$$
\IWg( B, a, u ) = \int\limits_B F( u(x), a(x, u(x)) \abs{ u'(x) } )\, dx.
$$

Мы снимаем требование ограничения роста, стоящее в теореме \ref{thm:bounded_growth},
и также доказываем аналогичный результат для симметричной перестановки,
устанавливая необходимые и достаточные условия для выполнения неравенства
\begin{equation}
\label{eq:toprove_symm}
\IWg( \symm{u} ) \le \IWg( u ).
\end{equation}

Мы продолжаем ссылаться на условие (\ref{eq:almostConcave}),
однако оно приобретает следующий вид:
\begin{equation}
a(s, v) + a(t, v) \ge a(1 - t + s, v), \qquad s, t: -1 \le s \le t \le 1,\ v \in \Real_+.
\end{equation}
