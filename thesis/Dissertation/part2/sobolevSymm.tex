\section{Доказательство неравенства (\ref{toprove_symm})}
\label{sobolevSymm}

\begin{thm}
\label{symmThm}
Пусть $F \in \mathfrak{F}$, функция $u \in \W(-1, 1)$ неотрицательна,
и непрерывная весовая функция $a: [-1, 1] \times \Real_+ \to \Real_+$ чётна и выпукла по первому аргументу.
Тогда справедливо неравенство $(\ref{toprove_symm})$.
\end{thm}

\begin{proof}
Для липшицевых функций $u$ утверждение теоремы доказано в \cite{Brock}.
Таким образом, необходимо лишь перейти к $\W$-функциям.

Структура выпуклого по $x$ веса гораздо проще структуры веса,
который мы рассматривали для случая монотонной перестановки.
Выпуклый вес убывает при $x < 0$ и возрастает при $x > 0$ независимо от $v$.
Тем самым, мы сразу входим в условия $(H6)$ из теоремы \ref{unbounded_growth_thm}.
Чтобы войти в условия $(H7)$, применим лемму \ref{zeroApprox} с множеством $W = \set{u(0)}$.
Это дает нам возможность сразу воспользоваться шагом 1 доказательства,
получив неравенство (\ref{toprove_symm}) в общем виде.
Заметим, что шаг 1 использует лишь условия $(H1)$, $(H6)$, $(H7)$, так что нет нужды проверять остальные.
\end{proof}
