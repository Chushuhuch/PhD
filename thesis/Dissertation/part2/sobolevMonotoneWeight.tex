\section{Доказательство неравенства (\ref{eq:to_prove_weighted}) для кусочно монотонных весов}
\label{sec:monotone_weight}

В этом параграфе мы получим неравенство (\ref{eq:to_prove_weighted}) при дополнительном условии
монотонности весовой функции при $x \in [-1, 0]$ и при $x \in [0, 1]$.

\begin{lm}
\label{lm:monotone_weight_appr}
Пусть $a$ --- непрерывная функция, $a(\cdot, u)$ возрастает на $[-1, 0]$ и убывает на $[0, 1]$ для всех $u \ge 0$.
Тогда для любой функции $u \in \W[-1, 1]$, $u \ge 0$, найдётся последовательность $\{u_k\} \subset Lip[-1, 1]$, удовлетворяющая
\begin{equation}
u_k \to u \text{ в } \W[-1, 1] \quad \text{ и } \quad \IWg(a, u_k) \to \IWg(a, u).
\end{equation}
\end{lm}

Для доказательства мы модифицируем схему теоремы 2.4 из \cite{ASC},
в которой аналогичный факт доказывается для интегрантов, не зависящих от свободной переменной.
Частично доказательство совпадает доказательством в \cite{ASC}, но для удобства читателя мы приводим здесь его полностью.

\begin{proof}[Доказательство леммы \ref{lm:monotone_weight_appr}]
Можно считать, что $\IWg( a, u ) < \infty$.

Мы докажем утверждение для функционала
$$
\IWg_1( u ) = \int\limits_0^1 F \bigl( u(x), a(x, u(x)) |u'(x)| \bigr) \, dx.
$$
Вторая часть с интегрированием по $[-1, 0]$ сводится к $\IWg_1$ заменой переменной.

Для $h \in \Nat$ покроем множество $\{ x \in [0, 1]: |u'(x)| > h \}$ открытым множеством $A_h$.
Не умаляя общности, можно считать, что $A_{h + 1} \subset A_{h}$ и $\abs{A_h} \to 0$ при $h \to \infty$.

Обозначим $v_h$ неотрицательную непрерывную функцию, заданную на $[0, 1]$,
совпадающую с $u$ на множестве $[0, 1] \setminus A_h$, и линейную на интервалах, составляющих $A_h$.
Тогда $v_h \to u$ в $\W[-1, 1]$.
Теперь изменим $v_h$ так, чтобы сделать их липшицевыми.

Представим $A_h = \cup_k \Omega_{h,k}$, где $\Omega_{h,k} = ( b_{h,k}^-, b_{h,k}^+ )$.
Обозначим
$$
\alpha_{h,k} := \abs{\Omega_{h,k}}, \quad
\beta_{h,k} := v_h(b_{h,k}^+) - v_h(b_{h,k}^-) = u(b_{h,k}^+) - u(b_{h,k}^-).
$$
Тогда $v'_h = \frac{\beta_{h,k}}{\alpha_{h,k}}$ в $\Omega_{h,k}$.
Заметим, что
$$
\sum_k \abs{\beta_{h,k}} \le \int\limits_{A_h} \abs{u'} \, dx \le \norm{u'}_{\Ls[-1, 1]} < \infty,
$$
а значит,
$\sum_k \abs{\beta_{h,k}} \to 0$ при $h \to 0$ по теореме Лебега.

Определим функцию $\phi_h \in \W[0, 1]$ следующим образом:
$$
\begin{aligned}
\phi_h( 0 ) &= 0 & & \\
\phi_h' &=  1 & \text{ в } & [0, 1] \setminus A_h,\\
\phi_h' &=  \max \Bigl( \frac{ \abs{\beta_{h,k}} }{ \alpha_{h,k} }, 1 \Bigr) & \text{ в } & \Omega_{h,k}.
\end{aligned}
$$	

Заметим, что $\int_0^1 \abs{\phi_h'} \, dx \le 1 + \sum_k \abs{\beta_{h,k}} < \infty$.

Покажем, что $\phi_h' \to 1$ в $\Ls(0, 1)$:
$$
\int \abs{\phi_h' - 1} \, dx = \sum\limits_k \Bigl( \max \Bigl( \frac{ \abs{\beta_{h,k}} }{ \alpha_{h,k} }, 1 \Bigr) - 1 \Bigr) \alpha_{h,k} \le
\sum\limits_k \abs{\beta_{h,k}} \to 0.
$$
Отсюда следует, что $\phi_h$ удовлетворяет условиям предложения \ref{prop:conv_to_one}.

Рассмотрим теперь $\phi_h^{-1}: [0, 1] \to [0, 1]$ --- ограничение обратной к $\phi_h$ функции на $[0, 1]$.
Тогда $0 \le ( \phi_h^{-1} )' \le 1$ и
$$
\begin{aligned}
\phi_h^{-1} ( 0 ) &= 0 & & \\
( \phi_h^{-1} )' &=  1 & \text{ в } & [0, 1] \setminus \phi_h( A_h ),\\
( \phi_h^{-1} )' &=  \min \Bigl( \frac{ \alpha_{h,k} }{ \abs{ \beta_{h,k} } }, 1 \Bigr) & \text{ в } & [0, 1] \cap \phi_h( \Omega_{h,k} ).
\end{aligned}
$$

Возьмём $u_h = v_h( \phi_h^{-1} )$.
Заметим, что $u_h(0) = u(0)$, и
\begin{align*}
u_h' &=  v_h'( \phi_h^{-1} ) \cdot ( \phi_h^{-1} )' = u'( \phi_h^{-1} ) & \text{ в } & [0, 1] \setminus \phi_h( A_h ),\\
u_h' &=  v_h'( \phi_h^{-1} ) \cdot ( \phi_h^{-1} )' = 
\sign{ \beta_{h,k} } \cdot \min \Bigl( 1, \frac{ \abs{ \beta_{h,k} } }{ \alpha_{h,k} } \Bigr) & \text{ в } & [0, 1] \cap \phi_h( \Omega_{h,k} ).
\end{align*}
Тем самым, $u_h$ липшицева, поскольку $u'$ ограничена в $[0, 1] \setminus A_h$.

Покажем, что $u_h \to u$ в $\W[0, 1]$.
Для этого достаточно оценить

$$
\norm{u_h' - u'}_{\Ls} \le \int\limits_{[0, 1] \setminus \phi_h(A_h)} \abs{u_h' - u'} +
\int\limits_{[0, 1] \cap \phi_h(A_h)} \abs{u_h'} + \int\limits_{[0, 1] \cap \phi_h(A_h)} \abs{u'} =: P_h^1 + P_h^2 + P_h^3.
$$
$$
P_h^1 = \int\limits_{[0, 1] \setminus \phi_h( A_h )} \abs{u'( \phi_h^{-1} ) - u'} \, dx =
\int\limits_{\phi_h^{-1} ( [0, 1] ) \setminus A_h} \abs{u' - u'( \phi_h )} \, dz \le
\int\limits_{[0, 1]} \abs{u' - u'( \phi_h )} \, dz.
$$
В силу предложения \ref{prop:conv_to_one}, $P_h^1 \to 0$.
Далее,
$$
P_h^2 \le \abs{\phi_h( A_h )} = \sum\limits_k \abs{\phi_h( \Omega_{h,k} )} = \sum\limits_k \max (\abs{\beta_{h,k}}, \alpha_{h,k})
\le \sum\limits_k \alpha_{h,k} + \sum\limits_k \abs{\beta_{h,k}} \to 0.
$$
Наконец, $P_h^3 \to 0$ по абсолютной непрерывности интеграла, и утверждение доказано.

Осталось показать, что $\IWg_1( u_h ) \to \IWg_1( u )$.

\begin{multline*}
\IWg_1( u_h ) = \int\limits_{[0, 1] \setminus \phi_h( A_h )} F \bigl( u_h( x ), a( x, u_h(x) ) |u_h'( x )| \bigr) \, dx
\\ + \int\limits_{[0, 1] \cap \phi_h( A_h )} F \bigl( u_h( x ), a( x, u_h(x) ) |u_h'( x )| \bigr) \, dx =: \hat{P_h^1} + \hat{P_h^2}.
\end{multline*}
Поскольку $u \in \W[0, 1]$, имеем $u \in L_\infty( [0, 1] )$.
Обозначим $\norm{u}_\infty = r$,
тогда $\norm{u_h}_\infty < 2r$ при достаточно больших $h$.
Кроме того, $\abs{u_h'} \le 1$ почти всюду в $\phi_h( A_h )$.
Тогда $\hat{P_h^2} \le M_F \abs{\phi_h( A_h )} \to 0$, где
$$
M_F = \max\limits_{[-2r, 2r] \times [-M_a, M_a]} F;\quad M_a = \max\limits_{[0, 1] \times [-2r, 2r]} a.
$$

Далее,
\begin{multline*}
\hat{P_h^1} = \int\limits_{ [0, 1] \setminus \phi_h( A_h ) }
	F \bigl( u( \phi_h^{-1}( x ) ), a( x, u( \phi_h^{-1}( x ) ) |u'( \phi_h^{-1}( x ) ) ( \phi_h^{-1} )'| ) \bigr) \, dx
\\ =\int\limits_{ \phi_h^{-1}( [0, 1] ) \setminus A_h } F \bigl( u( z ), a( \phi_h( z ), u( z ) ) |u'( z )| \bigr) \, dz
\\ = \int\limits_{ [0, 1] } F \bigl( u( z ), a( \phi_h( z ), u( z ) ) |u'( z )| \bigr) \chi_{ \phi_h^{-1}( [0, 1] ) \setminus A_h } \, dz.
\end{multline*}
Последнее равенство, вообще говоря, не имеет смысла, так как $\phi_h( z )$ может принимать значения вне $[0, 1]$.
Определим $a( z, u ) = a( 1, u )$ при $z > 1$, теперь выражение корректно.
Заметим, что $\chi_{\phi_h^{-1}( [0, 1] ) \setminus A_h}$ возрастают,
так как множества $\phi_h^{-1}( [0, 1] )$ возрастают и $A_h$ убывают,
то есть $\phi_{h_1}^{-1}( [0, 1] ) \subset \phi_{h_2}^{-1}( [0, 1] )$ и $A_{h_1} \supset A_{h_2}$ при $h_1 \le h_2$.
На отрезке $[0, 1]$ (и даже $\phi_h( [0, 1] )$) функция $a$ убывает, а также $\phi_h( z )$ убывает по $h$,
значит $a( \phi_h( z ) )$ будет расти по $h$.
В таком случае можно применить теорему о монотонной сходимости и получить
$$
\hat{P_h^1} \to \int\limits_{[0, 1]} F \bigl( u( z ), a( z, u( z ) ) |u'( z )| \bigr) \, dz.
$$

\end{proof}

\begin{rem}
\label{rem:monotone_weight_appr}
Очевидно, что те же рассуждения с закреплением функции $u$ на левом конце можно провести на любом интервале $[x_0, x_1]$,
где вес $a$ убывает по $x$.
То есть можно получить последовательность $\{u_h\}$, удовлетворяющую
\begin{gather*}
u_h(x_0) = u(x_0); \qquad u_h \to u \text{ в } \W[x_0, x_1];\\
\int\limits_{x_0}^{x_1} F \bigl( u_h(x), a(x, u_h(x)) \abs{u_h'(x)} \bigr) \to \int\limits_{x_0}^{x_1} F \bigl( u(x), a(x, u(x)) \abs{u'(x)} \bigr).
\end{gather*}
Аналогично, если $a$ возрастает по $x$, можно аппроксимировать $u$ с закреплением на правом конце.
\end{rem}

\begin{cor}
Пусть функция $a$ непрерывна, чётна, убывает на $[0, 1]$ и удовлетворяет неравенству $(\ref{eq:almostConcave})$.
Тогда для любой $u \in \W[-1, 1]$ выполнено $\IWg( a, \symm{u} ) \le \IWg( a, u )$.
\end{cor}

\begin{proof}
Неравенство немедленно следует из теоремы \ref{thm:uplift} и леммы \ref{lm:monotone_weight_appr}.
\end{proof}
