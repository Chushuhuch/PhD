\section{Необходимые условия}
\label{sec:necessary_variable}

\begin{thm}
Пусть $\J(\symm{u}) \le \J(u)$ выполнено для любой кусочно линейной функции $u \ge 0$.
Тогда $p(x) \equiv const$.
\end{thm}

\begin{proof}
Рассмотрим $x_0 \in (-1, 1)$.
Для любых $\alpha > 0$ и $\eps > 0$, удовлетворяющих $[x_0 - \eps, x_0 + \eps] \subset [-1, 1]$, определим функцию
$$
u_{\alpha,\eps}(x) = \alpha ( \eps - |x - x_0| )_+.
$$
Тогда $\symm{u_{\alpha,\eps}}(x) = \alpha ( \eps - |x| )_+$, и
$$
\J(u_{\alpha, \eps}) = \int\limits_{x_0 - \eps}^{x_0 + \eps} \alpha^{p(x)} dx, \qquad
\J(\symm{u_{\alpha, \eps}}) = \int\limits_{-\eps}^{\eps} \alpha^{p(x)} dx.
$$
Возьмём неравенство
$$
\frac {\J(\symm{u_{\alpha, \eps}})}{2 \eps} \le \frac {\J(u_{\alpha, \eps})} {2 \eps},
$$
и перейдём к пределу при $\eps \to 0$.
Поскольку $p$ непрерывна, мы получим $\alpha^{p(0)} \le \alpha^{p(x_0)}$.
При $\alpha > 1$ и $\alpha < 1$ это даёт $p(0) \le p(x_0)$ и $p(0) \ge p(x_0)$ соответственно.
\end{proof}

\begin{thm}
\label{thm:necessary_variable}
Если неравенство $\I(\symm{u}) \le \I(u)$ выполняется для всех кусочно линейных $u \ge 0$,
то $p$ чётна и выпукла.
Более того, выпукла следующая функция:
$$
K(s, x) = s ( 1 + s^{-2} )^{\frac {p(x)}{2}}, \qquad s > 0,\ x \in [-1, 1].
$$
\end{thm}

\begin{proof}
Возьмём две точки $-1 < x_1 < x_2 < 1$
и рассмотрим финитную кусочно линейную функцию с ненулевой производной лишь в окрестностях $x_1$ и $x_2$.
А именно, для произвольных $s, t > 0$ и достаточно малого $\eps > 0$
$$
\left\{
\begin{aligned}
u_\eps(x) &= 0,                        & x \in &[-1, x_1 - s \eps] \cup [x_2 + s \eps, 1] \\
u_\eps(x) &= \eps + \frac{x - x_1}{s}, & x \in &[x_1 - s \eps, x_1 + s \eps] \\
u_\eps(x) &= 2 \eps,                   & x \in &[x_1 + s \eps, x_2 + s \eps] \\
u_\eps(x) &= \eps + \frac{x_2 - x}{t}, & x \in &[x_2 - t \eps, x_2 + t \eps].
\end{aligned}
\right.
$$

\begin{figure}
    \begin{subfigure}{1.0\textwidth}
        \begin{center}
            \begin{picture}(400,180)
                \put(0,50){\vector(1,0){400}}
                \put(200,30){\vector(0,1){160}}
                \put(20,48){\line(0,1){4}}
                \put(0,28){$-1$}
                \put(380,48){\line(0,1){4}}
                \put(376,28){$1$}

                \thicklines
                \put(20,50){\line(1,0){56}}
                \put(76,50){\line(1,1){96}}
                \put(134,48){\line(0,1){4}}
                \put(120,28){$x_1$}

                \put(172,146){\line(1,0){96}}

                \put(268,146){\line(2,-3){64}}
                \put(300,48){\line(0,1){4}}
                \put(286,28){$x_2$}

                \put(332,50){\line(1,0){48}}
                \thinlines

                \put(220,154){$2 \eps$}
                \put(90,140){$u(x)$}
            \end{picture}
            \caption{График $u_\eps$}
        \end{center}
    \end{subfigure}

\bigskip
\bigskip
\bigskip

    \begin{subfigure}{1.0\textwidth}
        \begin{center}
            \begin{picture}(400,180)
                \put(0,50){\vector(1,0){400}}
                \put(200,30){\vector(0,1){160}}
                \put(20,48){\line(0,1){4}}
                \put(0,28){$-1$}
                \put(380,48){\line(0,1){4}}
                \put(376,28){$1$}

                \thicklines
                \put(20,50){\line(1,0){52}}
                \put(72,50){\line(5,6){80}}
                \put(112,48){\line(0,1){4}}
                \put(97,28){${x_1 - x_2 \over 2}$}

                \put(152,146){\line(1,0){96}}

                \put(248,146){\line(5,-6){80}}
                \put(328,50){\line(1,0){52}}
                \put(288,48){\line(0,1){4}}
                \put(273,28){${x_2 - x_1 \over 2}$}
                \thinlines

                \put(220,154){$2 \eps$}
                \put(70,140){$\symm{u}(x)$}
            \end{picture}
            \caption{График $\symm{u_\eps}$}
        \end{center}
    \end{subfigure}

    \bigskip

    \caption{К доказательству теоремы \ref{thm:necessary_variable}}
\end{figure}

Тогда
$$
\left\{
\begin{aligned}
\symm{u_\eps}(x) &= 0,
        & x \in &[-1, \frac{x_1 - x_2}{2} - \frac{s + t}{2} \eps] \cup [\frac{x_2 - x_1}{2} + \frac{s + t}{2} \eps, 1] \\
\symm{u_\eps}(x) &= \eps + \frac{2 x - (x_2 - x_1)}{s + t},
        & x \in &[\frac{x_1 - x_2}{2} - \frac{s + t}{2} \eps, \frac{x_1 - x_2}{2} + \frac{s + t}{2} \eps] \\
\symm{u_\eps}(x) &= 2 \eps,
        & x \in &[\frac{x_1 - x_2}{2} + \frac{s + t}{2} \eps, \frac{x_2 - x_1}{2} - \frac{s + t}{2} \eps] \\
\symm{u_\eps}(x) &= \eps + \frac{(x_2 - x_1) - 2 x}{s + t},
        & x \in &[\frac{x_2 - x_1}{2} - \frac{s + t}{2} \eps, \frac{x_2 - x_1}{2} + \frac{s + t}{2} \eps].
\end{aligned}
\right.
%\symm{u_\eps}(x) = \min\big( 2 \eps, ( \eps + ( t + s )^{-1} ( x_2 - x_1 - 2 |x| ) )_+ \big).
$$

Множества, на которых $u_\eps' = 0$ и $\symm{u_\eps}' = 0$ имеют одинаковую меру.
Поэтому неравенство $\I(\symm{u_\eps}) \le \I(u_\eps)$ эквивалентно следующему:
\begin{multline*}
\int\limits_{ x_1 - s \eps }^{ x_1 + s \eps } \big( 1 + \frac 1 {s^2 } \big)^{\frac {p(x)} 2} dx
+ \int\limits_{ x_2 - t \eps }^{ x_2 + t \eps } \big( 1 + \frac 1 {t^2 } \big)^{\frac {p(x)} 2} dx
\\ \ge \int\limits_{ \frac {x_1 - x_2} 2  - \frac {s + t}2 \eps }^{ { \frac {x_1 - x_2} 2 } + { \frac {s + t} 2 } \eps }
\Big( 1 + \frac{1}{ ( {\frac { s + t} 2 } )^2 } \Big)^{\frac {p(x)} 2} dx
     + \int\limits_{ {\frac {x_2 - x_1} 2 } - { \frac {s + t} 2 } \eps }^{ { \frac {x_2 - x_1} 2 } + { \frac {s + t} 2 } \eps }
     \Big( 1 + \frac{1}{ ( { \frac {s + t} 2 } )^2 } \Big)^{\frac {p(x)} 2} dx.
\end{multline*}

Разделим это неравенство на $2 \eps$ и устремим $\eps \to 0$, получив в пределе
\begin{multline}
\label{eq:var_nec_pre_conv}
s \Big( 1 + { \frac 1 {s^2} } \Big)^{\frac {p(x_1)} 2} + t \Big( 1 + { \frac 1 {t^2} } \Big)^{\frac {p(x_2)} 2}\\
\ge { \frac {s + t} 2 } \Big( 1 + \frac{1}{ ( { \frac {s + t} 2 } )^2 } \Big)^{\frac {p( { \frac {x_1 - x_2} 2 } )} 2}
  + { \frac {s + t} 2 } \Big( 1 + \frac{1}{ ( { \frac {s + t} 2 } )^2 } \Big)^{\frac {p( { \frac {x_2 - x_1} 2 } )} 2}.
\end{multline}

Для начала, положим $s = t$ в неравенстве (\ref{eq:var_nec_pre_conv}).
Получаем
\begin{equation}
\label{eq:var_nec_s_eq_t}
( 1 + { \frac 1 {s^2} } )^{\frac {p(x_1)} 2} + ( 1 + { \frac 1 {s^2} } )^{\frac {p(x_2)} 2}
\ge ( 1 + { \frac 1 {s^2} } )^{\frac {p( { \frac {x_2 - x_1} 2 } )}  2}
+ ( 1 + { \frac 1 {s^2} } )^{\frac {p( { \frac {x_1 - x_2}2 } )} 2}.
\end{equation}

Обозначим $\sigma: = {\frac 1 {s^2}}$ и применим разложение по Тейлору к неравенству (\ref{eq:var_nec_s_eq_t}) в точке $\sigma = 0$:
$$
\sigma p(x_1) + \sigma p(x_2) \ge \sigma p( {\frac { x_2 - x_1}2 } ) + \sigma p( {\frac { x_1 - x_2} 2 } ) + r(\sigma),
$$
где $r(\sigma) = o(\sigma)$ при $\sigma \to 0$.
Таким образом, для любых $x_1, x_2 \in [-1, 1]$ имеем
$$
p(x_1) + p(x_2) \ge p( { \frac {x_2 - x_1} 2 } ) + p( { \frac {x_1 - x_2} 2 } ).
$$
По лемме \ref{lm:convex_even_criterion} получаем, что $p$ чётна и выпукла.

Теперь подставим $-x_2$ вместо $x_2$ в (\ref{eq:var_nec_pre_conv}).
Поскольку $p$ чётна, получаем
$K(s, x_1) + K(t, x_2) \ge 2 K( { \frac {s + t}2 }, {\frac { x_1 + x_2}2 } )$.
\end{proof}
