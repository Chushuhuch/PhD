\section{Многомерный аналог неравенства $\I(\symm{u}) \le \I(u)$}

В этом параграфе функция $u \in \Wf(\Omega)$, где $\Omega=\omega \times [-1, 1]$.
Как и в первой главе, мы используем обозначения $x = (x', y)$, где $x' \in \omega$, $y \in [-1, 1]$.

Введём многомерный аналог функционала $\I$:
$$
\Imul(u)=\int\limits_{\Omega} ( 1 + | \nabla u(x) |^2 )^{ \frac{p(x)}{2} } dx.
$$

\begin{thm}
Если $\Imul(u^*) \le \Imul(u)$ для любой неотрицательной функции $u \in \Wf(\Omega)$,
то $p(x',y)$ не зависит от $y$.
\end{thm}

\begin{proof}
Для начала, мы докажем, что, аналогично теореме \ref{necessary_variable},
$p$ должна быть чётной и выпуклой по $y$, а функция
$$
\K_{x'}(c, d, y) = c \bigl( {\textstyle1 + {1 + d^2 \over c^2}} \bigr)^{p(x', y) \over 2}
$$
должна быть выпуклой по совокупности аргументов на $[-1, 1] \times \Real \times \Real_+$.

Действительно, рассмотрим две точки
$$
x_1 = (x_0', y_1),\ x_2 = (x_0', y_2), \qquad \text{где} \quad x_0' \in \omega,\ -1 < y_1 < y_2 < 1.
$$
Зададим функцию $u \in \Wf(\Omega)$ с ненулевым градиентом только в окрестностях $x_1, x_2$
и в окрестности боковой границы цилиндра с осью $[x_1, x_2]$ следующим образом
$$
u(x) = \min\Bigl(
  \bigl( {y - y_1 \over c_1} + (x' - x_0') \cdot \B_1' \bigr)_+,
  \bigl( {y_2 - y \over c_2} + (x' - x_0') \cdot \B_2' \bigr)_+,
  \delta \big( w - |x' - x_0'| \big)_+, h
\Bigr).
$$
Здесь параметры $c_1, c_2 > 0$ --- обратные производные по $y$ в <<основаниях>> цилиндра,
$\B_1', \B_2' \in \Real^{n - 1}$ --- градиенты по $x'$ в <<основаниях>> цилиндра,
$\delta > 0$ --- модуль градиента на боковой поверхности цилиндра,
$w > 0$ --- радиус цилиндра,
а $h > 0$ --- максимальное значение функции.

Зафиксировав $y_1$, $y_2$, $c_1$, $c_2$, $\B_1'$, $\B_2'$, мы выбираем $h$ и $\delta$ как функции малого параметра $w$.
Мы требуем $\varkappa \equiv {h \over \delta} := {w \over 2}$ ($\varkappa$ --- ширина бокового слоя с ненулевой производной).

Левое основание носителя $u$ задаётся системой
$$
{y - y_1 \over c_1} + (x' - x_0') \cdot \B_1' = 0; \qquad |x' - x_0'| \le w.
$$
То есть это $(n - 1)$-мерный вытянутый эллипсоид вращения
с большой полуосью $\sqrt{w^2 + c_1^2 w^2 |\B_1'|^2}$ и радиусом $w$.
Значит $\nabla u = (\B_1', \frac{1}{c_1})$ на множестве $A_1$, которое является усечённым конусом с этим эллипсоидом в основании.
Прямое вычисление показывает
$$
\meas A_1 = C_1 \delta c_1 w^n
$$
(здесь и далее $C$ с индексом или без --- некоторые константы, зависящие только от $n$).

Аналогично $\nabla u = (\B_2', -\frac{1}{c_2})$ на множестве $A_2$, $\meas A_2 = C_1 \delta c_2 w^n$.

После симметризации на <<основаниях>> $\nabla u^* = (\frac{c_1 \B_1' + c_2 \B_2'}{c_1 + c_2}, \pm \frac{2}{c_1 + c_2})$.
Тем самым, множества $A_1$ и $A_2$ переходят в $A_1'$ и $A_2'$, и выполнено
$$
\meas A_1' = \meas A_2' = C_1 \delta \frac{c_1 + c_2}{2} w^n.
$$

Далее обозначим за $A_\delta$ боковой слой с ненулевым градиентом.
Прямая оценка даёт
$$
\meas A_\delta \le C\big((y_2 - y_1) w^{n - 1} + (c_1 |\B_1'| + c_2 |\B_2'|)w^n\big).
$$
Также обозначим
$$
Z = \meas \{ x \in \Omega \,|\, \nabla u(x) = 0 \} = \meas \{ x \in \Omega \,|\, \nabla u^*(x) = 0 \}.
$$

При $\frac{1}{c_1} = \frac{1}{c_2} = w^2$, $\delta = w^4$ и $\B_1' = \B_2' = 0$
из предположений теоремы следует
\begin{eqnarray*}
0 &\le& (\Imul(u) - Z ) - (\Imul(u^*) - Z )
 \le \big( (1 + w^4)^{\frac{p(\bar{x}_1)}{2}} - 1 \big) \cdot \meas A_1 \\
&+& \big( (1 + w^4)^{\frac{p(\bar{x}_2)}{2}} - 1 \big) \cdot \meas A_2
+ \big( (1 + w^8)^{\frac{P}{2}} - 1 \big) \cdot \meas A_\delta
\\ &- &\big( (1 + w^4)^{\frac{p(\hat{x}_1)}{2}} - 1 + (1 + w^4)^{\frac{p(\hat{x}_2)}{2}} - 1 \big) \cdot \meas A_1'
\\ &\le& w^4 \big({\textstyle\frac{p(\bar{x}_1)}{2} + \frac{p(\bar{x}_2)}{2} + o(w)}\big) C_1 w^{n + 2}
+ w^8 \big({\textstyle\frac{P}{2} + o(w)} \big) (y_2 - y_1 ) C
w^{n-1}
\\ &&- w^4 \big({\textstyle\frac{p(\hat{x}_1)}{2} + \frac{p(\hat{x}_2)}{2} + o(w)}\big) C_1 w^{n + 2}.
\end{eqnarray*}
Здесь $P = \max p(x', y)$, $\bar{x}_1 \in A_1$, $\bar{x}_2 \in A_2$, $\hat{x}_1 \in A_1'$, $\hat{x}_2 \in A_2'$.

Мы переходим к пределу при $w \to 0$ и получаем
$$
0 \le p(x_0',y_1) + p(x_0', y_2) - p(x_0', \frac{y_1 - y_2}{2}) -
p(x_0', \frac{y_2 - y_1}{2}).
$$
Применив лемму \ref{almostConvexLm}, получаем, что $p$ чётна и выпукла по $y$.

Теперь зафиксируем произвольные положительные $c_1$, $c_2$, $d_1$ и $d_2$,
положим $\B_1'=\frac {d_1}{c_1}\, \E$, $\B_2'=\frac {d_2}{c_2}\, \E$ (здесь $\E$ --- некоторый единичный вектор в гиперплоскости $x'$)
и возьмём $\delta = w^2$.
Тогда получаем
\begin{eqnarray*}
0 &\le& (\Imul(u) - Z) - (\Imul(u^*) - Z)
\le \big( \textstyle{ \frac{1}{c_1} \K_{\bar{x}_1'}(c_1, d_1, \bar{y}_1)-1 } \big) \cdot  \meas A_1
\\ &+& \big( \textstyle{ \frac{1}{c_2} \K_{\bar{x}_2'}(c_2, d_2, \bar{y}_2)-1 } \big) \cdot \meas A_2
  + \big( (1 + w^4)^{\frac{P}{2}} - 1 \big) \cdot \meas A_\delta
\\ &-& \big( \textstyle{ \frac{2}{c_1 + c_2} \K_{\hat{x}_1'}(\frac{c_1 + c_2}{2}, \frac{d_1 + d_2}{2}, \hat{y}_1) - 1 }\big) \cdot \meas A_1'
\\ &-& \big( \textstyle{ \frac{2}{c_1 + c_2} \K_{\hat{x}_2'}(\frac{c_1 + c_2}{2}, \frac{d_1 + d_2}{2}, \hat{y}_2) - 1 }\big) \cdot \meas A_2'
\\ &\le& \big( {\textstyle \K_{\bar{x}_1'}(c_1, d_1, \bar{y}_1) + \K_{\bar{x}_2'}(c_2, d_2, \bar{y}_2)} \big) C_1 w^{n + 2}
+ ({\textstyle\frac{P}{2} + o(w)}) C (y_2 - y_1) w^{n + 3}
\\ &-& \big( {\textstyle\K_{\hat{x}_1'}(\frac{c_1 + c_2}{2}, \frac{d_1 + d_2}{2}, \hat{y}_1) +\K_{\hat{x}_2'}(\frac{c_1 +c_2}{2},
\frac{d_1 + d_2}{2}, \hat{y}_2)} \big) C_1 w^{n + 2}.
\end{eqnarray*}
Здесь $P = \max p(x', y)$, $(\bar{x}_1', \bar{y}_1)\in A_1$,
$(\bar{x}_2', \bar{y}_2) \in A_2$, $(\hat{x}_1', \hat{y}_1) \in
A_1'$, $(\hat{x}_2', \hat{y}_2) \in A_2'$.

Мы переходим к пределу при $w \to 0$ и получаем
\begin{multline*}
\K_{x_0'}(c_1, d_1, y_1) + \K_{x_0'}(c_2, d_2, y_2)
\\ \ge \K_{x_0'}\big({\textstyle \frac{c_1 + c_2}{2}, \frac{d_1 + d_2}{2},\frac{y_1 - y_2}{2}
}\big) + \K_{x_0'}\big({\textstyle \frac{c_1 + c_2}{2}, \frac{d_1 + d_2}{2}, \frac{y_2 -y_1}{2}}\big).
\end{multline*}
Отсюда следует, что $\K_{x_0'}$ выпукла, поскольку она чётна по $y$.

Наконец, заметим (здесь мы опускаем для краткости $x'$ в записи $\K_{x'}$), что
$$
\K(c,d,y)=K(\frac c{\sqrt{1 + d^2}}, y) \cdot \sqrt{1 + d^2},
$$
где функция $K$ введена в теореме \ref{necessary_variable_thm}.
Прямое вычисление показывает, что
$$
\det(\K''(c, d, y)) = \frac{1}{ (1 + d^2)^{ \frac{3}{2} } } \cdot \bigl[ (\partial^2_{yy} K \partial^2_{ss} K
- (\partial^2_{sy} K)^2) (K - s \partial_s K) - \partial^2_{ss} K (\partial_y K)^2 d^2 \bigr]
$$
(здесь $s = \frac{c}{\sqrt{1 + d^2}}$).
То есть если $\partial_y K \not\equiv 0$, можно выбрать достаточно большое $d$, чтобы получить $\det(\K''(c, d, y)) < 0$ и, тем самым, противоречие.
\end{proof}
