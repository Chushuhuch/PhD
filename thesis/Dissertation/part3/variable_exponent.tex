\section{Введение}

Положим $u \in \Wf[-1, 1]$, $u \ge 0$.
Обозначим через $symm{u}$ симметричную перестановку функции $u$.
Тогда выполнено классическое неравенство Пойя-Сегё:

\begin{equation}
\label{PS}
I(\symm{u}) \le I(u), \qquad \text{где } I(u) = \int\limits_{-1}^1 |u'(x)|^p dx, \quad p \ge 1
\end{equation}

Обобщения неравенства (\ref{PS}) обсуждаются в большом количестве работ (см. обзор \cite{Talenti} и цитированную в нем литературу).
В частности, в статье \cite{Brock} показано, что неравенство (\ref{PS}) выполняется для функционалов вида
$$\int_{-1}^1 F( u(x), a( x, u( x ) ) \abs{ u'( x ) } ) dx,$$
где $a$ выпукла, а $F$ непрерывна и выпукла по второму аргументу, а также для многомерного аналога.
В работе \cite{1dim} восполнены пробелы в доказательстве \cite{Brock}, и результат доказан для естественного класса функций $u$.
Аналогичные результаты для монотонной перестановки также получены в \cite{DAN} и \cite{1dim}.

Рассмотрим функционалы с переменным показателем суммирования:
\begin{equation*}
\J(u) = \int\limits_{-1}^1 |u'(x)|^{p(x)} dx, \qquad \I(u) = \int\limits_{-1}^1 ( 1 + | u'(x) |^2 )^{p(x) \over 2} dx.
\end{equation*}
Здесь $p(x) \ge 1$ --- непрерывная функция на $[-1, 1]$, $u \in \Wf[-1, 1]$.
Подобные функционалы встречаются в некоторых задачах математической физики,
в частности при моделировании электрореологических жидкостей.
Более подробное описание задач и связанных с ними подходов может быть найдено в книгах \cite{Ruzicka} и \cite{ZhikovBook}.

\section{Необходимые условия}

\begin{thm}
\label{uniform}
Предположим, что неравенство $\J(\symm{u}) \le \J(u)$ выполнено для любой кусочно линейной функции $u$.
Тогда $p(x) \equiv const$.
\end{thm}
\begin{proof}
Зафиксируем произвольное $x_0 \in (-1, 1)$.
Для каждого $\alpha > 0$ и $\eps > 0$ такого, что $[x_0 - \eps, x_0 + \eps] \subset [-1, 1]$,
определим функцию
$$
u_{\alpha,\eps}(x) = \alpha ( \eps - |x - x_0| )_+.
$$
Тогда $symm{u_{\alpha,\eps}}(x) = \alpha ( \eps - |x| )_+$, и

%Кроме того,
%\begin{eqnarray*}
%\begin{aligned}
%& |u_{\alpha,\eps}'(x)|     = 0,      & x & \in (-1, x_0 - \eps) \cup (x_0 + \eps, 1) \\
%& |u_{\alpha,\eps}'(x)|     = \alpha, & x & \in (x_0 - \eps, x_0 + \eps) \\
%& |\symm{u_{\alpha,\eps}}'(x)| = 0,      & x & \in (-1, -\eps) \cup (\eps, 1) \\
%& |\symm{u_{\alpha,\eps}}'(x)| = \alpha, & x & \in (-\eps, \eps) \\
%\end{aligned}
%\end{eqnarray*}

%Тогда
$$
\J(u_{\alpha, \eps}) = \int\limits_{x_0 - \eps}^{x_0 + \eps} \alpha^{p(x)} dx, \qquad
\J(\symm{u_{\alpha, \eps}}) = \int\limits_{-\eps}^{\eps} \alpha^{p(x)} dx.
$$

Перейдя к пределу при $\eps \to 0$ в неравенстве
$$
{\J(\symm{u_{\alpha, \eps}}) \over 2 \eps} \le {\J(u_{\alpha, \eps}) \over 2 \eps},
$$
получим $\alpha^{p(0)} \le \alpha^{p(x_0)}$, в силу непрерывности $\alpha^{p(x)}$.
Заметим, что при $\alpha > 1$ и $\alpha < 1$ это даёт неравенства $p(0) \le p(x_0)$ и $p(0) \ge p(x_0)$ соответственно.
\end{proof}

Таким образом, прямое обобщение неравенства (\ref{PS}) невозможно.

\begin{thm}
\label{necessary_conditions_variable}
Предположим, что неравенство $\I(\symm{u}) \le \I(u)$ выполнено для любой кусочно линейной функции $u$.
Тогда функция $p$ чётна и выпукла. Более того, выпукла функция двух переменных
$$K(s, x) = s ( 1 + s^{-2} )^{p(x) \over 2} \qquad s > 0,\ x \in [-1, 1].$$
\end{thm}

%Для доказательства нам потребуется
%\begin{prop}
%\label{convProp}
%\textrm{\cite[лемма 10]{1dim}.}
%Предположим, что функция $p$ непрерывна на $[-1, 1]$, и для нее выполнено
%$$
%\forall s, t \in [-1, 1] \qquad p(s) + p(t) \ge p \bigl( {s - t \over 2} \bigr) + p \bigl( {t - s \over 2} \bigr).
%$$
%Тогда функция $p$ чётна и выпукла.
%\end{prop}

\begin{proof}[Набросок доказательства]
Зафиксируем две точки на отрезке $-1 < x_1 < x_2 < 1$
и рассмотрим финитную кусочно линейную функцию с ненулевыми производными только в окрестностях $x_1$ и $x_2$.
А именно, для произвольных $s, t > 0$ и достаточно малого $\eps$ положим
$$
u_\eps(x) = \min \bigl( 2 \eps, ( \eps + s^{-1} (x - x_1))_+, ( \eps + t^{-1} (x_2 - x) )_+ \bigr).
$$
Тогда
$$
\symm{u_\eps}(x) = \min \bigl( 2 \eps, ( \eps + ( t + s )^{-1} ( x_2 - x_1 - 2 |x| ) )_+ \bigr).
$$
%\begin{equation}
%\left\{
%\begin{aligned}
%u(x) &= 0, & x &\in [-1, x_1 - s \eps] \cup [x_2 + t \eps, 1]\\
%u(x) &= 2 \eps, & x &\in [x_1 + s \eps, x_2 - t \eps]\\
%u(x) &= \eps + { x - x_1 \over s }, & x &\in [x_1 - s \eps, x_1 + s \eps]\\
%u(x) &= \eps - { x - x_2 \over t }, & x &\in [x_2 - t \eps, x_2 + t \eps]
%\end{aligned}
%\right.
%\end{equation}

%\begin{center}
%\begin{picture}(200,90)
%\refstepcounter{pictureCounter}
%\label{uGraph}
%
%\put(0,25){\vector(1,0){200}}
%\put(100,15){\vector(0,1){80}}
%\put(10,24){\line(0,1){2}}
%\put(0,14){$-1$}
%\put(190,24){\line(0,1){2}}
%\put(188,14){$1$}
%
%\put(38,25){\line(1,1){48}}
%\put(38,24){\line(0,1){2}}
%\put(20,14){$x_1 - s$}
%\put(86,24){\line(0,1){2}}
%\put(68,30){$x_1 + s$}
%
%\put(86,73){\line(1,0){48}}
%
%\put(134,73){\line(2,-3){32}}
%\put(134,24){\line(0,1){2}}
%\put(116,30){$x_2 - t$}
%\put(166,24){\line(0,1){2}}
%\put(148,14){$x_2 + t$}
%
%\put(110,77){$2 \eps$}
%\put(45,70){$u(x)$}
%
%\put(85,1){Fig. \arabic{pictureCounter}}
%\end{picture}
%\end{center}

%Тогда функция $symm{u}$ имеет следующий вид:
%\begin{equation}
%\left\{
%\begin{aligned}
%\symm{u}(x) &= 0, & x &\in [-1, { x_2 - x_1 \over 2 } - { s + t \over 2 } \eps] \cup [{ x_1 - x_2 \over 2 } + { s + t \over 2 } \eps, 1]\\
%\symm{u}(x) &= 2 \eps, & x &\in [{ x_2 - x_1 \over 2 } + { s + t \over 2 } \eps, { x_1 - x_2 \over 2 } - { s + t \over 2 } \eps]\\
%\symm{u}(x) &= \eps + \frac{ x - { x_2 - x_1 \over 2 } }{ { s + t \over 2 } }, & x &\in [{ x_2 - x_1 \over 2 } - { s + t \over 2 } \eps, { x_2 - x_1 \over 2 } + { s + t \over 2 } \eps]\\
%\symm{u}(x) &= \eps - \frac{ x - { x_1 - x_2 \over 2 } }{ { s + t \over 2 } }, & x &\in [{ x_1 - x_2 \over 2 } - { s + t \over 2 } \eps, { x_1 - x_2 \over 2 } + { s + t \over 2 } \eps]\\
%\end{aligned}
%\right.
%\end{equation}
%
%\begin{center}
%\begin{picture}(200,90)
%\refstepcounter{pictureCounter}
%\label{uGraph}
%
%\put(0,25){\vector(1,0){200}}
%\put(100,15){\vector(0,1){80}}
%\put(10,24){\line(0,1){2}}
%\put(0,14){$-1$}
%\put(190,24){\line(0,1){2}}
%\put(188,14){$1$}
%
%\put(36,25){\line(5,6){40}}
%
%\put(76,73){\line(1,0){48}}
%
%\put(124,73){\line(5,-6){40}}
%
%\put(110,77){$2 \eps$}
%\put(35,70){$symm{u}(x)$}
%
%\put(85,1){Fig. \arabic{pictureCounter}}
%\end{picture}
%\end{center}

Полагая $s = t$, из неравенства $\I(\symm{u}) \le \I(u)$ получим $p(x_1) + p(x_2) \le p({ x_1 - x_2 \over 2 }) + p({ x_2 - x_1 \over 2 })$.
Можно проверить (\cite[Lemma 10]{1dim}), что отсюда следует чётность и выпуклость функции $p$.
Далее, рассматривая произвольные $s$ и $t$, получаем выпуклость функции $K$.
%Тогда, в силу сделанных предположений, верно следующее:
%\begin{multline*}
%\int\limits_{ x_1 - s \eps }^{ x_1 + s \eps } ( 1 + { 1 \over s^2 } )^{p(x) \over 2} dx
%+ \int\limits_{ x_2 - t \eps }^{ x_2 + t \eps } ( 1 + { 1 \over t^2 } )^{p(x) \over 2} dx \ge
%\\ \ge \int\limits_{ { x_2 - x_1 \over 2 } - { s + t \over 2 } \eps }^{ { x_2 - x_1 \over 2 } + { s + t \over 2 } \eps } ( 1 + \frac{1}{ ( { s + t \over 2 } )^2 } )^{p(x) \over 2} dx
%     + \int\limits_{ { x_1 - x_2 \over 2 } - { s + t \over 2 } \eps }^{ { x_1 - x_2 \over 2 } + { s + t \over 2 } \eps } ( 1 + \frac{1}{ ( { s + t \over 2 } )^2 } )^{p(x) \over 2} dx
%\end{multline*}
%
%Или, переходя к пределу при $\eps \to 0$,
%\begin{equation}
%\label{preConv}
%s ( 1 + { 1 \over s^2 } )^{p(x_1) \over 2} + t ( 1 + { 1 \over t^2 } )^{p(x_2) \over 2}
%\ge { s + t \over 2 } ( 1 + \frac{1}{ ( { s + t \over 2 } )^2 } )^{p( { x_2 - x_1 \over 2 } ) \over 2}
%  + { s + t \over 2 } ( 1 + \frac{1}{ ( { s + t \over 2 } )^2 } )^{p( { x_1 - x_2 \over 2 } ) \over 2}.
%\end{equation}
%
%Положим в неравенстве (\ref{preConv}) $s = t$:
%$$( 1 + { 1 \over s^2 } )^{p(x_1) \over 2} + ( 1 + { 1 \over s^2 } )^{p(x_2) \over 2}
%\ge ( 1 + { 1 \over s^2 } )^{p( { x_2 - x_1 \over 2 } ) \over 2} + ( 1 + { 1 \over s^2 } )^{p( { x_1 - x_2 \over 2 } ) \over 2}.$$
%
%Обозначив $\sigma = {1 \over s^2}$ и воспользовавшись разложением в ряд Тейлора по $\sigma$ в точке $\sigma = 0$, получим
%$$\sigma p(x_1) + \sigma p(x_2) \ge \sigma p( { x_2 - x_1 \over 2 } ) + \sigma p( { x_1 - x_2 \over 2 } ) + r(\sigma),$$
%
%где $r(s) = o(\sigma)$ при $\sigma \to 0$. То есть, для любых $x_1, x_2 \in [-1, 1]$ выполнено
%$$p(x_1) + p(x_2) \ge p( { x_2 - x_1 \over 2 } ) + p( { x_1 - x_2 \over 2 } ).$$
%
%Воспользовавшись предложением \ref{convProp}, получаем, что $p$ чётна и выпукла.
%Тогда, заменим в (\ref{preConv}) $x_2$ на $-x_2$ и воспользовавшись чётностью $p$, получаем
%$K(s, x_1) + K(t, x_2) \ge 2 K( { s + t \over 2 }, { x_1 + x_2 \over 2 } )$.
\end{proof}

\section{Доказательство неравенства $\I(\symm{u}) \le \I(u)$}

В этом параграфе мы покажем, что условия, приведенные в теореме \ref{necessary_conditions_variable} являются не только необходимыми, но и достаточными.

\begin{lm}
\label{quasiConv}
Пусть $m$ --- чётное положительное число, $s_k > 0$ ($k = 1 \dots m$), $-1 \le x_1 \le \dots \le x_m \le 1$.
Тогда, если $K(s, x)$ чётна по $x$ и выпукла по совокупности аргументов, то
\begin{equation}
\label{K_ineq}
\sum\limits_{k = 1}^{m} K(s_k, x_k) \ge
2 K \Bigl( {1 \over 2} \sum\limits_{k = 1}^{m} s_k, {1 \over 2} \sum\limits_{k = 1}^{m} (-1)^k x_k \Bigr).
\end{equation}
\end{lm}

\begin{proof}
Заметим, что неравенство (\ref{K_ineq}) равносильно такому же неравенству для функции $M(s, x) = K(s, x) - s$.
Прямое вычисление показывает, что функция $M$ убывает по $s$.
Тогда
\begin{multline*}
\sum\limits_{k = 1}^{m} M(s_k, x_k)
\ge M(s_1, x_1) + M(s_m, x_m)
\overset{a}{\ge} 2 M \bigl( {s_1 + s_m \over 2}, {x_m - x_1 \over 2} \bigr) \ge \\
\overset{b}{\ge} 2 M \Bigl( {1 \over 2} \sum\limits_{k = 1}^{m} s_k, {x_m - x_1 \over 2} \Bigr)
\overset{c}{\ge} 2 M \Bigl( {1 \over 2} \sum\limits_{k = 1}^{m} s_k, {1 \over 2} \sum\limits_{k = 1}^{m} (-1)^k x_k \Bigr).
\end{multline*}

Неравенство (a) сделует из того, что $M$ чётна по $x$ и выпукла,
(b) --- из убывания $M$ по $s$,
(c) --- из возрастания $M$ по $x$ при $x \ge 0$.
\end{proof}

\begin{lm}
\label{linear}
Пусть функция $K(s, x)$ чётна по $x$ и выпукла по совокупности аргументов.
Тогда $\I(\symm{u}) \le \I(u)$ для любой кусочно линейной функции $u \in \Wf[-1, 1]$.
\end{lm}

\begin{proof}
Обозначим $L \subset [-1, 1]$ множество точек перелома функции $u$ (включая концы отрезка).
Возьмём $U = u([-1, 1]) \setminus u(L)$, образ функции $u$ без образов точек излома.
Это множество представляется в виде объединения конечного числа непересекающихся интервалов $U = \cup_j U_j$.
Заметим, что для каждого $j$ множество $u^{-1}(U_j)$ разбивается на чётное число интервалов,
на каждом из которых функция $u$ совпадает с некоторой линейной функцией $y^j_k$, $k = 1, \dots, m_j$.
Для удобства считаем, что носители $y^j_k$ для каждого $j$ идут по порядку,
то есть $\sup dom(y^j_k) \le \inf dom(y^j_{k + 1})$.
Обозначим $b^j_k = |y^j_k{}'(x)|$.
Также обозначим
$$
Z = \meas\{ x \in (-1, 1) | u'(x) = 0 \} = \textrm{meas}\{ x \in (-1, 1) | \symm{u}'(x) = 0 \}.
$$
Тогда
\begin{multline*}
\I(u) - Z = \sum\limits_j \int\limits_{u^{-1}(U_j)} (1 + u'^2(x))^{p(x) \over 2} dx
= \sum\limits_j \sum\limits_k \int\limits_{dom(y^j_k)} (1 + {y^j_k}'^2(x))^{p(x) \over 2} dx =
\\ = \sum\limits_j \int\limits_{U_j} \sum\limits_k {1 \over b^j_k} (1 + b^j_k{}^2)^{p((y^j_k)^{-1}(y)) \over 2} dy
= \sum\limits_j \int\limits_{U_j} \sum\limits_k K \Bigl( {1 \over b^j_k}, (y^j_k)^{-1}(y) \Bigr) dy.
\end{multline*}

Любая точка $y \in U$ имеет два прообраза относительно функции $symm{u}$,
поэтому на множестве $U$ можно определить $(\symm{u})^{-1}: U \to [0, 1]$.
Для каждого $j$ можно выразить $(\symm{u})^{-1}$ и модуль её производной на участке $U_j$ следующим образом:
\begin{eqnarray*}
(\symm{u})^{-1} (y) & = & {1 \over 2} \sum\limits_{k = 1}^{m_j} (-1)^k (y^j_k)^{-1}(y); \\
|((\symm{u})^{-1})'(y)| & = & {1 \over |\symm{u}'((\symm{u})^{-1}(y))|} = {1 \over 2} \sum\limits_{k = 1}^{m_j} {1 \over b^j_k} =: { 1 \over \symm{b_j} }.
\end{eqnarray*}
Ввиду чётности $symm{u}$ имеем
\begin{multline*}
\I(\symm{u}) - Z = 2 \int\limits_{(\symm{u})^{-1}(U)} (1 + \symm{u}'^2(x))^{p(x) \over 2} dx =
\\ = 2 \int\limits_U |((\symm{u})^{-1})'(y)| \cdot \bigl( 1 + { 1 \over ((\symm{u})^{-1})'(y)^2 } \bigr)^{p((\symm{u})^{-1}(y)) \over 2} dy =
\\ = 2 \sum\limits_j \int\limits_{U_j} {1 \over \symm{b_j}} \bigl( 1 + \symm{b_j}^2 \bigr)^{{1 \over 2} p \bigl( {1 \over 2} \sum\limits_{k = 1}^{m_j} (-1)^k (y^j_k)^{-1}(y) \bigr)} dy =
\\ = 2 \sum\limits_j \int\limits_{U_j} K \Bigl( {1 \over 2} \sum\limits_{k = 1}^{m_j} {1 \over b^j_k}, {1 \over 2} \sum\limits_{k = 1}^{m_j} (-1)^k (y^j_k)^{-1}(y) \Bigr) dy.
\end{multline*}
Зафиксируем $j$ и $y$.
Тогда для доказательства леммы достаточно выполнения
$$
\sum\limits_{k = 1}^{m_j} K \Bigl( {1 \over b^j_k}, (y^j_k)^{-1}(y) \Bigr) \ge
2 K \Bigl( {1 \over 2} \sum\limits_{k = 1}^{m_j} {1 \over b^j_k}, {1 \over 2} \sum\limits_{k = 1}^{m_j} (-1)^k (y^j_k)^{-1}(y) \Bigr).
$$
Но это неравенство обеспечивается леммой \ref{quasiConv}.
\end{proof}

Теперь можно доказать неравенство для функций $u$ общего вида.

\begin{thm}
\label{variable_exponent_thm}
Пусть $p$ чётна, а $K$ выпукла.
Тогда для любой функции $u \in \Wf[-1, 1]$ выполнено $\I(\symm{u}) \le \I(u)$.
\end{thm}

\begin{proof}
Поскольку $p(x)$ ограничена, можно построить последовательность кусочно постоянных функций $v_n$, сходящуюся к $u'$ в $L^{p(x)}$
(см. \cite[Теорема~1.4.1]{Sharapudinov}).
Обозначим $u_n$ первообразные к $v_n$.
Изменяя, если необходимо, $v_n$ в окрестности концов отрезка, можно считать, что $u_n \ge 0$ и $u_n(\pm 1) = 0$.

Из вложения $L^{p(x)}[-1, 1]$ в $L^1[-1, 1]$ следует $u_n \to u$ в $\Wf[-1, 1]$.
Также, поскольку $| \sqrt{ 1 + x^2 } - \sqrt{ 1 + y^2 } | \le | x - y |$ для любых аргументов,
из $v_n \to u'$ в $L^{p(x)}$ следует сходимость $\I( u_n ) \to \I( u )$.

%Будем считать, что частичные интегралы функций $v_n$ на отрезках $[-1, x]$ неотрицательны,
%иначе в качестве $v$ возьмём производные модуля первообразных $v_n$.
%Это не ухудшит сходимость ({\color{red} неочевидно !!!!!!!!!!}) и обеспечит требуемое.
%Обозначим $w_n(x) = v_n(x) - \int\limits_{-1}^1 v_n(x) dx$ и $u_n \in \Wf[-1, 1]$ --- первообразные функций $w_n$.
%
%Покажем, что $w_n \to u'$ в $L^{p(x)}$:
%\begin{multline*}
%\norm{ w_n - u' }_{L^{p(x)}}
%\le \norm{ v_n - u' }_{L^{p(x)}} + \norm{ \int\limits_{-1}^1 v_n( x ) dx }_{L^{p(x)}} = \\
%= \norm{ v_n - u' }_{L^{p(x)}} + \norm{ \int\limits_{-1}^1 v_n( x ) - u'( x ) dx }_{L^{p(x)}}
%\le \norm{ v_n - u' }_{L^{p(x)}} + \norm{ \norm{ v_n - u' }_{L^1} }_{L^{p(x)}} \le \\
%\le \norm{ v_n - u' }_{L^{p(x)}} + \norm{ С \norm{ v_n - u' }_{L^{p(x)}} }_{L^{p(x)}} \to 0
%\end{multline*}
%В последнем неравенстве мы воспользовались вложением $L^{q(x)}$ в $L^{p(x)}$ при $q(x) \le p(x)$ для $q \equiv 1$.
%Тем самым, $u_n' \to u'$ в $L^{p(x)}$, а значит, $u_n \to u$ в $\Wf[-1, 1]$
%и $\sqrt{ 1 + u_n'^2 } \to \sqrt{ 1 + u'^2 }$ в $L^{p(x)}$, так как $\sqrt{ 1 + x^2 }$ --- сжимающее отображение.
%Последнее равносильно
%$$
%A( \sqrt{ 1 + u_n'^2 } - \sqrt{ 1 + u'^2 } ) \to 0, \text{\ где\ }
%\norm{ f }_{L^{p(x)}} = \inf\{ \lambda: A \Bigl( { f \over \lambda } \Bigr) \le 1 \},
%$$
%поскольку функция $A$ задаёт метрику, согласованную с нормой в $L^{p(x)}$.
%Отсюда следует, что $A( \sqrt{ 1 + u_n'^2 } ) \to A( \sqrt{ 1 + u'^2 } )$,
%что совпадает с утверждением $\I(u_n) \to \I(u)$.

Согласно \cite[Theorem 1]{Brock}, из $u_n \to u$ в $\Wf[-1, 1]$ следует $symm{u_n} \rightharpoondown \symm{u}$ в $\Wf[-1, 1]$.
Кроме того, функционал $\I$ секвенциально слабо полунепрерывен снизу по теореме Тонелли (см., напр., \cite[Теорема 3.5]{BGH}).
Поэтому
$$
\I(\symm{u}) \le \liminf \I(\symm{u_n}) \le \lim \I(u_n) = \I(u).
$$
\end{proof}

\section{Некоторые достаточные условия}

Вычисление показывает, что функция $K$ выпукла по $s$.
Если $p$ выпукла, то $K$ выпукла также по $x$.
А выпуклость $K$ по совокупности переменных равносильна выполнению неравенства
$D_{ss} K(s, x) D_{xx} K(s, x) - (D_{sx} K(s, x))^2 \ge 0$ в смысле мер.
%Распишем это неравенство.
%Обозначим $H(s) = 1 + {1 \over s^2}$.
%Тогда $K(s, x) = s H(s)^{p(x)}$, $H'(s) = -{2 \over s^3}$, $H''(s) = {6 \over s^4}$.
%\begin{eqnarray*}
%D_s K(s, x) & =  & H(s)^{p(x)} + s p(x) H(s)^{p(x) - 1} H'(s) = H(s)^{p(x)} - 2 p(x) H(s)^{p(x) - 1} ( H(s) - 1 ) \\
%D_x K(s, x) & = & s H(s)^{p(x)} p'(x) \ln{H(s)}
%\end{eqnarray*}
%\begin{eqnarray*}
%D_{ss} K(s, x) = p(x) H(s)^{p(x) - 1} H'(s) - 2p(x) (p(x) - 1) H(s)^{p(x) - 2} H'(s) (H(s) - 1) - 2p(x) H(s)^{p(x) - 1} H'(s)
%\end{eqnarray*}

Это неравенство приводится к следующему виду:
\begin{equation}
\label{KconvexIneq}
q''(x) \ge {q'(x)^2 \over q(x)} A(q(x), w),
\end{equation}
где $q(x) = p(x) - 1$, $w = s^{-2}$,
$$A(q, w) = \frac{
q ( 4 w - ( w + 3 ) \ln( w + 1 ) ) - {w - 1 \over w} \ln( w + 1 ) + 4 {w \over \ln( w + 1 )} - 4
}{
2 ( q w + 1 )
} \cdot {q \over q + 1}.$$
%$$A(q, w) = \frac{
%( {1 \over w} - q w - 3 q - 1 ) \ln( w + 1 ) + 4 {w \over \ln( w + 1 )} + 4 ( q w - 1 )
%}{
%2 ( q w + 1 )
%} \cdot {q \over q + 1}.$$

\begin{lm}
Для любого $\alpha > 0$ и $q \in C[-1, 1]$ выпуклость функции $q^\alpha$
равносильна выполнению неравенства $q''(x) \ge (1 - \alpha) {q'(x)^2 \over q(x)}$ в смысле мер.
\end{lm}

Лемма доказывается прямым вычислением.

Из леммы следует, что если $A(q, w) \le M$, где $M < 1$ --- некоторая константа,
то неравенство (\ref{KconvexIneq}) следует из выпуклости функции $(p(x) - 1)^{1 - M}$.

Численное исследование показывает, что при всех $q \ge 0$ и $w > 0$ выполнено неравенство $A(q, w) < 0.63$.
Если же дополнительно $q < 1.36$, то $A(q, w) < 0.5$.
Отсюда с учётом теоремы \ref{variable_exponent_thm} следует
\begin{thm}
Пусть функция $p(x)$ чётна.

1) Если $(p(x) - 1)^{0.37}$ --- выпуклая функция, то для всех $u \in \Wf[-1, 1]$ справедливо неравенство $\I(\symm{u}) \le \I(u)$.

2) Если $p(x) < 2.36$ и $\sqrt{p(x) - 1}$ --- выпуклая функция, то для всех $u \in \Wf[-1, 1]$ справедливо неравенство $\I(\symm{u}) \le \I(u)$.
\end{thm}
