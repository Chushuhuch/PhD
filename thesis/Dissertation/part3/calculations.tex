\section{Численные оценки функции $B(w, q)$}
\label{sec:calculations}

Для доказательства неравенства (\ref{eq:var_suf_B_bound}) мы делим положительный квадрант
$(w,q) \in \Real_+ \times \Real_+$ на пять участков, см. рис.~\ref{fig:estimate_B_everywhere_partition}:
\begin{eqnarray*}
&&R_1 = [0, 6] \times [0, 1],\quad R_2 =[0, 1] \times [1,\infty],
\quad R_3 = [1, 4] \times [1, \infty],\\
&&R_4 = [6,\infty]\times [0,1], \quad
R_5=[4,\infty]\times[1,\infty].
\end{eqnarray*}
На каждом из участков мы доказываем неравенство численно-аналитическим методом.

\begin{figure}[ht]
\begin{picture}(380,200)

\put(370,5){\large $w$} \put(20,5){$0$} \put(60,5){$1$}
\put(260,5){$6$} \put(180,5){$4$}

\put(10,20){$0$} \put(10,60){$1$} \put(10,185){\large $q$}

\put(20,20){\vector(1,0){360}} \put(20,60){\line(1,0){350}}

\put(20,20){\vector(0,1){170}} \put(60,60){\line(0,1){120}}
\put(260,20){\line(0,1){40}} \put(180,60){\line(0,1){120}}

\put(120,35){\LARGE $R_1$}
\put(31,120){\LARGE $R_2$} \put(110,120){\LARGE $R_3$}
\put(310,35){\LARGE $R_4$}
\put(270,120){\LARGE $R_5$}

\end{picture}

\caption{К доказательству неравенства (\ref{eq:var_suf_B_bound})}
\label{fig:estimate_B_everywhere_partition}
\end{figure}

Для $(w, q) \in R_1$ мы строим кусочно постоянную функцию $B_1(w, q)$, оценивающую $B(w, q)$ сверху.
Для этого мы делим $R_1$ на прямоугольники
$$
Q \equiv \{w_0 \le w \le w_1;\quad q_0 \le q \le q_1\} \subset R_1
$$
и находим постоянное значение $B_1$ на каждом из них,
заменяя члены формулы для $B(w,q)$ их экстремальными значениями в этом прямоугольнике:
\begin{multline*}
B(w, q)
= \frac{ q ( 4 w - ( w + 3 ) \ln( w + 1 ) ) - (1 - {1 \over w}) \ln( w + 1 ) + 4 {w \over \ln( w + 1 )} - 4 }{ 2 ( q w + 1 ) }
\cdot {q \over q + 1}
\\ = \frac{
    [4 q w] - [q (w + 3) \ln(w + 1)] - [\ln(w + 1)] +
    [{\ln(w + 1) \over w}] + [4 {w \over \ln(w + 1)}] - 4 }{
    [2 (q w + 1)] } \cdot [{q \over q + 1}]
\\ \overset{(*)}{\le} \frac{
    [4 q_1 w_1] - [q_0 (w_0 + 3) \ln(w_0 + 1)] - [\ln(w_0 + 1)] +
    [{\ln(w_0 + 1) \over w_0}] + [4 {w_1 \over \ln(w_1 + 1)}] - 4 }{
    [2 (q_0 w_0 + 1)] }
    \\ \times [{q_1 \over q_1 + 1}]
=: B_1|_Q.
\end{multline*}
Неравенство $(*)$ вытекает из монотонности в $R_1$ каждой из функций, заключённых в квадратные скобки, по обеим переменным.

Для $(w,q) \in R_2$ мы полагаем $r = {1 \over q}$ и строим кусочно постоянную функцию $B_2(w, r)$,
оценивающую $B(w, {1 \over r})$ сверху.
Для каждого прямоугольника
$$
Q \equiv \set{ w_0 \le w \le w_1;\quad {1 \over r_1} \le q = {1 \over r} \le {1 \over r_0} } \subset R_2
$$
берём
\begin{multline*}
B(w, {1 \over r})
= \Bigl(
    [4 ( {w \over \ln(w + 1)} - 1 ) r]
    + [3 ( w - \ln(w + 1) )]
    - [w \ln(w + 1)]
    - [r \ln(w + 1)] \\
    + [w]
    + [r {\ln(w + 1) \over w}]
\Bigr) \Big/ \Bigl(
    [2 (w + r)] \cdot [1 + r]
\Bigr) \\
\le \Bigl(
        [4 ( {w_1 \over \ln(w_1 + 1)} - 1 ) r_1]
        + [3 ( w_1 - \ln(w_1 + 1) )]
        - [w_0 \ln(w_0 + 1)]
        - [r_0 \ln(w_0 + 1)] \\
        + [w_1]
        + [r_1 {\ln(w_0 + 1) \over w_0}]
    \Bigr) \Big/ \Bigl(
        [2 (w_0 + r_0)] \cdot [1 + r_0]
    \Bigr)
=: B_2|_Q.
\end{multline*}

Аналогично, для $(w, q) \in R_4$ мы полагаем $v = {1 \over w}$ и строим кусочно постоянную функцию $B_4(v,q)$,
оценивающую $B({1 \over v},q)$ сверху.
Для каждого прямоугольника
$$
Q \equiv \set{ {1 \over v_1} \le w = {1 \over v} \le {1 \over v_0};\quad q_0 \le q \le q_1 } \subset R_4
$$
берём
\begin{multline*}
B({1 \over v}, q)
= \Bigl(
        [4 q]
        - [q (1 + 3 v) \ln({1 \over v} + 1)]
        - [v \ln({1 \over v} + 1)]
        + [v^2 \ln({1 \over v} + 1)]
        \\ + [{4 \over \ln({1 \over v} + 1)}]
        - [4 v]
    \Bigr) \Big/ \Bigl(
        [2 (v + q)]
    \Bigr)
\times [\frac{q}{q + 1}]
\\ \le \Bigl(
       [4 q_1]
       - [q_0 (1 + 3 v_0) \ln({1 \over v_0} + 1)]
       - [v_0 \ln({1 \over v_0} + 1)]
       + [v_1^2 \ln({1 \over v_1} + 1)]
       \\ + [{4 \over \ln({1 \over v_1} + 1)}]
       - [4 v_0]
    \Bigr) \Big/ \Bigl(
       [2 (v_0 + q_0)]
    \Bigr)
\times [\frac{q_1}{q_1 + 1}]
=: B_4|_Q.
\end{multline*}

Наконец, для $(w,q) \in R_5$ мы используем обозначения $v = {1 \over w}$, $r = {1 \over q}$
и строим кусочно постоянную функцию $B_5(w, r)$,
оценивающую $B({1 \over v},{1 \over r})$ сверху.
Для каждого прямоугольника
$$
Q \equiv \set{ {1 \over v_1} \le w = {1 \over v} \le {1 \over v_0}; \quad {1 \over r_1} \le q = {1 \over r} \le {1 \over r_0} } \subset R_5
$$
берём
\begin{multline*}
B({1 \over v}, {1 \over r})
= \frac{
    [{4 r \over \ln({1 \over v} + 1)}]
    + [4 (1 - r v)]
    - [ (r (v - v^2) + 3 v + 1) \ln({1 \over v} + 1) ]
}{
    2 (1 + r)(1 + r v)
}
\\ \le \frac{
   [{4 r_1 \over \ln({1 \over v_1} + 1)}]
   + [4 (1 - r_0 v_0)]
   - [ (r_0 (v_1 - v_1^2) + 3 v_1 + 1) \ln({1 \over v_1} + 1) ]
}{
   2 (1 + r_0)(1 + r_0 v_0)
}
=: B_5|Q.
\end{multline*}

Оценочные функции $B_1$, $B_2$, $B_4$, $B_5$ были вычислены с $15$ значащими цифрами на достаточно мелких разбиениях на прямоугольники.
Были получены следующие результаты.

\begin{center}
\begin{tabular} {|c|c|c|c|c|}
\hline
Участок & & Шаг разбиения по $w(v)$ & Шаг разбиения по $q(r)$ & Неравенство \\
\hline
$R_1$   & & $6 \cdot 10^{-2}$      & $10^{-1}$              & $B_1 \le 0.51$ \\
\hline
$R_2$   & & $10^{-2}$              & $10^{-2}$              & $B_2 \le 0.617$ \\
\hline
$R_4$   & & $2 \cdot 10^{-2}$      & $10^{-1}$              & $B_4 \le 0.50$ \\
\hline
$R_5$   & & $2 \cdot 10^{-3}$      & $10^{-2}$              & $B_5 \le 0.605$ \\
\hline\end{tabular}
\end{center}
\medskip
Во всех случаях было получено $B(w,q) \le 0.62$.

Анализ $B(w,q)$ в $R_3$ приходится производить более аккуратно.
Мы снова берём $r = {1 \over q}$ и утверждаем, что $B(w, {1 \over r})$ убывает по $r$.
Для доказательства мы строим кусочно постоянную функцию $B_3(w, r)$, оценивающую $\partial_r B(w, {1 \over r})$ сверху.
Для каждого прямоугольника
$$
Q \equiv \set{ w_0 \le w \le w_1;\quad {1 \over r_1} \le q = {1 \over r} \le {1 \over r_0} } \subset R_2
$$
берём
\begin{multline*}
\partial_r B(w, {1 \over r})
 = \Bigl(
    [4 {w \over \ln(w + 1)}]
    [w - r^2]
    + 4 \bigl(
        [r^2]
        - [2 r w + w^2 + 2 w]
        \\ + (
            [r^2 + 2 r (3 + w) + 4 + 3 w + w^2]
            - [{r^2 \over w}]
        )
        [\ln(w + 1)]
    \bigr)
\Bigr) \Big/ \Bigl(
    [2 (1 + r)^2 (w + r)^2]
\Bigr)
\\ \le \Bigl(
    [4 {w_1 \over \ln(w_1 + 1)}]
    [w_1 - r_0^2]
    + 4 \bigl(
        [r_1^2]
        - [2 r_0 w_0 + w_0^2 + 2 w_0]
        \\ + (
            [r_1^2 + 2 r_1 (3 + w_1) + 4 + 3 w_1 + w_1^2]
            - [{r_0^2 \over w_1}]
        )
        [\ln(w_1 + 1)]
    \bigr)
\Bigr) \Big/ \Bigl(
    [2 (1 + r_0)^2 (w_0 + r_0)^2]
\Bigr)
\\ =: B_5|_Q.
\end{multline*}
Функция $B_5$ была вычислена на достаточно мелком разбиении с $15$ значащими цифрами.
Был получен следующий результат.

\begin{center}
\begin{tabular} {|c|c|c|c|c|}
\hline
Участок & & Шаг разбиения по $w$ & Шаг разбиения по $r$ & Неравенство \\
\hline
$R_3$   & & $5 \cdot 10^{-3}$    & $10^{-3}$            & $B_3 \le -0.08$ \\
\hline
\end{tabular}
\end{center}
\medskip
Тем самым, поскольку в $B_3$ есть точки, в которых значение функции $B$ больше $0.62$,
мы получаем, что $B(w, {1 \over r})$ достигает максимума в $R_3$ при $r = 0$.

Для нахождения максимума мы берём
$$
B(w, \infty) = 2 - {1 \over 2} ( \ln(w + 1) + 3 {\ln(w + 1) \over w} )
$$
и утверждаем, что $B(w, \infty)$ вогнута при $w \in [1, 4]$.
Чтобы доказать это, мы строим кусочно постоянную функцию $B_\infty(w)$, оценивающую $\partial^2_{ww} B(w, \infty)$ сверху.
На каждом отрезке $[w_0, w_1] \subset [1, 4]$ берём
\begin{multline*}
\partial^2_{ww} B(w, \infty)
= \frac{
    \frac{
        [w (w^2 + 9 w + 6)]
    }{
        [(w + 1)^2]
    }
    - [6 \ln(w + 1)]
}{
    [2 w^3]
}
\\ \le \frac{
    \frac{
        [w_1 (w_1^2 + 9 w_1 + 6)]
    }{
        [(w_0 + 1)^2]
    }
    - [6 \ln(w_0 + 1)]
}{
    [2 w_0^3]
}
=: B_{\infty}|_{[w_0, w_1]}.
\end{multline*}
Функция $B_\infty(w)$ была вычислена на достаточно мелком разбиении с $15$ значащими цифрами.
Был получен следующий результат.

\begin{center}
\begin{tabular} {|c|c|c|c|}
\hline
Участок         & & Шаг разбиения     & Неравенство \\
\hline
$1 \le w \le 4$ & & $3 \cdot 10^{-3}$ & $B_\infty \le -0.13$ \\
\hline
\end{tabular}
\end{center}
Тем самым, точка максимума единственна.
С использованием стандартных численных методов было получено, что максимум достигается при
$$
w \approx 1.816960565240,
$$
причём
$$
\max B(w,\infty) \approx 0.627178211634.
$$
Неравенство (\ref{eq:var_suf_B_bound}) доказано.

Для доказательства неравенства (\ref{eq:var_suf_B_half_bound}) мы делим
$(w, q) \in \Real_+ \times [0, 1.36]$ на четыре участка, см. рис.~\ref{fig:estimate_B_half_partition}:
\begin{eqnarray*}
&&R_6 = [0, 3] \times [0, 1.36],\quad R_7 = [3, 5] \times [0, 1.3], \\
&&R_8 = [3, 5] \times [1.3, 1.36],\quad R_9 =[5, \infty] \times [0,1.36].
\end{eqnarray*}

\begin{figure}[ht]
\begin{picture}(380,120)

\put(370,5){\large $w$} \put(30,5){$0$} \put(165,5){$3$} \put(255,5){$5$}

\put(19,20){$0$} \put(11,70){$1.3$} \put(5,80){$1.36$} \put(10,110){\large $q$}

\put(30,20){\vector(1,0){350}}
\put(30,80){\line(1,0){340}} \put(165,75){\line(1,0){90}}

\put(30,20){\vector(0,1){90}}
\put(165,20){\line(0,1){60}} \put(255,20){\line(0,1){60}}

\put(90,43){\LARGE $R_6$} \put(305,43){\LARGE $R_9$}
\put(201,40){\LARGE $R_7$} \put(225,85){\LARGE $R_8$}

\put(225,95){\vector(-1,-1){15}}

\end{picture}

\caption{К доказательству неравенства (\ref{eq:var_suf_B_half_bound})}
\label{fig:estimate_B_half_partition}
\end{figure}

На этих участках мы используем кусочно постоянные функции $B_1$ и $B_4$, введённые ранее.
Значения функций были вычислены при достаточно мелком разбиении с $15$ значащими цифрами.
В $R_8$ потребовался шаг разбиения меньше $10^{-5}$, поэтому мы повторили вычисления с $18$ значащими цифрами.
Был получен следующий результат.

\begin{center}
\begin{tabular} {|c|c|c|c|c|}
\hline
Участок & & Шаг разбиения по $w(v)$ & Шаг разбиения по $q$ & Неравенство \\
\hline
$R_6$   & & $3 \cdot 10^{-3}$       & $1.36 \cdot 10^{-3}$ & $B_1 \le 0.498$ \\
\hline
$R_7$   & & $2 \cdot 10^{-3}$       & $1.3 \cdot 10^{-3}$  & $B_1 \le 0.498$ \\
\hline
$R_8$   & & $2 \cdot 10^{-4}$       & $6 \cdot 10^{-6}$    & $B_1 \le 0.49996$ \\
\hline
$R_9$   & & $2 \cdot 10^{-3}$       & $1.36 \cdot 10^{-2}$ & $B_4 \le 0.4992$ \\
\hline
\end{tabular}
\end{center}

Доказательство завершено.
