\section{Некоторые достаточные условия}

Условие выпуклости функции $K$ на самом деле есть некоторое условие на функцию $p$.
Вычисление показывает, что всегда $\partial^2_{ss} K > 0$, а также если $p$ выпукла, то и $\partial^2_{xx} K \ge 0$.
Поэтому выпуклость $K$ равносильна выполнению неравенства $\det(K'') \ge 0$ в смысле мер.

Прямое вычисление приводит к
\begin{multline*}
\det(K'') = \frac{(1 + w)^{q - 1}}{4}\\
\times\Big( w (w q + 1) (q + 1) \ln(1 + w) (q'^2 \ln(1 + w) + 2 q'') - q'^2 ( (1 - q w) \ln(1 + w) - 2 w )^2 \Big),
\end{multline*}
где $q = q(x) = p(x) - 1$ и $w = w(s) = \frac{1}{s^2}$.

Поэтому неравенство $\det(K'') \ge 0$ даёт следующее неравенство на функцию $q$:
\begin{equation}
\label{eq:var_suff_qq''}
q q'' \ge q'^2 {\mathcal B}(q),
\end{equation}
где ${\mathcal B}(q) \equiv \sup\limits_{w>0} B(w,q)$ и
$$
B(w, q) = \frac{
q ( 4 w - ( w + 3 ) \ln( w + 1 ) ) - {w - 1 \over w} \ln( w + 1 ) + 4 {w \over \ln( w + 1 )} - 4
}{
2 ( q w + 1 )
} \cdot {q \over q + 1}.
$$

Следующее утверждение проверяется прямым счётом.

\begin{lm}
\label{lm:var_mul_convexity_criterion}
Пусть $q \ge 0$ --- непрерывная функция на $[-1, 1]$.
Тогда неравенство $q q'' \ge q'^2 \mathcal M$ в смысле распределений при $\mathcal M \in (0, 1)$
равносильно выпуклости функции $q^{1 - \mathcal M}$.
\end{lm}

Сформулируем теперь простые достаточные условия для выполнения неравенства $\I(u^*) \le \I(u)$:
\begin{thm}
Пусть $p(x)\ge1$ --- чётная непрерывная функция на $[-1, 1]$.

\textbf{\textup{i)}}
Если функция $(p(x)-1)^{0.37}$ выпукла, то неравенство $\I(u^*) \le \I(u)$ выполнено для любой неотрицательной $u \in \Wf[-1, 1]$.

\textbf{\textup{ii)}}
Если $p(x) \le 2.36$ для всех $x \in [-1, 1]$ и функция $\sqrt{p(x) - 1}$ выпукла,
то неравенство $\I(u^*) \le \I(u)$ выполнено для любой неотрицательной $u \in \Wf(-1,1)$.
\end{thm}

\begin{proof}
Следующие неравенства доказаны в параграфе \ref{sec:calculations}:
\begin{eqnarray}
\label{eq:var_suf_B_bound}
&& \sup\limits_{q \ge 0}{\mathcal B}(q) = \limsup\limits_{q \to +\infty}{\mathcal B}(q) \le 0.63; \\
\label{eq:var_suf_B_half_bound}
&& \sup\limits_{0 \le q \le 1.36}{\mathcal B}(q) \le 0.5.
\end{eqnarray}
По лемме \ref{lm:var_mul_convexity_criterion} неравенство (\ref{eq:var_suff_qq''}) следует для обоих пунктов теоремы.
Применение теоремы \ref{thm:variable_exponent} завершает доказательство.
\end{proof}
