\chapter*{Заключение}						% Заголовок
\addcontentsline{toc}{chapter}{Заключение}	% Добавляем его в оглавление

%% Согласно ГОСТ Р 7.0.11-2011:
%% 5.3.3 В заключении диссертации излагают итоги выполненного исследования, рекомендации, перспективы дальнейшей разработки темы.
%% 9.2.3 В заключении автореферата диссертации излагают итоги данного исследования, рекомендации и перспективы дальнейшей разработки темы.
%% Поэтому имеет смысл сделать эту часть общей и загрузить из одного файла в автореферат и в диссертацию:

%Основные результаты работы заключаются в следующем.
%\input{common/concl}
%И какая-нибудь заключающая фраза.
%
%Последний параграф может включать благодарности.  В заключение автор
%выражает благодарность и большую признательность научному руководителю
%Иванову~И.\:И. за поддержку, помощь, обсуждение результатов и научное
%руководство. Также автор благодарит Сидорова~А.\:А. и Петрова~Б.\:Б. за
%помощь в~работе с~образцами, Рабиновича~В.\:В. за предоставленные
%образцы и~обсуждение результатов, Занудятину~Г.\:Г. и авторов шаблона
%*Russian-Phd-LaTeX-Dissertation-Template* за помощь в оформлении
%диссертации. Автор также благодарит много разных людей и
%всех, кто сделал настоящую работу автора возможной.


В данной диссертации получены следующие результаты.
Получены необходимые условия на вес для выполнения неравенства Пойа-Сегё с весом для монотонной перестановки.
Доказано неравенство Пойа-Сегё с весом для монотонной перестановки в случае ограниченного (степенного) роста интегранта .
Доказано неравенство Пойа-Сегё с весом в одномерном случае без ограничений, лишь при необходимых условиях.
Доказана необходимость условий, налагаемых в работе \cite{Brock} на вес для выполнения неравенства Пойа-Сегё с весом для симметризации.
В одномерном случае закрыт пробел в работе \cite{Brock}:
доказано неравенство Пойа-Сегё с весом для симметризации без дополнительных ограничений.
Представлены необходимые и достаточные условия выполнения неравенства Пойа-Сегё с весом для монотонной перестановки
на функциях, закреплённых на левом конце.
Неравенство доказано в многомерном случае для интегрантов ограниченного роста по производной
и в одномерном случае без дополнительных ограничений.
Представлены необходимые и достаточные условия выполнения неравенства Пойа-Сегё с переменным показателем
суммирования в одномерном случае.
Показано, что прямое многомерное обобщение отсутствует.
