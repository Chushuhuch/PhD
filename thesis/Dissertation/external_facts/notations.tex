Перечислим обозначения функциональных пространств, используемых в работе.
Все перечисленные пространства состоят из функций, определённых на множестве $E \subset \Real^n$.

$C(E)$ --- множество непрерывных функций.

$C^1(E)$ --- множество непрерывно дифференцируемых функций.

$Lip(E)$ --- множество липшицевых функций.

$\Ls(E)$ --- множество суммируемых функций.

$W{}^1_q(E)$ --- множество дифференцируемых в соболевском смысле функций, суммируемых в степени $q$ вместе с первыми производными.

$\Wf(E)$ --- замыкание множества гладких финитных функций в $\W(E)$.

$L^{p(x)}(E)$ --- пространство Орлича.
Измеримая функция $u \in L^{p(x)}$, если интеграл $\int\limits_E \abs{u(x)}^{p(x)} dx$ конечен.
Норма в этом пространстве определяется как
$$
\norm{u}_{L^{p(x)}} = \inf \Bigset{ \alpha > 0: \int\limits_E \bigabs{ {u(x) \over \alpha} }^{p(x)} dx \le 1 }.
$$

Мы используем обозначение $f_\pm = \max( \pm f, 0 )$.
