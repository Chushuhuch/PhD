\section{Свойства меры и функций}

\begin{prop}[{\cite[теорема 3.5]{BGH}}, теорема Тонелли о полунепрерывности]
\label{prop:wlsc}
Пусть $I$ --- ограниченный интервал на $\Real$ и $F(x, u, p)$ --- лагранжиан, удовлетворяющий следующим условиям:

\textbf{\textup{i)}}
$F$ и $F_p$ непрерывны по $(x, u, p)$,

\textbf{\textup{ii)}}
$F$ неотрицательна,

\textbf{\textup{iii)}}
$F$ выпукла по $p$.

Тогда функционал $\int\limits_I F(x, u(x), u'(x))dx$ секвенциально слабо полунепрерывен снизу в $\W(I)$.
\end{prop}

\begin{prop}[{\cite[теорема 3.13]{Rudin}}]
\label{prop:convex_combination_convergence}
Пусть $X$ --- метрическое локально выпуклое пространство.
Если $\{x_k\}$ --- последовательность в $X$, слабо сходящаяся к некоторому $x \in X$,
то найдётся последовательность $\{y_k\}$, удовлетворяющая условиям:

\textbf{\textup{i)}}
каждый $y_k$ является выпуклой комбинацией конечного количества $x_k$,

\textbf{\textup{ii)}}
$y_k$ сходятся сильно к $x$ в пространстве $X$.
\end{prop}

\begin{prop}[{\cite[\S6.6, теорема 1]{Gariepy}}]
\label{prop:app_lip_with_smooth}
Пусть $f: \Real^n \to \Real$ --- липшицева функция.
Тогда для любого $\eps$ найдётся непрерывно дифференцируемая функция $\tilde f: \Real^n \to \Real$, удовлетворяюшая
$$
\meas\set{ x: \tilde f(x) \neq f(x) \text{ или } \nabla \tilde f(x) \neq \nabla f(x) } < \eps.
$$
Более того, для некоторой константы, зависящей только от $n$, выполнено
$$
\sup\limits_{\Real^n} \abs{ \nabla \tilde f(x) } \le C Lip(f),
$$
где $Lip(f)$ --- константа липшицевости функции $f$.
\end{prop}

\begin{prop}[{\cite[лемма 2.7]{ASC}}]
\label{prop:conv_to_one}
Пусть $\phi_h: [-1, 1] \to \Real$ --- последовательность липшицевых функций, удовлетворяющих условиям:
$\phi_h' \ge 1$ для почти всех $x$ и всех $h$, $\phi_h( x ) \to x$ для почти каждого $x$.
Тогда для любой $f \in \Ls(\Real)$ выполнено $f(\phi_h) \to f$ в $\Ls(\Real)$.
\end{prop}

\begin{prop}[{\cite[теорема 6.19]{LiebLoss}}]
\label{prop:level_derivative}
Для любой $u \in \W[-1, 1]$ и произвольного множества $A \subset \Real$ нулевой меры выполнено
$u'(x) = 0$ для почти всех $x \in u^{-1}(A)$.
\end{prop}

\begin{prop}[{\cite[\S2.1]{Sharapudinov}}]
\label{prop:step_dense_orlicz}
Пусть $p(x)$ --- измеримая функция на отрезке $[-1, 1]$, удовлетворяющая
$1 \le p(x) \le \sup p(x) < \infty$.
Тогда ступенчатые функции плотны в пространстве Орлича $L^{p(x)}$.
\end{prop}
