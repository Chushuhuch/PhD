\section{Введение}
Определим множество $\mathfrak{F}$ непрерывных функций $F: \omega \times \Real_+ \times \Real_+ \to \Real_+$
(здесь и далее $\Real_+ = [0, \infty)$), выпуклых и возрастающих по третьему аргументу,
удовлетворяющих $F( \cdot, \cdot, 0 ) \equiv 0$.

Рассмотрим функционал:
\begin{equation}
\label{functional}
I( u ) = \int_\Omega F( x', u(x), \norm{ \mathcal{D} u } ) dx,
\end{equation}
где $F \in \mathfrak{F}$,
$\norm{\cdot}$ --- некоторая норма в $\Real^n$, симметричная по последней координате,
то есть удовлетворяющая $\norm{( x', y )} = \norm{( x', -y )}$,
$$\mathcal{D} u = ( a_1( x', u( x ) ) D_1 u, \dots, a_{n - 1}( x', u( x ) ) D_{n - 1} u, a( x, u( x ) ) D_n u )$$
--- градиент $u$ с весом (обратите внимание, что только вес при $D_n u$ зависит от $y$),
$a( \cdot, \cdot ): \Omega \times \Real_+ \to \Real_+$ и $a_i( \cdot, \cdot ): \omega \times \Real_+ \to \Real_+$ --- непрерывные функции.
Здесь и далее индекс $i$ пробегает от $1$ до $n - 1$.

Хорошо известно, что при $a_i = a \equiv 1$ справедливо неравенство
\begin{equation}
\label{toprove}
I( \symm{u} ) \le I( u ), \qquad \qquad u \in \W(\Omega), u \ge 0
\end{equation}
(см., например, \cite{Kawohl} и цитированную там литературу).

В настоящей работе мы приводим необходимые условия на вес $a$ для выполнения неравенства (\ref{toprove})
и доказываем неравенство (\ref{toprove}) при некоторых дополнительных ограничениях.

В работе \cite{Brock} рассматривается неравенство, аналогичное (\ref{toprove}).
А именно,
\begin{equation}
\label{breq}
I( \overline{u} ) \le I( u ), \qquad \qquad u \in \Wf( \Omega ), u \ge 0,
\end{equation}
где $\overline{u}$ --- симметризация по Штейнеру (симметричная перестановка) функции $u$ по переменной $y$.
Автор \cite{Brock} доказывает неравенство в случае, когда вес $a$ --- четная и выпуклая по $y$ функция.
Для этого неравенство (\ref{breq}) доказывается для кусочно линейных функций $u$
и утверждается, что в общем случае неравенство получается при помощи аппроксимации.
Однако предельный переход в \cite{Brock} должным образом не обоснован,
и можно считать, что результат получен лишь для липшицевых функций $u$.
Отметим, что в статье \cite{Brock} также только вес при $D_n u$ зависит от $y$, избавиться от этого ограничения не удается.

В одномерном случае неравенства (\ref{toprove}) и (\ref{breq}) рассматриваются в работе \cite{1dim},
где получены необходимые и достаточные условия их выполнения.
В частности, пробел в работе \cite{Brock} был закрыт для $n = 1$.
При некоторых дополнительных ограничениях этот результат был анонсирован в \cite{DAN}.

В статье \cite{Landes} рассматривается неравенство (\ref{toprove}) с аналогичным функционалом в двумерном случае
при дополнительном ограничении $u(\cdot, -1) \equiv 0$.
Отметим, что наши условия на весовые функции слабее, чем в \cite{Landes}.

Статья разделена на четыре параграфа.
В \S2 приведены условия, необходимые для выполнения неравенства (\ref{toprove}).
В \S3 приведено доказательство неравенства (\ref{toprove}) для кусочно линейных функций $u$.
\S4 содержит доказательство неравенства в более общих случаях.
