\section{Условия, необходимые для выполнения неравенства (\ref{toprove})}
\begin{thm}
\label{necessary_conditions_constant}
\textbf{1}. Let inequality (\ref{toprove}) hold for some $F \in \mathfrak {F}$
and an arbitrary piecewise linear $u$. Then the weight function $\mathfrak a$ is even with respect to the first argument,
that is $\mathfrak a(x, v) \equiv \mathfrak a(-x, v)$.

\textbf{2}. Let inequality (\ref{toprove}) hold for arbitrary $F \in \mathfrak{F}$
and arbitrary piecewise linear $u$. Then the weight function $\mathfrak a$ satisfies
\begin{equation}
\label{almostConcave}
\mathfrak a(s, v) + \mathfrak a(t, v) \ge \mathfrak a(1 - t + s, v), \qquad -1 \le s \le t \le 1, v \in \Real_+.
\end{equation}
\end{thm}

\begin{proof}
\textbf{1.} Suppose that $\mathfrak a(x, v) \not \equiv \mathfrak a(-x, v)$.
Then there is $\bar{x} \in (-1, 1 )$ and $\bar{v} \in \Real_+$ such that
$$\mathfrak a(\bar{x}, \bar{v}) < \mathfrak a(-\bar{x}, \bar{v}).$$
Therefore, there is $\eps> 0$ such that
$$\bar{x} - \eps \le x \le \bar{x}, \bar{v} \le v \le \bar{v} + \eps \quad \Longrightarrow \quad \mathfrak a(x, v) < \mathfrak a(-x, v).$$
Now we introduce the following function:
$$
\left\{
\begin{aligned}
u(x) &= \bar{v} + \eps, & x \in [-1,\bar{x}-\eps]\\
u(x) &= \bar{v} + \bar{x} - x, & x \in (\bar{x} - \eps, \bar{x})\\
u(x) &= \bar{v}, & x \in [\bar{x}, 1]
\end{aligned}
\right.
$$
Then $\mon{u}(x, v) = u(-x, v)$ and
\begin{multline*}
I(\mathfrak a, u)-I(\mathfrak a, \mon{u}) \\
= \int\limits_{\bar{x}-\eps}^{\bar{x}} F\big( \bar{v} + \bar{x} - x, \mathfrak a(x, \bar{v} + \bar{x} - x) \big) \, dx -
\int\limits_{-\bar{x}}^{-\bar{x}+\eps} F\big( \bar{v} + \bar{x} + x, \mathfrak a(x, \bar{v} + \bar{x} + x) \big) \, dx \\
= \int\limits_{\bar{x}-\eps}^{\bar{x}} \big( F\big( \bar{v} + \bar{x} - x, \mathfrak a(x, \bar{v} + \bar{x} - x) \big) -
F\big( \bar{v} + \bar{x} - x, \mathfrak a(-x, \bar{v} + \bar{x} - x) \big) \big) \, dx < 0,
\end{multline*}
which contradicts the assumption. Thus, the first statement is proved.

\textbf{2.} Suppose that assumption (\ref{almostConcave}) is not satisfied.
Then, by continuity of $\mathfrak a$, there exist $-1 \le s \le t \le 1$, $\eps, \delta> 0$ and $\bar{v} \in \Real_+$, such that
for any $0 \le y \le \eps$ and $\bar{v} \le v \le \bar{v} + \eps$ the following inequality holds:
$$\mathfrak a(s + y, v) + \mathfrak a(t - y, v) + \delta < \mathfrak a( 1 - t + s + 2y, v).$$

Consider the function $u$ (see Fig. \ref{uGraph}):
\begin{equation}
\label{parLinU}
\left\{
\begin{aligned}
u(x) &= \bar{v}, & x \in [-1, s] \cup [t, 1]\\
u(x) &= \bar{v} + x - s, & x \in [s, s + \eps]\\
u(x) &= \bar{v} + \eps, & x \in [s + \eps, t - \eps]\\
u(x) &= \bar{v} + t - x, & x \in [t - \eps, t]
\end{aligned}
\right.
\end{equation}

\begin{center}
\begin{picture}(200,90)
\refstepcounter{pictureCounter}
\label{uGraph}
\put(10,65){\line(1,0){50}}
\put(60,65){\line(1,1){10}}
\put(70,75){\line(1,0){40}}
\put(110,75){\line(1,-1){10}}
\put(120,65){\line(1,0){70}}
\put(0,25){\vector(1,0){200}}
\put(100,15){\vector(0,1){80}}
\put(99,65){\line(1,0){2}}
\put(92,62){$\bar{v}$}
\put(60,24){\line(0,1){2}}
\put(58,14){$s$}
\put(120,24){\line(0,1){2}}
\put(119,14){$t$}
\put(10,24){\line(0,1){2}}
\put(6,14){$-1$}
\put(190,24){\line(0,1){2}}
\put(188,14){$1$}
\put(20,70){$u(x)$}
\put(85,1){Fig. \arabic{pictureCounter}}
\end{picture}
\end{center}
Then
$$
\left\{
\begin{aligned}
\mon{u}(x) &= \bar{v}, & x \in [-1, 1 - t + s]\\
\mon{u}(x) &= \bar{v} + \frac{ x - ( 1 - t + s ) }{2}, & x \in [1 - t + s, 1 - t + s + 2\eps]\\
\mon{u}(x) &= \bar{v} + \eps, & x \in [1 - t + s + 2\eps, 1]
\end{aligned}
\right.
$$
(see Fig. \ref{uStarGraph}).

\begin{center}
\begin{picture}(200,90)
\refstepcounter{pictureCounter}
\label{uStarGraph}
\put(10,65){\line(1,0){120}}
\put(130,64){\line(2,1){20}}
\put(150,75){\line(1,0){40}}
\put(0,25){\vector(1,0){200}}
\put(100,15){\vector(0,1){80}}
\put(99,65){\line(1,0){2}}
\put(92,67){$\bar{v}$}
\put(130,24){\line(0,1){2}}
\put(110,14){$1 - t + s$}
\put(10,24){\line(0,1){2}}
\put(6,14){$-1$}
\put(190,24){\line(0,1){2}}
\put(188,14){$1$}
\put(20,70){$\mon{u}(x)$}
\put(85,1){Fig. \arabic{pictureCounter}}
\end{picture}
\end{center}

We have
\begin{multline*}
I(\mathfrak a, \mon{u}) = \int\limits_0^{2\eps} F \big( u(1 - t + s + z), \frac{\mathfrak a(1 - t + s + z, u(1 - t + s + z))}{2} \big) \, dz\\
= \int\limits_0^\eps 2 F \big(\bar{v} + y, \frac{\mathfrak a(1 - t + s + 2y, \bar{v} + y)}{2} \big) \, dy\\
0 \le I( \mathfrak a, u ) - I( \mathfrak a, \mon{u} ) =
\int\limits_0^\eps \big( F\big(\bar{v} + y, \mathfrak a(s + y, \bar{v} + y)\big) + F\big(\bar{v} + y, \mathfrak a( t - y, \bar{v} + y)\big)\\
- 2 F \big(\bar{v} + y, \frac{ \mathfrak a(1 - t + s + 2y, \bar{v} + y) }{2} \big) \big) \, dy\\
< \int\limits_0^\eps \big( F\big(\bar{v} + y, \mathfrak a(s + y, \bar{v} + y)\big) + F\big(\bar{v} + y, \mathfrak a(t - y, \bar{v} + y)\big)\\
- 2 F \big( \bar{v} + y, \frac{ \mathfrak a(s + y, \bar{v} + y) + \mathfrak a(t - y, \bar{v} + y) + \delta }{2} \big) \big) \, dy =: J.
\end{multline*}

Let us consider the function $F(v, p) = p ^ \alpha$.
For $\alpha = 1$, the following inequality trivially holds:
\begin{equation}
\label{anticonvex}
\frac{F(v, p) + F(v, q)}{ 2 } - F\big(v, \frac{p + q}{ 2 } + \frac{\delta}{ 2 }\big) <0.
\end{equation}
We are interested in $p, q$ from the compact $[0 , A]$,
where
\begin{equation}
\label{weightMax}
A=\max \limits_{(x, v)} \mathfrak a,\qquad (x, v) \in [-1, 1 ] \times u([-1, 1] ).
\end{equation}
Therefore, there is an $\alpha> 1$, for which inequality (\ref{anticonvex})
still holds.
For example, any $1 < \alpha < (\log_2 \frac{ 2 A}{A + \delta})^{-1}$ is suitable.

Thus, we obtain a function $F$ strictly convex with respect to the second argument
for which $J \le 0$. This contradiction proves the second statement.
\end{proof}

\begin{rem}
\label{landesNecessary}
It can be seen that, proving the second statement of Theorem \ref{necessary_conditions_constant},
one can replace the function $u$ on the interval $[-1, s]$ by any increasing function.
Thus, in the case where $u$ is pinned at the left end \textrm{($u(-1) = 0$)},
assumption (\ref{almostConcave}) is also necessary for inequality (\ref{toprove}) to hold.
\end{rem}

\begin{rem}
Let $\mathfrak a(\cdot, v)$ be even.
Then assumption (\ref{almostConcave}) is equivalent to subadditivity of the function $\mathfrak a(1 - \cdot, v)$.
In particular, if a nonnegative function $\mathfrak a$ is even and concave with respect to the first argument, then it satisfies assumption (\ref{almostConcave}).
\end{rem}
