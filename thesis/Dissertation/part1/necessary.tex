\section{Условия, необходимые для выполнения неравенства (\ref{toprove_weighted})}

\todo{add remark that we omit n-1 dimensions here}

\begin{thm}
\label{necessary_conditions_weighted}
\textbf{1.}
Если неравенство $(\ref{toprove_weighted})$ выполняется для некоторой $F \in \mathfrak{F}$
и произвольной кусочно линейной $u$,
то вес $a$ чётен по первому аргументу,
то есть $a(x, v) \equiv a(-x, v)$.

\textbf{2.}
Если неравенство $(\ref{toprove_weighted})$ выполняется для произвольной $F \in \mathfrak{F}$
и произвольной кусочно линейной $u$, то вес $a$ удовлетворяет неравенству
\begin{equation}
\label{almostConcave}
a(s, v) + a(t, v) \ge a(1 - t + s, v), \qquad -1 \le s \le t \le 1, v \in \Real_+.
\end{equation}
\end{thm}

\begin{proof}
\textbf{1.}
Предположим, что $a(x, v) \not\equiv a(-x, v)$.
Тогда найдутся такие $x_0 \in (-1, 1)$ и $v_0 \in \Real_+$, что
$$a(x_0, v_0) < a(-x_0, v_0).$$
Поэтому существует $\eps > 0$ такое, что
$$a(x, v) < a(-x, v), \qquad x_0 - \eps \le x \le x_0, v_0 \le v \le v_0 + \eps,$$
и можно взять следующую функцию:
$$
\left\{
\begin{aligned}
u_1(x) &= v_0 + \eps, & x \in [-1,x_0-\eps]\\
u_1(x) &= v_0 + x_0 - x, & x \in (x_0 - \eps, x_0)\\
u_1(x) &= v_0, & x \in [x_0, 1]
\end{aligned}
\right.
$$
Тогда $\mon{u_1}(x) = u_1(-x)$ и
\begin{multline*}
\IWg(a, u_1) - \IWg(a, \mon{u_1}) \\
= \int\limits_{x_0-\eps}^{x_0} F \bigl( v_0 + x_0 - x, a(x, v_0 + x_0 - x) \bigr) \, dx -
\int\limits_{-x_0}^{-x_0+\eps} F \bigl( v_0 + x_0 + x, a(x, v_0 + x_0 + x) \bigr) \, dx \\
= \int\limits_{x_0-\eps}^{x_0} \bigl( F \bigl( v_0 + x_0 - x, a(x, v_0 + x_0 - x) \bigr) -
F \bigl( v_0 + x_0 - x, a(-x, v_0 + x_0 - x) \bigr) \bigr) \, dx < 0,
\end{multline*}
что противоречит предположениям теоремы.
Утверждение \textbf{1} доказано.

\textbf{2.}
Предположим, что условие (\ref{almostConcave}) не выполняется.
Тогда в силу непрерывности функции $a$ найдутся такие $-1 \le s \le t \le 1$, $\eps, \delta> 0$ и $v_0 \in \Real_+$, что
для любых $0 \le y \le \eps$ и $v_0 \le v \le v_0 + \eps$ справедливо неравенство
$$a(s + y, v) + a(t - y, v) + \delta < a( 1 - t + s + 2y, v).$$

Рассмотрим функцию $u_2$ (см. рис. \ref{uGraph}):
\begin{equation}
\label{parLinU}
\left\{
\begin{aligned}
u_2(x) &= v_0, & x \in [-1, s] \cup [t, 1]\\
u_2(x) &= v_0 + x - s, & x \in [s, s + \eps]\\
u_2(x) &= v_0 + \eps, & x \in [s + \eps, t - \eps]\\
u_2(x) &= v_0 + t - x, & x \in [t - \eps, t]
\end{aligned}
\right.
\end{equation}

\begin{center}
\begin{picture}(200,90)
\refstepcounter{pictureCounter}
\label{uGraph}
\put(10,65){\line(1,0){50}}
\put(60,65){\line(1,1){10}}
\put(70,75){\line(1,0){40}}
\put(110,75){\line(1,-1){10}}
\put(120,65){\line(1,0){70}}
\put(0,25){\vector(1,0){200}}
\put(100,15){\vector(0,1){80}}
\put(99,65){\line(1,0){2}}
\put(85,62){$v_0$}
\put(60,24){\line(0,1){2}}
\put(58,14){$s$}
\put(120,24){\line(0,1){2}}
\put(119,14){$t$}
\put(10,24){\line(0,1){2}}
\put(6,14){$-1$}
\put(190,24){\line(0,1){2}}
\put(188,14){$1$}
\put(20,70){$u_2(x)$}
\put(85,1){рис. \arabic{pictureCounter}}
\end{picture}
\end{center}
Тогда
$$
\left\{
\begin{aligned}
\mon{u_2}(x) &= v_0, & x \in [-1, 1 - t + s]\\
\mon{u_2}(x) &= v_0 + \frac{ x - ( 1 - t + s ) }{2}, & x \in [1 - t + s, 1 - t + s + 2\eps]\\
\mon{u_2}(x) &= v_0 + \eps, & x \in [1 - t + s + 2\eps, 1]
\end{aligned}
\right.
$$
(см. рис. \ref{uStarGraph}).

\begin{center}
\begin{picture}(200,90)
\refstepcounter{pictureCounter}
\label{uStarGraph}
\put(10,65){\line(1,0){120}}
\put(130,64){\line(2,1){20}}
\put(150,75){\line(1,0){40}}
\put(0,25){\vector(1,0){200}}
\put(100,15){\vector(0,1){80}}
\put(99,65){\line(1,0){2}}
\put(85,67){$v_0$}
\put(130,24){\line(0,1){2}}
\put(110,14){$1 - t + s$}
\put(10,24){\line(0,1){2}}
\put(6,14){$-1$}
\put(190,24){\line(0,1){2}}
\put(188,14){$1$}
\put(20,70){$\mon{u_2}(x)$}
\put(85,1){рис. \arabic{pictureCounter}}
\end{picture}
\end{center}

Имеем
\begin{multline*}
\IWg(a, \mon{u_2}) = \int\limits_0^{2\eps} F \bigl( u_2(1 - t + s + z), \frac{a(1 - t + s + z, u_2(1 - t + s + z))}{2} \bigr) \, dz\\
= \int\limits_0^\eps 2 F \bigl(v_0 + y, \frac{a(1 - t + s + 2y, v_0 + y)}{2} \bigr) \, dy\\
0 \le \IWg( a, u_2 ) - \IWg( a, \mon{u_2} ) =
\int\limits_0^\eps \bigl( F \bigl( v_0 + y, a(s + y, v_0 + y) \bigr) + F \bigl( v_0 + y, a( t - y, v_0 + y) \bigr)\\
- 2 F \bigl( v_0 + y, \frac{ a(1 - t + s + 2y, v_0 + y) }{2} \bigr) \bigr) \, dy\\
< \int\limits_0^\eps \bigl( F \bigl( v_0 + y, a(s + y, v_0 + y) \bigr) + F \bigl( v_0 + y, a(t - y, v_0 + y) \bigr)\\
- 2 F \bigl( v_0 + y, \frac{ a(s + y, v_0 + y) + a(t - y, v_0 + y) + \delta }{2} \bigr) \bigr) \, dy =: \Delta\IWg.
\end{multline*}

Рассмотрим теперь функцию $F(v, p) = p^\alpha$.
Очевидно, что при $\alpha = 1$ выполнено неравенство
\begin{equation}
\label{anticonvex}
\frac{F(v, p) + F(v, q)}{ 2 } - F \bigl( v, \frac{p + q}{ 2 } + \frac{\delta}{ 2 } \bigr) < 0.
\end{equation}
Нас интересуют $p, q$, лежащие на компакте $[0, A]$, где
\begin{equation*}
%\label{weightMax}
A = \max \limits_{(x, v)} a(x, v), \qquad (x, v) \in [-1, 1 ] \times u_2([-1, 1] ).
\end{equation*}
Значит найдётся и $\alpha > 1$ такое, что неравенство (\ref{anticonvex}) будет выполняться.
Например, подходит любое $1 < \alpha < ( \log_2 \frac{2 A}{A + \delta} )^{-1}$.

Тем самым, мы подобрали строго выпуклую по второму аргументу функцию $F$, для которой $\Delta\IWg \le 0$.
Это противоречие доказывает утверждение \textbf{2}.
\end{proof}

\begin{rem}
Пусть $a(\cdot, v)$ чётна.
Тогда условие (\ref{almostConcave}) эквивалентно субаддитивности функции $a(1 - \cdot, v)$.
В частности, если неотрицательная функция $a$ чётна и вогнута по первому аргументу, она удовлетворяет (\ref{almostConcave}).
\end{rem}

\begin{thm}
\label{landesNecessaryRem}
Если неравенство $(\ref{toprove_weighted})$ выполняется для произвольной $F \in \mathfrak{F}$
и произвольной кусочно линейной $u$, закреплённой на левом конце: $u( -1 ) = 0$,
то вес $a$ удовлетворяет неравенству $(\ref{almostConcave})$.
\end{thm}

\begin{proof}
Будем следовать схеме доказательства пункта 2 теоремы \ref{necessary_conditions_weighted}.
Мы ставим дополнительное ограничение $s > -1$ (ввиду непрерывности весовой функции от этого требования легко избавиться).
Также в качестве функции $u$ берём функцию, возрастающую от нуля на отрезке $[-1, s]$,
а на отрезке $[s, 1]$ совпадающую с $u_2$ из теоремы \ref{necessary_conditions_weighted}.
Тогда функция $\mon{u}$ на отрезке $[-1, s]$ совпадает с $u$, а на отрезке $[s, 1]$ совпадает с $\mon{u_2}$.
Тем самым, значения $\IWg(u)$ и $\IWg(\mon{u})$ увеличиваются на одну и ту же величину,
и рассуждения, начиная с вычисления $\Delta \IWg$, полностью повторяются.
\end{proof}
