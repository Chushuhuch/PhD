\section{Условия, необходимые для выполнения неравенства (\ref{toprove_weighted})}
\begin{thm}
\label{necessary_conditions_constant}
\textbf{1.}
Если неравенство $(\ref{toprove_weighted})$ выполняется для некоторой $F \in \mathfrak{F}$
и произвольной кусочно линейной $u$,
то вес $a$ чётен по первому аргументу,
то есть $a(x, v) \equiv a(-x, v)$.

\textbf{2.}
Если неравенство $(\ref{toprove_weighted})$ выполняется для произвольной $F \in \mathfrak{F}$
и произвольной кусочно линейной $u$, то вес $a$ удовлетворяет неравенству
\begin{equation}
\label{almostConcave}
a(s, v) + a(t, v) \ge a(1 - t + s, v), \qquad -1 \le s \le t \le 1, v \in \Real_+.
\end{equation}
\end{thm}

\begin{proof}
\textbf{1.}
Предположим, что $a(x, v) \not\equiv a(-x, v)$.
Тогда найдутся такие $\bar{x} \in (-1, 1)$ и $\bar{v} \in \Real_+$, что
$$a(\bar{x}, \bar{v}) < a(-\bar{x}, \bar{v}).$$
Поэтому существует $\eps > 0$, такое, что
$$\bar{x} - \eps \le x \le \bar{x}, \bar{v} \le v \le \bar{v} + \eps \quad \Longrightarrow \quad a(x, v) < a(-x, v),$$
и можно взять следующую функцию:
$$
\left\{
\begin{aligned}
u(x) &= \bar{v} + \eps, & x \in [-1,\bar{x}-\eps]\\
u(x) &= \bar{v} + \bar{x} - x, & x \in (\bar{x} - \eps, \bar{x})\\
u(x) &= \bar{v}, & x \in [\bar{x}, 1]
\end{aligned}
\right.
$$
Тогда $\mon{u}(x, v) = u(-x, v)$ и
\begin{multline*}
\IWg(a, u) - \IWg(a, \mon{u}) \\
= \int\limits_{\bar{x}-\eps}^{\bar{x}} F \bigl( \bar{v} + \bar{x} - x, a(x, \bar{v} + \bar{x} - x) \bigr) \, dx -
\int\limits_{-\bar{x}}^{-\bar{x}+\eps} F \bigl( \bar{v} + \bar{x} + x, a(x, \bar{v} + \bar{x} + x) \bigr) \, dx \\
= \int\limits_{\bar{x}-\eps}^{\bar{x}} \bigl( F \bigl( \bar{v} + \bar{x} - x, a(x, \bar{v} + \bar{x} - x) \bigr) -
F \bigl( \bar{v} + \bar{x} - x, a(-x, \bar{v} + \bar{x} - x) \bigr) \bigr) \, dx < 0,
\end{multline*}
что противоречит условию.
Утверждение \textbf{1} доказано.

\textbf{2.}
Предположим, что условие (\ref{almostConcave}) не выполняется.
Тогда в силу непрерывности функции $a$ найдутся такие $-1 \le s \le t \le 1$, $\eps, \delta> 0$ и $\bar{v} \in \Real_+$, что
для любых $0 \le y \le \eps$ и $\bar{v} \le v \le \bar{v} + \eps$ справедливо неравенство
$$a(s + y, v) + a(t - y, v) + \delta < a( 1 - t + s + 2y, v).$$

Рассмотрим функцию $u$ (см. рис. \ref{uGraph}):
\begin{equation}
\label{parLinU}
\left\{
\begin{aligned}
u(x) &= \bar{v}, & x \in [-1, s] \cup [t, 1]\\
u(x) &= \bar{v} + x - s, & x \in [s, s + \eps]\\
u(x) &= \bar{v} + \eps, & x \in [s + \eps, t - \eps]\\
u(x) &= \bar{v} + t - x, & x \in [t - \eps, t]
\end{aligned}
\right.
\end{equation}

\begin{center}
\begin{picture}(200,90)
\refstepcounter{pictureCounter}
\label{uGraph}
\put(10,65){\line(1,0){50}}
\put(60,65){\line(1,1){10}}
\put(70,75){\line(1,0){40}}
\put(110,75){\line(1,-1){10}}
\put(120,65){\line(1,0){70}}
\put(0,25){\vector(1,0){200}}
\put(100,15){\vector(0,1){80}}
\put(99,65){\line(1,0){2}}
\put(92,62){$\bar{v}$}
\put(60,24){\line(0,1){2}}
\put(58,14){$s$}
\put(120,24){\line(0,1){2}}
\put(119,14){$t$}
\put(10,24){\line(0,1){2}}
\put(6,14){$-1$}
\put(190,24){\line(0,1){2}}
\put(188,14){$1$}
\put(20,70){$u(x)$}
\put(85,1){рис. \arabic{pictureCounter}}
\end{picture}
\end{center}
Тогда
$$
\left\{
\begin{aligned}
\mon{u}(x) &= \bar{v}, & x \in [-1, 1 - t + s]\\
\mon{u}(x) &= \bar{v} + \frac{ x - ( 1 - t + s ) }{2}, & x \in [1 - t + s, 1 - t + s + 2\eps]\\
\mon{u}(x) &= \bar{v} + \eps, & x \in [1 - t + s + 2\eps, 1]
\end{aligned}
\right.
$$
(см. рис. \ref{uStarGraph}).

\begin{center}
\begin{picture}(200,90)
\refstepcounter{pictureCounter}
\label{uStarGraph}
\put(10,65){\line(1,0){120}}
\put(130,64){\line(2,1){20}}
\put(150,75){\line(1,0){40}}
\put(0,25){\vector(1,0){200}}
\put(100,15){\vector(0,1){80}}
\put(99,65){\line(1,0){2}}
\put(92,67){$\bar{v}$}
\put(130,24){\line(0,1){2}}
\put(110,14){$1 - t + s$}
\put(10,24){\line(0,1){2}}
\put(6,14){$-1$}
\put(190,24){\line(0,1){2}}
\put(188,14){$1$}
\put(20,70){$\mon{u}(x)$}
\put(85,1){рис. \arabic{pictureCounter}}
\end{picture}
\end{center}

Имеем
\begin{multline*}
\IWg(a, \mon{u}) = \int\limits_0^{2\eps} F \bigl( u(1 - t + s + z), \frac{a(1 - t + s + z, u(1 - t + s + z))}{2} \bigr) \, dz\\
= \int\limits_0^\eps 2 F \bigl(\bar{v} + y, \frac{a(1 - t + s + 2y, \bar{v} + y)}{2} \bigr) \, dy\\
0 \le \IWg( a, u ) - \IWg( a, \mon{u} ) =
\int\limits_0^\eps \bigl( F \bigl( \bar{v} + y, a(s + y, \bar{v} + y) \bigr) + F \bigl( \bar{v} + y, a( t - y, \bar{v} + y) \bigr)\\
- 2 F \bigl( \bar{v} + y, \frac{ a(1 - t + s + 2y, \bar{v} + y) }{2} \bigr) \bigr) \, dy\\
< \int\limits_0^\eps \bigl( F \bigl( \bar{v} + y, a(s + y, \bar{v} + y) \bigr) + F \bigl( \bar{v} + y, a(t - y, \bar{v} + y) \bigr)\\
- 2 F \bigl( \bar{v} + y, \frac{ a(s + y, \bar{v} + y) + a(t - y, \bar{v} + y) + \delta }{2} \bigr) \bigr) \, dy =: \Delta\IWg.
\end{multline*}

Рассмотрим теперь функцию $F(v, p) = p^\alpha$.
Очевидно, что при $\alpha = 1$ выполнено неравенство
\begin{equation}
\label{anticonvex}
\frac{F(v, p) + F(v, q)}{ 2 } - F \bigl( v, \frac{p + q}{ 2 } + \frac{\delta}{ 2 } \bigr) < 0.
\end{equation}
Нас интересуют $p, q$, лежащие на компакте $[0, А]$, где
\begin{equation*}
%\label{weightMax}
A = \max \limits_{(x, v)} a, \qquad (x, v) \in [-1, 1 ] \times u([-1, 1] ).
\end{equation*}
Значит найдётся такое $\alpha > 1$, что неравенство (\ref{anticonvex}) будет продолжать выполняться.
Например, подходит любое $1 < \alpha < ( \log_2 \frac{2 A}{A + \delta} )^{-1}$.

Тем самым, мы подобрали строго выпуклую по второму аргументу функцию $F$, для которой $\Delta\IWg \le 0$.
Это противоречие доказывает утверждение \textbf{2}.
\end{proof}

\begin{rem}
\label{landesNecessaryRem}
Видно, что в доказательстве второго пункта теоремы
функцию $u$ на отрезке $[-1, s]$ можно заменить на любую возрастающую функцию.
Тем самым, условие $(\ref{almostConcave})$ является необходимым и для выполнения неравенства $(\ref{toprove_weighted})$
в случае закрепленных на левом конце функций: $u( -1 ) = 0$.
\end{rem}

\begin{rem}
Пусть $a(\cdot, v)$ чётна.
Тогда условие (\ref{almostConcave}) эквивалентно субаддитивности функции $a(1 - \cdot, v)$.
В частности, если неотрицательная функция $a$ чётна и вогнута по первому аргументу, она удовлетворяет (\ref{almostConcave}).
\end{rem}
