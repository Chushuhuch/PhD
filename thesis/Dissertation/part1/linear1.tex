\section{Результат на кусочно линейных функциях}

\todo{Если включать эту секцию, сверить с английским вариантом}

В этом параграфе мы докажем неравенство (\ref{toprove_weighted}) для кусочно линейных функций.
Не умаляя общности, будем считать, что $F(\cdot, 0) \equiv 0$.

\begin{theorem}
\label{linth}
Пусть функция $a$ четна и удовлетворяет условию $(\ref{almostConcave})$.
Тогда, если $u$ --- неотрицательная кусочно линейная функция, то $I(a, u) \ge I(a, \mon{u})$.
\end{theorem}

\begin{proof}
Положим $-1 = x_1 < x_2 < \dots < x_K = 1$ --- множество точек перелома функции $u$.
Рассмотрим множество $U$ значений функции $u$, не являющихся образами конечных точек
линейных участков: $U := u( [-1, 1] ) \setminus \{ u(x_1), \dots, u(x_K) \}$.
Множество $U$ представляет собой объединение конечного числа интервалов $U = \cup_{j = 1}^N G_j$.

Обозначим $m_j$ число прообразов значения $u_0 \in G_j$,
то есть число решений уравнения $u(y) = u_0$ 
(очевидно, это число постоянно для $u_0 \in G_j$).
Легко видеть, что эти прообразы являются линейными функциями $u_0$:
$y = y_k^j(u_0)$, $k = 1, \dots, m_j$,
и $y_k^j{}'(u(y)) = \frac{1}{u'(u)}$.
Мы будем считать, что $y_1^j(u_0) < y_2^j(u_0) < \dots < y_{m_j}^j(u_0)$.

Решение уравнения $\mon{u}(\mon{y})=u_0$ ($u_0 \in U$) можно выразить через $y_k^j$:

\begin{center}
\begin{tabular}{l|l|l} 
\multirow{2}{*}{$u(-1)<u_0$ \rule[-34pt]{0pt}{65pt}} & $m_j$ четно   & $\mon{y}=1-\sum\limits_{k=1}^{m_j} (-1)^k y_k^j$ \rule[-17pt]{0pt}{40pt} \\
                                                     & $m_j$ нечетно & $\mon{y}=-\sum\limits_{k=1}^{m_j} (-1)^k y_k^j$ \rule[-17pt]{0pt}{40pt} \\ \hline
\multirow{2}{*}{$u(-1)>u_0$ \rule[-34pt]{0pt}{65pt}} & $m_j$ четно   & $\mon{y}=-1+\sum\limits_{k=1}^{m_j} (-1)^k y_k^j$ \rule[-17pt]{0pt}{40pt} \\
                                                     & $m_j$ нечетно & $\mon{y}=\sum\limits_{k=1}^{m_j} (-1)^k y_k^j$ \rule[-17pt]{0pt}{40pt} \\
\end{tabular}
\end{center}

Положим $\mon{y}(v) = (\mon{u})^{-1}(v)$.
Тогда $\mon{y}'(v) = \sum_{k=1}^{m_j} \abs{y_k^j{}'(v)}$ при $u \in G_j$, поскольку знаки в выражении для
$\mon{y}$ и знаки $y_k^j{}'$ чередуются, и $\mon{y}'(v)\ge 0$.

Множества нулей $u'(x)$ и $\mon{u}'(x)$ могут иметь ненулевую меру.
Однако, они не вносят вклада в интеграл, поскольку $F(u(x), 0) = 0$.

Рассмотрим оставшиеся части интегралов:
\begin{multline*}
I(a, u) = \sum_{j=1}^N \int_{u^{-1}(G_j)} F(u(x), a(x, u(x)) \abs{u'(x)}) dx
\\ = \sum_{j=1}^N \int_{G_j} \sum_{k=1}^{m_j} F\Big(v, \frac{a(y_k^j(v), v)}{\bigabs{y_k^j{}'(v)}}\Big) \bigabs{y_k^j{}'(v)} dv,
\end{multline*}
\begin{multline*}
I(a, \mon{u}) = \sum_{j=1}^N \int_{(\mon{u})^{-1}(G_j)} F(\mon{u}(x), \bigabs{a(x, u(x)) \mon{u}'(x)}) dx
\\ = \sum_{j=1}^N \int_{G_j} F\Big(v, \frac{a(\mon{y}(v), v)}{\sum_{k=1}^{m_j} \bigabs{y_k^j{}'(v)}}\Big)
\sum_{k=1}^{m_j} \bigabs{ y_k^j{}'(v) } dv.
\end{multline*}

Зафиксируем $j$ и $v$ в правых частях и докажем неравенство для подынтегральных выражений.
Обозначим $b_k := |y_k^j{}'(v)|$, $y_k := y_k^j(v)$, $\mon{y} := \mon{y}(v)$, $m := m_j$. Тогда
требуемое утверждение имеет вид:
$$T:=\sum_{k=1}^m b_k F\Big( v, \frac{ a(y_k, v) }{b_k} \Big) \ge F\Big( v, \frac{ a(\mon{y}, v) }{ \sum_{k=1}^m b_k  } \Big) \sum_{k=1}^m b_k$$
С помощью неравенства Йенсена для функции $F(v, \cdot)$ получаем
$$T \ge F\Big( v, \frac{ \sum_{k=1}^m a(y_k, v) }{ \sum_{k=1}^m b_k } \Big) \sum_{k=1}^m b_k.$$
Тогда нам достаточно установить $\sum_{k=1}^m a(y_k, v) \ge a(\mon{y}, v)$, что верно с учетом леммы \ref{weightSum} и замечания
\ref{almostConcaveMultRem}.
\end{proof}

\begin{rem}
\label{landesLinear}
В работе $\cite{Landes}$ неравенство $(\ref{toprove_weighted})$ доказывается при дополнительном условии $u(-1) = 0$
для весовых функций $a$, убывающих по $x$.
Несложно видеть, что при этом условии доказательство теоремы $\ref{linth}$ проходит для весов, удовлетворяющих условию
$(\ref{almostConcave})$ без условия четности,
поскольку в этом случае $u(-1) < u_0$, и нам требуются только два неравенства из четырех,
которые и дает лемма $\ref{weightSum}$.
Очевидно также, что условие $(\ref{almostConcave})$ слабее, чем условие убывания $a$ по $x$.
\end{rem}
