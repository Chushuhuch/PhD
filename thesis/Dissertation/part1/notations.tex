\section{Обозначения}

Пусть $\Omega = \omega \times (-1, 1)$,
где $\omega$ --- ограниченная область в $\Real^{n - 1}$ с липшицевой границей.
Обозначим $x = ( x_1, \dots, x_{n - 1}, y ) = ( x', y )$.

Напомним теорему о послойном представлении измеримой неотрицательной функции $u$, заданной на $\overline{\Omega}$
(см. \cite[Теорема 1.13]{LiebLoss}).
Положим $\mathcal{A}_t(x') := \{ y \in [-1,1] :\ u( x', y ) > t \}$.
Тогда имеет место равенство
$$
u(x', y) = \int_0^\infty \charf{\mathcal{A}_t(x')}(y) dt,
$$
где $\charf{A}$ --- характеристическая функция множества $A$.

Определим симметричную перестановку измеримого множества $E \subset [-1, 1]$ и
симметричную перестановку (симметризацию по Штейнеру) неотрицательной функции $u \in \W(\overline{\Omega})$:
\begin{eqnarray*}
\symm{E} := [-\frac{\abs{E}}{2}, \frac{\abs{E}}{2}]; \qquad
\symm{u}(x', y) = \int\limits_0^\infty \charf{ \symm{( \mathcal{A}_t(x') )} }(y) dt.
\end{eqnarray*}

В тех же условиях определим монотонную перестановку множества $E$ и функции $u \in \W(\overline{\Omega})$:
\begin{eqnarray*}
\mon{E} := [1 - \meas E, 1]; \qquad
\mon{u}(x', y) = \int\limits_0^\infty \charf{ \mon{ \mathcal{A}_t(x') } }(y) dt.
\end{eqnarray*}

Определим множество $\mathfrak{F}$ непрерывных функций $F: \overline{\omega} \times \Real_+ \times \Real_+ \to \Real_+$
(здесь и далее $\Real_+ = [0, \infty)$),
выпуклых и строго возрастающих по третьему аргументу, удовлетворяющих $F( \cdot, \cdot, 0 ) \equiv 0$.

Рассмотрим функционал:
\begin{equation}
\IWg( u ) = \int\limits_\Omega F( x', u(x), \norm{ \mathcal{D} u } ) dx,
\end{equation}
где $F \in \mathfrak{F}$,
$\norm{\cdot}$ --- некоторая норма в $\Real^n$, симметричная по последней координате,
то есть удовлетворяющая $\norm{( x', y )} = \norm{( x', -y )}$,
$$\mathcal{D} u = ( a_1( x', u( x ) ) D_1 u,\ \dots,\ a_{n - 1}( x', u( x ) ) D_{n - 1} u,\ a( x, u( x ) ) D_n u )$$
--- градиент $u$ с весом (обратите внимание, что только вес при $D_n u$ зависит от $y$),
$a( \cdot, \cdot ): \overline{\Omega} \times \Real_+ \to \Real_+$ и $a_i( \cdot, \cdot ): \overline{\omega} \times \Real_+ \to \Real_+$ --- непрерывные функции.
Здесь и далее индекс $i$ пробегает от $1$ до $n - 1$.

В этой главе мы рассматриваем следующее неравенство:
\begin{equation}
\label{eq:to_prove_weighted}
\IWg( \mon{u} ) \le \IWg( u )
\end{equation}
Мы устанавливаем необходимые для выполнения неравенства условия на весовую функцию $a$.
Также мы доказываем неравенство при необходимых условиях и дополнительном ограничении на рост интегранта по производной.
