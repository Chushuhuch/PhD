\section{Доказательство неравенства (\ref{eq:to_prove_weighted}) для кусочно ли\-ней\-ных функций}

\begin{lm}
\label{lm:weight_sum}
Пусть $a$ удовлетворяет (\ref{eq:almostConcave}).

\textbf{\textup{i)}}
Для любых $x' \in \omega$, $-1 \le t_1 \le t_2 \le \ldots \le t_n \le 1$, $v \in \Real_+$ выполнены следующие неравенства
\begin{align*}
\sum_{k = 1}^n a(x', t_k, v) & \ge a( x', 1 - \sum_{k = 1}^n (-1)^k t_k, v ), & \text{ для чётных $n$}, & \\
\sum_{k = 1}^n a(x', t_k, v) & \ge a( x', - \sum_{k = 1}^n (-1)^k t_k, v ),   & \text{ для нечётных $n$}. &
\end{align*}

\textbf{\textup{ii)}}
Предположим дополнительно, что функция $a$ чётна.
Тогда для всех $x' \in \omega$, $-1 \le t_1 \le t_2 \le \ldots \le t_n \le 1$, $v \in \Real_+$ также выполнены следующие неравенства
\begin{align*}
\sum_{k = 1}^n a(t_k, v) & \ge a( x', -1 + \sum_{k = 1}^n (-1)^k t_k, v ), & \text{ для чётных $n$}, & \\
\sum_{k = 1}^n a(t_k, v) & \ge a( x', \sum_{k = 1}^n (-1)^k t_k, v ),      & \text{ для нечётных $n$}. &
\end{align*}
\end{lm}

\begin{proof}
\textbf{\textup{i)}}
Будем доказывать по индукции.
Для $n = 1$ утверждение тривиально.
Пусть теперь $n$ чётное.
Тогда, по предположению индукции,
$$
\sum_{k=1}^{n - 1} a(x', t_k, v) \ge a( x', -\sum_{k = 1}^{n - 1} (-1)^k t_k, v ).
$$
Значит
\begin{multline*}
\sum_{k = 1}^{n - 1} a( x', t_k, v ) + a( x', t_n, v )
\ge a( x', -\sum_{k = 1}^{n - 1} (-1)^k t_k, v ) + a( x', t_n, v )
\\ \ge a( x', 1 - \sum_{k = 1}^{n} (-1)^k t_k, v ).
\end{multline*}
В случае нечётного $n$ воспользуемся предположением индукции в следующем виде:
$$
\sum_{k=2}^n a(x', t_k, v) \ge a( x', 1 + \sum_{k = 2}^n (-1)^k t_k, v ).
$$
Тогда
\begin{multline*}
a( x', t_1, v ) + \sum_{k = 2}^n a( x', t_k, v )
\ge a( x', t_1, v ) + a( x', 1 + \sum_{k = 2}^{n} (-1)^k t_k, v )
\\ \ge a( x', t_1 - \sum_{k = 2}^{n} (-1)^k t_k, v ) = a( x', -\sum_{k = 1}^{n} (-1)^k t_k, v ).
\end{multline*}

\textbf{\textup{ii)}} Доказательство этой части очевидно.
\end{proof}

\begin{lm}
\label{lm:weighted_linear}
Пусть функция $a(x', \cdot, u)$ чётна и удовлетворяет условию $(\ref{eq:almostConcave})$.
Тогда, если $u$ --- неотрицательная кусочно линейная функция, то $\IWg( u ) \ge \IWg( \mon{u} )$.
\end{lm}

\begin{proof}
Пусть функция $u$ имеет изломы на множестве $C$ ($\partial \Omega \subset C \subset \Omega$).
Возьмём
$$
U := \{ ( x', u( x', y ) ): x' \in \omega, y \in (-1, 1), (x', y) \not\in C \}.
$$
Тогда открытое множество $U$ разбивается в объединение конечного числа связных открытых множеств $G_j$.
Обозначим $m_j$ число прообразов значения $( x', u_0 ) \in G_j$, то есть число решений уравнения $u( x', y ) = u_0$
(очевидно, это число постоянно для $( x', u_0 ) \in G_j$).
Легко видеть, что эти прообразы являются линейными функциями $( x', u_0 )$:
$y = y_k^j( x', u_0 )$, $k = 1, \dots, m_j$,
и $D_n y_k^j( x', u( x', y ) ) = \frac{1}{D_n u( x', y )}$.
Мы будем считать, что $y_1^j(x', u_0) < y_2^j(x', u_0) < \dots < y_{m_j}^j(x', u_0)$.

Уравнение $\mon{u}(x'_0, \mon{y}) = u_0$ задаёт $\mon{y}$ как функцию $( x'_0, u_0 ) \in G_j$.
Её можно выразить через $y_k^j$ (в частности, $\mon{y}$ кусочно линейна):

\begin{equation}
\label{eq:wlin_rearranged_expression}
\begin{tabular}{l|l|l}
\multirow{2}{*}{$u( x'_0, -1 ) < u_0$ \rule[-34pt]{0pt}{65pt}} & $m_j$ чётно   & $\mon{y} = 1-\sum\limits_{k=1}^{m_j} (-1)^k y_k^j$ \rule[-17pt]{0pt}{40pt} \\
                                                               & $m_j$ нечётно & $\mon{y} = -\sum\limits_{k=1}^{m_j} (-1)^k y_k^j$ \rule[-17pt]{0pt}{40pt} \\ \hline
\multirow{2}{*}{$u( x'_0, -1 ) > u_0$ \rule[-34pt]{0pt}{65pt}} & $m_j$ чётно   & $\mon{y} = -1+\sum\limits_{k=1}^{m_j} (-1)^k y_k^j$ \rule[-17pt]{0pt}{40pt} \\
                                                               & $m_j$ нечётно & $\mon{y} = \sum\limits_{k=1}^{m_j} (-1)^k y_k^j$ \rule[-17pt]{0pt}{40pt} \\
\end{tabular}
\end{equation}

Отсюда ясно, что
\begin{equation*}
D_n \mon{y}(x', u(x', y)) = \frac{1}{D_n \mon{u}(x',y)}=\sum_{k=1}^{m_j}{|D_n y_k^j (x',u(x',y))|}
\end{equation*}
и $D_i \mon{u}( x', y ) = \pm \sum_{k = 1}^{m_j} ( -1 )^k D_i y_k^j( x', u( x', y ) )$, где знак перед правой частью зависит только от $j$.

Тогда 
\begin{multline}
\label{eq:wlin_functional_of_u}
\IWg( u )=\sum_{j=1}^N {
    \int\limits_{G_j}{
        F( x', u(x), \norm{
            a_i(x',u(x)) D_i u(x), a(x,u(x)) D_n u(x)
        })
    dx}
}
\\ = \sum_{j=1}^N{
    \int\limits_{u(G_j)}{
        \sum_{k=1}^{m_j}{
            F \Bigl( x', u, \frac{
                \norm{
                    a_i(x',u) D_i y_k^j(x',u), a(x', y_k^j(x',u), u)
                }
            }{\abs{ D_n y_k^j (x',u) }} \Bigr)
        }
        }} {{\abs{ D_n y_k^j(x',u) }
    dx' du}
},
\end{multline}
\begin{multline}
\label{eq:wlin_functional_of_u_rearranged}
\IWg( \mon{u} )=
\sum_{j=1}^N{
    \int\limits_{G_j}{
        F(x', \mon{u}, \norm{
            a_i(x', \mon{u}(x)) D_i \mon{u}(x), a(x, \mon{u}(x)) D_n \mon{u}(x)
        })
    dx}
}
\\ = \sum_{j=1}^N {
    \int\limits_{u(G_j)}{
        F \Bigl( x', \mon{u}, \frac{
            \norm{
                a_i(x', \mon{u}) D_i \mon{y}(x', \mon{u}), a(x', \mon{y}(x', \mon{u}), \mon{u})
            }
        }{\sum_{k=1}^{m_j} \abs{ D_n y_k^j(x', \mon{u}) }} \Bigr) }} \times
        \\ \times {{ \sum_{k=1}^{m_j} \abs{ D_n y_k^j(x', \mon{u}) }
    dx' d\mon{u}}
}.
\end{multline}
Зафиксируем $j$, $x'$ и $u$ и обозначим
$b_k=\abs{ D_n y_k^j }$, $c_{ki}=D_i y_k^j$, $\mon{c}_i=D_i \mon{y}$, $y_k=y_k^j(x',u)$, $\mon{y} = \mon{y}(x', u)$, $m = m_j$.
Тогда справедлива следующая цепочка неравенств:

\begin{multline}
\label{eq:wlin_ineq_chain}
\sum_{k=1}^m{ b_k F \Bigl(\frac{ \norm{ a_ic_{ki}, a(y_k) } }{b_k} \Bigr) }
\overset{a}{\ge} F \Bigl( \frac{ \sum_{k=1}^m \norm{ a_i c_{ki}, a(y_k)} }{ \sum_{k=1}^m b_k } \Bigr) \sum_{k=1}^m b_k \\
\overset{b}{=}  F \Bigl( \frac{ \sum_{k=1}^m \norm{ ( -1 )^k a_i c_{ki}, a(y_k) } }{ \sum_{k=1}^m b_k} \Bigr) \sum_{k=1}^m b_k
\overset{c}{\ge}  F \Bigl( \frac{ \norm{ \sum_{k = 1}^m ( ( -1 )^k a_i c_{ki}, a( y_k ) ) } }{ \sum_{k=1}^m b_k} \Bigr) \sum_{k=1}^m b_k \\
= F \Bigl( \frac{ \norm{ \sum_{k = 1}^m ( -1 )^k a_i c_{ki}, \sum_{k = 1}^m a( y_k ) } }{ \sum_{k=1}^m b_k} \Bigr) \sum_{k=1}^m b_k
\overset{d}{\ge} F \Bigl( \frac{ \norm{ \sum_{k = 1}^m ( -1 )^k a_i c_{ki}, a( \mon{y} ) } }{ \sum_{k=1}^m b_k} \Bigr) \sum_{k=1}^m b_k \\
\overset{e}{=}   F \Bigl( \frac{ \norm{ \pm a_i \sum_{k = 1}^m ( -1 )^k c_{ki}, a( \mon{y} ) } }{ \sum_{k=1}^m b_k} \Bigr) \sum_{k=1}^m b_k
= F \Bigl( \frac{ \norm{ a_i \mon{c}_i, a(\mon{y}) } }{\sum_{k=1}^m b_k} \Bigr) \sum_{k=1}^m b_k.
\end{multline}
Здесь в переходе (a) применено неравенство Йенсена, в переходах (b) и (e) использована чётность нормы, в (c) использовано неравенство треугольника,
в (d) --- лемма \ref{lm:weight_sum} и чётность веса $a$ по $y$.

Из (\ref{eq:wlin_ineq_chain}) видно, что подынтегральное выражение в (\ref{eq:wlin_functional_of_u}) не меньше
подынтегрального выражения в (\ref{eq:wlin_functional_of_u_rearranged}).
Тем самым, доказательство завершено.
\end{proof}

\begin{lm}
Пусть функция $a(x', \cdot, u)$ удовлетворяет условию $(\ref{eq:almostConcave})$.
Тогда, если $u$ --- неотрицательная кусочно линейная функция, удовлетворяющая $u(\cdot, -1) \equiv 0$,
то $\IWg( u ) \ge \IWg( \mon{u} )$.
\end{lm}

\begin{proof}
Заметим, что в доказательстве леммы \ref{lm:weighted_linear}
мы используем чётность веса только в переходе (d) цепочки неравенств (\ref{eq:wlin_ineq_chain}).
Поскольку при $u(\cdot, -1) \equiv 0$ всегда выполнено $u(x'_0, -1) < u_0$,
с учётом соотношений (\ref{eq:wlin_rearranged_expression}) лемма \ref{lm:weight_sum} как раз обеспечивает требуемые для перехода (d) неравенства.
\end{proof}
