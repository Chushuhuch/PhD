\section{Extension of class of functions for which inequality (\ref{toprove}) holds}
The next statement is more or less standard.
However, the set $\{u: I(\mathfrak a, u) < \infty\}$ is not even convex subset of $\W(-1, 1)$.
Thus we give here a full proof for the reader's convenience.

\begin{lm}
Let the function $\mathfrak a$ be continuous. Then the functional $I(\mathfrak a, u)$ is weakly lower semicontinuous in $\W(-1, 1)$.
\label{lowersemi}
\end{lm}

\begin{proof}
Let $u_m \rightharpoondown u$ in $\W(-1, 1)$.
Let us denote $A = \varliminf I( \mathfrak a, u_m ) \ge 0$.
We are going to prove $I(\mathfrak a, u) \le A$.
In the case $A = \infty$ the assertion is trivial, so we can assume $A < \infty$.
Switching to a subsequence, we obtain $A = \lim I( \mathfrak a, u_m )$.

The weak convergence implies, that there exists
$R_0$ such that $\norm{ u_m }_{\W(-1, 1)} \le R_0$.
Moreover, switching to a subsequence, we can assume that $u_m \to u$ in $L_1(-1, 1)$
and $u_m(x) \to u(x)$ almost everywhere.
Then, by Egorov's theorem, for any $\eps$ there exists a set
$G_\eps^1$ such that $\abs{ G_\eps^1 } < \eps$ and $u_m \rightrightarrows u$ in $[-1, 1] \setminus G_\eps^1$.

The uniform convergence of $u_m$ implies that there exists $K$ such that for each $m>K$
the inequality $\abs{u_m} \le \abs{u} + \eps$ holds in $[-1, 1] \setminus G_\eps^1$.
Let $G_\eps^2 = \{x \in [-1, 1] \setminus G_\eps^1 : \abs{u(x)} \ge \frac{R_0 + \eps}{\eps} \}$.
Then
$$R_0 \ge \int\limits_{-1}^1 \abs{ u(x) } \, dx \ge \int\limits_{G_\eps^2} \abs{ u(x) } \, dx \ge
\int\limits_{G_\eps^2} \frac{R_0 + \eps}{\eps} \, dx = \abs{G_\eps^2} \frac{R_0 + \eps}{\eps}$$
That is, $\abs{G_\eps^2} \le \eps \frac{R_0}{R_0 + \eps} < \eps$.
Thus, the functions $u_m$ converge uniformly and are uniformly bounded outside the set $G_\eps := G_\eps^1 \cup G_\eps^2$.

The continuity of $F$ and $\mathfrak a$ implies that for any $\eps$ and $R$, there exists
$N( \eps, R )$, such that if $x \in [-1, 1] \setminus G_\eps$, $\abs{ M } \le R$ and $m > N( \eps, R )$, then
$$| F\big( u_m( x ), \mathfrak a( x, u_m( x ) ) M \big) - F\big( u( x ), \mathfrak a( x, u( x ) ) M \big) | < \eps.$$

Let $E_{m,\eps} := \{ x \in [-1, 1]: \abs{ u_m'( x ) } \ge \frac{ R_0 }{ \eps } \}$.
Then
$$R_0 \ge \int\limits_{-1}^1 \abs{ u_m'( x ) } \, dx \ge \int\limits_{ E_{m,\eps} } \abs{ u_m'( x ) } \, dx \ge
\int\limits_{ E_{m,\eps} } \frac{ R_0 }{ \eps } \, dx = \frac{ R_0 }{ \eps } \abs{ E_{m,\eps} }.$$
Therefore $\abs{ E_{m,\eps} } \le \eps$.

Finally we set $L_{m,\eps} := [-1, 1] \setminus ( E_{m,\eps} \cup G_\eps )$.
Note, that $\abs{ L_{m,\eps} } \ge 2 - 3 \eps$.

We put $R := \frac{ R_0 }{ \eps }$, $N( \eps ) := N( \eps, \frac{ R_0 }{ \eps } )$.
For any $\eps > 0$, $x \in L_{m,\eps}$ and $m > N( \eps )$ we have
$$\Big | F\big( u_m( x ), \mathfrak a( x, u_m( x ) ) \abs{u_m'( x )} \big) - F\big( u( x ), \mathfrak a( x, u( x ) ) \abs{u_m'( x )} \big) \Big | < \eps,$$
thus
\begin{equation}
\label{frDer}
\int\limits_{L_{m,\eps}} \Big | F\big( u_m( x ), \mathfrak a( x, u_m( x ) ) \abs{u_m'( x )} \big) - F\big( u( x ), \mathfrak a( x, u( x ) ) \abs{u_m'( x )} \big) \Big | \, dx < 2 \eps.
\end{equation}

We put $\eps_j = \frac{ \eps }{ 2^j }$ ($j \ge 1$), $m_j = N( \eps_j ) + j \to \infty$ and $L_\eps = \bigcap L_{m_j,\eps_j}$.
Then $\sum \eps_j = \eps$ and therefore $\abs{ [-1, 1] \setminus L_\eps } < 3 \eps$.
Since (\ref{frDer}) implies
$$\int\limits_{L_\eps} \Big | F\big( u_{m_j}( x ), \mathfrak a( x, u_{m_j}( x ) ) |u_{m_j}'( x )| \big) - F\big( u( x ), \mathfrak a( x, u( x ) ) |u_{m_j}'( x )| \big) \Big | \, dx < 2 \eps_j,$$
we obtain
\begin{multline*}
A = \lim I (\mathfrak a, u_{m_j}) = \lim \int\limits_{-1}^1 F\big(u_{m_j}(x), \mathfrak a(x, u_{m_j}(x)) | u_{m_j }'(x) |\big) \, dx \\
\ge \varliminf \int\limits_{-1}^1 \chi_{L_\eps}(x) F\big(u (x), \mathfrak a(x, u(x)) | u_{m_j}'(x) |\big) \, dx
=: \varliminf J_\eps(u_{m_j}').
\end{multline*}

The functional
$$J_\eps( v ) = \int\limits_{-1}^1 \chi_{L_\eps}( x ) F\big( u( x ), \mathfrak a( x, u( x ) ) |v( x )| \big) \, dx$$
is convex.
Switching to a subsequence $u_k$ again, we can assume that
$\varliminf J_\eps( u_{m_j}' ) = \lim J_\eps( u_k' )$.
Since $u_k' \rightharpoondown u'$ in $L_1$, we can choose a sequence of convex combinations of $u_k'$,
which converges to $u'$ strongly (see \cite[Theorem 3.13]{Rudin}).
Namely, there are $\alpha_{k,l} \ge 0$ for
$k \in \Nat$, $l \le k$, such that $\sum_{l = 1}^k \alpha_{k,l} = 1$ for every $k$ and
$w_k := \sum_{l = 1}^k \alpha_{k,l} u_{l}' \to u'$ in $L_1$.
Also, without loss of generality we can assume that the minimal index $l$ of a nonzero coefficient $\alpha_{k,l}$
tends to infinity as $k$ tends to infinity.
Then
$$\lim J_\eps( u_k' ) = \lim \sum_{l = 1}^k \alpha_{k,l} J_\eps( u_{l}' ).$$

By the convexity of $J_\eps$, we have
$$\sum_{l = 1}^k \alpha_{k,l} J_\eps( u_{l}' ) \ge J_\eps( w_k ).$$

Finally, since $w_k \to u'$ in $L_1(-1, 1)$, we can assume, by switching to a subsequence, that $w_k(x) \to u'(x)$ almost everywhere.
Moreover, since $\abs{ u_j'( x ) } < \frac{ R_0 }{\eps}$ holds for $x \in L_\eps$, then $\abs{ w_k( x ) } < \frac{ R_0 }{\eps}$.
Hence,
$$F\big( u( x ), \mathfrak a( x, u( x ) ) |w_k( x )| \big) \le \max\limits_{(x, M)} F\big( u( x ), \mathfrak a( x, u( x ) ) M \big) < \infty,$$
where the maximum is taken over a compact set
$(x,M) \in [-1, 1] \times [-\frac{ R_0 }{\eps},\frac{ R_0 }{\eps}]$.
Therefore, by the Lebesgue theorem, $\lim J_\eps(w_k) = J_\eps(u')$.
Thus,
$$A \ge \lim J_\eps( u_k' ) = \lim \sum_{l = 1}^k \alpha_{k,l} J_\eps( u_{l}' ) \ge
\varliminf J_\eps( w_k ) = J_\eps( u' ).$$

Since $\eps > 0$ is arbitrary, $A \ge I(\mathfrak a, u)$ follows.
\end{proof}

\begin{lm}
\label{uplift}
Пусть $A \subset W^1_1(\Omega)$.
И пусть $B \subset A$ таково, что $\forall v \in B$ выполнено $I( \mon{v} ) \le I( v )$.
Предположим, что для каждого $u \in A$ найдется последовательность $u_k \in B$ такая,
что $u_k \to u$ в $W^1_1(\Omega)$ и $I( u_k ) \to I( u )$.
Тогда $\forall u \in A$ будет выполнено $I( \mon{u} ) \le I( u )$.
\end{lm}
\begin{proof}
Let us pick some $u \in A$ and find an approximating sequence $\{u_k\} \subset B$.
By hypothesis, $I(\mathfrak a, \mon{u_k}) \le I(\mathfrak a, u_k) \to I(\mathfrak a, u)$.
By \cite[Theorem 1]{Brock}
$$u_k \to u \text{ in } \W(-1, 1) \quad \Longrightarrow \quad \overline{u_k} \rightharpoondown \overline{u} \text{ in } \W(-1, 1).$$
Since $\mon{u_k}( x ) = \overline{u_k}( \frac{x - 1}{2} )$ and $\mon{u}( x ) = \overline{u}( \frac{x - 1}{2} )$,
we have $\mon{u_k} \rightharpoondown \mon{u}$ in $\W(-1, 1)$.
By Lemma \ref{lowersemi}, we obtain 
$$I(\mathfrak a, \mon{u}) \le \liminf I(\mathfrak a, \mon{u_k}) \le \lim I(\mathfrak a, u_k) = I(\mathfrak a, u).$$
\end{proof}

\begin{cor}
Let the weight $\mathfrak a$ be continuous, and let inequality $(\ref{toprove})$ hold for nonnegative piecewise linear functions $u$.
Then it holds for all nonnegative Lipschitz functions.
\end{cor}
\begin{proof}
By Theorem 1 in Section 6.6 \cite{Gariepy}, any Lipschitz function $u$ can be approximated by $u_k \in C^1[-1, 1]$ such that
$$u_k \rightrightarrows u, \qquad u_k' \to u' \text{ a.e.}, \qquad |u_k'| \le const.$$
Relation (\ref{convergence}) holds by the Lebesgue theorem.
In turn, $u_k$ can be approximated in the same way by piecewise linear functions.
Using Theorem \ref{linth} and applying Lemma \ref{uplift}, we complete the proof.
\end{proof}

