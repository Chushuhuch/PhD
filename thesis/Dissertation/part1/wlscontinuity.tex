\section{О расширении класса функций, для которых выполняется неравенство (\ref{eq:to_prove_weighted})}
\label{sec:wlsc}

Следующее утверждение более или менее стандартно и близко к теореме Тонелли (предложение \ref{prop:wlsc}).
Однако, в нашем случае множество $\{u: \IWg(u) < \infty\}$ даже не является выпуклым подмножеством $\W(\overline{\Omega})$.
Поэтому здесь мы приводим полное доказательство для удобства читателя.

\begin{lm}
\label{lm:lowersemi}
Пусть функция $a$ непрерывна.
Тогда функционал $\IWg(u)$ слабо полунепрерывен снизу в $\W(\overline{\Omega})$.
\end{lm}

\begin{proof}
Пусть $u_m \rightharpoondown u$ в $\W(\overline{\Omega})$.
Обозначим $\IWg_{lim} = \varliminf \IWg(u_m) \ge 0$.
Наша задача --- доказать $\IWg(u) \le \IWg_{lim}$.
Если $\IWg_{lim} = \infty$, то утверждение тривиально, поэтому можно считать $\IWg_{lim} < \infty$.
Переходя к подпоследовательности, добиваемся $\IWg_{lim} = \lim \IWg(u_m)$.
Из слабой сходимости $u_m \rightharpoondown u$ заключаем, что найдётся $R_0$ такое, что $\norm{ u_m }_{\W(\overline{\Omega})} \le R_0$.
Более того, переходя к подпоследовательности, можно считать,
что $u_m \to u$ в $\Ls(\overline{\Omega})$ и $u_m(x) \to u(x)$ почти всюду.
Тогда по теореме Егорова для любого $\eps$ найдётся множество
$G_\eps^1$ такое, что $\meas G_\eps^1 < \eps$ и $u_m \rightrightarrows u$ в $\Omega \setminus G_\eps^1$.

Из равномерной сходимости $u_m$ следует существование такого $K$, что для каждого $m > K$
неравенство $\abs{u_m} \le \abs{u} + \eps$ выполнено для аргументов из $\Omega \setminus G_\eps^1$.
Возьмём $G_\eps^2 = \{x \in \Omega \setminus G_\eps^1 : \abs{u(x)} \ge \frac{R_0 + \eps}{\eps} \}$.
Тогда
$$
R_0 \ge \int\limits_{\Omega} \abs{ u(x) } \, dx \ge \int\limits_{G_\eps^2} \abs{ u(x) } \, dx \ge
\int\limits_{G_\eps^2} \frac{R_0 + \eps}{\eps} \, dx = \meas G_\eps^2 \cdot \frac{R_0 + \eps}{\eps}.
$$
То есть $\meas G_\eps^2 \le \eps \frac{R_0}{R_0 + \eps} < \eps$.
Тем самым, последовательность $u_m$ равномерно сходится и равномерно ограничена вне множества $G_\eps := G_\eps^1 \cup G_\eps^2$.

Из непрерывности $F$, $a_i$ и $a$ следует, что для произвольных $\eps$ и $R$ найдётся такое
$N( \eps, R )$, что если $x \in \Omega \setminus G_\eps$, $\abs{ M_i } \le R$, $\abs{ M } \le R$ и $m > N( \eps, R )$, то
\begin{multline*}
| F \bigl( x', u_m( x ), \norm{ a_i( x', u_m(x) ) M_i, a( x, u_m(x) ) M } \bigr)
\\ - F \bigl( x', u( x ), \norm{ a_i( x', u(x) ) M_i, a( x, u( x ) ) M } \bigr) | < \eps.
\end{multline*}

Рассмотрим множества
$$
E_{m,\eps} := \{ x \in \Omega: \max \set{ \abs{ D_i u_m(x) }, \abs{ D_n u_m(x) } } \ge \frac{ R_0 }{ \eps } \}.
$$
Имеем
$$
R_0
\ge \int\limits_{\Omega} \abs{ \nabla u_m( x ) } \, dx
\ge \int\limits_{ E_{m,\eps} } \abs{ \nabla u_m( x ) } \, dx
\ge \int\limits_{ E_{m,\eps} } \frac{ R_0 }{ \eps } \, dx
= \meas E_{m,\eps} \cdot \frac{ R_0 }{ \eps }.
$$
Поэтому $\meas E_{m,\eps} \le \eps$.

Теперь можно ввести $L_{m,\eps} := \Omega \setminus ( E_{m,\eps} \cup G_\eps )$.
Тогда
$$
\meas L_{m,\eps} \ge \meas \Omega - 3 \eps.
$$

Зафиксируем $R := \frac{ R_0 }{ \eps }$, $N( \eps ) := N( \eps, \frac{ R_0 }{ \eps } )$.
Для любых $\eps > 0$, $x \in L_{m,\eps}$ и $m > N( \eps )$ получим
\begin{multline*}
\Bigabs{
    F \bigl( x', u_m( x ), \norm{ a_i( x', u_m(x) ) D_i u_m(x), a( x, u_m(x) ) D_n u_m(x) } \bigr)
    \\ - F \bigl( x', u( x ), \norm{ a_i( x', u(x) ) D_i u_m(x), a( x, u( x ) ) D_n u_m(x) } \bigr)
} < \eps,
\end{multline*}
откуда
\begin{multline}
\label{eq:wls_appr_diff}
\int\limits_{L_{m,\eps}} \Bigabs{
    F \bigl( x', u_m( x ), \norm{ a_i( x', u_m(x) ) D_i u_m(x), a( x, u_m(x) ) D_n u_m(x) } \bigr)
    \\ - F \bigl( x', u( x ), \norm{ a_i( x', u(x) ) D_i u_m(x), a( x, u( x ) ) D_n u_m(x) } \bigr)
} \, dx < \meas \Omega \cdot \eps.
\end{multline}

Возьмём $\eps_j = \frac{ \eps }{ 2^j }$ ($j \ge 1$), $m_j = N( \eps_j ) + j \to \infty$ и $L_\eps = \bigcap L_{m_j,\eps_j}$.
Тогда $\sum \eps_j = \eps$ и, тем самым, $\meas (\Omega \setminus L_\eps) < 3 \eps$.
Поскольку из (\ref{eq:wls_appr_diff}) следует
\begin{multline*}
\int\limits_{L_\eps} \Bigabs{
    F \bigl( x', u_m( x ), \norm{ a_i( x', u_m(x) ) D_i u_m(x), a( x, u_m(x) ) D_n u_m(x) } \bigr)
    \\ - F \bigl( x', u( x ), \norm{ a_i( x', u(x) ) D_i u_m(x), a( x, u( x ) ) D_n u_m(x) } \bigr)
} \, dx < \meas \Omega \cdot \eps_j.
\end{multline*}
мы получаем
\begin{multline*}
\IWg_{lim}
= \lim \IWg (u_{m_j})
\\ = \lim \int\limits_{\Omega}
    F \bigl( x', u_{m_j}(x), \norm{ a_i(x', u_{m_j}(x) D_i u_{m_j}, a(x, u_{m_j}(x)) D_n u_{m_j }(x) } \bigr) \, dx
\\ \ge \varliminf \int\limits_{\Omega} \charf{L_\eps}(x)
    F \bigl( x', u(x), \norm{ a_i(x', u(x)) D_i u_{m_j}(x), a(x, u(x)) D_n u_{m_j}(x) } \bigr) \, dx
\\ =: \varliminf \IWg_\eps( \nabla u_{m_j} ).
\end{multline*}

Наш новый функционал
$$
\IWg_\eps( v ) = \int\limits_{\Omega} \charf{L_\eps}( x )
    F \bigl( x', u( x ), \norm{ a_i( x', u(x) ) v_i(x), a( x, u(x) ) v_n( x ) } \bigr) \, dx
$$
выпуклый.
Вновь переходя к подпоследовательности $u_k$, можно считать, что
$\varliminf \IWg_\eps( u_{m_j}' ) = \lim \IWg_\eps( u_k' )$.
Так как $\nabla u_k' \rightharpoondown\nabla  u'$ в $\Ls$,
по предложению \ref{prop:convex_combination_convergence} можно подобрать последовательность выпуклых комбинаций $\nabla u_k$,
которые будут сходиться к $\nabla u$ сильно.
А именно: найдутся $\alpha_{k,l} \ge 0$ для
$k \in \Nat$, $l \le k$ такие, что $\sum_{l = 1}^k \alpha_{k,l} = 1$ для каждого $k$ и
$w_k := \sum_{l = 1}^k \alpha_{k,l} \nabla u_{l} \to \nabla u$ в $\Ls$.
Кроме того, очевидно, можно потребовать, чтобы минимальный индекс $l$ ненулевого коэффициента $\alpha_{k,l}$
стремился к бесконечности при $k \to +\infty$.
Тогда
$$
\lim \IWg_\eps( \nabla u_k ) = \lim \sum_{l = 1}^k \alpha_{k,l} \IWg_\eps( \nabla u_{l} ).
$$

В силу выпуклости $\IWg_\eps$ имеем
$$\sum_{l = 1}^k \alpha_{k,l} \IWg_\eps( \nabla u_{l} ) \ge \IWg_\eps( w_k ).$$

Наконец, поскольку $(w_k)_i \to D_i u$ и $(w_k)_n \to D_n u$ в $\Ls(\overline{\Omega})$,
переходя к подпоследовательности, можем считать, что $w_k(x) \to \nabla u(x)$ почти всюду.
Кроме того, так как для  $x \in L_\eps$ выполнено $\abs{ D_i u_j( x ) } < \frac{ R_0 }{\eps}$ и $\abs{ D_n u_j( x ) } < \frac{ R_0 }{\eps}$,
то $\abs{ (w_k)_i( x ) } < \frac{ R_0 }{\eps}$ и $\abs{ (w_k)_n( x ) } < \frac{ R_0 }{\eps}$.
Значит,
\begin{multline*}
\charf{L_\eps}(x)
    F \bigl( x', u( x ), \norm{ a_i( x', u(x) ) (w_k)_i(x), a( x, u( x ) ) (w_k)_n(x) } \bigr)
\\ \le \max\limits_{(x, M_i, M)} \charf{L_\eps}(x)
    F \bigl( x', u( x ), \norm{ a_i( x', u(x) ) M_i, a( x, u(x) ) M } \bigr) < \infty,
\end{multline*}
где максимум берется по компактному множеству
$( x, M_i, M ) \in \overline{\Omega} \times [-\frac{ R_0 }{\eps},\frac{ R_0 }{\eps}]^n$.
Поэтому применима теорема Лебега, и мы получаем $\lim \IWg_\eps(w_k) = \IWg_\eps( \nabla u )$.
Таким образом,
$$
\IWg_{lim} \ge \lim \IWg_\eps( \nabla u_k ) = \lim \sum_{l = 1}^k \alpha_{k,l} \IWg_\eps( \nabla u_{l} )
\ge \varliminf \IWg_\eps( w_k ) = \IWg_\eps( \nabla u ).$$

Ввиду произвольности $\eps > 0$ имеем $\IWg_{lim} \ge \IWg(u)$.
\end{proof}

\begin{thm}
\label{thm:uplift}
Пусть $B \subset A \subset \W(\overline{\Omega})$.
Предположим, что для каждого $u \in A$ найдётся последовательность $u_k \in B$ такая,
что $u_k \to u$ в $\W(\overline{\Omega})$ и $\IWg( u_k ) \to \IWg( u )$.
Тогда

\textbf{\textup{i)}}
Если для любой функции $v \in B$ выполнено $\IWg( \symm{v} ) \le \IWg( v )$,
то для любой функции $u \in A$ будет выполнено $\IWg( \symm{u} ) \le \IWg( u )$.

\textbf{\textup{ii)}}
Если для любой функции $v \in B$ выполнено $\IWg( \mon{v} ) \le \IWg( v )$,
то для любой функции $u \in A$ будет выполнено $\IWg( \mon{u} ) \le \IWg( u )$.
\end{thm}

\begin{proof}
\textbf{\textup{i)}}
Возьмём некоторую $u \in A$ и для неё найдём приближающую последовательность $\set{u_k} \subset B$.
По условию $\IWg(\symm{u_k}) \le \IWg(u_k) \to \IWg(u)$.
По предложению \ref{prop:rearr_weak_conv} найдётся подпоследовательность $u_{k_l}$, для которой
$$
\symm{u_{k_l}} \rightharpoondown \symm{u} \text{ in } \W(\overline{\Omega}).
$$
Из леммы \ref{lm:lowersemi} получаем
$$
\IWg(\mon{u}) \le \varliminf \IWg(\symm{u_{k_l}}) \le \lim \IWg(u_{k_l}) = \IWg(u).
$$

\textbf{\textup{ii)}}
Поскольку $\mon{u_k}( x ) = \symm{u_k}( \frac{x - 1}{2} )$ и $\mon{u}( x ) = \symm{u}( \frac{x - 1}{2} )$,
из сходимости $u_k \to u$ в $\W(\overline{\Omega})$ также следует сходимость $\mon{u_{k_l}} \rightharpoondown \mon{u}$ в $\W(\overline{\Omega})$
для некоторой подпоследовательности $u_{k_l}$.
Тем самым, рассуждения из доказательства предыдущего пункта могут быть дословно повторены.
\end{proof}
