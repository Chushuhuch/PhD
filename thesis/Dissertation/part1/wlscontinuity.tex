\section{О расширении класса функций, для которых выполняется неравенство (\ref{toprove_weighted})}

\todo{проверить ещё раз, что всё ок в многомерном случае}

\todo{кое-где вместо 2 вылезет $\abs{\Omega}$}

Следующее утверждение более-менее стандартно.
Однако, множество $\{u: \IWg(u) < \infty\}$ даже не является выпуклым подмножеством $\W(\Omega)$.
Поэтому здесь мы приводим полное доказательство для удобства читателя.
\todo{это ок фраза для диссертации вообще?}

\begin{lm}
Пусть функция $a$ непрерывна.
Тогда функционал $\IWg(u)$ слабо полунепрерывен снизу в $\W(\Omega)$.
\label{lowersemi}
\end{lm}

\begin{proof}
Пусть $u_m \rightharpoondown u$ в $\W(\Omega)$.
Обозначим $A = \varliminf \IWg(u_m) \ge 0$.
Наша задача --- доказать $\IWg(u) \le A$.
Если $A = \infty$, то утверждение тривиально, поэтому можно считать $A < \infty$.
Переходя к подпоследовательности, добиваемся $A = \lim \IWg(u_m)$.
Из слабой сходимости $u_m \rightharpoondown u$ заключаем, что найдётся $R_0$ такое, что $\norm{ u_m }_{\W(\Omega)} \le R_0$.
Более того, переходя к подпроследовательности, можно считать,
что $u_m \to u$ в $L_1(\Omega)$ и $u_m(x) \to u(x)$ почти всюду.
Тогда по теореме Егорова для любого $\eps$ найдётся множество
$G_\eps^1$ такое, что $\abs{ G_\eps^1 } < \eps$ и $u_m \rightrightarrows u$ в $\Omega \setminus G_\eps^1$.

Из равномерной сходимости $u_m$ следует существование такого $K$, что для каждого $m > K$
неравенство $\abs{u_m} \le \abs{u} + \eps$ выполнено для аргументов из $\Omega \setminus G_\eps^1$.
Возьмём $G_\eps^2 = \{x \in \Omega \setminus G_\eps^1 : \abs{u(x)} \ge \frac{R_0 + \eps}{\eps} \}$.
Тогда
$$
R_0 \ge \int\limits_{\Omega} \abs{ u(x) } \, dx \ge \int\limits_{G_\eps^2} \abs{ u(x) } \, dx \ge
\int\limits_{G_\eps^2} \frac{R_0 + \eps}{\eps} \, dx = \abs{G_\eps^2} \frac{R_0 + \eps}{\eps}.
$$
То есть $\abs{G_\eps^2} \le \eps \frac{R_0}{R_0 + \eps} < \eps$.
Тем самым, последовательность $u_m$ равномерно сходится и равномерно ограничена вне множества $G_\eps := G_\eps^1 \cup G_\eps^2$.

Из непрерывности $F$ и $a$ следует, что для произвольных $\eps$ и $R$ найдётся такое
$N( \eps, R )$, что если $x \in \Omega \setminus G_\eps$, $\abs{ M } \le R$ и $m > N( \eps, R )$, то
$$
| F \bigl( u_m( x ), a( x, u_m( x ) ) M \bigr) - F \bigl( u( x ), a( x, u( x ) ) M \bigr) | < \eps.
$$

Рассмотрим множества $E_{m,\eps} := \{ x \in \Omega: \abs{ u_m'( x ) } \ge \frac{ R_0 }{ \eps } \}$.
Имеем
$$R_0 \ge \int\limits_{\Omega} \abs{ u_m'( x ) } \, dx \ge \int\limits_{ E_{m,\eps} } \abs{ u_m'( x ) } \, dx \ge
\int\limits_{ E_{m,\eps} } \frac{ R_0 }{ \eps } \, dx = \frac{ R_0 }{ \eps } \abs{ E_{m,\eps} }.$$
Поэтому $\abs{ E_{m,\eps} } \le \eps$.

Теперь можно ввести $L_{m,\eps} := \Omega \setminus ( E_{m,\eps} \cup G_\eps )$.
Тогда $\abs{ L_{m,\eps} } \ge 2 - 3 \eps$.

Зафиксируем $R := \frac{ R_0 }{ \eps }$, $N( \eps ) := N( \eps, \frac{ R_0 }{ \eps } )$.
Для любых $\eps > 0$, $x \in L_{m,\eps}$ и $m > N( \eps )$ получим
$$\Bigabs{ F \bigl( u_m( x ), a( x, u_m( x ) ) \abs{u_m'( x )} \bigr) - F \bigl( u( x ), a( x, u( x ) ) \abs{u_m'( x )} \bigr) } < \eps,$$
откуда
\begin{equation}
\label{wls_appr_diff}
\int\limits_{L_{m,\eps}} \Bigabs{ F \bigl( u_m( x ), a( x, u_m( x ) ) \abs{u_m'( x )} \bigr) - F \bigl( u( x ), a( x, u( x ) ) \abs{u_m'( x )} \bigr) } \, dx < 2 \eps.
\end{equation}

Возьмём $\eps_j = \frac{ \eps }{ 2^j }$ ($j \ge 1$), $m_j = N( \eps_j ) + j \to \infty$ и $L_\eps = \bigcap L_{m_j,\eps_j}$.
Тогда $\sum \eps_j = \eps$ и, тем самым, $\abs{ \Omega \setminus L_\eps } < 3 \eps$.
Поскольку из (\ref{wls_appr_diff}) следует
$$\int\limits_{L_\eps} \Bigabs{ F \bigl( u_{m_j}( x ), a( x, u_{m_j}( x ) ) |u_{m_j}'( x )| \bigr) - F \bigl( u( x ), a( x, u( x ) ) |u_{m_j}'( x )| \bigr) } \, dx < 2 \eps_j,$$
мы получаем
\begin{multline*}
A = \lim \IWg (u_{m_j}) = \lim \int\limits_{\Omega} F \bigl( u_{m_j}(x), a(x, u_{m_j}(x)) | u_{m_j }'(x) | \bigr) \, dx \\
\ge \varliminf \int\limits_{\Omega} \chi_{L_\eps}(x) F \bigl(u (x), a(x, u(x)) | u_{m_j}'(x) | \bigr) \, dx
=: \varliminf \IWg_\eps(u_{m_j}').
\end{multline*}

Наш новый функционал
$$
\IWg_\eps( v ) = \int\limits_{\Omega} \chi_{L_\eps}( x ) F \bigl( u( x ), a( x, u( x ) ) |v( x )| \bigr) \, dx
$$
выпуклый.
Вновь переходя к подпоследовательности $u_k$, можно считать, что
$\varliminf \IWg_\eps( u_{m_j}' ) = \lim \IWg_\eps( u_k' )$.
Так как $u_k' \rightharpoondown u'$ в $L_1$, то можно подобрать последовательность выпуклых комбинаций $u_k'$,
которые будут сходиться к $u'$ сильно (см. \cite[Теорема 3.13]{Rudin}).
А именно: найдутся $\alpha_{k,l} \ge 0$ для
$k \in \Nat$, $l \le k$ такие, что $\sum_{l = 1}^k \alpha_{k,l} = 1$ для каждого $k$ и
$w_k := \sum_{l = 1}^k \alpha_{k,l} u_{l}' \to u'$ в $L_1$.
Кроме того, очевидно, можно потребовать, чтобы минимальный индекс $l$ ненулевого коэффициента $\alpha_{k,l}$
стремился к бесконечности по $k$.
Тогда
$$\lim \IWg_\eps( u_k' ) = \lim \sum_{l = 1}^k \alpha_{k,l} \IWg_\eps( u_{l}' ).$$

В силу выпуклости $\IWg_\eps$ имеем
$$\sum_{l = 1}^k \alpha_{k,l} \IWg_\eps( u_{l}' ) \ge \IWg_\eps( w_k ).$$

Наконец, поскольку $w_k \to u'$ в $L_1(\Omega)$, переходя к подпоследовательности, можем считать, что $w_k(x) \to u'(x)$ почти всюду.
Кроме того, так как для  $x \in L_\eps$ выполнено $\abs{ u_j'( x ) } < \frac{ R_0 }{\eps}$, то и $\abs{ w_k( x ) } < \frac{ R_0 }{\eps}$.
Значит,
$$F \bigl( u( x ), a( x, u( x ) ) |w_k( x )| \bigr) \le \max\limits_{(x, M)} F \bigl( u( x ), a( x, u( x ) ) M \bigr) < \infty,$$
где максимум берется по компактному множеству
$( x, M ) \in \Omega \times [-\frac{ R_0 }{\eps},\frac{ R_0 }{\eps}]$.
\todo{оно не компактно}
Поэтому применима теорема Лебега, и мы получаем $\lim \IWg_\eps(w_k) = \IWg_\eps(u')$.
Таким образом,
$$A \ge \lim \IWg_\eps( u_k' ) = \lim \sum_{l = 1}^k \alpha_{k,l} \IWg_\eps( u_{l}' ) \ge
\varliminf \IWg_\eps( w_k ) = \IWg_\eps( u' ).$$

Ввиду произвольности $\eps > 0$ имеем $A \ge \IWg(u)$.
\end{proof}

\begin{lm}
\label{uplift}
Пусть $A \subset W^1_1(\Omega)$.
И пусть $B \subset A$ таково, что $\forall v \in B$ выполнено $\IWg( \mon{v} ) \le \IWg( v )$.
Предположим, что для каждого $u \in A$ найдётся последовательность $u_k \in B$ такая,
что $u_k \to u$ в $W^1_1(\Omega)$ и $\IWg( u_k ) \to \IWg( u )$.
Тогда $\forall u \in A$ будет выполнено $\IWg( \mon{u} ) \le \IWg( u )$.
\end{lm}
\begin{proof}
Возьмём некоторую $u \in A$ и для нее найдем приближающую последовательность $\set{u_k} \subset B$.
По условию $\IWg(\mon{u_k}) \le \IWg(u_k) \to \IWg(u)$.
В \cite[теорема 1]{Brock} показано, что
$$
u_k \to u \text{ в } \W(\Omega) \quad \Longrightarrow \quad \symm{u_k} \rightharpoondown \symm{u} \text{ in } \W(\Omega).
$$
Поскольку $\mon{u_k}( x ) = \symm{u_k}( \frac{x - 1}{2} )$ and $\mon{u}( x ) = \symm{u}( \frac{x - 1}{2} )$,
имеем $\mon{u_k} \rightharpoondown \mon{u}$ в $\W(\Omega)$.
Из леммы \ref{lowersemi} получаем
$$
\IWg(\mon{u}) \le \liminf \IWg(\mon{u_k}) \le \lim \IWg(u_k) = \IWg(u).
$$
\end{proof}

\todo{Это утверждение совпадает с первой частью следующей теоремы, но более подробно объяснено.
Вставить формулировку теоремы, на которую ссылаемся, и перенести в следующий пункт}
\begin{cor}
Пусть вес $a$ непрерывен, и неравенство $(\ref{toprove_weighted})$ верно для неотрицательных кусочно линейных функций $u$.
Тогда оно верно для всех неотрицательных липшицевых функций.
\end{cor}
\begin{proof}
\todo{Не нужно ли тут что-то от границы $\Omega$?}
Ввиду теоремы 1 из \S6.6 \cite{Gariepy},
любая липшицева функция $u$ может быть приближены последовательностью $u_k \in C^1(\overline{\Omega})$ в следующем смысле:
$$
u_k \rightrightarrows u, \qquad u_k' \to u' \text{ п.в.}, \qquad |u_k'| \le const.
$$
Тогда по теореме Лебега $u_k \to u$ в $\W(\Omega)$ и $\IWg( u_k ) \to \IWg( u )$.
В свою очередь, $u_k$ могут быть аналогичным образом приближены кусочно линейными функциями.
Применив лемму \ref{weighted_linear_lm} и лемму \ref{uplift}, получаем требуемое.
\end{proof}
