\section{Переход к соболевским функциям}

\begin{thm}
\label{bounded_growth_thm}
Пусть функция $a(x', \cdot, u)$ четна и удовлетворяет условию $(\ref{almostConcave})$ для всех $x'$ и $u$.
Тогда

\textbf{1.} Неравенство (\ref{toprove}) верно для произвольной неотрицательной $u \in Lip(\Omega)$.

\textbf{2.} Предположим, что $\partial \omega \in Lip$ и
для любых $x' \in \omega, z \in \Real_+, p \in \Real$
функция $F$ удовлетворяет неравенству
$$F( x', z, p ) \le C ( 1 + |z|^{q^*} + |p|^q ),$$
где $\frac{1}{q^*} = \frac{1}{q} - \frac{1}{n}$, если $q < n$, либо $q^*$ любое в противном случае.
Если $q \le n$, то дополнительно предположим, что веса $a$ и $a_i$ ограничены.
Тогда неравенство (\ref{toprove}) верно для произвольной неотрицательной $u \in W^1_q(\Omega)$.
\end{thm}
\begin{proof}
\textbf{1.} Мы можем приблизить липшицевы $u$ кусочно линейными функциями $u_k$ вместе с производными почти всюду.
Поскольку $u_k$ равномерно ограничены вместе с производными,
то и $F(x', u_k(x), \norm{\mathcal{D} u_k})$ равномерно ограничены.
Тогда мы можем воспользоваться теоремой Лебега, получив $u_k \to u$ в $W^1_1(\Omega)$ и $I( u_k ) \to I( u )$.
Воспользовавшись предложением \ref{uplift}, получаем требуемое.

\textbf{2.} Рассмотрим произвольную $u \in W^1_q(\Omega)$.
Для нее можно построить последовательность кусочно линейных функций $u_k$, приближающих ее в $W^1_q(\Omega)$.
Действительно, поскольку $\partial \Omega \in Lip$,
$u$ можно продолжить финитным образом на внутренность большого шара в $\Real^n$
и приблизить гладкими финитными функциями.
Далее шар триангулируется, и значения функции линейно интерполируются.
Очевидно, в процессе все функции остаются неотрицательными.

Тогда, ввиду предложения \ref{uplift}, достаточно добиться $I( u_k ) \to I( u )$.
Доказательство этой сходимости можно свести к теореме Красносельского о непрерывности
оператора Немыцкого (см. \cite[гл. 5, \textsection 17]{Krasnoselsky}).
Однако для удобства читателя мы приводим здесь рассуждение целиком.

Покажем, что веса $a_i(x', u(x))$ и $a(x, u(x))$ ограничены.
Если $q \le n$, то это выполнено по предположению теоремы. Если же нет, то $W^1_q(\Omega)$ вкладывается в $C(\overline{\Omega})$,
тем самым, $u_k(x)$ равномерно ограничены, а значит, и $a_i(x', u_k(x))$ и $a(x, u_k(x))$ равномерно ограничены.
Поэтому $\norm{ \mathcal{D} u_k( x ) } \le C_1 |\nabla u_k( x )|$.
То есть,
$$F( x', u_k( x ), \norm{ \mathcal{D} u_k( x ) } ) \le C_2 ( 1 + |u_k( x )|^{q^*} + |\nabla u_k( x )|^q ).$$

%Обозначим
%$$H( x, ( f( x ), g_i( x ), h( x ) ) ) := F( x', f( x ), \norm{ ( a_i( x', f( x ) ) g_i( x ), a( x, f( x ) ) h( x ) ) } ).$$
%Заметим, что $( f( x ), g_i( x ), h( x ) ) \mapsto H( x, ( f( x ), g_i( x ), h( x ) ) )$
%суть оператор Немыцкого,
%а также $H( x, ( u( x ), D_i u( x ), D_n( x ) ) ) = F( x', u( x ), \norm{ \mathcal{D} u( x ) } )$.
%Ввиду неравенства (\ref{Fbound}) мы попадаем в условия теоремы Красносельского.
%Тем самым, $H$ непрерывно действует из $L_{q^*}( \Omega ) \times ( L_q( \Omega ) )^n$ в $L_1( \Omega )$.
%А значит, $I( u_k ) \to I( u )$.

Рассмотрим множества $A_m$, состоящие из $x \in \Omega$, для которых при всех $k \ge m$ выполнено
$1 + |u_k(x)|^{q^*} + |\nabla u_k( x )|^q \le 2 ( 1 + |u(x)|^{q^*} + |\nabla u( x )|^q )$.
Очевидно, что $A_m \subset A_{m + 1}$.
Переходя к подпоследовательности, можем считать, что $u_k \to u$ и $\nabla u_k \to \nabla u$ почти всюду.
А значит $|A_m| \to |\Omega|$.
Тогда
\begin{eqnarray*}
\mathcal{X}\{A_k\} F( x', u_k( x ), \norm{ \mathcal{D} u_k( x ) } ) &\le& 2 ( 1 + |u( x )|^{q^*} + |\nabla u( x )|^q ), \\
\mathcal{X}\{A_k\} F( x', u_k( x ), \norm{ \mathcal{D} u_k( x ) } ) &\to& F( x', u( x ), \norm{ \mathcal{D} u( x ) } )
\end{eqnarray*}
почти всюду.
По теореме вложения $\Vert u_k \Vert_{q^*} \le C_3 \Vert u_k \Vert_{W^1_q}$.
Тем самым, мы нашли суммируемую мажоранту и получаем
$\int_{A_k} \mathcal{X}\{A_k\} F( x', u_k( x ), \norm{ \mathcal{D} u_k( x ) } ) dx \to I( u )$ по теореме Лебега .

Теперь оценим остаток:
\begin{multline*}
\int_{\Omega \setminus A_k} F( x', u_k( x ), \norm{\mathcal{D} u_k( x )} ) dx
\le \int_{\Omega \setminus A_k} C_2 ( 1 + |u_k( x )|^{q^*} + |\nabla u_k( x )|^q ) dx \\
\le C_4 \Big( \int_{\Omega \setminus A_k} ( 1 + |u( x )|^{q^*} + |\nabla u( x )|^q ) dx
+ \int_{\Omega \setminus A_k} ( 1 + |u( x ) - u_k( x )|^{q^*} + |\nabla ( u - u_k )( x )|^q ) \Big) dx.
\end{multline*}

Первое слагаемое стремится к нулю по абсолютной непрерывности интеграла.
Для второго слагаемого выполнено
\begin{multline*}
\int_{\Omega \setminus A_k} ( 1 + |u( x ) - u_k( x )|^{q^*} + |\nabla ( u - u_k )( x ) )|^q ) dx \\
\le ( | \Omega \setminus A_m( k ) | + \Vert u - u_k \Vert_{W^1_q}^{q^*} + \Vert u - u_k \Vert_{W^1_q}^q ) \to 0.
\end{multline*}

Тем самым, сходимость $I( u_k ) \to I( u )$ доказана.
\end{proof}

Аналогично, с учетом замечаний (\ref{lanNec}) и (\ref{lanLin}), доказывается
\begin{thm}
Пусть $u(\cdot, -1) \equiv 0$ и функция $a(x', \cdot, u)$ удовлетворяет условию $(\ref{almostConcave})$ для всех $x'$ и $u$.
Тогда верны выводы теоремы \ref{bounded_growth_thm}.
\end{thm}
