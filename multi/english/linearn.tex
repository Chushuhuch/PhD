\section{The proof of (\ref{toprove}) for piecewise linear functions}

In this section we prove inequality (\ref{toprove}) for piecewise linear functions.

\begin{lm}
Let the function $a(x', \cdot, u)$ is even and satisfies the condition $(\ref{almostConcave})$ ��� ���� $(x', u)$.
Then, if $u$ --- a nonnegative piecewise linear function, then $I( u ) \ge I( u^* )$.
\end{lm}

\begin{proof}
The proof follows the scheme of proof of Theorem 2 in \cite{1dim}.
Let the function $u$ has breaks on the set $\partial \Omega \subset C \subset \Omega$.
Take $$U := \{ ( x', u( x', y ) ): y \in (-1, 1), (x', y) \not\in C \}.$$
Then the open set $U$ is partitioned into a union of a finite number of connected open sets $G_j$.
Let $m_j$ the number of inverse images of values ??of $( x', u_0 ) \in G_j$, that is, the number of solutions $u( x', y ) = u_0$
(Obviously, this number is for $( x', u_0 ) \in G_j$).
It is easy to see that these pre-images are linear functions $( x', u_0 )$:
$y = y_k^j( x', u_0 )$, $k = 1, \dots, m_j$,
and $D_n y_k^j( x', u( x', y ) ) = \frac{1}{D_n u( x', y )}$.
We will assume that the $y_1^j(x', u_0) < y_2^j(x', u_0) < \dots < y_{m_j}^j(x', u_0)$.

The equation $u^*(x'_0, y^*) = u_0$ sets $y^*$ as a function of $( x'_0, u_0 ) \in G_j$.
It can be expressed in terms of $y_k^j$ (in particular, $y^*$ is piecewise linear):

\begin{center}
\begin{tabular}{l|l|l}
\multirow{2}{*}{$u( x'_0, -1 ) < u_0$ \rule[-34pt]{0pt}{65pt}} & $m_j$ �����   & $y^* = 1-\sum\limits_{k=1}^{m_j} (-1)^k y_k^j$ \rule[-17pt]{0pt}{40pt} \\
                                                               & $m_j$ ������� & $y^* = -\sum\limits_{k=1}^{m_j} (-1)^k y_k^j$ \rule[-17pt]{0pt}{40pt} \\ \hline
\multirow{2}{*}{$u( x'_0, -1 ) > u_0$ \rule[-34pt]{0pt}{65pt}} & $m_j$ �����   & $y^* = -1+\sum\limits_{k=1}^{m_j} (-1)^k y_k^j$ \rule[-17pt]{0pt}{40pt} \\
                                                               & $m_j$ ������� & $y^* = \sum\limits_{k=1}^{m_j} (-1)^k y_k^j$ \rule[-17pt]{0pt}{40pt} \\
\end{tabular}
\end{center}

Hence it is clear that
\begin{equation*}
D_n y^*(x', u(x', y)) = \frac{1}{D_n u^*(x',y)}=\sum_{k=1}^{m_j}{|D_n y_k^j (x',u(x',y))|}
\end{equation*}
and $D_i u^*( x', y ) = \pm \sum_{k = 1}^{m_j} ( -1 )^k D_i y_k^j( x', u( x', y ) )$, where the sign before the right-hand side depends only on $j$.

Then
\begin{multline}
\label{intu}
I( u )=\sum_{j=1}^N {
    \int\limits_{G_j}{
        F( x', u(x), \norm{
            a_i(x',u(x)) D_i u(x), a(x,u(x)) D_n u(x)
        })
    dx}
}
\\ = \sum_{j=1}^N{
    \int\limits_{u(G_j)}{
        \sum_{k=1}^{m_j}{
            F \Big( x', u, \frac{
                \norm{
                    a_i(x',u) D_i y_k^j(x',u), a(x', y_k^j(x',u), u)
                }
            }{\abs{ D_n y_k^j (x',u) }} \Big)
        }
        }} {{\abs{ D_n y_k^j(x',u) }
    dx' du}
},
\end{multline}
\begin{multline}
\label{intus}
I( u^* )=
\sum_{j=1}^N{
    \int\limits_{G_j}{
        F(x', u^*, \norm{
            a_i(x',u^*(x)) D_i u^*(x), a(x,u^*(x)) D_n u^*(x)
        })
    dx}
}
\\ = \sum_{j=1}^N {
    \int\limits_{u(G_j)}{
        F \Big( x', u^*, \frac{
            \norm{
                a_i(x',u^*) D_i y^*(x',u^*), a(x',y^*(x',u^*),u^*)
            }
        }{\sum_{k=1}^{m_j} \abs{ D_n y_k^j(x',u^*) }} \Big) }} \times
        \\ \times {{ \sum_{k=1}^{m_j} \abs{ D_n y_k^j(x',u^*) }
    dx' du^*}
}.
\end{multline}
Fix a $j$, $x'$ and $u$ and denoted
$b_k=\abs{ D_n y_k^j }$, $c_{ki}=D_i y_k^j$, $c^*_i=D_i y^*$, $y_k=y_k^j(x',u)$, $y^*=y^*(x',u)$, $m = m_j$.
Then we have the following chain of inequalities:

\begin{multline}
\label{chain}
\sum_{k=1}^m{ b_k F\Big(\frac{ \norm{ a_ic_{ki}, a(y_k) } }{b_k}\Big) }
\overset{a}{\ge} F\Big( \frac{ \sum_{k=1}^m \norm{ a_i c_{ki}, a(y_k)} }{ \sum_{k=1}^m b_k } \Big) \sum_{k=1}^m b_k \\
\overset{b}{=}  F\Big( \frac{ \sum_{k=1}^m \norm{ ( -1 )^k a_i c_{ki}, a(y_k) } }{ \sum_{k=1}^m b_k} \Big) \sum_{k=1}^m b_k
\overset{c}{\ge}  F\Big( \frac{ \norm{ \sum_{k = 1}^m ( ( -1 )^k a_i c_{ki}, a( y_k ) ) } }{ \sum_{k=1}^m b_k} \Big) \sum_{k=1}^m b_k \\
= F\Big( \frac{ \norm{ \sum_{k = 1}^m ( -1 )^k a_i c_{ki}, \sum_{k = 1}^m a( y_k ) } }{ \sum_{k=1}^m b_k} \Big) \sum_{k=1}^m b_k
\overset{d}{\ge} F\Big( \frac{ \norm{ \sum_{k = 1}^m ( -1 )^k a_i c_{ki}, a( y^* ) } }{ \sum_{k=1}^m b_k} \Big) \sum_{k=1}^m b_k \\
\overset{e}{=}   F\Big( \frac{ \norm{ \pm a_i \sum_{k = 1}^m ( -1 )^k c_{ki}, a( y^* ) } }{ \sum_{k=1}^m b_k} \Big) \sum_{k=1}^m b_k
= F\Big( \frac{ \norm{ a_i c^*_i, a(y^*) } }{\sum_{k=1}^m b_k} \Big) \sum_{k=1}^m b_k.
\end{multline}
Here, in the transition (a) applied Jensen's inequality, in passages (b) and (e) the parity of the rules used in (c) used by the triangle inequality,
in (d) --- proposal \ref{weightSum} and parity of weight $a$ by $y$.

From (\ref{chain}) shows that the integrand in (\ref{intu}) is not less than
integrand in (\ref{intus}).
Thus, the proof is complete.
\end{proof}

\begin{rem}
\label{lanLin}
If $u(\cdot, -1) \equiv 0$, then the assertion is true without the condition of parity weights.
Parity is used only in transition (d) chain of inequalities (\ref{chain}),
and, since the $u(\cdot, -1) \equiv 0$ is always satisfied $u(x'_0, -1) < u_0$,
the proposal \ref{weightSum} just wanted to ensure the transition (d) inequalities.
\end{rem}