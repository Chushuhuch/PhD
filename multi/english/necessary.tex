\section{The conditions necessary for inequality (\ref{toprove})}

Hereafter, we will refer to some of the assertions in \cite{1dim},
that the evidence in the multidimensional case, literally coincide with the evidence in the one-dimensional case.

In addition, in this section we will for simplicity omit the parameters $x'$ and $v$ of the function $a$,
implying that all inequalities hold for any $x' \in \omega, v \in \Real_+$.
As part of this agreement will put $a( x', y, v ) \equiv a( y )$.

\begin{prop}
{\rm \cite[Theorem 1]{1dim}.}
{\bf 1.} If inequality $(\ref{toprove})$ holds for a $F \in \mathfrak{F}$
and arbitrary piecewise linear $u$, the weight of $a$ is even on the $y$,
ie $a(y) \equiv a(-y)$.

{\bf 2.} If inequality $(\ref{toprove})$ holds for arbitrary $F \in \mathfrak{F}$
and arbitrary piecewise linear $u$, the weight of $a$ satisfies
\begin{equation}
\label{almostConcave}
a( s ) + a( t ) \ge a( 1 - t + s ), \qquad -1 \le s \le t \le 1.
\end{equation}
\end{prop}

\begin{rem}
If the function $a$ is non-negative, and even and concave with respect to the first argument, it satisfies the condition $(\ref{almostConcave})$.
Indeed: for such a function in all $(s, t)$ holds $a( 1 ) - a( s ) \le a( t ) - a( -1 + t - s )$.
Since $a( 1 ) \ge 0$, then
We receive $a( s ) + a( t ) \ge a( -1 + t - s ) = a( 1 - t + s )$.
The converse is not true in general, that is not any kind of an even
negative function satisfying the $(\ref{almostConcave})$, is concave.
\end{rem}

\begin{rem}
\label{lanNec}
If you put $u(\cdot, -1) \equiv 0$, then the condition (\ref{almostConcave}) is necessary for inequality (\ref{toprove}).
\end{rem}

\begin{prop}
\label{weightSum}
{\rm \cite[Lemma 1]{1dim}.}
Consider a continuous function $a \ge 0$ defined on $[-1,1]$
and satisfying the condition (\ref{almostConcave}).
Then for any $-1 \le t_1 \le t_2 \le \ldots \le t_m \le 1$ right
\begin{align*}
\sum_{k=1}^m a(t_k) & \ge a( 1 - \sum_{k = 1}^m (-1)^k t_k ), & \text{ ��� ������ $m$},&\\
\sum_{k=1}^m a(t_k) & \ge a( -\sum_{k = 1}^m (-1)^k t_k ),    & \text{ ��� �������� $m$}.&
\end{align*}
\end{prop}
