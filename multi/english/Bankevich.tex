\documentclass[12pt]{article}
\renewcommand{\baselinestretch}{1}
%\renewcommand{\baselinestretch}{1.5}
\textwidth=158mm
\textheight=232mm
\voffset=-24mm
%\pagestyle{empty}

\usepackage{amsmath,amssymb,amsthm,amsfonts}
\usepackage{multirow}
\newcommand{\Real}{\mathbb R}
\newcommand{\Nat}{\mathbb N}
\newcommand{\norm}[1]{\pmb{\Vert}#1\pmb{\Vert}}
\newcommand{\abs}[1]{\left\vert#1\right\vert}
\newcommand{\bignorm}[1]{\bigl\Vert#1\bigr\Vert}
\newcommand{\bigabs}[1]{\bigl\vert#1\bigr\vert}
\newcommand{\set}[1]{\left\{#1\right\}}
\renewcommand{\phi}{\varphi}
\newcommand{\eps}{\varepsilon}
\renewcommand{\ge}{\geqslant}
\renewcommand{\le}{\leqslant}
\renewcommand{\liminf}{\underline{\lim}}
\newcommand{\grad}{\triangledown}
\newcommand{\card}{{\rm card}}
\newtheorem{thm}{Theorem}
\newtheorem{prop}{Proposition}
\newtheorem{lm}{Lemma}
\newtheorem{rem}{Remark}
\newtheorem{cor}{Corollary}
\newcommand{\To}{\longrightarrow}
\newcommand{\W}{W_1^1}
\newcommand{\Wf}{\stackrel{o\ }{\W}}
\newcommand{\sign}{\mathop{\rm sign}\nolimits}
\newcommand{\dist}{\mathop{\rm dist}\nolimits}
\newcounter{pictureCounter}
\begin{document}

\title{On monotonicity of some functionals under monotone rearrangement with respect to one variable}
\author{
S.~Bankevich
\footnote{Sergey.Bankevich@gmail.com}
}

\maketitle

\section{Introduction}
Let $\Omega = \omega \times [-1,1]$
where $\omega$ is a bounded domain in $\Real^{n - 1}$.
Also let $x = ( x_1, \dots, x_{n - 1}, y ) = ( x', y )$.

Recall the layer cake representation for a measurable function $u: \Omega \to \Real_+$
(see \cite[Theorem 1.13]{LL}).
Hereinafter $\Real_+ = [0, \infty)$.
Let $\mathcal{A}_t(x') := \{ y \in [-1,1]:\ u( x', y ) > t \}$.
Then
$$u(x', y) = \int\limits_0^\infty \mathcal{X}\{\mathcal{A}_t(x')\}(y) dt,$$
where $\mathcal{X}\{A\}$ is an indicator function of a set $A$.

Define the monotone rearrangement of a measurable set $E \subset [-1, 1]$ and
the monotone rearrangement of a nonnegative function $u \in \W(\Omega)$ as follows:
\begin{eqnarray*}
E^* := [1 - \abs{E}, 1]; \qquad
u^*(x', y) = \int\limits_0^\infty \mathcal{X}\{ ( \mathcal{A}_t(x') )^* \}(y) dt.
\end{eqnarray*}

We define the set $\mathfrak{F}$ of continuous functions $F: \omega \times \Real_+ \times \Real_+ \to \Real_+$,
which are convex and increasing in the third argument, and satisfy $F( \cdot, \cdot, 0 ) \equiv 0$.

Consider the following functional:
\begin{equation}
\label{functional}
I( u ) = \int\limits_\Omega F( x', u(x), \norm{ \mathcal{D} u } ) dx.
\end{equation}
Here $F \in \mathfrak{F}$ and
$\norm{\cdot}$ is a norm in $\Real^n$, which is symmetric in the last coordinate,
i.e. satisfying $\norm{( x', y )} = \norm{( x', -y )}$ for any $( x',  y ) \in \Real^n$.
$$\mathcal{D} u = ( a_1( x', u( x ) ) D_1 u, \dots, a_{n - 1}( x', u( x ) ) D_{n - 1} u, a( x, u( x ) ) D_n u )$$
is a weighted gradient of $u$ (note that only the weight of $D_n u$ depends on $y$).
$a( \cdot, \cdot ): \Omega \times \Real_+ \to \Real_+$ and $a_i( \cdot, \cdot ): \omega \times \Real_+ \to \Real_+$ are continuous functions.
Hereinafter, the index $i$ runs from $1$ to $n - 1$.

It is well known that for $a_i = a \equiv 1$ the following inequality holds
\begin{equation}
\label{toprove}
I( u^* ) \le I( u ), \qquad \qquad u \in \W(\Omega),\ u \ge 0
\end{equation}
(see e.g. \cite{Kawohl} and references therein).

In this paper we present the necessary conditions on weight $a$ for inequality (\ref{toprove}) to hold.
Also we prove inequality (\ref{toprove}) under certain additional restrictions.

An inequality similar to (\ref{toprove}) is considered in paper \cite{Br}.
Namely,
\begin{equation}
\label{breq}
I( \overline{u} ) \le I( u ), \qquad \qquad u \in \Wf( \Omega ),\ u \ge 0,
\end{equation}
where $\overline{u}$ stands for Steiner symmetrization (symmetrical rearrangement) of function $u$ with respect to $y$.
The author of \cite{Br} proves the inequality for any weight $a$, which is even and convex with respect to $y$.
To do this inequality (\ref{breq}) is proved for piecewise linear $u$.
It is further claimed that in the general case the inequality is obtained by approximation.
However, the passage to the limit is not properly justified in \cite{Br},
and in fact the result was obtained only for Lipschitz $u$ functions.
Note that the article \cite{Br} has only the weight of $D_n u$ depend on $y$ either,
this restriction appears to be essential.

Inequalities (\ref{toprove}) and (\ref{breq}) in a one-dimensional case are considered in work \cite{1dim}.
It establishes the necessary and sufficient conditions for the inequalities.
In particular, the gap in \cite{Br} was closed for $n = 1$.
Under certain additional restrictions the result was announced in \cite{DAN}.

The article \cite{Lan} considered inequality (\ref{toprove}) for a similar functional in the two-dimensional case
with the additional restriction of $u(\cdot, -1) \equiv 0$.
We note that our conditions on the weight functions are weaker than in \cite{Lan}.

The article is divided into four sections.
\S2 contains necessary conditions for inequality (\ref{toprove}).
\S3 contains the proof of inequality (\ref{toprove}) for piecewise linear functions $u$.
\S4 contains the proof of the inequality in more general cases.

\section{The necessary conditions for inequality (\ref{toprove})}

Here we refer to several assertions from \cite{1dim}.
They are proved in \cite{1dim} for one-dimensional case, but the proofs of a multidimensional analogues are the same.

For simplicity we omit parameters $x'$ and $v$ of the weight $a$ in this section.
This means that all inequalities hold for any $x' \in \omega, v \in \Real_+$.
Also $a( x', y, v ) \equiv a( y )$.

\begin{prop}
{\rm \cite[Theorem 1]{1dim}.}
{\bf 1.} Suppose inequality $(\ref{toprove})$ holds for some $F \in \mathfrak{F}$
and arbitrary piecewise linear $u$.
Then the weight $a$ is even in $y$.

{\bf 2.} Suppose inequality $(\ref{toprove})$ holds for arbitrary $F \in \mathfrak{F}$
and arbitrary piecewise linear $u$.
Then the weight $a$ satisfies
\begin{equation}
\label{almostConcave}
a( s ) + a( t ) \ge a( 1 - t + s ), \qquad -1 \le s \le t \le 1.
\end{equation}
\end{prop}

\begin{rem}
Any nonnegative weight $a$, which is even and concave with respect to $y$, satisfies condition $(\ref{almostConcave})$.
Indeed, for any $(s, t)$ we have $a( 1 ) - a( s ) \le a( t ) - a( -1 + t - s )$.
Since $a( 1 ) \ge 0$ we get $a( s ) + a( t ) \ge a( -1 + t - s ) = a( 1 - t + s )$.
The converse in general is not true, that is there are even nonnegative functions satisfying $(\ref{almostConcave})$ which are not concave.
\end{rem}

\begin{rem}
\label{lanNec}
Suppose $u(\cdot, -1) \equiv 0$.
Then condition (\ref{almostConcave}) is necessary for inequality (\ref{toprove}) to hold.
\end{rem}

\begin{prop}
\label{weightSum}
{\rm \cite[Lemma 1]{1dim}.}
Consider a continuous function $a \ge 0$ defined on $[-1,1]$
and satisfying (\ref{almostConcave}).
Then for any $-1 \le t_1 \le t_2 \le \ldots \le t_m \le 1$ we have
\begin{align*}
\sum\limits_{k=1}^m a(t_k) & \ge a( 1 - \sum\limits_{k = 1}^m (-1)^k t_k ), & \text{ for even $m$},&\\
\sum\limits_{k=1}^m a(t_k) & \ge a( -\sum\limits_{k = 1}^m (-1)^k t_k ),    & \text{ for odd $m$}.&
\end{align*}
\end{prop}

\section{The proof of (\ref{toprove}) for piecewise linear functions}

\begin{lm}
Suppose the function $a(x', \cdot, u)$ is even and satisfies $(\ref{almostConcave})$ for any $(x', u)$.
Then for any nonnegative piecewise linear $u$ we have $I( u ) \ge I( u^* )$.
\end{lm}

\begin{proof}
The proof follows the scheme of proof of Theorem 2 in \cite{1dim}.
Let $\partial \Omega \subset C \subset \Omega$ be a minimal closed set, such that on any connected subset of $\Omega \setminus C$ the function $u$ is linear.
Define $$U := \{ ( x', u( x', y ) ): y \in (-1, 1), (x', y) \not\in C \}.$$
Then the open set $U$ can be partitioned into a finite union of disjoint connected open sets $G_j$.
Let $m_j$ be the number of inverse images of $( x', u_0 ) \in G_j$, that is the number of solutions to $u( x', y ) = u_0$
(obviously, this number is same for any $( x', u_0 ) \in G_j$).
It is easy to see that the inverse images themselves are linear functions of $( x', u_0 )$:
$y = y_k^j( x', u_0 )$, $k = 1, \dots, m_j$,
and $D_n y_k^j( x', u( x', y ) ) = \frac{1}{D_n u( x', y )}$.
Without loss of generality, we assume that $y_1^j(x', u_0) < y_2^j(x', u_0) < \ldots < y_{m_j}^j(x', u_0)$.

The equation $u^*(x'_0, y^*) = u_0$ defines $y^*$ as a function of $( x'_0, u_0 ) \in G_j$.
The $y^*$ function can be expressed in terms of $y_k^j$ (in particular, $y^*$ is piecewise linear):

\begin{center}
\begin{tabular}{l|l|l}
\multirow{2}{*}{$u( x'_0, -1 ) < u_0$ \rule[-34pt]{0pt}{65pt}} & even $m_j$ & $y^* = 1-\sum\limits_{k=1}^{m_j} (-1)^k y_k^j$ \rule[-17pt]{0pt}{40pt} \\
                                                               & odd $m_j$  & $y^* = -\sum\limits_{k=1}^{m_j} (-1)^k y_k^j$ \rule[-17pt]{0pt}{40pt} \\ \hline
\multirow{2}{*}{$u( x'_0, -1 ) > u_0$ \rule[-34pt]{0pt}{65pt}} & even $m_j$ & $y^* = -1+\sum\limits_{k=1}^{m_j} (-1)^k y_k^j$ \rule[-17pt]{0pt}{40pt} \\
                                                               & odd $m_j$  & $y^* = \sum\limits_{k=1}^{m_j} (-1)^k y_k^j$ \rule[-17pt]{0pt}{40pt} \\
\end{tabular}
\end{center}

Hence it is clear that for $(x', y) \in G_j$
\begin{equation*}
D_n y^*(x', u(x', y)) = \frac{1}{D_n u^*(x',y)}=\sum\limits_{k=1}^{m_j}{|D_n y_k^j (x',u(x',y))|}
\end{equation*}
and $D_i u^*( x', y ) = \pm \sum\limits_{k = 1}^{m_j} ( -1 )^k D_i y_k^j( x', u( x', y ) )$, where the sign before the right-hand side depends only on $j$.

Then
\begin{multline}
\label{intu}
I( u )=\sum\limits_{j=1}^N {
    \int\limits_{G_j}{
        F( x', u(x), \norm{
            a_i(x',u(x)) D_i u(x), a(x,u(x)) D_n u(x)
        })
    dx}
}
\\ = \sum\limits_{j=1}^N{
    \int\limits_{u(G_j)}{
        \sum\limits_{k=1}^{m_j}{
            F \Big( x', u, \frac{
                \norm{
                    a_i(x',u) D_i y_k^j(x',u), a(x', y_k^j(x',u), u)
                }
            }{\abs{ D_n y_k^j (x',u) }} \Big)
        }
        }} {{\abs{ D_n y_k^j(x',u) }
    dx' du}
},
\end{multline}
\begin{multline}
\label{intus}
I( u^* )=
\sum\limits_{j=1}^N{
    \int\limits_{G_j}{
        F(x', u^*, \norm{
            a_i(x',u^*(x)) D_i u^*(x), a(x,u^*(x)) D_n u^*(x)
        })
    dx}
}
\\ = \sum\limits_{j=1}^N {
    \int\limits_{u(G_j)}{
        F \Big( x', u^*, \frac{
            \norm{
                a_i(x',u^*) D_i y^*(x',u^*), a(x',y^*(x',u^*),u^*)
            }
        }{\sum\limits_{k=1}^{m_j} \abs{ D_n y_k^j(x',u^*) }} \Big) }}
        \\ \times {{ \sum\limits_{k=1}^{m_j} \abs{ D_n y_k^j(x',u^*) }
    dx' du^*}
}.
\end{multline}
Now we fix a $j$, $x'$ and $u$ and denote
$b_k := \abs{ D_n y_k^j }$, $c_{ki} := D_i y_k^j$, $c^*_i := D_i y^*$, $y_k := y_k^j(x',u)$, $y^* := y^*(x',u)$, $m := m_j$.
Then the following chain of inequalities holds:

\begin{multline}
\label{chain}
\sum\limits_{k=1}^m{ b_k F\Big(\frac{ \norm{ a_ic_{ki}, a(y_k) } }{b_k}\Big) }
\overset{a}{\ge} F\Big( \frac{ \sum\limits_{k=1}^m \norm{ a_i c_{ki}, a(y_k)} }{ \sum\limits_{k=1}^m b_k } \Big) \sum\limits_{k=1}^m b_k \\
\overset{b}{=}  F\Big( \frac{ \sum\limits_{k=1}^m \norm{ ( -1 )^k a_i c_{ki}, a(y_k) } }{ \sum\limits_{k=1}^m b_k} \Big) \sum\limits_{k=1}^m b_k
\overset{c}{\ge}  F\Big( \frac{ \norm{ \sum\limits_{k = 1}^m ( ( -1 )^k a_i c_{ki}, a( y_k ) ) } }{ \sum\limits_{k=1}^m b_k} \Big) \sum\limits_{k=1}^m b_k \\
= F\Big( \frac{ \norm{ \sum\limits_{k = 1}^m ( -1 )^k a_i c_{ki}, \sum\limits_{k = 1}^m a( y_k ) } }{ \sum\limits_{k=1}^m b_k} \Big) \sum\limits_{k=1}^m b_k
\overset{d}{\ge} F\Big( \frac{ \norm{ \sum\limits_{k = 1}^m ( -1 )^k a_i c_{ki}, a( y^* ) } }{ \sum\limits_{k=1}^m b_k} \Big) \sum\limits_{k=1}^m b_k \\
\overset{e}{=}   F\Big( \frac{ \norm{ \pm a_i \sum\limits_{k = 1}^m ( -1 )^k c_{ki}, a( y^* ) } }{ \sum\limits_{k=1}^m b_k} \Big) \sum\limits_{k=1}^m b_k
= F\Big( \frac{ \norm{ a_i c^*_i, a(y^*) } }{\sum\limits_{k=1}^m b_k} \Big) \sum\limits_{k=1}^m b_k.
\end{multline}
Here, in (a) we applied Jensen's inequality, in (b) and (e) we applied evenness of the norm, in (c) we used the triangle inequality,
in (d) we applied proposition \ref{weightSum} and evenness of the weight $a$.

(\ref{chain}) shows that the integrand in (\ref{intu}) is not less than the integrand in (\ref{intus}).
This finishes the proof.
\end{proof}

\begin{rem}
\label{lanLin}
If $u(\cdot, -1) \equiv 0$ the lemma holds without the evenness of the weight.
Indeed, the evenness is used only in (d) from (\ref{chain}).
Since $u(\cdot, -1) \equiv 0$ implies $u(x'_0, -1) < u_0$, Proposition \ref{weightSum} alone is enough for (d) to be true.
\end{rem}

\section{Inequality (\ref{toprove}) for Sobolev functions}

The following propositions are proved in \cite{1dim} in one-dimensional case.
Their proofs in the multidimensional case do not differ.
\begin{prop} {\rm \cite[Lemma 5]{1dim}.}
Functional $I( u )$ is weakly lower semicontinuous in $W^1_1(\Omega)$.
\end{prop}

\begin{prop}
\label{uplift}
{\rm \cite[Lemma 6]{1dim}.}
Let $A \subset W^1_1(\Omega)$. Let $B \subset A$ such that
$\forall v \in B$ ��������� $I( v^* ) \le I( v )$. Suppose that for every $u \in A$
there is a sequence $u_k \in B$ such that $u_k \to u$ in $W^1_1(\Omega)$ and $I( u_k ) \to I( u )$.
Then $I( u^* ) \le I( u )$ for any $u \in A$.
\end{prop}

\begin{thm}
\label{mainThm}
Suppose function $a(x', \cdot, u)$ is even and satisfies condition $(\ref{almostConcave})$ for any $(x', u)$.
Then

{\bf 1.} Inequality (\ref{toprove}) holds for any nonnegative $u \in Lip(\Omega)$.

{\bf 2.} Suppose in addition that $\partial \omega \in Lip$ and
for any $x' \in \omega, z \in \Real_+, p \in \Real$
$$F( x', z, p ) \le C ( 1 + |z|^{q^*} + |p|^q ),$$
where $\frac{1}{q^*} = \frac{1}{q} - \frac{1}{n}$  if $q < n$, and $q^*$ is arbitrary otherwise.
If $q \le n$ we also assume that the weights $a$ and $a_i$ are bounded.
Then inequality (\ref{toprove}) holds for any nonnegative $u \in W^1_q(\Omega)$.
\end{thm}

\begin{proof}
{\bf 1.} We can approximate a Lipschitz function $u$ with piecewise linear functions $u_k$ with its derivative almost everywhere.
Since $u_k$ and their derivatives are uniformly bounded, $F(x', u_k(x), \norm{\mathcal{D} u_k})$ are uniformly bounded either.
Then we can use Lebesgue's dominated convergence theorem, concluding $u_k \to u$ in $W^1_1(\Omega)$ and $I( u_k ) \to I( u )$.
Applying Proposition \ref{uplift}, we obtain the required result.

{\bf 2.} Consider arbitrary $u \in W^1_q(\Omega)$.
There exists a sequence of piecewise linear functions $u_k$, approximating it in $W^1_q(\Omega)$.
Indeed, since $\partial \Omega \in Lip$, $u$ can be extended to a large ball to become a finite function.
We approximate the extension by smooth finite functions.
Next we triangulate the ball, and interpolate the function linearly.
Obviously, all the functions in the process remain nonnegative.

Thus, in view of Proposition \ref{uplift}, it is enough to achieve $I( u_k ) \to I( u )$.
The proof of this convergence could be reduced to Krasnoselskii's theorem on the continuity of the Nemytskii operator (see. \cite[ch. 5, \textsection 17]{Kr}).
However, we present here the full proof for the reader's convenience.

Let us show that the weights $a_i(x', u(x))$ and $a(x, u(x))$ are bounded.
If $q \le n$, it is stated by the theorem assumptions.
Otherwise $W^1_q(\Omega)$ is embedded into $C(\overline{\Omega})$,
hence $u_k(x)$ are uniformly bounded,
and $a_i(x', u_k(x))$ and $a(x, u_k(x))$ are uniformly bounded either.
Therefore $\norm{ \mathcal{D} u_k( x ) } \le C_1 |\nabla u_k( x )|$.
In particular,
$$F( x', u_k( x ), \norm{ \mathcal{D} u_k( x ) } ) \le C_2 ( 1 + |u_k( x )|^{q^*} + |\nabla u_k( x )|^q ).$$

Consider the sets $$A_m = \{ x \in \Omega: \forall k \ge m \quad 1 + |u_k(x)|^{q^*} + |\nabla u_k( x )|^q \le 2 ( 1 + |u(x)|^{q^*} + |\nabla u( x )|^q ) \}.$$
Obviously, $A_m \subset A_{m + 1}$.
Passing to a subsequence, we can assume that $u_k \to u$ and $\nabla u_k \to \nabla u$ almost everywhere.
Then $|A_m| \to |\Omega|$, and
\begin{eqnarray*}
\mathcal{X}\{A_k\} F( x', u_k( x ), \norm{ \mathcal{D} u_k( x ) } ) &\le& 2 ( 1 + |u( x )|^{q^*} + |\nabla u( x )|^q ), \\
\mathcal{X}\{A_k\} F( x', u_k( x ), \norm{ \mathcal{D} u_k( x ) } ) &\to& F( x', u( x ), \norm{ \mathcal{D} u( x ) } )
\end{eqnarray*}
almost everywhere.
By the Sobolev embedding theorem we get $\Vert u_k \Vert_{q^*} \le C_3 \Vert u_k \Vert_{W^1_q}$.
Thus, we found an integrable majorant for $\mathcal{X}\{A_k\} F( x', u_k( x ), \norm{ \mathcal{D} u_k( x ) } )$, and
$$\int\limits_{A_k} \mathcal{X}\{A_k\} F( x', u_k( x ), \norm{ \mathcal{D} u_k( x ) } ) dx \to I( u )$$
follows by Lebesgue's dominated convergence theorem.

Next, we estimate the remainder:
\begin{multline*}
\int\limits_{\Omega \setminus A_k} F( x', u_k( x ), \norm{\mathcal{D} u_k( x )} ) dx
\le \int\limits_{\Omega \setminus A_k} C_2 ( 1 + |u_k( x )|^{q^*} + |\nabla u_k( x )|^q ) dx \\
\le C_4 \Big( \int\limits_{\Omega \setminus A_k} ( 1 + |u( x )|^{q^*} + |\nabla u( x )|^q ) dx
+ \int\limits_{\Omega \setminus A_k} ( 1 + |u( x ) - u_k( x )|^{q^*} + |\nabla ( u - u_k )( x )|^q ) \Big) dx.
\end{multline*}

The first term tends to zero due to the absolute continuity of the integral.
And the second term satisfies
\begin{multline*}
\int\limits_{\Omega \setminus A_k} ( 1 + |u( x ) - u_k( x )|^{q^*} + |\nabla ( u - u_k )( x ) )|^q ) dx \\
\le ( | \Omega \setminus A_m( k ) | + \Vert u - u_k \Vert_{W^1_q}^{q^*} + \Vert u - u_k \Vert_{W^1_q}^q ) \to 0.
\end{multline*}

Thus, the convergence $I( u_k ) \to I( u )$ is proved.
\end{proof}

The following theorem is proved similarly taking into account Remarks \ref{lanNec} and \ref{lanLin}.
\begin{thm}
Suppose $u(\cdot, -1) \equiv 0$ and the function $a(x', \cdot, u)$ satisfies condition $(\ref{almostConcave})$ for any $(x', u)$.
Then the conclusions of Theorem \ref{mainThm} hold.
\end{thm}

\vskip 40pt

I am grateful to Professor A.~I.~Nazarov for the problem statement, multiple valuable comments and for the encouragement.
Also, I am grateful to Professor V.~G.~Osmolovsky for comments which helped to improve the text.

This work was supported by RFBR grant 15-01-07650.

\begin{thebibliography}{99}
\bibitem{LL} E.~Lieb, M.~Loss: Analysis, second edition, American Mathematical Soc., 2001. 346~pp.
\bibitem{Kawohl} B.~Kawohl, ``Rearrangements and convexity of level sets in PDE'',
Lecture notes in mathematics {\bf1150}. Berlin; Springer Verlag, 134~p. (1985).
\bibitem{Br} F.~Brock, ``Weighted Dirichlet-type inequalities for Steiner symmetrization'',
Calc. Var. and PDEs~{\bf8}, 15--25 (1999).
\bibitem{1dim} S.~V.~Bankevich, A.~I.~Nazarov, ``On monotonicity of some functionals under rearrangements'', Calc. Var. and PDEs~{\bf53}, No. 3, 627--647 (2015)
(http://link.springer.com/article/10.1007/s00526-014-0761-6).
\bibitem{DAN} S.~Bankevich, A.~Nazarov, ``A generalization of the P\'olya--Szeg\"o inequality for one-dimensional functionals'',
Doklady RAN~{\bf438}, N1, p.~11--13 (2011) (in Russian).
English translation in:
Doklady Mathematics~{\bf83}, N3, p.~287--289 (2011).
\bibitem{Lan} R.~Landes, ``Some remarks on rearrangements and functionals with non-constant density'',
Math.~Nachr.~{\bf280}, N5--6, 560--570 (2007).
\bibitem{Kr} M.~A.~Krasnoselskii, E.~I.~Pustylnik, P.~E.~Sobolevskii, P.~P.~Zabreiko ``Integral Operators in Spaces of Summable Functions'',
Nauka, Moscow., 500~p. (1966) (in Russian).
English translation in:
Noordhoff International Publishing, Leyden, 520~p. (1976).
\end{thebibliography}

\end{document}
