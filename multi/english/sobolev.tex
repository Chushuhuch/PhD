\section{Inequality (\ref{toprove}) for Sobolev functions}

The following propositions are proved in \cite{1dim} in one-dimensional case.
Their proofs in the multidimensional case do not differ.
\begin{prop} {\rm \cite[Lemma 5]{1dim}.}
Functional $I( u )$ is weakly lower semicontinuous in $W^1_1(\Omega)$.
\end{prop}

\begin{prop}
\label{uplift}
{\rm \cite[Lemma 6]{1dim}.}
Let $A \subset W^1_1(\Omega)$. Let $B \subset A$ such that
$\forall v \in B$ ��������� $I( v^* ) \le I( v )$. Suppose that for every $u \in A$
there is a sequence $u_k \in B$ such that $u_k \to u$ in $W^1_1(\Omega)$ and $I( u_k ) \to I( u )$.
Then $I( u^* ) \le I( u )$ for any $u \in A$.
\end{prop}

\begin{thm}
\label{mainThm}
Suppose function $a(x', \cdot, u)$ is even and satisfies condition $(\ref{almostConcave})$ for any $(x', u)$.
Then

{\bf 1.} Inequality (\ref{toprove}) holds for any nonnegative $u \in Lip(\Omega)$.

{\bf 2.} Suppose in addition that $\partial \omega \in Lip$ and
for any $x' \in \omega, z \in \Real_+, p \in \Real$
$$F( x', z, p ) \le C ( 1 + |z|^{q^*} + |p|^q ),$$
where $\frac{1}{q^*} = \frac{1}{q} - \frac{1}{n}$  if $q < n$, and $q^*$ is arbitrary otherwise.
If $q \le n$ we also assume that the weights $a$ and $a_i$ are bounded.
Then inequality (\ref{toprove}) holds for any nonnegative $u \in W^1_q(\Omega)$.
\end{thm}

\begin{proof}
{\bf 1.} We can approximate a Lipschitz function $u$ with piecewise linear functions $u_k$ with its derivative almost everywhere.
Since $u_k$ and their derivatives are uniformly bounded, $F(x', u_k(x), \norm{\mathcal{D} u_k})$ are uniformly bounded either.
Then we can use Lebesgue's dominated convergence theorem, concluding $u_k \to u$ in $W^1_1(\Omega)$ and $I( u_k ) \to I( u )$.
Applying Proposition \ref{uplift}, we obtain the required result.

{\bf 2.} Consider an arbitrary $u \in W^1_q(\Omega)$.
There exists a sequence of piecewise linear functions $u_k$, approximating it in $W^1_q(\Omega)$.
Indeed, since $\partial \Omega \in Lip$, $u$ can be extended to a large ball to become a finite function.
We approximate the extension by smooth finite functions.
Next we triangulate the ball, and interpolate the function linearly.
Obviously, all the functions in the process remain nonnegative.

Thus, in view of Proposition \ref{uplift}, it is enough to achieve $I( u_k ) \to I( u )$.
The proof of this convergence could be reduced to the Krasnoselsky's theorem on the continuity of Nemytskij operator (see. \cite[ch. 5, \textsection 17]{Kr}).
However, we present here the full proof for the reader's convenience.

Let us show that the weights $a_i(x', u(x))$ and $a(x, u(x))$ are bounded.
If $q \le n$, it is stated by the theorem assumptions.
Otherwise $W^1_q(\Omega)$ is embedded into $C(\overline{\Omega})$,
hence $u_k(x)$ are uniformly bounded,
and $a_i(x', u_k(x))$ and $a(x, u_k(x))$ are uniformly bounded either.
Therefore $\norm{ \mathcal{D} u_k( x ) } \le C_1 |\nabla u_k( x )|$.
In particular,
$$F( x', u_k( x ), \norm{ \mathcal{D} u_k( x ) } ) \le C_2 ( 1 + |u_k( x )|^{q^*} + |\nabla u_k( x )|^q ).$$

Consider the sets $$A_m = \{ x \in \Omega: \forall k \ge m \quad 1 + |u_k(x)|^{q^*} + |\nabla u_k( x )|^q \le 2 ( 1 + |u(x)|^{q^*} + |\nabla u( x )|^q ) \}.$$
Obviously, $A_m \subset A_{m + 1}$.
Passing to a subsequence, we can assume that $u_k \to u$ and $\nabla u_k \to \nabla u$ almost everywhere.
Then $|A_m| \to |\Omega|$, and
\begin{eqnarray*}
\mathcal{X}\{A_k\} F( x', u_k( x ), \norm{ \mathcal{D} u_k( x ) } ) &\le& 2 ( 1 + |u( x )|^{q^*} + |\nabla u( x )|^q ), \\
\mathcal{X}\{A_k\} F( x', u_k( x ), \norm{ \mathcal{D} u_k( x ) } ) &\to& F( x', u( x ), \norm{ \mathcal{D} u( x ) } )
\end{eqnarray*}
almost everywhere.
By the Sobolev embedding theorem we get $\Vert u_k \Vert_{q^*} \le C_3 \Vert u_k \Vert_{W^1_q}$.
Thus, we found an integrable majorant for $\mathcal{X}\{A_k\} F( x', u_k( x ), \norm{ \mathcal{D} u_k( x ) } )$, and
$$\int\limits_{A_k} \mathcal{X}\{A_k\} F( x', u_k( x ), \norm{ \mathcal{D} u_k( x ) } ) dx \to I( u )$$
follows by Lebesgue's dominated convergence theorem.

Next, we estimate the remainder:
\begin{multline*}
\int\limits_{\Omega \setminus A_k} F( x', u_k( x ), \norm{\mathcal{D} u_k( x )} ) dx
\le \int\limits_{\Omega \setminus A_k} C_2 ( 1 + |u_k( x )|^{q^*} + |\nabla u_k( x )|^q ) dx \\
\le C_4 \Big( \int\limits_{\Omega \setminus A_k} ( 1 + |u( x )|^{q^*} + |\nabla u( x )|^q ) dx
+ \int\limits_{\Omega \setminus A_k} ( 1 + |u( x ) - u_k( x )|^{q^*} + |\nabla ( u - u_k )( x )|^q ) \Big) dx.
\end{multline*}

The first term tends to zero due to the absolute continuity of the integral.
And the second term satisfies
\begin{multline*}
\int\limits_{\Omega \setminus A_k} ( 1 + |u( x ) - u_k( x )|^{q^*} + |\nabla ( u - u_k )( x ) )|^q ) dx \\
\le ( | \Omega \setminus A_m( k ) | + \Vert u - u_k \Vert_{W^1_q}^{q^*} + \Vert u - u_k \Vert_{W^1_q}^q ) \to 0.
\end{multline*}

Thus, the convergence $I( u_k ) \to I( u )$ is proved.
\end{proof}

The following theorem is proved similarly taking into account Remarks (\ref{lanNec}) and (\ref{lanLin}).
\begin{thm}
Suppose $u(\cdot, -1) \equiv 0$ and the function $a(x', \cdot, u)$ satisfies condition $(\ref{almostConcave})$ for any $(x', u)$.
Then the conclusions of Theorem \ref{mainThm} hold.
\end{thm}