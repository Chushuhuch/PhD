\section{Transition to Sobolev functions}

The following assertions are proved in the one in \cite{1dim}.
Evidence in the multidimensional case are quite similar.
\begin{prop} {\rm \cite[Lemma 5]{1dim}.}
Functional $I( u )$ is weakly lower semicontinuous in $W^1_1(\Omega)$.
\end{prop}

\begin{prop}
\label{uplift}
{\rm \cite[Lemma 6]{1dim}.}
Let $A \subset W^1_1(\Omega)$. Let $B \subset A$ such that
$\forall v \in B$ ��������� $I( v^* ) \le I( v )$. Suppose that for every $u \in A$
there is a sequence $u_k \in B$ such that $u_k \to u$ in $W^1_1(\Omega)$ and
$I( u_k ) \to I( u )$. Then $\forall u \in A$ will be executed $I( u^* ) \le I( u )$.
\end{prop}

\begin{thm}
\label{mainThm}
Let the function $a(x', \cdot, u)$ is even and satisfies the condition $(\ref{almostConcave})$ for all $x'$ and $u$.
Then

{\bf 1.} Inequality (\ref{toprove}) is true for any non-negative $u \in Lip(\Omega)$.

{\bf 2.} Suppose that $\partial \omega \in Lip$ and
for any $x' \in \omega, z \in \Real_+, p \in \Real$
function $F$ satisfies
$$F( x', z, p ) \le C ( 1 + |z|^{q^*} + |p|^q ),$$
where $\frac{1}{q^*} = \frac{1}{q} - \frac{1}{n}$, if $q < n$, or $q^*$, any otherwise.
If $q \le n$, then we assume that the weight $a$ and $a_i$ limited.
Then inequality (\ref{toprove}) is true for any non-negative $u \in W^1_q(\Omega)$.
\end{thm}
\begin{proof}
{\bf 1.} We can bring Lipschitz $u$ piecewise linear functions $u_k$ along with the derivatives almost everywhere.
Since $u_k$ uniformly bounded together with its derivatives,
then $F(x', u_k(x), \norm{\mathcal{D} u_k})$ are uniformly bounded.
Then we can use the theorem of Lebesgue, we received $u_k \to u$ in $W^1_1(\Omega)$ and $I( u_k ) \to I( u )$.
Using Proposition \ref{uplift}, we obtain the required result.

{\bf 2.} Consider a $u \in W^1_q(\Omega)$.
For it is possible to construct a sequence of piecewise linear functions $u_k$, approximating it in $W^1_q(\Omega)$.
In fact, since $\partial \Omega \in Lip$,
$u$ finite way, you can continue on the inside of a large bowl in the $\Real^n$
and bring smooth finite functions.
Next ball is triangulated, and the values ??of the function are linearly interpolated.
Obviously, all the process functions are nonnegative.

Then, by Proposition \ref{uplift}, it is enough to achieve $I( u_k ) \to I( u )$.
The proof of this convergence can be reduced to the theorem on the continuity of Krasnoselsky
Nemytskij operator (see. \cite[��. 5, \textsection 17]{Kr}).
However, for the reader's convenience, we present here the whole argument.

We show that the weight $a_i(x', u(x))$ and $a(x, u(x))$ restricted.
If $q \le n$, it is performed on the assumption of the theorem. If not, then $W^1_q(\Omega)$ imbedded in $C(\overline{\Omega})$,
thus, $u_k(x)$ are uniformly bounded, and hence $a_i(x', u_k(x))$ and $a(x, u_k(x))$ uniformly bounded.
Therefore $\norm{ \mathcal{D} u_k( x ) } \le C_1 |\nabla u_k( x )|$.
I.e,
$$F( x', u_k( x ), \norm{ \mathcal{D} u_k( x ) } ) \le C_2 ( 1 + |u_k( x )|^{q^*} + |\nabla u_k( x )|^q ).$$

Consider the sets $A_m$, consisting of $x \in \Omega$, that for all $k \ge m$ fulfilled
$1 + |u_k(x)|^{q^*} + |\nabla u_k( x )|^q \le 2 ( 1 + |u(x)|^{q^*} + |\nabla u( x )|^q )$.
Obviously, $A_m \subset A_{m + 1}$.
Passing to a subsequence, we can assume that $u_k \to u$ � $\nabla u_k \to \nabla u$ almost everywhere.
So $|A_m| \to |\Omega|$.
Then
\begin{eqnarray*}
\mathcal{X}\{A_k\} F( x', u_k( x ), \norm{ \mathcal{D} u_k( x ) } ) &\le& 2 ( 1 + |u( x )|^{q^*} + |\nabla u( x )|^q ), \\
\mathcal{X}\{A_k\} F( x', u_k( x ), \norm{ \mathcal{D} u_k( x ) } ) &\to& F( x', u( x ), \norm{ \mathcal{D} u( x ) } )
\end{eqnarray*}
almost everywhere.
By embedding theorem $\Vert u_k \Vert_{q^*} \le C_3 \Vert u_k \Vert_{W^1_q}$.
Thus, we found an integrable majorant and get
$\int_{A_k} \mathcal{X}\{A_k\} F( x', u_k( x ), \norm{ \mathcal{D} u_k( x ) } ) dx \to I( u )$ by Theorem Lebesgue.

We now estimate the content:
\begin{multline*}
\int_{\Omega \setminus A_k} F( x', u_k( x ), \norm{\mathcal{D} u_k( x )} ) dx
\le \int_{\Omega \setminus A_k} C_2 ( 1 + |u_k( x )|^{q^*} + |\nabla u_k( x )|^q ) dx \\
\le C_4 \Big( \int_{\Omega \setminus A_k} ( 1 + |u( x )|^{q^*} + |\nabla u( x )|^q ) dx
+ \int_{\Omega \setminus A_k} ( 1 + |u( x ) - u_k( x )|^{q^*} + |\nabla ( u - u_k )( x )|^q ) \Big) dx.
\end{multline*}

The first term tends to zero in the absolute continuity of the integral.
For the second term is made
\begin{multline*}
\int_{\Omega \setminus A_k} ( 1 + |u( x ) - u_k( x )|^{q^*} + |\nabla ( u - u_k )( x ) )|^q ) dx \\
\le ( | \Omega \setminus A_m( k ) | + \Vert u - u_k \Vert_{W^1_q}^{q^*} + \Vert u - u_k \Vert_{W^1_q}^q ) \to 0.
\end{multline*}

Thus, the convergence of $I( u_k ) \to I( u )$ is proved.
\end{proof}

Similarly, based on the comments (\ref{lanNec}) and (\ref{lanLin}), proved
\begin{thm}
Let $u(\cdot, -1) \equiv 0$ and the function $a(x', \cdot, u)$ satisfies the condition $(\ref{almostConcave})$ for all $x'$ and $u$.
Then the true conclusions of Theorem \ref{mainThm}.
\end{thm}