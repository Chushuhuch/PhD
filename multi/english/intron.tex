\section{Introduction}
Let $\Omega = \omega \times [-1,1]$
where $\omega$ is a bounded domain in $\Real^{n - 1}$.
Also let $x = ( x_1, \dots, x_{n - 1}, y ) = ( x', y )$.

Recall the layer cake representation for a measurable function $u: \Omega \to \Real_+$
(see \cite[Theorem 1.13]{LL}).
Hereinafter $\Real_+ = [0, \infty)$.
Let $\mathcal{A}_t(x') := \{ y \in [-1,1]:\ u( x', y ) > t \}$.
Then
$$u(x', y) = \int_0^\infty \mathcal{X}\{\mathcal{A}_t(x')\}(y) dt,$$
where $\mathcal{X}\{A\}$ is a characteristic function of a set $A$.

Define the monotone rearrangement of a measurable set $E \subset [-1, 1]$ and
the monotone rearrangement of a nonnegative function $u \in \W(\Omega)$ as follows:
\begin{eqnarray*}
E^* := [1 - \abs{E}, 1]; \qquad
u^*(x', y) = \int_0^\infty \mathcal{X}\{ ( \mathcal{A}_t(x') )^* \}(y) dt.
\end{eqnarray*}

We define the set $\mathfrak{F}$ of continuous functions $F: \omega \times \Real_+ \times \Real_+ \to \Real_+$,
which are convex and increasing in the third argument, and satisfy $F( \cdot, \cdot, 0 ) \equiv 0$.

Consider the following functional:
\begin{equation}
\label{functional}
I( u ) = \int_\Omega F( x', u(x), \norm{ \mathcal{D} u } ) dx,
\end{equation}
where $F \in \mathfrak{F}$ and
$\norm{\cdot}$ is a norm in $\Real^n$, which is symmetric in the last coordinate,
i.e. satisfying $\norm{( x', y )} = \norm{( x', -y )}$ for any $( x',  y ) \in \Real^n$.
$$\mathcal{D} u = ( a_1( x', u( x ) ) D_1 u, \dots, a_{n - 1}( x', u( x ) ) D_{n - 1} u, a( x, u( x ) ) D_n u )$$
is a weighted gradient of $u$ (note that only the weight of $D_n u$ depends on $y$).
$a( \cdot, \cdot ): \Omega \times \Real_+ \to \Real_+$ and $a_i( \cdot, \cdot ): \omega \times \Real_+ \to \Real_+$ are continuous functions.
Hereinafter, the index $i$ runs from $1$ to $n - 1$.

It is well known that for $a_i = a \equiv 1$ the following inequality holds 
\begin{equation}
\label{toprove}
I( u^* ) \le I( u ), \qquad \qquad u \in \W(\Omega), u \ge 0
\end{equation}
(see e.g. \cite{Kawohl} and references therein).

In this paper we present the necessary conditions on weight $a$ for inequality (\ref{toprove}) to hold.
Also we prove inequality (\ref{toprove}) under certain additional restrictions.

The author of paper \cite{Br} considered an inequality similar to (\ref{toprove}).
Namely,
\begin{equation}
\label{breq}
I( \overline{u} ) \le I( u ), \qquad \qquad u \in \Wf( \Omega ),\ u \ge 0,
\end{equation}
where $\overline{u}$ stands for Steiner symmetrization (symmetrical rearrangement) of function $u$ with respect to $y$.
The author of \cite{Br} proves the inequality for weight $a$, even and convex with respect to $y$.
To do this inequality (\ref{breq}) is proved for piecewise linear $u$.
It is further claimed that in the general case the inequality is obtained by approximation.
However, the passage to the limit is not properly justified in \cite{Br},
and in fact the result was obtained only for Lipschitz $u$.
Note that the article \cite{Br} has only the weight of $D_n u$ depend on $y$ either,
this restriction appears to be essential.

Inequalities (\ref{toprove}) and (\ref{breq}) in a one-dimensional case are considered in the work \cite{1dim}.
It establishes the necessary and sufficient conditions for the inequalities.
In particular, the gap in \cite{Br} was closed for $n = 1$.
Under certain additional restrictions the result was announced in \cite{DAN}.

The article \cite{Lan} considered inequality (\ref{toprove}) for a similar functional in the two-dimensional case
with the additional restriction of $u(\cdot, -1) \equiv 0$.
We note that our conditions on the weight functions are weaker than in \cite{Lan}.

The article is divided into four sections.
\S2 contains necessary conditions for inequality (\ref{toprove}).
\S3 contains the proof of inequality (\ref{toprove}) for piecewise linear functions $u$.
\S4 contains the proof of the inequality in more general cases.