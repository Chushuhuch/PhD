\section{The transition to $\W$ functions with an additional restriction on weight}
\label{ASC}

In this section we obtain the inequality (\ref{toprove}) under the additional condition
monotony of the weight function at $x \in [-1, 0]$ and $x \in [0, 1]$.

\begin{lm}
\label{Wapprox}
Let $a$ --- continuous function, $a(\cdot, u)$ is increasing on $[-1, 0]$ and decreasing on $[0, 1]$ for all $u \ge 0$.
Then any function $u \in \W(-1, 1)$, $u \ge 0$,
Lipschitz functions approaching the functional $I$,
then there exists a sequence $u_k \in Lip[-1, 1]$, such that the relation $(\ref{convergence})$.
\end{lm}

\begin{proof}
We can assume that $I( a, u ) < \infty$.

Weight increases $a$ in $x$ at $x \in [-1, 0]$
and decreases for $x \in [0, 1]$.
We prove the assertion for the functional
$$I_2( u ) = \int_0^1 F( u(x), a(x, u(x)) |u'(x)| ) dx,$$
and the case of $[-1, 0]$ to first reduce the $$I_1( u ) = \int_{-1}^0 F( u( x ), a( x, u(x) ) |u'(x)| ) dx =
\int_0^1 F( u( -z ), a( -z, u(-z) ) |u'( -z )| ) dz.$$

To prove this, we will modify the schema from \cite [ Theorem 2.4]{ASC}.
Proof overlaps with \cite{ASC}, but for the reader's convenience, we present it here in full.

We need the following auxiliary assertion.

\begin{prop}
\label{convToOne}
$\cite[Lemma 2.7]{ASC}$.
Let $\phi_h: [-1, 1] \to \Real$ --- a sequence of Lipschitz functions satisfying the conditions :
$\phi_h' \ge 1$ for almost all $x$ and all $h$, $\phi_h( x ) \to x$ for almost every $x$.
Then, for any $f \in L_1(\Real)$ $f(\phi_h) \to f$ in $L_1(\Real)$.
\end{prop}

For $h \in \Nat$ cover the set $\{ x \in [0, 1]: |u'(x)| > h \}$
open set $A_h$.
Without loss of generality, we can assume that
$A_{h + 1} \subset A_{h}$ and $\abs{A_h} \to 0$ for $h \to \infty$.
As we take $v_h$ function coinciding with $u$ is the set $A_h$.
On connected sites will make $A_h$ $v_h$ linear.
Then $v_h \to u$ in $\W$.
Change little $v_h$, to make an approximation of Lipschitz.

Imagine $A_h = \cup_k \Omega_{h,k}$, where $\Omega_{h,k} = ( b_{h,k}^-, b_{h,k}^+ )$.
denote
$$\alpha_{h,k} := \abs{\Omega_{h,k}},\quad
\beta_{h,k} := v_h(b_{h,k}^+) - v_h(b_{h,k}^-) = u(b_{h,k}^+) - u(b_{h,k}^-).$$
Then $v'_h = \frac{\beta_{h,k}}{\alpha_{h,k}}$ in $\Omega_{h,k}$.
Note that
$$\sum_k \abs{\beta_{h,k}} \le \int_{A_h} \abs{u'} dx \le \norm{u'}_{L_1(-1, 1)}< \infty,$$
and hence
$\sum_k \abs{\beta_{h,k}} \to 0$ for $h \to 0$ by Lebesgue.

We define a function $\phi_h \in \W(0, 1)$ as follows:
$$
\begin{aligned}
\phi_h( 0 ) &= 0 & & \\
\phi_h' &=  1 & \text{ in } & [0, 1] \setminus A_h,\\
\phi_h' &=  \max \Big( \frac{ \abs{\beta_{h,k}} }{ \alpha_{h,k} }, 1 \Big) & \text{ in } & \Omega_{h,k}.
\end{aligned}
$$	

Note that $\int_0^1 \abs{\phi_h'} dx \le 1 + \sum_k \abs{\beta_{h,k}} < \infty$.

\medskip

We show that $\phi_h' \to 1$ in $L_1(0, 1)$ :
$$\int \abs{\phi_h' - 1} dx = \sum \Big( \max \Big( \frac{\abs{\beta_{h,k}}}{\alpha_{h,k}}, 1 \Big) - 1 \Big) \alpha_{h,k} \le
\sum \abs{\beta_{h,k}} \to 0.$$
It follows that $\phi_h$ satisfies the conditions of Proposition \ref{convToOne}.

We now consider the $\phi_h^{-1}: [0, 1] \to [0, 1]$ --- restriction inverse to $\phi_h$ functions on $[0, 1]$.
For it is true $0 \le ( \phi_h^{-1} )' \le 1$ and

$$
\begin{aligned}
\phi_h^{-1} ( 0 ) &= 0 & & \\
( \phi_h^{-1} )' &=  1 & \text{ in } & [0, 1] \setminus \phi_h( A_h ),\\
( \phi_h^{-1} )' &=  \min \Big( \frac{ \alpha_{h,k} }{ \abs{ \beta_{h,k} } }, 1 \Big) & \text{ in } & [0, 1] \cap \phi_h( \Omega_{h,k} ).
\end{aligned}
$$

Take $u_h = v_h( \phi_h^{-1} )$.
Note that $u_h(0) = u(0)$, and
\begin{align*}
u_h' &=  v_h'( \phi_h^{-1} ) \cdot ( \phi_h^{-1} )' = u'( \phi_h^{-1} ) & \text{ in } & [0, 1] \setminus \phi_h( A_h ),\\
u_h' &=  v_h'( \phi_h^{-1} ) \cdot ( \phi_h^{-1} )' = 
\sign{ \beta_{h,k} } \cdot \min \Big( 1, \frac{ \abs{ \beta_{h,k} } }{ \alpha_{h,k} } \Big) & \text{ in } & [0, 1] \cap \phi_h( \Omega_{h,k} ).
\end{align*}

Thus, $u_h$ Lipschitz as $u$ is a $A_h$ bounded derivative.

We show that $u_h \to u$ in $\W(0, 1)$.
To do this, it suffices to estimate

$$\norm{u_h' - u'}_{L_1} \le \int_{[0, 1] \setminus \phi_h(A_h)} \abs{u_h' - u'} + 
\int_{[0, 1] \cap \phi_h(A_h)} \abs{u_h'} + \int_{[0, 1] \cap \phi_h(A_h)} \abs{u'} =: P_h^1 + P_h^2 + P_h^3.$$
$$P_h^1 = \int_{[0, 1] \setminus \phi_h( A_h )} \abs{u'( \phi_h^{-1} ) - u'} dx =
\int_{\phi_h^{-1} ( [0, 1] ) \setminus A_h} \abs{u' - u'( \phi_h )} dz \le
\int_{[0, 1]} \abs{u' - u'( \phi_h )} dz.$$
By Proposition \ref{convToOne}, $P_h^1 \to 0$.

Further,
$$P_h^2 \le \abs{\phi_h( A_h )} = \sum \abs{\phi_h( \Omega_{h,k} )} = \sum \max (\abs{\beta_{h,k}}, \alpha_{h,k})
\le \sum \alpha_{h,k} + \sum \abs{\beta_{h,k}} \to 0.$$
Finally, $P_h^3 \to 0$ in absolute continuity of the integral, and the assertion is proved.

It remains to show that $I_2( u_h ) \to I_2( u )$.

$$I_2( u_h ) = \!\!\!\!\int\limits_{[0, 1] \setminus \phi_h( A_h )}\!\!\!\! F( u_h( x ), a( x, u_h(x) ) |u_h'( x )| ) dx +\
\!\!\!\!\int\limits_{[0, 1] \cap \phi_h( A_h )}\!\!\!\! F( u_h( x ), a( x, u_h(x) ) |u_h'( x )| ) dx.$$
These terms denote $\hat{P_h^1}$ and $\hat{P_h^2}$.
Since $u \in \W(0, 1)$, then $u \in L_\infty( [0, 1] )$. We denote $\norm{u}_\infty = r$,
then $\norm{u_h}_\infty < 2r$ for sufficiently large $h$. Additionally, $\abs{u_h'} \le 1$
almost everywhere in $\phi_h( A_h )$. Then $\hat{P_h^2} \le M_F \abs{\phi_h( A_h )} \to 0$, where
$$M_F = \max\limits_{[-2r, 2r] \times [-M_a, M_a]} F;\quad M_a = \max\limits_{[0, 1] \times [-2r, 2r]} a.$$

Further,
\begin{multline*}
\hat{P_h^1} = \int\limits_{ [0, 1] \setminus \phi_h( A_h ) }
	F( u( \phi_h^{-1}( x ) ), a( x, u( \phi_h^{-1}( x ) ) |u'( \phi_h^{-1}( x ) ) ( \phi_h^{-1} )'| ) dx
\\ =\int\limits_{ \phi_h^{-1}( [0, 1] ) \setminus A_h } F( u( z ), a( \phi_h( z ), u( z ) ) |u'( z )| ) dz
\\ = \int\limits_{ [0, 1] } F( u( z ), a( \phi_h( z ), u( z ) ) |u'( z )| ) \chi_{ \phi_h^{-1}( [0, 1] ) \setminus A_h }dz.
\end{multline*}
The last equality, generally speaking, does not make sense, since $\phi_h( z )$ can take values is $[0, 1]$.
We define $a( z, u ) = a( 1, u )$ at $z > 1$. Now the expression correctly.
Note that $\chi_{\phi_h^{-1}( [0, 1] ) \setminus A_h}$ increases, as set
$\phi_h^{-1}( [0, 1] )$ increases and $A_h$ decay, ie
$\phi_{h_1}^{-1}( [0, 1] ) \subset \phi_{h_2}^{-1}( [0, 1] )$ and $A_{h_1} \supset A_{h_2}$ for $h_1 \le h_2$.
On $[0, 1]$ (or even $\phi_h( [0, 1] )$) $a$ decreases, then $a( \phi_h( z ) )$ is expected to grow by $h$,
since $\phi_h( z )$ is decreasing in $h$. In this case it is possible to apply the theorem
monotone convergence and get
$$\hat{P_h^1} \to \int_{[0, 1]} F( u( z ), a( z, u( z ) ) |u'( z )| ) dz.$$

\end{proof}

\begin{rem}
Obviously, the same reasoning with binding function $u$ on the left end can be spent at any interval $[x_0, x_1]$, where
weight $a$ is decreasing in $x$. That is to get over this interval sequence
\begin{gather*}
u_h \to u \text{ in } \W(x_0, x_1);\\
\int_{x_0}^{x_1} F( u_h(x), a(x, u_h(x)) \abs{u_h'(x)} ) \to \int_{x_0}^{x_1} F( u(x), a(x, u(x)) \abs{u'(x)} ).
\end{gather*}
Similarly, if $a$ is increasing in $x$, can be approximated by $u$ with fixing on the right end.
\end{rem}

\begin{cor}
Let the function $a$ is continuous, even, satisfies $(\ref{almostConcave})$
and decreasing on $[0, 1]$. Then for every $u \in \W(-1, 1)$ holds $I( a, u^* ) \le I( a, u )$.
\end{cor}

\begin{proof}
Inequality follows immediately from Lemma \ref{uplift} and \ref{Wapprox}.
\end{proof}

