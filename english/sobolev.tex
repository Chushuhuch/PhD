\section{The inequality for $u \in \W(-1, 1)$ with an additional restriction on weight}
\label{ASC}

In this section we prove the inequality (\ref{toprove}) under the additional condition:
weight is monotonic in $x$ for $x \in [-1, 0]$ and $x \in [0, 1]$.

\begin{lm}
\label{Wapprox}
Let $\mathfrak a$ be a continuous function
and let $\mathfrak a(\cdot, u)$ be increasing on $[-1, 0]$ and decreasing on $[0, 1]$ for all $u \ge 0$.
Then for any function $u \in \W(-1, 1)$, $u \ge 0$,
there exists a sequence $\{u_k\} \subset Lip[-1, 1]$, such that the relation $(\ref{convergence})$ holds.
\end{lm}

\begin{proof}
We can assume that $I( \mathfrak a, u ) < \infty$.

We prove the assertion for the functional
$$I_1( u ) = \int\limits_0^1 F\big( u(x), \mathfrak a(x, u(x)) |u'(x)| \big) \, dx,$$
and the integral over $[-1, 0]$ can be reduced to $I_1$ by changing variable.

We modify the scheme from \cite[Theorem 2.4]{ASC}.
A part of the proof overlaps with \cite{ASC}, but we present a complete proof here for the reader's convenience.

We need the following auxiliary assertion.

\begin{prop}
\label{convToOne}
{\rm \cite[Lemma 2.7]{ASC}.}
Let $\phi_h: [-1, 1] \to \Real_+$ be a sequence of Lipschitz functions satisfying the conditions:
$\phi_h' \ge 1$ for almost every $x$ and all $h$, $\phi_h( x ) \to x$ for almost every $x$.
Then for any $f \in L_1(\Real)$ we have $f(\phi_h) \to f$ in $L_1(\Real)$.
\end{prop}

For $h \in \Nat$ we cover the set $\{ x \in [0, 1]: |u'(x)| > h \}$ with an open set $A_h$.
Without loss of generality, we can assume that
$A_{h + 1} \subset A_{h}$ and $\abs{A_h} \to 0$ for $h \to \infty$.

Denote by $v_h$ the nonnegative continuous function on $[0, 1]$,
coinciding with $u$ on $[0, 1] \setminus A_h$ and
linear on intervals forming $A_h$.
Then $v_h \to u$ in $\W$.
Now we modify $v_h$ to get Lipschitz functions.

Let $A_h = \cup_k \Omega_{h,k}$, where $\Omega_{h,k} = ( b_{h,k}^-, b_{h,k}^+ )$.
Denote
$$\alpha_{h,k} := \abs{\Omega_{h,k}}, \quad
\beta_{h,k} := v_h(b_{h,k}^+) - v_h(b_{h,k}^-) = u(b_{h,k}^+) - u(b_{h,k}^-).$$
Then $v'_h = \frac{\beta_{h,k}}{\alpha_{h,k}}$ in $\Omega_{h,k}$.
Note that
$$\sum_k \abs{\beta_{h,k}} \le \int\limits_{A_h} \abs{u'} \, dx \le \norm{u'}_{L_1(-1, 1)}< \infty,$$
and hence
$\sum_k \abs{\beta_{h,k}} \to 0$ as $h \to 0$ by the Lebesgue theorem.

We define the function $\phi_h \in \W(0, 1)$ as follows:
$$
\begin{aligned}
\phi_h( 0 ) &= 0 & & \\
\phi_h' &=  1 & \text{ in } & [0, 1] \setminus A_h,\\
\phi_h' &=  \max \Big( \frac{ \abs{\beta_{h,k}} }{ \alpha_{h,k} }, 1 \Big) & \text{ in } & \Omega_{h,k}.
\end{aligned}
$$	

Note that $\int_0^1 \abs{\phi_h'} \, dx \le 1 + \sum_k \abs{\beta_{h,k}} < \infty$.

Next, $\phi_h' \to 1$ in $L_1(0, 1)$:
$$\int \abs{\phi_h' - 1} \, dx = \sum\limits_k \Big( \max \Big( \frac{\abs{\beta_{h,k}}}{\alpha_{h,k}}, 1 \Big) - 1 \Big) \alpha_{h,k} \le
\sum\limits_k \abs{\beta_{h,k}} \to 0.$$
Thus $\phi_h$ satisfies the conditions of Proposition \ref{convToOne}.

Consider now $\phi_h^{-1}: [0, 1] \to [0, 1]$ --- the restriction to $[0, 1]$ of the inverse to $\phi_h$.
Then
$$
\begin{aligned}
\phi_h^{-1} ( 0 ) &= 0 & & \\
( \phi_h^{-1} )' &=  1 & \text{ in } & [0, 1] \setminus \phi_h( A_h ),\\
( \phi_h^{-1} )' &=  \min \Big( \frac{ \alpha_{h,k} }{ \abs{ \beta_{h,k} } }, 1 \Big) & \text{ in } & [0, 1] \cap \phi_h( \Omega_{h,k} ).
\end{aligned}
$$

Let $u_h = v_h( \phi_h^{-1} )$.
Note that $u_h(0) = u(0)$, and
\begin{align*}
u_h' &=  v_h'( \phi_h^{-1} ) \cdot ( \phi_h^{-1} )' = u'( \phi_h^{-1} ) & \text{ in } & [0, 1] \setminus \phi_h( A_h ),\\
u_h' &=  v_h'( \phi_h^{-1} ) \cdot ( \phi_h^{-1} )' = 
\sign{ \beta_{h,k} } \cdot \min \Big( 1, \frac{ \abs{ \beta_{h,k} } }{ \alpha_{h,k} } \Big) & \text{ in } & [0, 1] \cap \phi_h( \Omega_{h,k} ).
\end{align*}
Thus, $u_h$ is Lipschitz since $u'$ is bounded in $[0, 1] \setminus A_h$.

We claim that $u_h \to u$ in $\W(0, 1)$. Indeed, it is sufficient to estimate

$$\norm{u_h' - u'}_{L_1} \le \int\limits_{[0, 1] \setminus \phi_h(A_h)} \abs{u_h' - u'} + 
\int\limits_{[0, 1] \cap \phi_h(A_h)} \abs{u_h'} + \int\limits_{[0, 1] \cap \phi_h(A_h)} \abs{u'} =: P_h^1 + P_h^2 + P_h^3.$$
$$P_h^1 = \int\limits_{[0, 1] \setminus \phi_h( A_h )} \abs{u'( \phi_h^{-1} ) - u'} \, dx =
\int\limits_{\phi_h^{-1} ( [0, 1] ) \setminus A_h} \abs{u' - u'( \phi_h )} \, dz \le
\int\limits_{[0, 1]} \abs{u' - u'( \phi_h )} \, dz.$$
By Proposition \ref{convToOne}, $P_h^1 \to 0$.
Further,
$$P_h^2 \le \abs{\phi_h( A_h )} = \sum\limits_k \abs{\phi_h( \Omega_{h,k} )} = \sum\limits_k \max (\abs{\beta_{h,k}}, \alpha_{h,k})
\le \sum\limits_k \alpha_{h,k} + \sum\limits_k \abs{\beta_{h,k}} \to 0.$$
Finally, $P_h^3 \to 0$ by the absolute continuity of the integral, and the assertion is proved.

It remains to show that $I_1( u_h ) \to I_1( u )$.

\begin{multline*}
I_1( u_h ) = \int\limits_{[0, 1] \setminus \phi_h( A_h )} F\big( u_h( x ), \mathfrak a( x, u_h(x) ) |u_h'( x )| \big) \, dx +\\
\int\limits_{[0, 1] \cap \phi_h( A_h )} F\big( u_h( x ), \mathfrak a( x, u_h(x) ) |u_h'( x )| \big) \, dx =: \hat{P_h^1} + \hat{P_h^2}.
\end{multline*}
Since $u \in \W(0, 1)$ then $u \in L_\infty( [0, 1] )$.
Denote $\norm{u}_\infty = r$.
Then $\norm{u_h}_\infty < 2r$ for sufficiently large $h$.
Also, $\abs{u_h'} \le 1$ almost everywhere in $\phi_h( A_h )$.
Then $\hat{P_h^2} \le M_F \abs{\phi_h( A_h )} \to 0$, where
$$M_F = \max\limits_{[-2r, 2r] \times [-M_{\mathfrak a}, M_{\mathfrak a}]} F;\quad M_{\mathfrak a} = \max\limits_{[0, 1] \times [-2r, 2r]} \mathfrak a.$$

Further,
\begin{multline*}
\hat{P_h^1} = \int\limits_{ [0, 1] \setminus \phi_h( A_h ) }
	F\big( u( \phi_h^{-1}( x ) ), \mathfrak a( x, u( \phi_h^{-1}( x ) ) |u'( \phi_h^{-1}( x ) ) ( \phi_h^{-1} )'| ) \big) \, dx
\\ =\int\limits_{ \phi_h^{-1}( [0, 1] ) \setminus A_h } F\big( u( z ), \mathfrak a( \phi_h( z ), u( z ) ) |u'( z )| \big) \, dz
\\ = \int\limits_{ [0, 1] } F\big( u( z ), \mathfrak a( \phi_h( z ), u( z ) ) |u'( z )| \big) \chi_{ \phi_h^{-1}( [0, 1] ) \setminus A_h } \, dz.
\end{multline*}
The last equality, generally speaking, does not make sense, since $\phi_h( z )$ can take values outside $[0, 1]$.
Let us define $\mathfrak a( z, u ) = \mathfrak a( 1, u )$ for $z > 1$. Now the expression is correct.
Note that $\chi_{\phi_h^{-1}( [0, 1] ) \setminus A_h}$ increases,
since sets $\phi_h^{-1}( [0, 1] )$ increase and sets $A_h$ decay,
that is $\phi_{h_1}^{-1}( [0, 1] ) \subset \phi_{h_2}^{-1}( [0, 1] )$ and $A_{h_1} \supset A_{h_2}$ for $h_1 \le h_2$.
Since $\mathfrak a$ is decreasing on $[0, 1]$ (in fact, on $\phi_h( [0, 1] )$) and $\phi_h( z )$ is decreasing in $h$,
then $\mathfrak a( \phi_h( z ) )$ is increasing in $h$.
We apply the monotone convergence theorem and get
$$\hat{P_h^1} \to \int\limits_{[0, 1]} F\big( u( z ), \mathfrak a( z, u( z ) ) |u'( z )| \big) \, dz.$$

\end{proof}

\begin{rem}
Obviously, the proof works for any interval $[x_0, x_1]$ with function $u$ pinned at $x_0$,
provided the weight $\mathfrak a$ is decreasing in $x$ on $[x_0, x_1]$.
That is there exists $\{u_h\}$, such that
\begin{gather*}
u_h(x_0) = u(x_0); \qquad u_h \to u \text{ in } \W(x_0, x_1);\\
\int\limits_{x_0}^{x_1} F\big( u_h(x), \mathfrak a(x, u_h(x)) \abs{u_h'(x)} \big) \to \int\limits_{x_0}^{x_1} F\big( u(x), \mathfrak a(x, u(x)) \abs{u'(x)} \big).
\end{gather*}
Similarly, if $\mathfrak a$ is increasing in $x$, the same works for functions $u$ pinned at the right end of the segment.
\end{rem}

\begin{cor}
Suppose that the function $\mathfrak a$ is continuous, even in $x$, decreasing on $[0, 1]$ and satisfies $(\ref{almostConcave})$.
Then for every $u \in \W(-1, 1)$ the inequality (\ref{toprove}) holds.
\end{cor}

\begin{proof}
The statement follows from Lemmata \ref{uplift} and \ref{Wapprox} immediately.
\end{proof}

