\section{Introduction}

First, we recall the layer cake representation for a measurable function $u: [-1, 1] \to \Real_+$
(here and elsewhere $\Real_+ = [0,\infty)$).
Namely, if we set $\mathcal{A}_t: = \{x \in [-1,1]:\ u(x)> t \}$
then $u(x) = \int_0^\infty \chi_{\mathcal{A}_t} \, dt$.

We define the monotone rearrangement of a measurable set $E \subset [-1, 1]$ and the
monotone rearrangement of a non-negative function $u \in \W (-1, 1)$ as follows:
\begin{eqnarray*}
E^*: = [1 - \abs{E}, 1]; \qquad
u^*(x): = \int\limits_0^\infty \chi_{\mathcal{A}_t^*} \, dt.
\end{eqnarray*}

Under the same conditions we define the symmetric rearrangement 
(symmetrization) for sets and functions:
\begin{eqnarray*}
\overline{E} := [-\frac{\abs{E}}{2}, \frac{\abs{E}}{2}]; \qquad
\overline{u}(x) := \int\limits_0^\infty \chi_{\overline{\mathcal{A}_t}} \, dt.
\end{eqnarray*}

We denote by $\mathfrak{F}$ the set of continuous functions 
$F: \Real_+ \times \Real_+ \to \Real_+,$
which are convex and increasing with respect to the second argument.

Let us consider a functional
\begin{equation}
\label{functional}
I(\mathfrak a, u) = \int\limits_{-1}^1 F\big(u(x), \mathfrak a(x, u(x)) \abs{u'(x)}\big) \, dx,
\end{equation}
where $\mathfrak a: [-1, 1] \times \Real_+ \to \Real_+$ is a continuous function, $F \in \mathfrak{F}$.

It is well known that if $\mathfrak a \equiv const$ then the P\'olya-Szeg\"o type inequalities
\begin{eqnarray}
\label{toprove}
I(\mathfrak a, u^*) & \le & I(\mathfrak a, u), \qquad \qquad u \in \W(-1, 1);\\
\label{toproveSymm}
I(\mathfrak a, \overline{u}) & \le & I(\mathfrak a, u), \qquad \qquad u \in \Wf(-1, 1)
\end{eqnarray}
hold, see for example \cite{Kawohl} and references therein.

The inequality (\ref{toproveSymm}) and its multi-dimensional analogue
are proved in \cite{Br} provided that the function $\mathfrak a$ is even and convex 
with respect to $x$. However, the proof contains a gap,
and in fact this inequality was proved in \cite{Br} only for Lipschitz functions $u$.

Namely, while proving the inequality (\ref{toproveSymm}) for a natural class of functions,
the author of \cite{Br} approximates $u \in \Wf$ with finite integral (\ref{functional})
using piecewise linear functions $u_k$ and claims that $I(\mathfrak a, u_k) \to I(\mathfrak a, u)$.
However, this assertion is not justified and generally speaking is not true.
In 1926, M.~A.~Lavrentiev proposed the first example of an integral functional
for which the infimum over the domain is strictly less than the infimum over the set of 
Lipschitz functions.
Historical overview and simple examples of ``one-dimensional'' 
functionals for which the Lavrentiev phenomenon takes place can be found e.g. in \cite{BGH}.
Note that a deep investigation of the Lavrentiev phenomenon for some classes of multidimensional 
functionals was carried out by V.~V.~Zhikov (see, e.g., \cite{Zh1}, \cite{Zh2}).

In the paper \cite{ASC} the absence of the Lavrentiev phenomenon was proved for the functionals
$I(\mathfrak a, u) = \int_{-1}^1 F(u, u')$. Moreover it was shown that for every $u \in \W(-1, 1)$ 
there exists a sequence of Lipschitz functions $u_k$, such that
\begin{equation}
\label{convergence}
u_k \to u \text{ in } \W (-1, 1) \quad \text{ and } \quad I (\mathfrak a, u_k) \to I (\mathfrak a, u).
\end{equation}

We modify the proof from \cite{ASC} and prove the absence of the Lavrentiev phenomenon
for the functionals of the form (\ref{functional}).
This allows us to fill the gap in the proof from \cite{Br} in one-dimensional case.
In addition we prove that evenness and convexity of the weight is a necessary condition
for the inequality (\ref{toproveSymm}) to hold.

The bulk of our paper is devoted to the inequality (\ref{toprove}).
We find necessary and sufficient conditions on the weight $\mathfrak a$ for the inequality 
(\ref{toprove}) to hold%
\footnote{In particular, the inequality is satisfied if the weight function $\mathfrak a$ is even 
and concave in $x$.}.
Under certain additional assumptions this result was announced in \cite{DAN}.

We note also that the inequality (\ref{toprove}) was considered in \cite{Lan}
for functionals similar to (\ref{functional}) under additional constraint $u(-1) = 0$.
We obtain necessary and sufficient conditions for (\ref{toprove}) under this constraint.
(The author of \cite{Lan} assumed that the weight $\mathfrak a$ was decreasing in $x$.)

The article is divided into 8 sections.
In Section 2 we deduce the assumptions on the weight function $\mathfrak a$ which are necessary for the 
inequality (\ref{toprove}).
Auxiliary statements for weights satisfying necessary conditions are established in Section 3.
In Section 4 the inequality (\ref{toprove}) is proved for piecewise linear functions $u$.
In Section 5 we present the scheme for proving inequality (\ref{toprove}) for a wider class of 
functions $u$.
In Section 6 we prove inequality (\ref{toprove}), provided that the weight $\mathfrak a$ first increases, 
then decreases.
Section 7 is devoted to the proof of (\ref{toprove}) under necessary conditions only.
Finally, in the Section 8 we deal with symmetric rearrangement.
There we obtain necessary conditions on the weight and complete the proof of (\ref{toproveSymm}).
