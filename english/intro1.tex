\section{Introduction}
Recall that for a measurable non-negative function there is a layer cake representation.
Namely, we set $\mathcal{A}_t: = \{x \in [-1,1]:\ u(x)> t \}$.
Then we have the equation $u(x) = \int_0^\infty \chi_{\mathcal{A}_t}dt$.

Let us define a monotone rearrangement of a measurable set $E \subset [-1, 1]$ and
monotone rearrangement of non-negative function $u \in \W (-1, 1)$ :
\begin{eqnarray*}
E^*: = [1 - \abs{E}, 1] \qquad
u^*(x): = \int_0^\infty \chi_{\mathcal{A}_t^*}dt
\end{eqnarray*}

Under the same conditions we can define a symmetric rearrangement (or symmetrization) for sets and functions:
\begin{eqnarray*}
\overline{E} := [-\frac{\abs{E}}{2}, \frac{\abs{E}}{2}] \qquad
\overline{u}(x) := \int_0^\infty \chi_{\overline{\mathcal{A}_t}}dt
\end{eqnarray*}

Introduce the set $\mathfrak{F}$ of convex and increasing in the second argument
continuous functions $F: \Real_+ \times \Real_+ \to \Real_+$
(where $\Real_+ = [0,\infty)$ here and below).

Consider the functional :
\begin{equation}
\label{functional}
I(a, u) = \int_{-1}^1 F(u(x), a(x, u(x)) \abs{u'(x)}) dx,
\end{equation}
where $a: [-1, 1] \times \Real_+ \to \Real_+$ is a continuous function, $F \in \mathfrak{F}$.

It is well known that if $a \equiv const$ the inequalities
\begin{eqnarray}
\label{toprove}
I(a, u^*) & \le & I(a, u), \\
\label{toproveSymm}
I(a, \overline{u}) & \le & I(a, u)
\end{eqnarray}
hold (see for example \cite{Kawohl} and references therein).

The inequality (\ref{toproveSymm}) and its multi-dimensional analogue
are proved in \cite{Br}
provided that the function $a$ is even and convex with respect to $x$.
However, the proof contains a gap,
and in fact this inequality is proved only for Lipschitz functions $u$.

Namely, while proving the inequality (\ref{toproveSymm}) for a natural class of functions
the author of \cite{Br} approximates $u \in \Wf$ having finite integral (\ref{functional})
using piecewise linear functions $u_k$ and argues that $I(a, u_k) \to I(a, u)$.
However, this assertion is not substantiated and in general is even incorrect.
In 1926 M.~A.~Lavrentiev proposed the first example of an integral functional
for which the infimum over the domain is strictly less than the infimum over the set of Lipschitz functions.
For historical overview and simple examples of ``one-dimensional'' functionals
for which the Lavrentiev phenomenon takes place see \cite{BGH} for example.
Note that a deep study of the Lavrentiev phenomenon for some classes of multidimensional functionals
was conducted by V.~V.~Zhikov (see for example \cite{Zh1}, \cite{Zh2}).

In the paper \cite{ASC} the Lavrentiev phenomenon is proved not to take place for the functionals
$I(a, u) = \int_{-1}^1 F(u, u')$.
Moreover it is shown that for every $u \in \W(-1, 1)$ there exists a sequence
of Lipschitz functions $u_k$, such that
\begin{equation}
\label{convergence}
u_k \to u \text{ in } \W (-1, 1) \quad \text{ and } \quad I (a, u_k) \to I (a, u).
\end{equation}

We modify the proof from \cite{ASC} and prove the absence of the Lavrentiev phenomenon
for the functionals of the form (\ref{functional}).
This makes it possible to fill a gap in the proof from \cite{Br} for the one-dimensional case.
In addition we prove that evenness and convexity of the weight is a necessary and sufficient condition
for the inequality (\ref{toproveSymm}) to hold.

The bulk of the paper is devoted to the inequality (\ref{toprove}).
We find a necessary and sufficient condition on the weight $a$ for the inequality (\ref{toprove}) to hold%
\footnote{In particular, the inequality is satisfied if the weight function $a$ is even and concave in $x$.}.
Under certain additional conditions this result was announced in \cite{DAN}.

Also note that the inequality (\ref{toprove}) was considered in \cite{Lan}
for functionals similar to (\ref{functional}) with the additional constraint $u(-1) = 0$.
We obtain necessary and sufficient conditions for (\ref{toprove}) under this constraint.
(Author of \cite{Lan} assumed that the weight $a$ is decreasing in $x$.)

The article is divided into 8 sections.
In \S2 we set the conditions on the weight function $a$ which are necessary for the inequality (\ref{toprove}).
In \S3 auxiliary statements for weights satisfying the necessary conditions are gathered.
In \S4 the inequality (\ref{toprove}) is proved for piecewise linear functions $u$.
In \S5 we present the scheme for proving inequality (\ref{toprove}) for a wider class of functions $u$.
In \S6 we prove inequality (\ref{toprove}), provided that the weight $a$ first increases, then decreases.
\S7 is devoted to the proof of (\ref{toprove}) under only necessary conditions.
Finally, in the \S8 we consider the case of symmetric rearrangement.
There we get the necessary conditions on the weight and complete the proof of (\ref{toproveSymm}).
