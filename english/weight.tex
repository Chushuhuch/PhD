\section{Properties of the weighting function }

For brevity, in this section we omit the second argument of the function $a$.

\begin{lm}
\label{weightSum}
Consider a continuous function $a \ge 0$ defined on $[-1, 1]$
and satisfying the condition (\ref{almostConcave}).
Then for any $-1 \le t_1 \le t_2 \le \ldots \le t_n \le 1$
the following inequalities hold
\begin{align*}
\sum_{k = 1}^n a(t_k) & \ge a( 1 - \sum_{k = 1}^n (-1)^k t_k), & \text{ for even $n$}, & \\
\sum_{k = 1}^n a(t_k) & \ge a(- \sum_{k = 1}^n (-1)^k t_k), & \text{ for odd $n$}. &
\end{align*}
\end{lm}

\begin{proof}
We will prove the lemma by induction.
For $n = 1$ the assertion is trivial.
Now let $n$ be even. Then, by the induction hypothesis,
$$\sum_{k=1}^{n - 1} a(t_k) \ge a( -\sum_{k = 1}^{n - 1} (-1)^k t_k ).$$
Then
$$\sum_{k = 1}^{n - 1} a( t_k ) + a( t_n ) \ge a( -\sum_{k = 1}^{n - 1} (-1)^k t_k ) + a( t_n ) \ge
a( 1 - \sum_{k = 1}^{n} (-1)^k t_k ).$$
In the case of odd $n$ we have the following induction hypothesis:
$$\sum_{k=2}^n a(t_k) \ge a( 1 + \sum_{k = 2}^n (-1)^k t_k ).$$
Then
$$a( t_1 ) + \sum_{k = 2}^n a( t_k ) \ge a( t_1 ) + a( 1 + \sum_{k = 2}^{n} (-1)^k t_k ) \ge
a( -\sum_{k = 2}^{n} (-1)^k t_k + t_1 ) = a( -\sum_{k = 1}^{n} (-1)^k t_k ).$$
\end{proof}

\begin{rem}
Assume that in addition to the conditions of the lemma \ref{weightSum} the function $a$ is even.
Then the following inequalities also hold:
\label{almostConcaveMultRem}
\begin{align*}
\sum_{k = 1}^na(t_k) & \ge a(-1 + \sum_{k = 1}^n (-1)^k t_k), & \text{ for even $n$}, & \\
\sum_{k = 1}^na(t_k) & \ge a(\sum_{k = 1}^n (-1)^k t_k), & \text{ for odd $n$}. &
\end{align*}
\end{rem}

\begin{lm}
\label{periodicity}
{\bf 1.} Let $a$ satisfy $(\ref{almostConcave})$.
If there is $x_0 \in [-1, 1]$, such that $a(x_0) = 0$,
then either $a\Big |_{[x_0, 1]} \equiv 0$
or the set of zeros of $a$ is periodic on $[x_0, 1]$
and the period is a divisor of $1 - x_0$.

{\bf 2.} Let $a$ be even and satisfy $(\ref{almostConcave})$.
If there is $x_0 \in [-1, 1]$, such that $a(x_0) = 0$,
then either $a \equiv 0$
or the function $a$ is periodic on the $[-1, 1]$ interval
and the period is a divisor of $1 - x_0$.
\end{lm}

\begin{proof}
First of all, note that if $a(s) = a(t) = 0$ holds for some $s \le t$
then the inequality (\ref{almostConcave}) implies
$$0 = a(s) + a(t) \ge a( 1 - (t - s) ) \ge 0$$
i.e. $a(1 - (t - s)) = 0$.

Similarly, if $s \le 1 - t$ and $a(s) = a(1 - t) = 0$, then $a(s + t) = 0$.

These two facts imply the following.
If $a(s) = a(t) = 0$ then $a(s + k(t - s)) = 0$ for all positive integers $k$, for which $s + k(t - s) \le 1$.

{\bf 1.}
Substituting $s = t = x_0$, we obtain $a(1) = 0$.

If function $a$ has roots arbitrary close to $x_0$
then it has periodical roots on $[x_0, 1]$ with arbitrary small period.
And thus it vanishes on a set which is dense in $[x_0, 1]$.
By the continuity we have $a\Big |_{[x_0, 1]} = 0$.
Otherwise, let $x_1 > x_0$ be the root of function $a$ closest to $x_0$.
Then there must be a natural number $K$, such that $x_0 + K(x_1 - x_0) = 1$,
otherwise we can find more dense set of roots of $a$.

Suppose now that there exists a root $x_2 > x_0$ of function $a$,
which does not coincide with any of $x_0 + k ( x_1 - x_0 )$ with $k = 0 \dots K$.
Then there exists a root $x_3 \in (1 - (x_1 - x_0), 1)$
and therefore there is a root $x_4 \in (x_0, x_1)$, which leads to a contradiction.

{\bf 2.} The periodicity of zeros of the function $a$ follows from its evenness and from the first assertion of the lemma.
Denote the distance between consecutive zeros by $\Delta$.

Then for $-1 \le x \le 1 - \Delta$ the following holds
$$a(x) = a(x) + a(1 - \Delta) \ge a(x + \Delta).$$

On the other hand, $-1 \le -(x + \Delta) \le 1 - \Delta$, and
$$a(x + \Delta) = a(-(x + \Delta)) + a(1 - \Delta) \ge a(-x) = a(x).$$

Thus, $a(x) = a(x + \Delta)$.
\end{proof}

\begin{lm}
\label{maxSumConcave}
Suppose that $a_1$ and $a_2$ satisfy $(\ref{almostConcave})$.
Then the function $a(x) = \max (a_1(x), a_2(x))$ and $a_1(x) + a_2(x)$ also satisfy $(\ref{almostConcave})$.
\end{lm}
\begin{proof}
\begin{multline*}
a(1 + s + t) = \max(a_1( 1 + s + t), a_2(1 + s + t)) \le
\max(a_1(s) + a_1(t), a_2(s) + a_2(t)) \\
\le \max(a_1(s), a_2(s)) + \max(a_1(t), a_2(t)) =
a(s) + a(t).
\end{multline*}

The second part is obvious.
\end{proof}

\begin{lm}
\label{piecewiseLinearConcave}
Let the function $a$ satisfy $(\ref{almostConcave})$, $k \in \Nat$.
Then a piecewise linear function $a_k$,
interpolating function $a$ using the nodes
$(-1 + \frac{2i}{k})$, $i = 0, 1, \dots, k$,
satisfies $(\ref{almostConcave})$ either.
\end{lm}
\begin{proof}
Let $s = -1 + \frac{2i}{k}$, $t = -1 + \frac{2j}{k}$.
Then the inequality $(\ref{almostConcave})$ holds for $a_k$, because it does for $a$,
and their values at these points are equal.

Now let $s = -1 + \frac{2i}{k}$ and $t \in [-1 + \frac{2j}{k}, -1 + \frac{2(j + 1)}{k}]$.

Consider the linear function $h_1(t) = a_k( 1 - t + s ) - a_k(t) - a_k(s)$.
From already proved it follows that $h_1(-1 + \frac{2j}{k}) \le 0$ and $h_1(-1 + \frac{2(j + 1)}{k}) \le 0$.
While $h_1$ is linear, $h_1(t) \le 0$.
Thus, the inequality holds for every $s \in -1 + \frac{2i}{k}$ and $t \in [-1, 1]$.

Consider the function $h_2(y) = a_k(\frac{2j}{k}) - a_k(s - y) - a_k(t + y)$, where $1 - t + s = \frac{2j}{k}$.
Suppose, $s + y_0$ is one of the interpolation nodes, then $t + y_0$ is also an interpolation node.
Thereby, $h_2(y_0) = a(\frac{2j}{k}) - a(s - y) - a(t + y) \le 0$.
Between such $y_0$ the $h_2$ function is linear.
Also the boundary points of the domain of $h_2$ are nodes of interpolation.
Therefore $h_2(y) \le 0$ on its entire domain.

Consider the function $h_3(s) = a_k( 1 - t + s ) - a_k(t) - a_k(s)$ for any fixed $t \in [-1, 1]$.
This function is piecewise linear,
and it can bend at each $s$, which is an interpolation node itself of for which
$1 - t + s$ is an interpolation node.
However, we have already proved that at these points $h_3(s) \le 0$.
The boundary points of the domain of $h_3$ are the nodes of interpolation, thus, $h_3(s) \le 0$ everywhere.
\end{proof}

