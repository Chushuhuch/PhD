\section{Properties of the weight function}

For brevity, in this section we omit the second argument of the function $\mathfrak a$.
Thus, we assume, that $\mathfrak a \in C[-1, 1]$ and $\mathfrak a \ge 0$.

\begin{lm}
\label{weightSum}
Let $\mathfrak a$ satisfy (\ref{almostConcave}).

{\bf 1.} For any $-1 \le t_1 \le t_2 \le \ldots \le t_n \le 1$
the following inequalities hold
\begin{align*}
\sum_{k = 1}^n \mathfrak a(t_k) & \ge \mathfrak a( 1 - \sum_{k = 1}^n (-1)^k t_k), & \text{ for even $n$}, & \\
\sum_{k = 1}^n \mathfrak a(t_k) & \ge \mathfrak a(- \sum_{k = 1}^n (-1)^k t_k), & \text{ for odd $n$}. &
\end{align*}

{\bf 2.}
Assume that, in addition, the function $\mathfrak a$ is even.
Then the following inequalities also hold:
\begin{align*}
\sum_{k = 1}^n \mathfrak a(t_k) & \ge \mathfrak a(-1 + \sum_{k = 1}^n (-1)^k t_k), & \text{ for even $n$}, & \\
\sum_{k = 1}^n \mathfrak a(t_k) & \ge \mathfrak a(\sum_{k = 1}^n (-1)^k t_k), & \text{ for odd $n$}. &
\end{align*}
\end{lm}

\begin{proof}
{\bf 1.}
We prove the lemma by induction.
For $n = 1$ the assertion is trivial.
Now let $n$ be even. Then, by the induction hypothesis,
$$\sum_{k=1}^{n - 1} \mathfrak a(t_k) \ge \mathfrak a( -\sum_{k = 1}^{n - 1} (-1)^k t_k ).$$
Then
$$\sum_{k = 1}^{n - 1} \mathfrak a( t_k ) + \mathfrak a( t_n ) \ge \mathfrak a( -\sum_{k = 1}^{n - 1} (-1)^k t_k ) + \mathfrak a( t_n ) \ge
\mathfrak a( 1 - \sum_{k = 1}^{n} (-1)^k t_k ).$$
In the case of odd $n$ we have the following induction hypothesis:
$$\sum_{k=2}^n \mathfrak a(t_k) \ge \mathfrak a( 1 + \sum_{k = 2}^n (-1)^k t_k ).$$
Then
$$\mathfrak a( t_1 ) + \sum_{k = 2}^n \mathfrak a( t_k ) \ge \mathfrak a( t_1 ) + \mathfrak a( 1 + \sum_{k = 2}^{n} (-1)^k t_k ) \ge
\mathfrak a( -\sum_{k = 2}^{n} (-1)^k t_k + t_1 ) = \mathfrak a( -\sum_{k = 1}^{n} (-1)^k t_k ).$$

{\bf 2.} The proof of this part is trivial. 
\end{proof}

\begin{lm}
\label{periodicity}
{\bf 1.} Let $\mathfrak a$ satisfy $(\ref{almostConcave})$.
If there is $x_0 \in [-1, 1]$, such that $\mathfrak a(x_0) = 0$,
then either $\mathfrak a \Big|_{[x_0, 1]} \equiv 0$
or the set of zeros of $\mathfrak a$ is periodic on $[x_0, 1]$
and the period is a divisor of $1 - x_0$.

{\bf 2.} Let $\mathfrak a$ be even and satisfy $(\ref{almostConcave})$.
If there is $x_0 \in [-1, 1]$, such that $\mathfrak a(x_0) = 0$,
then either $\mathfrak a \equiv 0$
or the function $\mathfrak a$ is periodic on $[-1, 1]$
and the period is a divisor of $1 - x_0$.
\end{lm}

\begin{proof}
{\bf 1.}
Note that if $\mathfrak a(s) = \mathfrak a(t) = 0$ for some $s \le t$,
then inequality (\ref{almostConcave}) implies
$$0 = \mathfrak a(s) + \mathfrak a(t) \ge \mathfrak a( 1 - (t - s) ) \ge 0$$
i.e. $\mathfrak a(1 - (t - s)) = 0$.
Substituting $s = t = x_0$, we obtain $\mathfrak a(1) = 0$.

Similarly, if $s \le 1 - t$ and $\mathfrak a(s) = \mathfrak a(1 - t) = 0$, then $\mathfrak a(s + t) = 0$.

Thus, the set of roots of $\mathfrak a$ is symmetric on the segment $[x_0, 1]$ and
whenever $s$ and $s + \Delta$ ($\Delta \ge 0$) are roots of $\mathfrak a$,
values $s + k\Delta$ are roots of $\mathfrak a$ too, provided $s + k\Delta \le 1$.
This implies that the set of roots of $\mathfrak a$ is periodic on $[x_0, 1]$
or coincides with it.

{\bf 2.} The periodicity of zeros of the function $\mathfrak a$ follows from its evenness and from the first assertion of the lemma.
Denote the distance between consecutive zeros by $\Delta$.

Then for $-1 \le x \le 1 - \Delta$ the following holds
$$\mathfrak a(x) = \mathfrak a(x) + \mathfrak a(1 - \Delta) \ge \mathfrak a(x + \Delta).$$

On the other hand, $-1 \le -(x + \Delta) \le 1 - \Delta$, and
$$\mathfrak a(x + \Delta) = \mathfrak a(-(x + \Delta)) + \mathfrak a(1 - \Delta) \ge \mathfrak a(-x) = \mathfrak a(x).$$

Thus, $\mathfrak a(x) = \mathfrak a(x + \Delta)$.
\end{proof}

\begin{lm}
\label{maxSumConcave}
Suppose that $\mathfrak a_1$ and $\mathfrak a_2$ satisfy $(\ref{almostConcave})$.
Then the functions $\max (\mathfrak a_1(x), \mathfrak a_2(x))$ and $\mathfrak a_1(x) + \mathfrak a_2(x)$ also satisfy $(\ref{almostConcave})$.
\end{lm}
\begin{proof}
Set $\mathfrak a(x) = \max (\mathfrak a_1(x), \mathfrak a_2(x))$. Then
\begin{multline*}
\mathfrak a(1 - t + s) = \max(\mathfrak a_1( 1 - t + s), \mathfrak a_2(1 - t + s)) \le
\max(\mathfrak a_1(s) + \mathfrak a_1(t), \mathfrak a_2(s) + \mathfrak a_2(t)) \\
\le \max(\mathfrak a_1(s), \mathfrak a_2(s)) + \max(\mathfrak a_1(t), \mathfrak a_2(t)) =
\mathfrak a(s) + \mathfrak a(t).
\end{multline*}

The second part is obvious.
\end{proof}

\begin{lm}
\label{piecewiseLinearConcave}
Let the function $\mathfrak a$ satisfy $(\ref{almostConcave})$, $k \in \Nat$.
Then a piecewise linear function $\mathfrak a_k$,
interpolating $\mathfrak a$ using the nodes
$(-1 + \frac{2i}{k})$, $i = 0, 1, \dots, k$,
also satisfies $(\ref{almostConcave})$.
\end{lm}
\begin{proof}
{\bf 1.}
Let $s = -1 + \frac{2i}{k}$, $t = -1 + \frac{2j}{k}$.
Then inequality $(\ref{almostConcave})$ holds for $\mathfrak a_k$, because it does for $\mathfrak a$,
and their values at these points coincide.

{\bf 2.}
Now let $s = -1 + \frac{2i}{k}$ and $t \in [-1 + \frac{2j}{k}, -1 + \frac{2(j + 1)}{k}]$.

Consider the linear function $h_1(t) = \mathfrak a_k( 1 - t + s ) - \mathfrak a_k(t) - \mathfrak a_k(s)$.
It follows from part 1 that $h_1(-1 + \frac{2j}{k}) \le 0$ and $h_1(-1 + \frac{2(j + 1)}{k}) \le 0$.
Since $h_1$ is linear, $h_1(t) \le 0$.
Thus, the inequality holds for every $s = -1 + \frac{2i}{k}$ and $t \in [-1, 1]$.

{\bf 3.}
Let $s$ and $t$ satisfy $1 - t + s = \frac{2j}{k}$.

Consider the function $h_2(y) = \mathfrak a_k(\frac{2j}{k}) - \mathfrak a_k(s + y) - \mathfrak a_k(t + y)$.
If we choose $y_0$ such that $s + y_0$ is one of the nodes, then $t + y_0$ is also a node.
Therefore, $h_2(y_0) = \mathfrak a(\frac{2j}{k}) - \mathfrak a(s + y_0) - \mathfrak a(t + y_0) \le 0$.
Since $h_2$ is linear between such $y_0$'s, we obtain $h_2(y) \le 0$ for all admissible $y$.

{\bf 4.}
Finally, consider $h_3(s) = \mathfrak a_k( 1 - t + s ) - \mathfrak a_k(t) - \mathfrak a_k(s)$ for an arbitrary given $t \in [-1, 1]$.
Note that the parts 2 and 3 above imply $h_3(s) \le 0$ for any $s$
such that either $s$ or $1 - t + s$ is a node.
Since $h_3$ is linear between these points, $h_3(s) \le 0$ for all admissible $s$,
and the statement follows.
\end{proof}

