\section{Extension of class of functions for which $I(a, u^*) \le I(a, u)$ holds}
\begin{lm}
Let the function $a$ be continuous. Then the functional $I(a, u)$ is weakly lower semicontinuous in $\W(-1, 1)$.
\label{lowersemi}
\end{lm}

\begin{proof}
Let $u_m \rightharpoondown u$ in $\W(-1, 1)$.
Let's denote $A = \varliminf I( a, u_m ) \ge 0$.
We are going to prove $I(a, u) \le A$.
In the case $A = \infty$ the assertion is trivial, so we can assume $A < \infty$.
Switching to a subsequence, we obtain $A = \lim I( a, u_m )$.
Weak convergence implies, that there exists
$R_0$ such that $\norm{ u_m }_{\W(-1, 1)} \le R_0$.

It is known that $\W(-1, 1)$ is compactly embedded in $L_1(-1, 1)$.
Then, switching to a subsequence, we can assume that $u_m \to u$ in $L_1(-1, 1)$
and $u_m(x) \to u(x)$ almost everywhere.
Then, by Egorov's theorem, for any $\eps$ there exists a set
$G_\eps^1$ such that $\abs{ G_\eps^1 } < \eps$ and $u_m \rightrightarrows u$ in $[-1, 1] \setminus G_\eps^1$.

Uniform convergence of $u_m$ implies $\exists K: \forall m>K \abs{u_m} \le \abs{u} + \eps$ in $[-1, 1] \setminus G_\eps^1$.
Let $G_\eps^2 = \{x \in [-1, 1] \setminus G_\eps^1 : \abs{u(x)} \ge \frac{R_0 + \eps}{\eps} \}$.
Then $$R_0 \ge \int_{-1}^1 \abs{ u(x) } dx \ge \int_{G_\eps^2} \abs{ u(x) } dx \ge
\int_{G_\eps^2} \frac{R_0 + \eps}{\eps} dx = \abs{G_\eps^2} \frac{R_0 + \eps}{\eps}$$
That is, $\abs{G_\eps^2} \le \eps \frac{R_0}{R_0 + \eps} < \eps$.
The functions $u_m$, $m > K$ converge uniformly and are uniformly bounded outside the set $G_\eps := G_\eps^1 \cup G_\eps^2$.

Continuity of $F$ and $a$ implies that for any $\eps$ and $R$, there exists
$N( \eps, R )$, such that if $x \in [-1, 1] \setminus G_\eps$, $\abs{ M } \le R$ and $m > N( \eps, R )$ then
$$| F( u_m( x ), a( x, u_m( x ) ) M ) - F( u( x ), a( x, u( x ) ) M ) | < \eps.$$

Let $E_{m,\eps} := \{ x \in [-1, 1]: \abs{ u_m'( x ) } \ge \frac{ R_0 }{ \eps } \}$.
Then
$$R_0 \ge \int_{-1}^1 \abs{ u_m'( x ) } dx \ge \int_{ E_{m,\eps} } \abs{ u_m'( x ) } dx \ge
\int_{ E_{m,\eps} } \frac{ R_0 }{ \eps } dx = \frac{ R_0 }{ \eps } \abs{ E_{m,\eps} }.$$
Therefore $\abs{ E_{m,\eps} } \le \eps$.

Finally we set $L_{m,\eps} := [-1, 1] \setminus ( E_{m,\eps} \cup G_\eps )$.
Note, that $\abs{ L_{m,\eps} } \ge 2 - 3 \eps$.

We fix $R := \frac{ R_0 }{ \eps }$, $N( \eps ) := N( \eps, \frac{ R_0 }{ \eps } )$.
For any $\eps > 0$, $x \in L_{m,\eps}$ and $m > N( \eps )$ we have
$$\Big | F( u_m( x ), a( x, u_m( x ) ) \abs{u_m'( x )} ) - F( u( x ), a( x, u( x ) ) \abs{u_m'( x )} ) \Big | < \eps,$$
thus
$$\int_{L_{m,\eps}} \Big | F( u_m( x ), a( x, u_m( x ) ) \abs{u_m'( x )} ) - F( u( x ), a( x, u( x ) ) \abs{u_m'( x )} ) \Big | dx < 2 \eps.$$

We put $\eps_j = \frac{ \eps }{ 2^j }$ ($j \ge 1$), $m_j = N( \eps_j ) + j \to \infty$ and $L_\eps = \bigcap L_{m_j,\eps_j}$.
Then $\sum \eps_j = \eps$ and $\abs{ [-1, 1] \setminus L_\eps } < 3 \eps$.
Therefore
$$\int_{L_\eps} \Big | F( u_{m_j}( x ), a( x, u_{m_j}( x ) ) |u_{m_j}'( x )| ) - F( u( x ), a( x, u( x ) ) |u_{m_j}'( x )| ) \Big | dx < 2 \eps_j.$$

And then
\begin{multline*}
A = \lim I (a, u_{m_j}) = \lim \int_{-1}^1 F(u_{m_j}(x), a(x, u_{m_j}(x)) | u_{m_j }'(x) |) dx \\
\ge \varliminf \int_{-1}^1 \chi_{L_\eps}(x) F(u (x), a(x, u(x)) | u_{m_j}'(x) |) dx
=: \varliminf J_\eps(u_{m_j}').
\end{multline*}

The new functional
$$J_\eps( v ) = \int_{-1}^1 \chi_{L_\eps}( x ) F( u( x ), a( x, u( x ) ) |v( x )| ) dx$$
is convex.
Switching to a subsequence $u_k$ again, we can assume that
$\varliminf J_\eps( u_{m_j}' ) = \lim J_\eps( u_k' )$.
Since $u_k' \rightharpoondown u'$ in $L_1$, we can choose a sequence
convex combinations of $u_k'$, which will converge to $u'$ strongly (see \cite[Theorem 3.13]{Rudin}).
Namely, there are $\alpha_{k,l} \ge 0$ for
$k \in \Nat$, $l \le k$, such that $\sum_{l = 1}^k \alpha_{k,l} = 1$ for every $k$ and
$w_k = \sum_{l = 1}^k \alpha_{k,l} u_{l}' \to u'$ in $L_1$.
Also, without loss of generality we can assume, that the minimal index $l$ of a nonzero coefficient $\alpha_{k,l}$
tends to infinity as $k$ tends to infinity.
Then
$$\lim J_\eps( u_k' ) = \lim \sum_{l = 1}^k \alpha_{k,l} J_\eps( u_{l}' ).$$

By the convexity of $J_\eps$, we have
$$\sum_{l = 1}^k \alpha_{k,l} J_\eps( u_{l}' ) \ge J_\eps( w_k ).$$

Finally, since $w_k \to u'$ in $L_1(-1, 1)$, we can assume, by switching to a subsequence, that $w_k(x) \to u'(x)$ almost everywhere.
Moreover, since $\abs{ u_j'( x ) } < \frac{ R_0 }{\eps}$ holds for $x \in L_\eps$, then $\abs{ w_k( x ) } < \frac{ R_0 }{\eps}$.
Hence,
$$F( u( x ), a( x, u( x ) ) w_k( x ) ) \le \max\limits_{(x, M)} F( u( x ), a( x, u( x ) ) M ) < \infty,$$
where the maximum is taken over a compact set
$(x,M) \in [-1, 1] \times [-\frac{ R_0 }{\eps},\frac{ R_0 }{\eps}]$.
Therefore, the Lebesgue theorem applies, and $\lim J_\eps(w_k) = J_\eps(u')$ is proved.
Thus,
$$A \ge \lim J_\eps( u_k' ) = \lim \sum_{l = 1}^k \alpha_{k,l} J_\eps( u_{l}' ) \ge
\varliminf J_\eps( w_k ) = J_\eps( u' ).$$

Since $\eps > 0$ is arbitrary, $A \ge I(a, u)$ is proved.
\end{proof}

\begin{lm}
\label{uplift}
Let $A \subset \W(-1,1)$.
And let $B \subset A$ contains only $v$ for which $I(a, v^*) \le I(a, v)$ holds.
Suppose that for each $u \in A$
there is a sequence $u_k \in B$ such that $u_k \to u$ in $\W(-1, 1)$ and
$I(a, u_k) \to I(a, u)$.
Then $\forall u \in A$ the inequality $I(a, u^*) \le I(a, u)$ holds.
\end{lm}
\begin{proof}
Let's pick some $u \in A$ and find the appropriate $u_k \in B$.
By hypothesis, $I(a, u_k^*) \le I(a, u_k) \to I(a, u)$.
It is shown in \cite[Theorem 1]{Br} that $u_k \to u$ in $\W(-1, 1)$ implies
$\overline{u_k} \rightharpoondown \overline{u}$ in $\W(-1, 1)$.
But $u_k^*( x ) = \overline{u_k}( \frac{x - 1}{2} )$ and
$u^*( x ) = \overline{u}( \frac{x - 1}{2} )$.
Hence, $u_k^* \rightharpoondown u^*$.
Then by the weak lower semicontinuity of the functional we conclude $I(a, u^*) \le \liminf I(a, u_k^*)$.
Thus $I(a, u^*) \le I(a, u)$.
\end{proof}

\begin{cor}
Let the weight $a$ be continuous, and the inequality $(\ref{toprove})$ hold for non-negative piecewise linear functions $u$.
Then it holds for all non-negative Lipschitz functions.
\end{cor}
\begin{proof}
By \cite{Gariepy}, Theorem 1 in Section 6.6, any Lipschitz function can be approximated with its derivative
almost everywhere with continuously differentiable functions.
Since the derivatives of the approximating functions are uniformly bounded,
the sequence will converge in $\W(-1, 1)$ due to Lebesgue theorem,
and also $I$ will converge.
In turn, continuously differentiable functions can be uniformly approximated with piecewise linear functions
along with their derivative.
This convergence provides convergence in $\W(-1, 1)$ and the convergence of the functional $I$.
Applying Lemma \ref{uplift} proves the corollary.
\end{proof}

