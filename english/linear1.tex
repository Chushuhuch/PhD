\section{The result for piecewise linear functions}

\rm
In this section we prove the inequality (\ref{toprove}) for piecewise linear functions.
Without loss of generality, we assume that $F(\cdot, 0) \equiv 0$.

\begin{thm}
\label{linth}
Let the function $\mathfrak a$ be even and satisfy the condition $(\ref{almostConcave})$.
If $u$ is a nonnegative piecewise linear function then $I(\mathfrak a, u) \ge I(\mathfrak a, u^*)$.
\end{thm}

\begin{proof}
Let $-1 = x_1 < x_2 < \dots < x_K = 1$ be the nodes of $u$.
Consider the set $U$ equal to the range of $u$ with images of endpoints of linear pieces excluded:
$U := u( [-1, 1] ) \setminus \{ u(x_1), \dots, u(x_K) \}$.
It's obvious that the set $U$ is the union of a finite number of intervals $U = \cup_{j = 1}^N G_j$.

We denote by $m_j$ the number of preimages for $u_0 \in G_j$,
i.e. the number of solutions of the equation $u(y) = u_0$
(obviously, $m_j$ does not depend on $u_0 \in G_j$).
It is easy to see that the preimages are linear functions of $u_0$:
$y = y_k^j(u_0)$, $k = 1, \dots, m_j$,
and $y_k^j{}'(u(y)) = \frac{1}{u'(y)}$.
We assume that $y_1^j(u_0) < y_2^j(u_0) < \dots < y_{m_j}^j(u_0)$.

The solution of the equation $u^*(y^*)=u_0$ ($u_0 \in U$) can be expressed in terms of $y_k^j$:

\begin{center}
\begin{tabular}{l|l|l} 
\multirow{2}{*}{$u(-1)<u_0$ \rule[-34pt]{0pt}{65pt}} & $m_j$ is even & $y^*=1-\sum\limits_{k=1}^{m_j} (-1)^k y_k^j$ \rule[-17pt]{0pt}{40pt} \\
                                                     & $m_j$ is odd  & $y^*=-\sum\limits_{k=1}^{m_j} (-1)^k y_k^j$ \rule[-17pt]{0pt}{40pt} \\ \hline
\multirow{2}{*}{$u(-1)>u_0$ \rule[-34pt]{0pt}{65pt}} & $m_j$ is even & $y^*=-1+\sum\limits_{k=1}^{m_j} (-1)^k y_k^j$ \rule[-17pt]{0pt}{40pt} \\
                                                     & $m_j$ is odd  & $y^*=\sum\limits_{k=1}^{m_j} (-1)^k y_k^j$ \rule[-17pt]{0pt}{40pt} \\ 
\end{tabular}
\end{center}

Let $y^*(v) = (u^*)^{-1}(v)$.
Then $y^*{}'(v) = \sum_{k=1}^{m_j} \abs{y_k^j{}'(v)}$ for $v \in G_j$, as the signs in the expression for
$y^*$ and signs of $y_k^j{}'$ alternate, and $y^*{}'(v)\ge 0$.

The sets of zeros of $u'(x)$ and $u^*{}'(x)$ can have nonzero measure.
However, they do not contribute to the integral, since $F(u(x), 0) = 0$.

Consider the remaining parts of the integrals :
\begin{multline*}
I(\mathfrak a, u) = \sum_{j=1}^N \,\int\limits_{u^{-1}(G_j)} F(u(x), \mathfrak a(x, u(x)) \abs{u'(x)}) \, dx
\\ = \sum_{j=1}^N \,\int\limits_{G_j} \sum_{k=1}^{m_j} F\Big(v, \frac{\mathfrak a(y_k^j(v), v)}{\bigabs{y_k^j{}'(v)}}\Big) \bigabs{y_k^j{}'(v)} \, dv,
\end{multline*}
\begin{multline*}
I(\mathfrak a, u^*) = \sum_{j=1}^N \,\int\limits_{(u^*)^{-1}(G_j)} F(u^*(x), \bigabs{\mathfrak a(x, u(x)) u^*{}'(x)}) \, dx
\\ = \sum_{j=1}^N \,\int\limits_{G_j} F\Big(v, \frac{\mathfrak a(y^*(v), v)}{\sum_{k=1}^{m_j} \bigabs{y_k^j{}'(v)}}\Big)
\sum_{k=1}^{m_j} \bigabs{ y_k^j{}'(v) } \, dv.
\end{multline*}

We fix $j$ and $v$ in the right parts and prove the inequality for integrands.
We denote $b_k := |y_k^j{}'(v)|$, $y_k := y_k^j(v)$, $y^* := y^*(v)$, $m := m_j$.
Then the assertion takes the form:
$$T:=\sum_{k=1}^m b_k F\Big( v, \frac{ \mathfrak a(y_k, v) }{b_k} \Big)
\ge F\Big( v, \frac{ \mathfrak a(y^*, v) }{ \sum_{k=1}^m b_k  } \Big) \sum_{k=1}^m b_k.$$
By Jensen's inequality for the function $F(v, \cdot)$, we obtain
$$T \ge F\Big( v, \frac{ \sum_{k=1}^m \mathfrak a(y_k, v) }{ \sum_{k=1}^m b_k } \Big) \sum_{k=1}^m b_k.$$
Then it is sufficient to prove $\sum_{k=1}^m \mathfrak a(y_k, v) \ge \mathfrak a(y^*, v)$, which is true due to Lemma \ref{weightSum}.
\end{proof}

\begin{rem}
\label{landesLinear}
In the paper $\cite{Lan}$ the inequality $(\ref{toprove})$ is proved under the additional assumption $u(-1) = 0$
for the weight functions $\mathfrak a$, decreasing in $x$.
It is easy to see that under this assumption, the proof of Theorem $\ref{linth}$ works for weights satisfying
$(\ref{almostConcave})$ without the evenness assumption,
since in this case $u(-1) < u_0$, and we need only two of the four inequalities,
given by the first part of Lemma $\ref{weightSum}$.
It is also obvious that the assumption $(\ref{almostConcave})$ is weaker than the assumption of $\mathfrak a$ decreasing in $x$.
\end{rem}

