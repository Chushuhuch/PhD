\PassOptionsToPackage{table}{xcolor}
\documentclass{beamer}\usepackage[]{graphicx}\usepackage[]{color}
%% maxwidth is the original width if it is less than linewidth
%% otherwise use linewidth (to make sure the graphics do not exceed the margin)
\makeatletter
\def\maxwidth{ %
  \ifdim\Gin@nat@width>\linewidth
    \linewidth
  \else
    \Gin@nat@width
  \fi
}
\makeatother

\definecolor{fgcolor}{rgb}{0.345, 0.345, 0.345}
\newcommand{\hlnum}[1]{\textcolor[rgb]{0.686,0.059,0.569}{#1}}%
\newcommand{\hlstr}[1]{\textcolor[rgb]{0.192,0.494,0.8}{#1}}%
\newcommand{\hlcom}[1]{\textcolor[rgb]{0.678,0.584,0.686}{\textit{#1}}}%
\newcommand{\hlopt}[1]{\textcolor[rgb]{0,0,0}{#1}}%
\newcommand{\hlstd}[1]{\textcolor[rgb]{0.345,0.345,0.345}{#1}}%
\newcommand{\hlkwa}[1]{\textcolor[rgb]{0.161,0.373,0.58}{\textbf{#1}}}%
\newcommand{\hlkwb}[1]{\textcolor[rgb]{0.69,0.353,0.396}{#1}}%
\newcommand{\hlkwc}[1]{\textcolor[rgb]{0.333,0.667,0.333}{#1}}%
\newcommand{\hlkwd}[1]{\textcolor[rgb]{0.737,0.353,0.396}{\textbf{#1}}}%

\usepackage{framed}
\makeatletter
\newenvironment{kframe}{%
 \def\at@end@of@kframe{}%
 \ifinner\ifhmode%
  \def\at@end@of@kframe{\end{minipage}}%
  \begin{minipage}{\columnwidth}%
 \fi\fi%
 \def\FrameCommand##1{\hskip\@totalleftmargin \hskip-\fboxsep
 \colorbox{shadecolor}{##1}\hskip-\fboxsep
     % There is no \\@totalrightmargin, so:
     \hskip-\linewidth \hskip-\@totalleftmargin \hskip\columnwidth}%
 \MakeFramed {\advance\hsize-\width
   \@totalleftmargin\z@ \linewidth\hsize
   \@setminipage}}%
 {\par\unskip\endMakeFramed%
 \at@end@of@kframe}
\makeatother

\definecolor{shadecolor}{rgb}{.97, .97, .97}
\definecolor{messagecolor}{rgb}{0, 0, 0}
\definecolor{warningcolor}{rgb}{1, 0, 1}
\definecolor{errorcolor}{rgb}{1, 0, 0}
\newenvironment{knitrout}{}{} % an empty environment to be redefined in TeX

\usepackage{alltt}

\mode<presentation>
{
  \usetheme{Warsaw}

}

\beamertemplatenavigationsymbolsempty

\usepackage[]{inputenc}
\usepackage[english]{babel}
\usepackage{amsthm}
\usepackage{graphicx}
\usepackage{epstopdf}
\usepackage{grffile}
\usepackage{hyperref}
\usepackage{beamerthemesplit}
\definecolor{links}{HTML}{2A1B81}
\hypersetup{colorlinks,linkcolor=,urlcolor=links}
%\setsansfont[Ligatures={Common,TeX}]{TeX Gyre Heros}

{\renewcommand{\arraystretch}{1.1}

%\AtBeginSection[]
%{
%  \begin{frame}
%    \frametitle{Outline}
%    \tableofcontents[currentsection]
%  \end{frame}
%}

\newcommand*\oldmacro{}%
\let\oldmacro\insertshorttitle%
\renewcommand*\insertshorttitle{%
   \oldmacro\hfill%
   \insertframenumber\,/\,\inserttotalframenumber}

\setbeamertemplate{headline}{}
\setbeamertemplate{itemize items}[circle]
\setbeamertemplate{footline}
{
  \leavevmode%
  \hbox{%
  \begin{beamercolorbox}[wd=.3\paperwidth,ht=2.25ex,dp=1ex,center]{author in head/foot}%
    \usebeamerfont{author in head/foot}\insertshortauthor
  \end{beamercolorbox}%
  \begin{beamercolorbox}[wd=.7\paperwidth,ht=2.25ex,dp=1ex,center]{title in head/foot}%
    \usebeamerfont{title in head/foot}\insertshorttitle
  \end{beamercolorbox}%
%  \begin{beamercolorbox}[wd=.1\paperwidth,ht=2.25ex,dp=1ex,center]{date in head/foot}%
%    \insertframenumber{} / \inserttotalframenumber\hspace*{1ex}
%  \end{beamercolorbox}
  }%
  \vskip0pt%
}

\title{The P\'olya-Szeg\"o type inequality with variable exponent}

    \author{Sergey Bankevich}
\institute{Saint Petersburg State University, Russia}
%\date {
%February 12, 2016
%}

\usepackage[absolute,overlay]{textpos}
\setlength{\TPHorizModule}{1cm} % Horizontale Einheit
\setlength{\TPVertModule}{1cm} % Vertikale Einheit
\addtobeamertemplate{title page}{
}{}
\IfFileExists{upquote.sty}{\usepackage{upquote}}{}

\renewcommand{\ge}{\geqslant}
\renewcommand{\le}{\leqslant}
\newcommand{\Wf}{\stackrel{o\ }{W{}_1^1}}
\newtheorem{prop}{Proposition}

\begin{document}
\begin{frame}
  \titlepage
\end{frame}

%\section{Problem statement}

\begin{frame}{Problem statement}

Consider functional
{\Large 
\begin{equation}
I(u) = \int\limits_{-1}^1 |u'(x)|^{p(x)} dx.
\label{sfunc}
\end{equation}
}

Here $p \in C[-1, 1]$, $p(x) > 0$, $u \in \Wf[-1, 1]$.

\pause

\bigskip
We'd like to prove that
{\Large 
\begin{equation}
I(u^*) \le I(u).
\label{ineq}
\end{equation}
}

\end{frame}


\begin{frame}{Symmetrization}

$u^*$ is a Steiner symmetrization of $u$.

\end{frame}


\begin{frame}{Results}

It appears, that if the inequality (\ref{ineq}) holds for the functional (\ref{sfunc}) for any piecewise linear function $u$,
then $p \equiv const$.

Thus the inequality (\ref{ineq}) reduces to a classical one.

\end{frame}

\begin{frame}{Fixed functional}
To avoid exponentiating values from different side of $1$ we consider the following functional
{\Large
\begin{equation}
I(u) = \int\limits_{-1}^1 \big(1 + u'(x)^2 \big)^{p(x) \over 2} dx.
\label{func}
\end{equation}
}

\end{frame}

\begin{frame}{Necessary conditions}

\begin{prop}
Suppose inequality (\ref{ineq}) with functional (\ref{func}) holds for any piecewise linear function $u$.
Then $p(x) \ge 1$, $p$ is even and convex,
and the function
$$K(s, x) = s ( 1 + { 1 \over s^2 } )^{p(x) \over 2} \qquad s > 0,\ x \in [-1, 1]$$
is convex.
\end{prop}

In sufficiently smooth conditions the convexity of the function $K$ is equivalent to 
$D^2 K$ matrix of second derivatives to be positive definite.

\end{frame}

\begin{frame}{Necessary conditions: further analysis}
The fact that $D^2 K$ is positive definite reduces to an inequality
$$p''(x) \ge {p'(x)^2 \over p(x) - 1} A(s, p(x) - 1)$$
for some function $A(s, q)$.

We find that $A(s, q) < 0.63$, thus we get
\begin{prop}
Suppose inequality (\ref{ineq}) with functional (\ref{func}) holds for any piecewise linear function $u$,
and the function $p$ is smooth enough.
Then $(p(x) - 1)^{0.37}$ is convex.
\end{prop}

\end{frame}

\begin{frame}{Proof sketch}

\begin{theorem}
Suppose $K$ is convex.
Then the inequality (\ref{ineq}) holds for any $u \in \Wf[-1, 1]$, for which $I(u) < \infty$.
\end{theorem}

\end{frame}


\end{document}
