\PassOptionsToPackage{table}{xcolor}
\documentclass{beamer}\usepackage[]{graphicx}\usepackage[]{color}
%% maxwidth is the original width if it is less than linewidth
%% otherwise use linewidth (to make sure the graphics do not exceed the margin)
\makeatletter
\def\maxwidth{ %
  \ifdim\Gin@nat@width>\linewidth
    \linewidth
  \else
    \Gin@nat@width
  \fi
}
\makeatother

\definecolor{fgcolor}{rgb}{0.345, 0.345, 0.345}
\newcommand{\hlnum}[1]{\textcolor[rgb]{0.686,0.059,0.569}{#1}}%
\newcommand{\hlstr}[1]{\textcolor[rgb]{0.192,0.494,0.8}{#1}}%
\newcommand{\hlcom}[1]{\textcolor[rgb]{0.678,0.584,0.686}{\textit{#1}}}%
\newcommand{\hlopt}[1]{\textcolor[rgb]{0,0,0}{#1}}%
\newcommand{\hlstd}[1]{\textcolor[rgb]{0.345,0.345,0.345}{#1}}%
\newcommand{\hlkwa}[1]{\textcolor[rgb]{0.161,0.373,0.58}{\textbf{#1}}}%
\newcommand{\hlkwb}[1]{\textcolor[rgb]{0.69,0.353,0.396}{#1}}%
\newcommand{\hlkwc}[1]{\textcolor[rgb]{0.333,0.667,0.333}{#1}}%
\newcommand{\hlkwd}[1]{\textcolor[rgb]{0.737,0.353,0.396}{\textbf{#1}}}%

\usepackage{framed}
\makeatletter
\newenvironment{kframe}{%
 \def\at@end@of@kframe{}%
 \ifinner\ifhmode%
  \def\at@end@of@kframe{\end{minipage}}%
  \begin{minipage}{\columnwidth}%
 \fi\fi%
 \def\FrameCommand##1{\hskip\@totalleftmargin \hskip-\fboxsep
 \colorbox{shadecolor}{##1}\hskip-\fboxsep
     % There is no \\@totalrightmargin, so:
     \hskip-\linewidth \hskip-\@totalleftmargin \hskip\columnwidth}%
 \MakeFramed {\advance\hsize-\width
   \@totalleftmargin\z@ \linewidth\hsize
   \@setminipage}}%
 {\par\unskip\endMakeFramed%
 \at@end@of@kframe}
\makeatother

\definecolor{shadecolor}{rgb}{.97, .97, .97}
\definecolor{messagecolor}{rgb}{0, 0, 0}
\definecolor{warningcolor}{rgb}{1, 0, 1}
\definecolor{errorcolor}{rgb}{1, 0, 0}
\newenvironment{knitrout}{}{} % an empty environment to be redefined in TeX

\usepackage{alltt}

\mode<presentation>
{
  \usetheme{Warsaw}

}

\beamertemplatenavigationsymbolsempty

\usepackage[]{inputenc}
\usepackage[english]{babel}
\usepackage{amsthm}
\usepackage{graphicx}
\usepackage{epstopdf}
\usepackage{grffile}
\usepackage{hyperref}
\usepackage{beamerthemesplit}
\usepackage{graphicx, subfigure}
\definecolor{links}{HTML}{2A1B81}
\hypersetup{colorlinks,linkcolor=,urlcolor=links}
%\setsansfont[Ligatures={Common,TeX}]{TeX Gyre Heros}

{\renewcommand{\arraystretch}{1.1}

%\AtBeginSection[]
%{
%  \begin{frame}
%    \frametitle{Outline}
%    \tableofcontents[currentsection]
%  \end{frame}
%}

\newcommand*\oldmacro{}%
\let\oldmacro\insertshorttitle%
\renewcommand*\insertshorttitle{%
   \oldmacro\hfill%
   \insertframenumber\,/\,\inserttotalframenumber}

\setbeamertemplate{headline}{}
\setbeamertemplate{itemize items}[circle]
\setbeamertemplate{footline}
{
  \leavevmode%
  \hbox{%
  \begin{beamercolorbox}[wd=.3\paperwidth,ht=2.25ex,dp=1ex,center]{author in head/foot}%
    \usebeamerfont{author in head/foot}\insertshortauthor
  \end{beamercolorbox}%
  \begin{beamercolorbox}[wd=.7\paperwidth,ht=2.25ex,dp=1ex,center]{title in head/foot}%
    \usebeamerfont{title in head/foot}\insertshorttitle
  \end{beamercolorbox}%
%  \begin{beamercolorbox}[wd=.1\paperwidth,ht=2.25ex,dp=1ex,center]{date in head/foot}%
%    \insertframenumber{} / \inserttotalframenumber\hspace*{1ex}
%  \end{beamercolorbox}
  }%
  \vskip0pt%
}

\title{The P\'olya-Szeg\"o type inequality with variable exponent}

\author{Sergey Bankevich}
\institute{Saint Petersburg State University, Russia\\
\bigskip
Suzdal, 2016}
\date {
%February 12, 2016
}

\usepackage[absolute,overlay]{textpos}
\setlength{\TPHorizModule}{1cm} % Horizontale Einheit
\setlength{\TPVertModule}{1cm} % Vertikale Einheit
\addtobeamertemplate{title page}{
}{}
\IfFileExists{upquote.sty}{\usepackage{upquote}}{}

\renewcommand{\ge}{\geqslant}
\renewcommand{\le}{\leqslant}
\newcommand{\Wf}{\stackrel{o\ }{W{}_1^1}}
\newtheorem{prop}{Proposition}
\newtheorem{cor}{Corollary}

\begin{document}
\begin{frame}
  \titlepage
\end{frame}

\begin{frame}{The P\'olya-Szeg\"o inequality}

Let $u \in \Wf[-1, 1]$, $u \ge 0$.
Denote its symmetric rearrangement by $u^*$.
Recall the classical P\'olya-Szeg\"o inequality:
$$I(u^*) \le I(u), \quad \mbox{where } I(u) = \int\limits_{-1}^1 |u'(x)|^p dx, \quad p \ge 1.$$

\begin{figure}
  %\nonumber
  \begin{center}
    \mbox{
        \subfigure{
            \includegraphics[width = .5\textwidth]{symmetrization_exmpl_u.pdf}
        }
        \subfigure{
            \includegraphics[width = .5\textwidth]{symmetrization_exmpl_us.pdf}
        }
    }
  \end{center}
\end{figure}

\end{frame}


\begin{frame}{Problem statement}

Now we consider the functional
{\Large 
\begin{equation*}
I(u) = \int\limits_{-1}^1 |u'(x)|^{p(x)} dx.
\label{sfunc}
\end{equation*}
}

Here $p \in C[-1, 1]$, $p(x) \ge 1$, $u \in \Wf[-1, 1]$.

%\pause

\bigskip
What can we say about the same inequality?
{\Large 
\begin{equation*}
I(u^*) \le I(u).
\label{ineq}
\end{equation*}
}

\end{frame}


\begin{frame}{}

\begin{prop}
Suppose that inequality $I(u^*) \le I(u)$ holds for any piecewise linear function $u$.
Then $p(x) \equiv const$.
\end{prop}

To show this consider the following function $u$.
\begin{figure}
	\includegraphics[width = .8\textwidth]{necessary_condition_1.pdf}
\end{figure}
$I(u^*) \le I(u)$ implies $t^{p(x)} \ge t^{p(0)}$ when passing support of $u$ to $0$.
This gives us opposite inequalities for $t < 1$ and $t > 1$.

%Thus the inequality (\ref{ineq}) reduces to a classical one.

\end{frame}


\begin{frame}{Fixed functional}
Now we consider another functional
{\large
\begin{equation*}
J(u) = \int\limits_{-1}^1 \big(1 + u'(x)^2 \big)^{p(x) \over 2} dx.
\label{func}
\end{equation*}
}

\begin{theorem}
Suppose that inequality $J(u^*) \le J(u)$ holds for any piecewise linear function $u$.
Then $p$ is even and convex,
and the function
$$K(s, x) = s \Big( 1 + { 1 \over s^2 } \Big)^{p(x) \over 2} \qquad s > 0,\ x \in [-1, 1]$$
is convex.
\end{theorem}

Note that convexity of $K$ implies convexity of $p$.

\end{frame}


\begin{frame}{Necessary and sufficient conditions}

\vskip -0.5cm

\begin{theorem}
Suppose $p$ is even and $K$ is convex.
Then the inequality $J(u^*) \le J(u)$ holds for any $u \in \Wf[-1, 1]$, for which $J(u) < \infty$.
\end{theorem}

\begin{itemize}
\item The inequality is proved for piecewise linear functions $u$.
This is done by direct calculations.

\item Due to convexity of function $p$ the integrand is regular (Zhikov, 1994),
and there's a sequence of piecewise linear functions $u_n \to u$: $J(u_n) \to J(u)$.

\item $u_n \to u$ in $\Wf[-1, 1]$ implies $u_n^* \rightharpoondown u^*$ in $\Wf[-1, 1]$ (Brock, 1999).

\item $J$ is weakly lower semicontinuous in $\Wf[-1, 1]$ as the integrand is convex and power bounded.

\item $J(u^*) \le \varliminf J(u_n^*) \le \lim J(u_n) = J(u)$.
\end{itemize}

\end{frame}


\begin{frame}{Sufficient conditions for convexity of $K$}

The fact that $D^2 K$ is positive definite reduces to an inequality
$$q''(x) \ge {q'(x)^2 \over q} \cdot A(s, q(x))$$
for certain function $A(s, q)$, $q(x) = p(x) - 1$.

Note that \quad
$q''(x) \ge {q'^2(x) \over q(x)} \alpha \quad \Longleftrightarrow \quad q^{1 - \alpha}$ is convex, $\alpha \neq 1$.

Numeric calculations show that $A(s, q) < 0.63$, thus we get
\begin{cor}
Suppose $(p(x) - 1)^{0.37}$ is convex.
Then the inequality $J(u^*) \le J(u)$ holds for any $u \in \Wf[-1, 1]$, for which $J(u) < \infty$.
\end{cor}

\end{frame}


\end{document}
