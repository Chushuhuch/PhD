\documentclass[12pt,russian]{article}
\renewcommand{\baselinestretch}{1}
%\renewcommand{\baselinestretch}{1.5}
\textwidth=158mm
\textheight=232mm
\voffset=-24mm
%\pagestyle{empty}

\usepackage[russian]{babel}
\usepackage{amsmath,amssymb,amsthm,amsfonts}
\usepackage{multirow}
\usepackage{color}

\newcommand{\Real}{\mathbb R}
\newcommand{\Nat}{\mathbb N}
\newcommand{\norm}[1]{\pmb{\Vert}#1\pmb{\Vert}}
\newcommand{\abs}[1]{\left\vert#1\right\vert}
\newcommand{\bignorm}[1]{\bigl\Vert#1\bigr\Vert}
\newcommand{\bigabs}[1]{\bigl\vert#1\bigr\vert}
\newcommand{\set}[1]{\left\{#1\right\}}
\renewcommand{\phi}{\varphi}
\newcommand{\eps}{\varepsilon}
\renewcommand{\ge}{\geqslant}
\renewcommand{\le}{\leqslant}
\renewcommand{\liminf}{\underline{\lim}}
\newcommand{\grad}{\triangledown}
\newcommand{\card}{{\rm card}}
\newtheorem{thm}{Теорема}
\newtheorem{prop}{Предложение}
\newtheorem{lm}{Лемма}
\newtheorem{rem}{Замечание}
\newtheorem{cor}{Следствие}
\newcommand{\To}{\longrightarrow}
\newcommand{\Wf}{\stackrel{o\ }{W{}_1^1}}
\newcommand{\W}{W_1^1}
\newcommand{\sign}{\mathop{\rm sign}\nolimits}
\newcommand{\dist}{\mathop{\rm dist}\nolimits}
\newcommand{\comment}[1]{\colorbox{yellow}{#1}}
\newcounter{pictureCounter}
\begin{document}

\section{Постановка задачи}

Рассмотрим функционал:
\begin{equation}
\label{functional}
I(u) = \int\limits_{-1}^1 ( 1 + | u'(x) |^2 )^{p(x)} dx,
\end{equation}
где $p(x) > 0$ --- непрерывная функция на $[-1, 1]$, $u \in \Wf[-1, 1]$.

Неравенство:
\begin{equation}
\label{toprove}
I(u^*) \le I(u)
\end{equation}

\section{Необходимые условия}

\begin{prop}
Предположим, неравенство (\ref{toprove}) выполнено для любой кусочно линейной функции $u$.
Тогда функция $p$ четна и выпукла, а также выпукла функция
$$K(s, x) = s ( 1 + { 1 \over s^2 } )^{p(x)} \qquad s > 0,\ x \in [-1, 1].$$
То есть $K(s, x_1) + K(t, x_2) \ge 2 K({ s + t \over 2 }, { x_1 + x_2 \over 2 } )$ для любых $s, t > 0, x_1, x_2 \in [-1, 1]$.
\end{prop}

\begin{proof}
Зафиксируем две точки на отрезке $-1 < x_1 < x_2 < 1$
и рассмотрим финитную кусочно линейную функцию с ненулевыми производными только в окрестностях $x_1$ и $x_2$.
А именно, для достаточно малого $\eps$:
\begin{equation}
\left\{
\begin{aligned}
u(x) &= 0, & x &\in [-1, x_1 - s \eps] \cup [x_2 + t \eps, 1]\\
u(x) &= 2 \eps, & x &\in [x_1 + s \eps, x_2 - t \eps]\\
u(x) &= \eps + { x - x_1 \over s }, & x &\in [x_1 - s \eps, x_1 + s \eps]\\
u(x) &= \eps - { x - x_2 \over s }, & x &\in [x_2 - t \eps, x_2 + t \eps]
\end{aligned}
\right.
\end{equation}

\comment{update graph}

\begin{center}
\begin{picture}(200,90)
\refstepcounter{pictureCounter}
\label{uGraph}
\put(10,65){\line(1,0){50}}
\put(60,65){\line(1,1){10}}
\put(70,75){\line(1,0){40}}
\put(110,75){\line(1,-1){10}}
\put(120,65){\line(1,0){70}}
\put(0,25){\vector(1,0){200}}
\put(100,15){\vector(0,1){80}}
\put(99,65){\line(1,0){2}}
\put(92,62){$\bar{v}$}
\put(60,24){\line(0,1){2}}
\put(58,14){$s$}
\put(120,24){\line(0,1){2}}
\put(119,14){$t$}
\put(10,24){\line(0,1){2}}
\put(6,14){$-1$}
\put(190,24){\line(0,1){2}}
\put(188,14){$1$}
\put(20,70){$u(x)$}
\put(85,1){Fig. \arabic{pictureCounter}}
\end{picture}
\end{center}

Тогда функция $u^*$ имеет следующий вид:
\begin{equation}
\left\{
\begin{aligned}
u(x) &= 0, & x &\in [-1, { x_2 - x_1 \over 2 } - { s + t \over 2 } \eps] \cup [{ x_1 - x_2 \over 2 } + { s + t \over 2 } \eps, 1]\\
u(x) &= 2 \eps, & x &\in [{ x_2 - x_1 \over 2 } + { s + t \over 2 } \eps, { x_1 - x_2 \over 2 } - { s + t \over 2 } \eps]\\
u(x) &= \eps + \frac{ x - { x_2 - x_1 \over 2 } }{ { s + t \over 2 } }, & x &\in [{ x_2 - x_1 \over 2 } - { s + t \over 2 } \eps, { x_2 - x_1 \over 2 } + { s + t \over 2 } \eps]\\
u(x) &= \eps - \frac{ x - { x_1 - x_2 \over 2 } }{ { s + t \over 2 } }, & x &\in [{ x_1 - x_2 \over 2 } - { s + t \over 2 } \eps, { x_1 - x_2 \over 2 } + { s + t \over 2 } \eps]\\
\end{aligned}
\right.
\end{equation}

\comment{add graph}

Тогда, если выполнено неравенство (\ref{toprove}), верно следующее:
\begin{multline*}
\int\limits_{ x_1 - s \eps }^{ x_1 + s \eps } ( 1 + { 1 \over s^2 } )^{p(x)} dx
+ \int\limits_{ x_2 - t \eps }^{ x_2 + t \eps } ( 1 + { 1 \over t^2 } )^{p(x)} dx
\\ \ge \int\limits_{ { x_2 - x_1 \over 2 } - { s + t \over 2 } \eps }^{ { x_2 - x_1 \over 2 } + { s + t \over 2 } \eps } ( 1 + \frac{1}{ ( { s + t \over 2 } )^2 } )^{p(x)} dx
     + \int\limits_{ { x_1 - x_2 \over 2 } - { s + t \over 2 } \eps }^{ { x_1 - x_2 \over 2 } + { s + t \over 2 } \eps } ( 1 + \frac{1}{ ( { s + t \over 2 } )^2 } )^{p(x)} dx
\end{multline*}

Или, переходя к $\eps \to 0$,
\begin{equation}
\label{preConv}
s ( 1 + { 1 \over s^2 } )^{p(x_1)} + t ( 1 + { 1 \over t^2 } )^{p(x_2)}
\ge { s + t \over 2 } ( 1 + \frac{1}{ ( { s + t \over 2 } )^2 } )^{p( { x_2 - x_1 \over 2 } )}
  + { s + t \over 2 } ( 1 + \frac{1}{ ( { s + t \over 2 } )^2 } )^{p( { x_1 - x_2 \over 2 } )}.
\end{equation}

Положим в неравенстве (\ref{preConv}) $s = t$:
$$( 1 + { 1 \over s^2 } )^{p(x_1)} + ( 1 + { 1 \over s^2 } )^{p(x_2)}
\ge ( 1 + { 1 \over s^2 } )^{p( { x_2 - x_1 \over 2 } )} + ( 1 + { 1 \over s^2 } )^{p( { x_1 - x_2 \over 2 } )}.$$

Воспользовавшись разложением в ряд Тейлора по $s$ в точке $s = +\infty$, получим
$${ 1 \over s^2 } p(x_1) + { 1 \over s^2 } p(x_2) \ge { 1 \over s^2 } p( { x_2 - x_1 \over 2 } ) + { 1 \over s^2 } p( { x_1 - x_2 \over 2 } ) + r(s),$$

где $r(s) = o({ 1 \over s^2 })$ при $s \to +\infty$. То есть, для любых $x_1, x_2 \in [-1, 1]$ выполнено
$$p(x_1) + p(x_2) \ge p( { x_2 - x_1 \over 2 } ) + p( { x_1 - x_2 \over 2 } ).$$

Воспользовавшись леммой 10 из \cite{1dim}, получаем, что $p$ четна и выпукла.
Тогда, подставив в (\ref{preConv}) $x_1$ и $-x_2$ и воспользовавшись четностью $p$, получаем
$K(s, x_1) + K(t, x_2) \ge 2 K( { s + t \over 2 }, { x_1 + x_2 \over 2 } )$.
\end{proof}

Однако, выпуклось $K$ не следует из четности и выпуклости $p$.
В достаточно гладком случае она равносильна неравенству на определитель матрицы вторых производных:
$D_{ss} K(s, x) D_{xx} K(s, x) - (D_{sx} K(s, x))^2 \ge 0$.
Распишем это неравенство.
Обозначим $H(s) = 1 + {1 \over s^2}$.
Тогда $K(s, x) = s H(s)^{p(x)}$, $H'(s) = -{2 \over s^3}$, $H''(s) = {6 \over s^4}$.
\begin{eqnarray*}
D_s K(s, x) & =  & H(s)^{p(x)} + s p(x) H(s)^{p(x) - 1} H'(s) = H(s)^{p(x)} - 2 p(x) H(s)^{p(x) - 1} ( H(s) - 1 ) \\
D_x K(s, x) & = & s H(s)^{p(x)} p'(x) \ln{H(s)}
\end{eqnarray*}
\begin{eqnarray*}
D_{ss} K(s, x) = p(x) H(s)^{p(x) - 1} H'(s) - 2p(x) (p(x) - 1) H(s)^{p(x) - 2} H'(s) (H(s) - 1) - 2p(x) H(s)^{p(x) - 1} H'(s)
\end{eqnarray*}

Которое после преобразований приводится к следующему виду:
\begin{equation}
\label{KconvexIneq}
p''(x) \ge p'(x)^2 A(q(x), w).
\end{equation}

Здесь $q(x) = 2 p(x) - 1$, а $w = {1 \over u^2}$.

\begin{lm}
Для любого $\alpha \in (0, 1)$ выпуклость гладкой функции $q(x)^\alpha$
равносильна выполнению неравенства $p''(x) \ge {p'}^2(x) {2 (1 - \alpha) \over q(x)}$.
\end{lm}

Из леммы следует, что если $q(x) A(q(x), w) \le M$, где $M$ --- некоторая константа,
то неравенство (\ref{KconvexIneq}) следует из выпуклости функции $q^{1 - {M \over 2}}$.

Расчеты показывают, что $q(x) A(q(x), w) \le 1.26$, то есть выпуклость функции $K(u, x)$
следует из выпуклости функции $q^{0.37}$.
Кроме того, если внутри выпуклой оболочки носителя функции $u(x)$ функция $p(x)$ достаточно мала
(в соответствии с нашими вычислениями, достаточно $p(x) \le ???$),
то для выпуклости $K(u, x)$ достаточно выпуклости $\sqrt{q(x)}$.

\section{Доказательство в одномерном случае}

\begin{lm}
Если функция $K$ выпукла, то неравенство (\ref{toprove}) выполнено.
\end{lm}

\begin{proof}
Разобьем носитель функции $u(x)$ на конечное число областей $\Omega_j$ так,
чтобы каждая из них разбивалась на интервалы, на которых функция $u(x)$ линейна,
а также образы интервалов внутри области совпадали.
Положим, $\Omega_j$ разбивается на $m_j$ интервалов,
на которых функция $u(x)$ совпадает с линейными функциями $y_j^k$, $k = 1, \dots, m_j$.
Обозначим $b_j^k = y_j^k{}'(x)$.
Тогда
\begin{multline}
I(u) = \sum\limits_j \int\limits_{\Omega_j} (1 + u'^2(x))^{p(x)} dx
= \sum\limits_j \sum\limits_k \int\limits_{dom(y_j^k)} (1 + {y_j^k}'^2(x))^{p(x)} dx
\\ = \sum\limits_j \int\limits_{u(\Omega_j)} \sum\limits_k {1 \over |b_j^k|} (1 + (b_j^k)^2)^{p((y_j^k)^{-1}(y))} dy
\end{multline}

Заметим, что при симметризации участки функции $u(x)$ при $x \in \Omega_j$
объединяются в два линейных симметричных куска, причем их образ сохраняется (переписать).
$$(y_j^*)^{-1} (\pm y) = \pm {1 \over 2} \sum\limits_{k = 1}^{m_j} (-1)^k (y_j^k)^{-1}(y),
\ {1 \over b_{\pm}^*} = \pm {1 \over 2} \sum\limits_{k = 1}^{m_j} {(-1)^k \over b_j^k}.$$
Тогда
\begin{multline}
I(u^*) = \sum\limits_j \int\limits_{\Omega_j^*} (1 + u^*{}'^2(x))^{p(x)} dx
= \sum\limits_j \sum\limits_k \int\limits_{dom(y_j^*)} (1 + {y_j^*}'^2(x))^{p(x)} dx
\\ = \sum\limits_j \int\limits_{u(\Omega_j)} \big|{1 \over 2} \sum\limits_{k = 1}^{m_j} {(-1)^k \over b_j^k}\big|
\Big(1 + \frac{1}{ {1 \over 4} (\sum\limits_{k = 1}^{m_j} {(-1)^k \over b_j^k})^2 } \Big)^{p\big({1 \over 2} \sum\limits_{k = 1}^{m_j} (-1)^k (y_j^k)^{-1}(y)\big)} dy
\end{multline}

Зафиксируем $j$ и $y$.
Тогда для доказательства неравенства (\ref{toprove}) достаточно выполнения
$$\sum\limits_{k = 1}^{m_j} K({1 \over b_j^k}, (y_j^k)^{-1}(y)) \ge
K({1 \over 2} \sum\limits_{k = 1}^{m_j} {(-1)^k \over b_j^k}, {1 \over 2} \sum\limits_{k = 1}^{m_j} (-1)^k (y_j^k)^{-1}(y)).$$
А это следует из выпуклости $K(u, x)$ и леммы из статьи Брока (которая дословно распространяется на функции двух переменных).

\end{proof}

\input bibliography

\end{document}